\pdfobjcompresslevel=0
\documentclass[a4paper, 11pt, twoside, titlepage]{article}
\usepackage[T1]{fontenc}
\usepackage[utf8]{inputenc}
\usepackage[italian]{babel}
\usepackage{amsthm}
\usepackage{verbatim}
\usepackage{blindtext}
\usepackage{tcolorbox}
\usepackage{amssymb}
\usepackage{bm}
\usepackage{amsmath}
\newcommand{\tens}[1]{%
  \mathbin{\mathop{\otimes}\displaylimits_{#1}}%
}
\usepackage{mathrsfs}
\makeatletter
\def\amsbb{\use@mathgroup \M@U \symAMSb}
\makeatother
\DeclareFontFamily{U}{bbold}{}
\DeclareFontShape{U}{bbold}{m}{n}
{  <5> <6> bbold5
   <7> <8> bbold7
   <9> <10> <10.95> <12> <14.4> <17.28> <20.74> <24.88> bbold10
}{}
%\usepackage{bbold}

\makeatletter
\newcommand\aint{\mathop{\mathpalette\tb@int{t}}\!\int}
\newcommand\bint{\mathop{\mathpalette\tb@int{b}}\!\int}
\newcommand\tb@int[2]{%
  \sbox\z@{$\m@th#1\int$}%
  \if#2t%
    \rlap{\hbox to\wd\z@{%
      \hfil
      \vrule width .35em height \dimexpr\ht\z@+1.4pt\relax depth -\dimexpr\ht\z@+1pt\relax
      \kern.05em % a small correction on the top
    }}
  \else
    \rlap{\hbox to\wd\z@{%
      \vrule width .35em height -\dimexpr\dp\z@+1pt\relax depth \dimexpr\dp\z@+1.4pt\relax
      \hfil
    }}
  \fi
}
\makeatother

\usepackage{commath}
\usepackage{enumerate}
\usepackage{setspace}
\usepackage{braket}
\usepackage{dsfont}
\makeatletter
\newcommand*{\rom}[1]{\expandafter\@slowromancap\romannumeral #1@}
\makeatother
\usepackage{graphicx}
\usepackage[nottoc]{tocbibind}
\usepackage[headheight= 14pt,lmargin=1.2in, rmargin=1.2in]{geometry}
\usepackage{blindtext}
\usepackage{fancyhdr}

\fancypagestyle{style1}{
\fancyhf{}
\fancyhead{}% clear header
\fancyfoot{} % clear footer
\fancyhead[R]{\thesection.\ \Sectionname}
\fancyfoot[R]{\thepage} % page at left on even and right on odd pages
}

\pagestyle{fancy}
\let\Sectionmark\sectionmark
\def\sectionmark#1{\def\Sectionname{#1}\Sectionmark{#1}}
\let\Subsectionmark\subsectionmark
\def\subsectionmark#1{\def\Subsectionname{#1}\Subsectionmark{#1}}
\let\Subsubsectionmark\subsubsectionmark
\def\subsubsectionmark#1{\def\Subsubsectionname{#1}\Subsubsectionmark{#1}}
\renewcommand\headrulewidth{0.5pt} % no line between document and header 
\fancyhf{}
\fancyhead{}% clear header
\fancyfoot{} % clear footer
\fancyhead[R]{\thesubsection.\ \Subsectionname}
\fancyfoot[R]{\thepage} % page at left on even and right on odd pages

\fancypagestyle{style3}{
\fancyhf{}
\fancyhead{}% clear header
\fancyfoot{} % clear footer
\fancyhead[OR, EL]{\thesection.\ \Sectionname}
\fancyfoot[OR, EL]{\thepage} % page at left on even and right on odd pages

}

\fancypagestyle{style2}{
\fancyhf{}
\fancyhead{}% clear header
\fancyfoot{} % clear footer % page at left on even and right on odd pages
}
\usepackage{polynom}
\usepackage{textcomp}
\usepackage{tikz-cd}
\usepackage{tkz-tab}
\usepackage{xpatch}

% tkz-tab hardcodes $0$ for the zeros
\xpatchcmd{\tkzTabLine}{$0$}{$\bullet$}{}{}
% we want solid lines
\tikzset{t style/.style={style=solid}}
\usepackage{mathtools}
\usepackage{accents}
\newcommand{\ubar}[1]{\underaccent{\bar}{#1}}
\newcommand{\ut}[1]{\underaccent{\tilde}{#1}}
\renewcommand{\vec}[1]{\ut{#1}}
\usepackage{tzplot}
\usepackage{chngcntr}
\usepackage{apptools}
\usepackage{afterpage}

\newcommand\blankpage{%
    \null
    \thispagestyle{empty}%
    \addtocounter{page}{-1}%
    \newpage}
\numberwithin{equation}{section}
\AtAppendix{\counterwithin{Lemma}{section}}
\AtAppendix{\counterwithin{Theorem}{section}}
\AtAppendix{\counterwithin{Corollary}{section}}
\AtAppendix{\counterwithin{Proposition}{section}}
\AtAppendix{\counterwithin{Definition}{section}}
\newcommand{\deq}{\overset{\cdot}{=}}
\newcommand{\cl}{\mathscr{C}\hspace{-1.5pt}\ell}
\newcommand{\scl}{c\hspace{-0.5pt}\ell}
\newcommand{\froo}{\text{Fr}_{\text{oo}}}
\newcommand{\ccl}{\amsbb{C}\hspace{-1.5pt}\ell}
\newcommand{\diro}{\mathbf{D}}
\newcommand{\wcon}{\overset{\text{weakly}}{\to}}
\newcommand{\spinro}{{S_{\amsbb{C}}}_{\rho}(M)}
\newcommand{\spinrod}{{S_{\amsbb{C}}}_{\rho}^\ast(M)}
\newcommand{\spinrop}{{S_{\amsbb{C}}}_{\rho}^\oplus(M)}
\newcommand{\id}{\text{id}}
\newcommand{\Gal}{\text{Gal}}
\DeclareMathOperator{\sgn}{sgn}
\DeclareMathOperator{\rank}{rank} 
\DeclareMathOperator{\Tr}{Tr}
\DeclareMathOperator{\pr}{pr}
\DeclareMathOperator*{\esssup}{ess\,sup}
\theoremstyle{definition}
\newtheorem{definition}{Definizione}[section]
\theoremstyle{plain}
\newtheorem{theorem}{Teorema}[section]
\newtheorem{lemma}{Lemma}[section]
\newtheorem{exercise}{Esercizio}[section]
\newtheorem{corollary}{Corollario}[section]
\newtheorem{proposition}{Proposizione}[section]
\newtheorem{axiom}{Assioma}
\counterwithout{axiom}{section}
\theoremstyle{remark}
\newtheorem*{remark}{Osservazione}
\newtheorem*{example}{Esempio}
\usepackage{float}
\usepackage{pgfplots}
\pgfplotsset{compat=1.17, 
     %%%% common settings
     axis lines=middle,  
     axis line style= {-Straight Barb},
     axis on top,
     x=7mm, y=3.5mm, % <---
     %
     xlabel=$\mathrm{Re}$,
     xlabel style=below,
     xmin=-4.5, xmax=4.5,
     xtick={-4,-3,...,\Xmax}, % <---
     %
     ylabel=$\mathrm{Im}$,
     ylabel style=left,     
     ymin=-8.5, ymax=4.5,
     ytick={-8,-7,...,8},
     %
     tick style=black,
     tick label style = {inner sep=1pt, font=\tiny},
     %
     no marks,
     every axis plot post/.append style={very thick}
             }
\newcommand\Xmin{\pgfkeysvalueof{/pgfplots/xmin}}
     \newcommand\Xmax{\pgfkeysvalueof{/pgfplots/xmax}}
     \newcommand\Ymin{\pgfkeysvalueof{/pgfplots/ymin}}
     \newcommand\Ymax{\pgfkeysvalueof{/pgfplots/ymax}}
\usepgfplotslibrary{fillbetween}
\usepgfplotslibrary{external}
\tikzexternalize[prefix = Graphs/]
\usepackage[backend=biber]{biblatex}
\addbibresource{biblio.bib}
\usepackage{csquotes}
\usepackage[pdfa]{hyperref}
\usepackage{hyperxmp}
%% Color profile
\immediate\pdfobj stream attr{/N 3}  file{eciRGB_v2.icc}
    \pdfcatalog{%
        /OutputIntents [ <<
            /Type /OutputIntent
            /S/GTS_PDFA1
            /DestOutputProfile \the\pdflastobj\space 0 R
            /OutputConditionIdentifier (eciRGB v2)
            /Info(eciRGB v2)
        >> ]
    }
%%PDF metadata
\title{Analisi Matematica 1, Corso di Laurea in Informatica - Appunti delle Esercitazioni}
\author{Valentino Abram}
\hypersetup{%
            pdftitle={\xmpquote{Analisi Matematica 1\xmpcomma\ Corso di Laurea in Informatica\xmpcomma\ Appunti delle Esercitazioni}},
            pdfauthor={Valentino Abram},
             pdfsubject={Mathematical analysis},
             pdfkeywords={Lecture notes, Calculus 1},
             pdfcontactaddress={Via Sommarive 14},
             pdfcontactcity={\xmpquote{Povo\xmpcomma\ Trento}},
             pdfcontactpostcode={38123},
             pdfcontactcountry={Italy},
             pdfcontactemail={valentino.abram@unitn.it},
             pdflang={it},
             pdfmetalang={en},
             pdfapart=1,
             pdfaconformance=B,
             bookmarksopen=true,
             bookmarksopenlevel=3,
             hypertexnames=false,
                 linktocpage=true,
                 plainpages=false,
                 breaklinks
}
\usepackage[
       type={CC},
       modifier={by-sa},
       version={4.0},
       lang={italian}
     ]{doclicense}

\begin{document}
\pagenumbering{gobble}
\afterpage{\blankpage}
\pagestyle{empty}
\begin{titlepage}
    \centering
    \vspace*{\baselineskip}
    \rule{\textwidth}{1.6pt}\vspace*{-\baselineskip}\vspace*{2pt}
    \rule{\textwidth}{0.4pt}\\[\baselineskip]
    {\LARGE ANALISI MATEMATICA 1\\
    [0.2\baselineskip]
    ESERCITAZIONI}\\[0.2\baselineskip]
    \rule{\textwidth}{0.4pt}\vspace*{-\baselineskip}\vspace{3.2pt}
    \rule{\textwidth}{1.6pt}\\[\baselineskip]
    \scshape
    Appunti del Corso \\
    Analisi Matematica 1, Corso di Laurea in Informatica
    \vfill
    {\scshape Anno Accademico 2023 - 2024}
    
  \end{titlepage}
%\doclicenseThis%
\clearpage
\tableofcontents
\clearpage
%\blankpage
\pagestyle{style3}
\pagenumbering{arabic}
\section{Lezione 1}
\subsection{Polinomi: definizioni di base}
\begin{definition}
    \label{def:1.1}
    Un \emph{polinomio} $P(x)$ nella variabile $x$ di grado $n$ a coefficienti nel campo $\amsbb{K}$ (nel nostro caso $\amsbb{R}$ o $\amsbb{C}$) è un'espressione algebrica del tipo
    \[
    P(x) = a_0 + \sum_{k=1}^n a_k x^k, \qquad a_n\ne 0, \ a_0, \dots, a_n\in\amsbb{K}
    \]
    Gli elementi $a_k x^k$ sono detti \emph{monomi}, e se $a_k\ne 0$ vengono chiamati anche \emph{termini}.
\end{definition}
\begin{remark}
    Solitamente il grado di un polinomio $P(x)$ viene indicato con $\deg(P)$.\\
    I coefficienti di un polinomio di grado $n$, $\{a_k\}_{0\le k\le n}$, possono anche essere scritti come, dato $m>n$,
    \[
    \{a_k\}_{0\le k \le m}, \qquad a_k = 0 \ \text{per ogni} \ n<k\le m
    \]
    I simboli $x^k$ sono, per l'appunto, simboli: non hanno, per ora, alcun significato particolare.
\end{remark}
\begin{example}
    \[
    \begin{split}
    P(x) = a_5x^5+a_3x^3+a_0 &\qquad \deg(P) = 5\\
    P(x) = 0x^b+a_1x^1, \ b>1 & \qquad \deg(P)=1\\
    P(x) = a_0 = a_0x^0 &\qquad \deg(P)=0\\
    P(x)=0 &\qquad \deg(P)=-\infty \ \text{per convenzione}
    \end{split}
    \]
\end{example}
\begin{definition}
    \label{def:1.2}
    Siano $P(x)$ un polinomio di grado $n$ i cui coefficienti sono $\{a_k\}_{0\le k \le n} = \{a_0, \dots, a_n\}$ e $Q(x)$ un polinomio di grado $m$ i cui coefficenti sono $\{b_i\}_{0\le i \le m} = \{b_0, \dots, b_m\}$. La \emph{somma algebrica} di $P$ e $Q$, indicata con $P+Q$, è il polinomio di grado $d \le \max\{n,m\}$
    \[
    (P+Q)(x) = c_0 + \sum_{k=1}^{d}c_k x^k
    \]
    i cui coefficienti $\{c_k\}_{0\le k \le d}$ sono dati da:
    \begin{enumerate}[(i)]
        \item nel caso $n=m$,  
        \[
        c_k = a_k + b_k \ \text{per ogni} \ 0\le k \le n 
        \]
        \item nel caso $n>m$, 
        \[
        c_k = \begin{dcases}
            a_k + b_k\, & 0\le k \le m\\
            a_k\, & m<k\le n
        \end{dcases}
        \]
        \item nel caso $m>n$
        \[
        c_k = \begin{dcases}
            a_k + b_k\, & 0\le k \le n\\
            b_k\, & n<k\le m
        \end{dcases}
        \]
    \end{enumerate}
\end{definition}
\begin{remark}
    Può accadere che il grado $d$ di $P+Q$ sia minore di $\max\{n,m\}$, ad esempio se $n=m$ e $b_n = -a_n$.\\
    Si può facilmente verificare che la somma algebrica così definita è commutativa, associativa e che il polinomio $P(x)=0$ è l'elemento neutro della somma algebrica. Inoltre, per ogni polinomio $P(x)$ esiste un polinomio $Q(x)$ tale che $(P+Q)(x)=0$.
\end{remark}
\begin{definition}
    \label{def:1.3}
    Dato un polinomio $P(x)$ di grado $n$ avente coefficienti $\{a_k\}_{0\le k \le n}$ e un numero (detto \emph{scalare}) $\lambda\in \amsbb{K}$, la \emph{moltiplicazione per scalare} di $\lambda$ e $P$ è il polinomio $\lambda P$ di grado $d\le n$ i cui coefficienti $\{c_k\}_{0\le k \le d}$ sono dati da
    \[
    c_k = \lambda a_k \ \text{per ogni} \ 0\le k \le n
    \]
\end{definition}
\begin{remark}
    Si può facilmente verificare che la moltiplicazione per scalare è distributiva rispetto alla somma algebrica di polinomi.
\end{remark}
In realtà, poiché possiamo intendere un elemento del campo $\amsbb{K}$ come un polinomio di grado $0$, la moltiplicazione per scalare è un caso particolare della moltiplicazione tra polinomi:
\begin{definition}
    \label{def:1.4}
    Dati un polinomio $P(x)$ di grado $n$ e coefficienti $\{a_k\}_{0\le k \le n}$ e un polinomio $Q(x)$ di grado $m$ e coefficienti $\{b_k\}_{0\le k \le m}$, il \emph{prodotto di $P$ e $Q$}, indicato con $PQ$, è un polinomio di grado $n+m$ i cui coefficienti $\{c_k\}_{0\le k \le n+m}$ sono dati da
    \[
    c_k = \sum_{i+j=k} a_ib_j
    \]
    ossia
    \[
    c_0 = a_0b_0, \quad c_1 = a_0b_1 + a_1b_0, \quad c_2 = a_2b_0 + a_1 b_1 + a_0b_2 \ \dots
    \]
\end{definition}
\begin{remark}
    Si può verificare che il prodotto fra polinomi è associativo, commutativo e che è distributivo rispetto alla somma algebrica. Inoltre, il polinomio $P(x) = 1$ è l'elemento neutro del prodotto fra polinomi.\\
    Le operazioni così definite fanno sì che i simboli $x^k$ soddisfino le usuali proprietà delle potenze: $x^kx^l = x^{k+l}$ etc.
\end{remark}
Sino ad ora, abbiamo parlato dei polinomi come oggetti astratti su cui abbiamo definito delle operazioni (che rendono l'insieme dei polinomi su un dato campo un anello commutativo, che si indica con $\amsbb{K}[x]$). Possiamo però pensare di \emph{valutare i polinomi sugli elementi del campo $\amsbb{K}$}: dati un numero $a\in\amsbb{K}$ e un polinomio $P(x)$, possiamo calcolare il numero $P(a)$ sostituendo alla $x$ il numero $a$, in simboli
\[
\amsbb{K}\ni a \overset{P(x)}{\mapsto} P(a)\in\amsbb{K}
\]
$P(x)$ quindi induce un modo per assegnare ad ogni elemento del campo $\amsbb{K}$ un elemento del campo $\amsbb{K}$, ossia $P(x)$ induce una \emph{funzione polinomiale}.
\begin{example}
    Questo modo di intendere i polinomi consente di visualizzare meglio le operazioni fra polinomi precedentemente definite in modo molto astratto, applicando ai vari termini dei polinomi (che risultano essere numeri nell'accezione illustrata) le solite manipolazioni algebriche; ad esempio,
    \[
    \begin{split}
    (x^2+5x+2)(x^3+6x) &\overset{\text{prop. associativa}}{=} (x^2+5x+2)\left((x^3)+(6x)\right)\overset{\text{prop. distributiva}}{=}\\
    & = (x^2+5x+2)x^3+(x^2 + 5x + 2)6x \overset{\text{prop. distributiva}}{=}\\
    & = x^5+5x^4+2x^3+6x^3+30x^2+12x = \\
    & = x^5+5x^4+8x^3+30x^2+12x
    \end{split}
    \]
\end{example}
\subsection{Divisione fra polinomi}
\begin{theorem}
    \label{th:1.1}
    Siano $P_1(x)$ e $P_2(x)$ due polinomi di grado rispettivamente $n$ e $m$ con $n\ge m$, e $P_2(x)\not\equiv 0$ (ossia $P_2(x)$ diverso dal polinomio nullo). Esistono due polinomi $Q(x)$ (detto \emph{quoziente}) e $R(x)$ (detto \emph{resto}) tali che
    \[
    P_1(x) = P_2(x)Q(x)+R(x)
    \]
    con $\deg(Q) = \deg(P_1)-\deg(P_2)$ e $\deg(R)<m$.
\end{theorem}
\begin{example}
    Esiste un algoritmo che consente di determinare agilmente i polinomi $Q(x)$ e $R(x)$. Lo illustriamo nel caso esempio
    \[
    P_1(x) = x^5+2x^3+x+3 \qquad P_2(x) = x^3+3
    \]
    \begin{enumerate}[(i)]
        \item Scriviamo i polinomi mettendo i termini in ordine \emph{decrescente} e lasciando lo spazio per eventuali termini mancanti
        \[
        \polylongdiv[style=D, stage=0]{x^5+2x^3+x+3}{x^3+3}
        \]
        \item Effettuiamo la divisione fra i monomi di grado maggiore di $P_1(x)$ e $P_2(x)$, nel nostro caso $x^5$ e $x^3$, ottenendo $x^2$
        \[
        \polylongdiv[style=D, stage=2]{x^5+2x^3+x+3}{x^3+3}
        \]
        \item Moltiplichiamo il polinomio $P_2(x) = x^3+3$ per il risultato ottenuto e sottraiamolo a $P_1(x)$
        \[
        \polylongdiv[style=D,stage=4]{x^5+2x^3+x+3}{x^3+3}
        \]
        ottenendo il polinomio $R_1(x) = 2x^3-3x^2+x+3$;
        \item poiché $\deg(R_1)\ge \deg(P_2)$, ripetiamo la procedura precedente per i polinomi $R_1(x)$ e $P_2(x)$: effettuiamo la divisione fra i termini di grado maggiore, $2x^3$ e $x^3$
        \[
        \polylongdiv[style=D,stage=5]{x^5+2x^3+x+3}{x^3+3}
        \]
        moltiplichiamo $P_2(x)$ per il risultato ottenuto e sottraiamolo al polinomio $R_1(x)$, 
        \[
        \polylongdiv[style=D,stage=7]{x^5+2x^3+x+3}{x^3+3}
        \]
        ottenendo così $R_2(x) = -3x^2+x-3$;
        \item in questo caso $\deg(R_2)<\deg(P_2)$, e abbiamo quindi finito. Il polinomio $Q(x)$ sarà il polinomio ottenuto sommando i quozienti intermedi, nel nostro caso $Q(x) = x^2+2$, e il polinomio $R(x)$ sarà il polinomio ottenuto dall'ultima sottrazione, nel nostro caso $R(x)= R_2(x)$.
    \end{enumerate}
    Quindi 
    \[
    P_1(x) = P_2(x)(x^2+2)+(-3x^2+x-3)
    \]
\end{example}
\begin{definition}
    \label{def:1.5} 
    Dati due polinomi $P_1(x)$ e $P_2(x)$ che soddisfano le ipotesi del teorema \ref{th:1.1}, diremo che $P_1(x)$ è \emph{divisibile} per $P_2(x)$ se il polinomio di resto $R(x)$ è il polinomio nullo, ossia se esiste un polinomio $Q(x)$ tale che
    \[
    P_1(x) = P_2(x)Q(x)
    \]
\end{definition}
Nel caso in cui il polinomio divisore $P_2(x)$ sia un polinomio del tipo $P_2(x) = x-c$ con $c\in\amsbb{K}$, esiste un comodo criterio per determinare se $P_1(x)$ con $\deg(P_1)\ge1$ è divisibile per $P_2$:
\begin{proposition}
    \label{prop:1.1}
    Dati due polinomi $P_1(x)$ e $P_2(x)$ con $P_2(x) = x-c$, $c\in\amsbb{K}$ e $\deg(P_1)\ge 1$, $P_1$ è divisibile per $P_2$ se e solo se $P_1(c)=0$.
\end{proposition}
\begin{proof}
    Mostriamo le due implicazioni.
    \begin{enumerate}[(i)]
        \item Supponiamo che $P_1$ sia divisibile per $P_2$; allora per la definizione \ref{def:1.5} esiste un polinomio $Q(x)$ tale che
        \[
        P_1(x) = (x-c)Q(x)
        \]
        Se valutiamo ambo i membri dell'uguaglianza in $c$ otteniamo
        \[
        P_1(c) = (c-c)Q(x) = 0
        \]
        ossia $P_1(c)=0$.
        \item Supponiamo ora che $P_1(c)=0$. Poiché $P_1$ e $P_2$ soddisfano le ipotesi del teorema \ref{th:1.1}, sappiamo che esistono due polinomi $Q(x)$ e $R(x)$ tali che
        \[
        P_1(x) = (x-c)Q(x)+R(x)
        \]
        con $\deg(R)<\deg(P_2) = 1$, ossia $R(x) = r_0$ per qualche $r_0\in\amsbb{K}$; quindi
        \[
        P_1(x) = (x-c)Q(x)+r_0
        \]
        Sappiamo che $P_1(c)=0$; valutando ambo i membri dell'espressione precedente in $x=c$ otteniamo
        \[
        0=P_1(c) = (c-c)Q(x)+r_0 = r_0
        \]
        ossia $0=r_0 = R(x)$. Quindi per la definizione \ref{def:1.5} $P_1$ è divisibile per $P_2(x) = x-c$.\qedhere
    \end{enumerate}
\end{proof}
\begin{definition}
    \label{def:1.6} Se $a\in\amsbb{K}$ è tale per cui $P(a)=0$, $a$ è detto \emph{radice} del polinomio $P$.
\end{definition}
\begin{proposition}
    \label{prop:1.2}
    Dato un polinomio $P(x)$ di grado $n$, questo ammette al massimo $n$ radici distinte.
\end{proposition}
\begin{proof}
    Procediamo per induzione:
    \begin{enumerate}[(i)]
        \item \emph{Passo base $n=0$}: se $P(x)$ è un polinomio di grado $0$ allora $P(x) = a_0$ con $a_0\ne 0$; di conseguenza $P(x)$ non ammette radici, e il risultato è verificato.
        \item \emph{Passo induttivo}: supponiamo che il risultato valga per polinomi di grado $n$, e dimostriamo che vale anche per polinomi di grado $n+1$. Sia $P(x)$ un polinomio di grado $n+1$. Esistono due possibilità: $P(x)$ può non ammettere radici, nel qual caso il risultato è verificato, oppure può ammettere almeno una radice, sia essa $a\in\amsbb{K}$. Per la definizione \ref{def:1.6}, sappiamo che $P(a) = 0$; per la proposizione \ref{prop:1.1} allora vale che $P(x)$ è divisibile per $(x-a)$, ossia
        \[
        P(x)=(x-a)Q(x)
        \]
        con $\deg(Q) = \deg(P)-1 = n$. Per ipotesi induttiva, sappiamo che $\deg(Q)$ ammette al massimo $n$ radici distinte; di conseguenza $P(x)$ ammette come radici le radici di $Q(x)$, al massimo $n$ distinte, e $a$, e quindi ammette al massimo $n+1$ radici distinte. \qedhere
    \end{enumerate}
\end{proof}
\begin{example}
    Nel caso della divisione fra polinomi in cui il divisore è della forma $P_1(x) = x-c$, esiste un altro algoritmo, più veloce, che consente di effettuare la divisione, l'\emph{algoritmo di Ruffini}. Consideriamo i polinomi
    \[
    P_1(x) = x^3+4x+1 \qquad P_2(x) = x-2
    \]
    \begin{enumerate}[(i)]
        \item Scriviamo i polinomi mettendo i termini in ordine \emph{decrescente};
        \item Inseriamo in uno schema simile a quello seguente i coefficienti $\{1, 0, 4, 1\}$ di $P_1(x)$, nella riga superiore, avendo cura di inserire degli zeri per i coefficienti dei termini mancanti, e la radice $x=2$ del polinomio $P_2(x)$ nella riga inferiore
        \[
        \polyhornerscheme[x=2, tutor=true, stage=1, resultleftrule=true]{x^3+4x+1}
        \]
        \item Abbassiamo il primo coefficiente, 1, sotto la riga orizzontale; questo sarà il coefficiente del termine di grado $\deg(P_1)-1$
        \[
        \polyhornerscheme[x=2, tutor=true, stage=2, resultleftrule=true]{x^3+4x+1}
        \]
        \item Moltiplichiamo il coefficiente, 1, ottenuto in questo modo per la radice di $P_1(x)$, 2, e scriviamo il risultato sopra la linea di demarcazione sotto il coefficiente successivo a quello precedentemente utilizzato, in questo caso lo 0.  
        \[
        \polyhornerscheme[x=2, tutor=true, stage=3, resultleftrule=true]{x^3+4x+1}
        \]
        \item Sommiamo i due numeri incolonnati, riportando il risultato sotto la linea di demarcazione orizzontale
        \[
        \polyhornerscheme[x=2, tutor=true, stage=4, resultleftrule=true]{x^3+4x+1}
        \]
        \item Ripetiamo la procedura precedente, ossia moltiplichiamo il numero ottenuto, 2,  per la radice di $P_2(x)$ 2, e riportiamo il risultato sotto il coefficiente successivo
        \[
        \polyhornerscheme[x=2, tutor=true, stage=5, resultleftrule=true]{x^3+4x+1}
        \]
        ed effettuiamo la somma dei numeri incolonnati
        \[
        \polyhornerscheme[x=2, tutor=true, stage=6, resultleftrule=true]{x^3+4x+1}
        \]
        finché non terminiamo lo scorrimento dei coefficienti superiori:
        \[
        \polyhornerscheme[x=2, tutor=true, stage=7, resultleftrule=true]{x^3+4x+1}
        \]
        e infine
        \[
        \polyhornerscheme[x=2, tutor=true, stage=8, resultleftrule=true]{x^3+4x+1}
        \]
        \item Il numero sotto la linea di demarcazione orizzontale a destra del separatore verticale è il resto $R(x)$ della divisione, nel nostro caso 17, mentre i numeri a sinistra del separatore verticale sono i coefficienti dei termini del quoziente $Q(x)$: quindi
        \[
        P_1(x) = P_2(x)(x^2 + 2x+8)+17
        \]
    \end{enumerate}
\end{example}
\subsection{Fattorizzazione dei polinomi}
\begin{definition}
    \label{def:1.7}
    Un polinomio $P(x)$ di grado $n\ge 1$ è detto \emph{irriducibile} se non esiste un polinomio $D(x)$ di grado $m$ con $0<m<n$ che divide esattamente $P$.
\end{definition}
\begin{theorem}
    \label{th:1.2}
    Se $\amsbb{K}=\amsbb{R}$, ossia nel caso di polinomi a coefficienti reali, gli unici polinomi irriducibili sono i polinomi di grado 1 e i polinomi di grado 2 con discriminante\footnote{Ricordiamo che dato un polinomio di grado 2 $P(x) = a_2x^2 + a_1x+a_0$, il discriminante è definito essere $\Delta = a_1^2-4a_2a_0$} negativo.
\end{theorem}
Come possiamo fattorizzare un polinomio $P(x)$ a coefficienti reali? Idealmente, procediamo nel modo seguente: cerchiamo una radice $a\in\amsbb{R}$, dividiamo $P(x)$ per $x-a$ e ripetiamo il passaggio finché non otteniamo un polinomio irriducibile, ossia una delle due tipologie indicate dal teorema \ref{th:1.2}. Ci sono delle scorciatoie che agevolano il processo:
\begin{enumerate}[(i)]
    \item Se i coefficienti del polinomio sono numeri \emph{interi}, le radici \emph{intere}, \emph{se esistono}, sono da cercarsi fra i sottomultipli \emph{interi} del termine noto.
    \begin{example}
        Ad esempio, consideriamo il polinomio
        \[
        P(x) = x^3+3x^2-25x+21
        \]
        Tutti i coefficienti appartengono a $\amsbb{Z}$, e quindi le radici intere, se esistono, sono da cercarsi nell'insieme
        \[
        \{1, -1, 3, -3, 7, -7, 21, -21\}
        \]
        Per trovare le radici intere è quindi sufficiente valutare il polinomio $P(x)$ sui numeri dell'insieme precedente. Si verifica facilmente che
        \[
        P(1)=0, \qquad P(3)=0, \qquad P(-7)=0 
        \]
        Per la proposizione \ref{prop:1.2} possiamo quindi fermarci, poiché abbiamo trovato 3 radici distinte; abbiamo quindi che
        \[
        P(x) = (x-3)(x-1)(x+7)
        \]
    \end{example}
    Questo è vero per il motivo seguente: supponiamo che $P(x)$ sia un polinomio a coefficienti interi $\{a_k\}_{0\le k \le n}$, e sia $r\in\amsbb{Z}$ una sua radice intera; vale che
    \[
    0=P(r) = a_n r^n + a_{n-1}r^{n-1}+\dots + a_1 r + a_0
    \]
    ossia
    \[
    -a_0 = a_nr^n+a_{n-1}r^{n-1}+\dots + a_1 r = \overbrace{r}^{\in\amsbb{Z}}\underbrace{(a_n r^{n-1}+a_{n-1}r^{n-1}+\dots + a_1)}_{\in \amsbb{Z}}
    \]
    Risulta quindi evidente che $r$ è un sottomultiplo intero di $a_0$.
    \item Formule di calcolo di vario tipo:
    \begin{tcolorbox}
    \vspace{-0.3cm}
        \[
        \begin{split}
            &(a+b)^2 = a^2 + 2ab+b^2 \\
            &(a+b)^3 = a^3+3a^2b+3ab^2+b^3\\
            &(a^2-b^2) = (a+b)(a-b)\\
            &(a^3+b^3) = (a+b)(a^2-ab+b^2)\\
            &(a^3-b^3) = (a-b)(a^2+ab+b^2)\\
            &(a+b+c)^2 = a^2 + b^2 + c^2 +2ab+2bc + 2ac\\
            &(x^2+sx+p) = (x+n_1)(x+n_2) \ \text{con} \ s=n_1 + n_2 \ \text{e} \ p=n_1n_2
        \end{split}
        \]
    \end{tcolorbox}
    \end{enumerate}
    \begin{example}
        Consideriamo il polinomio $P(x) = x^4+1$. Osserviamo che $P(x)>0$ per ogni $x\in\amsbb{R}$, e quindi non ammette radici; tuttavia per il teorema \ref{th:1.2} \emph{non} è irriducibile: infatti,
        \[
        \begin{split}
                P(x) &= x^4+1 =x^4+1+\underbrace{2x^2-2x^2}_{\text{abbiamo aggiunto 0}} = \\
                & = (x^4+2x^2+1)-2x^2 \overset{\text{quadrato di binomio}}{=} \\
                &= (x^2+1)^2 - (\sqrt{2}x)^2 \overset{\text{differenza di quadrati}}{=} \\
                & = (x^2 + 1 + \sqrt{2}x)(x^2 + 2 -\sqrt{2}x)
        \end{split}
        \]
    \end{example}
    \begin{example}
        Consideriamo il polinomio $P(x) = x^3 -(a+b)x^2+(ab+b^2)x-ab^2$ ove $a,b$ sono due parametri reali. Vogliamo fattorizzarlo in funzione di $a,b$:
        \begin{enumerate}[(i)]
            \item se $a=b=0$ allora $P(x) = x^3$ ed è già fattorizzato;
            \item se $a=0, \ b\ne 0$ allora $P(x) = x^3-bx^2+b^2x$;
            \[
            P(x) = x^3-bx^2+b^2x = x\underbrace{(x^2-bx+b^2)}_{\Delta<0}
            \]
            e quindi risulta fattorizzato;
            \item se $a\ne0, \ b=0$ allora $P(x) = x^3-ax^2$, ossia 
            \[
            P(x) = x^2(x-a)
            \]
            e lo abbiamo fattorizzato;
            \item infine, se $a\ne0, \ b\ne 0$. Notiamo che, poiché il termine $x^3$ ha come coefficiente $1$, se cerchiamo radici costituite da numeri o contenenti termini numerici non riusciamo più a cancellare gli elementi di $\amsbb{R}$; di conseguenza, cerchiamo radici contenenti solo i parametri: proviamo con
            \[
            P(b) = b^3-(a+b)b^2+(ab+b^2)b-ab^2 = -ab^2 + b^3 \ne 0
            \]
            \[
            P(a) = a^3 -(a+b)a^2 +(ab+b^2)a -ab^2 = 0
            \]
            Quindi $(x-a)$ è divisore esatto di $P(x)$, ed effettuando la divisione con Ruffini otteniamo
            \[
            P(x) = (x-a)\underbrace{(x^2-bx+b^2)}_{\Delta<0}
            \]
        \end{enumerate}
    \end{example}
\subsection{Disequazioni con funzioni polinomiali e razionali}
\begin{exercise}
    \label{ex:1.1}
    Determinare l'insieme delle soluzioni di 
    \begin{equation}
        \label{eq:1.1}
        \frac{x^2-x+1}{x^3-1}\ge 0
    \end{equation}
\end{exercise}
\begin{proof}[Soluzione]
    Cerchiamo di fattorizzare il numeratore ed il denominatore della funzione razionale:
    \begin{enumerate}[(i)]
        \item notiamo che $P(x) = x^2-x+1$ è irriducibile per il teorema \ref{th:1.2}, infatti
        \[
        \Delta = 1-4 <-3
        \]
        Inoltre, $P(x)>0$ per ogni $x\in\amsbb{R}$;
        \item $Q(x)=x^3-1$ è invece una differenza di cubi, che possiamo decomporre come
        \[
        Q(x) = (x-1)\underbrace{(x^2+x+1)}_{\Delta<0}
        \]
        Notiamo che, poiché $Q$ è a denominatore, le sue radici vanno escluse dal dominio di esistenza della funzione razionale, e quindi eventualmente dalle soluzioni della disequazione.
    \end{enumerate}
    Quindi la disequazione (\ref{eq:1.1}) risulta essere
    \[
    \frac{\overbrace{x^2-x+1}^{>0 \ \forall x\in\amsbb{R}}}{(x-1)(x^2+x+1)}\ge 0
    \]
    e la positività della funzione razionale dipende unicamente dalla quantità a denominatore; andiamo quindi a studiarne il segno:\\
    
    \begin{center}
        \begin{tikzpicture}
            \tikzset{h style/.style = {fill=black!30}}
            \tkzTabInit[lgt=3,espcl=2,deltacl=0]
            { /.8, $(x-1)$ /.8, $(x^2+x+1)$ /.8, $Q(x)$ /.8}
            {,$1$, } % four main references
            \tkzTabLine {,-,z,+,} % seven denotations
            \tkzTabLine {,+,t,+,}
            \tkzTabLine {,-,t,h,}
            \path (N23) -- (N24) node[black,midway,inner sep=2pt,draw,circle,fill=white]{};
            \path (M23) -- (M24) node[black,midway]{+};
        \end{tikzpicture}
    \end{center}
    Di conseguenza l'insieme delle soluzioni di (\ref{eq:1.1}) è 
    \[
    \left\{x\in\amsbb{R} \ \colon \ x>1\right\}
    \]
\end{proof}
\newpage
\section{Lezione 2}
\subsection{Ripasso: maggioranti, minoranti, estremo superiore e inferiore}
\begin{definition}
    \label{def:2.1}
    Sia $S\subseteq \amsbb{R}$, $S\ne \varnothing$. Diremo che $x\in\amsbb{R}$ è un \emph{maggiorante} di $S$ se
    \[
    x\ge y \ \text{per ogni} \ y\in S
    \]
    Diremo invece che $x\in\amsbb{R}$ è un \emph{minorante} di $S$ se 
    \[
    x\le y \ \text{per ogni} \ y\in S
    \]
    Se $S$ ammette un minorante o un maggiorante, diremo rispettivamente che $S$ è \emph{limitato inferiormente} o \emph{superiormente}.
\end{definition}
\begin{definition}
    \label{def:2.2}
    Sia $S\subseteq \amsbb{R}$, $S\ne \varnothing$ un sottoinsieme limitato superiormente. Se esiste $\alpha\in\amsbb{R}$ tale che
    \begin{enumerate}[(i)]
        \item $\alpha$ è maggiorante di $S$
        \item se $\gamma <\alpha$ allora $\gamma$ \emph{non} è maggiorante di $S$
    \end{enumerate}
    allora $\alpha$ è unico ed è detto \emph{estremo superiore} di $S$, e scriveremo $\alpha = \sup S$.\\
    Analogamente, se $S$ è limitato inferiormente ed esiste $\beta\in\amsbb{R}$ tale che
    \begin{enumerate}[(i)]
        \item $\beta$ è minorante di $S$
        \item se $\gamma>\beta$ allora $\gamma$ \emph{non} è minorante di $S$
    \end{enumerate}
    allora $\beta$ è unico ed è detto \emph{estremo inferiore} di $S$, in notazione $\beta = \inf S$.\\
    Se $\alpha = \sup S \in S$, allora $\alpha$ verrà detto \emph{massimo}, e se $\beta = \inf S \in S$, $\beta$ verrà detto \emph{minimo}.
\end{definition}
\begin{theorem}[Least-upper-bound property]
    \label{th:2.1}
    Se $S\subseteq \amsbb{R}$, $S\ne \varnothing$ è limitato superiormente, allora $S$ ammette un estremo superiore.
\end{theorem}
\begin{corollary}
    \label{cor:2.1}
    Sia $S\subseteq \amsbb{R}$, $S\ne \varnothing$ limitato inferiormente. Se denotiamo con $L$ l'insieme dei minoranti di $L$, avremo che
    \[
    \beta = \sup L
    \]
    esiste e $\beta = \inf S$.
\end{corollary}
\begin{proof}
    Per ipotesi, sappiamo che $S$ è limitato inferiormente, e di conseguenza $S$ ammette almeno un minorante; abbiamo quindi che $L\ne \varnothing$. Inoltre, fissato un qualsiasi $y\in S$, 
    \[
    x\le y \ \text{per ogni} \ x\in L
    \]
    poiché $L$ è l'insieme dei minoranti di $S$ (cfr. definizione \ref{def:2.1}). Di conseguenza $y$ è un maggiorante di $L$ e $L$ è limitato superiormente, e per il teorema \ref{th:2.1} ammette un estremo superiore; sia $\beta = \sup L$.\\
    Vogliamo ora mostrare che $\beta$ è l'estremo inferiore di $S$, ossia che $\beta$ soddisfa i punti (i) e (ii) della seconda parte della  definizione \ref{def:2.2}. 
    \begin{enumerate}[(i)]
        \item Consideriamo la definizione \ref{def:2.2} di estremo superiore; per il punto (ii) vale che se $\gamma <\beta$ allora $\gamma$ non è maggiorante di $L$. Abbiamo notato prima che se $y\in S$, allora $y$ è un maggiorante di $L$, scritto in simboli
        \[
        \gamma\in S \implies \gamma \ \text{è un maggiorante di}\ L
        \]
        Questa scrittura è equivalente\footnote{In logica proposizionale $A\implies B$ è equivalente a $\neg B \implies \neg A$.} a
        \[
        \gamma\  \text{non è un maggiorante di} \ L \implies \gamma \notin S
        \]
        Di conseguenza, 
        \[
        \gamma < \beta \implies \gamma \ \text{non è un maggiornate di} \ L \implies \gamma \notin S
        \]
        che è equivalente a 
        \[
        \gamma \in S \implies \gamma \ge \beta 
        \]
        ossia $\beta$ è un minorante di $S$.
        \item Consideriamo ora $\gamma>\beta$; vogliamo mostrare che $\gamma$ non è un minorante di $S$. Ricordiamo che $\beta = \sup L$; di conseguenza se $\gamma>\beta$, allora
        \[
        \gamma > x \ \text{per ogni} \ x\in L
        \]
        In particolare ne consegue che $\gamma\notin L$; ma ricordiamo che $L$ è l'insieme dei minoranti di $S$, e quindi $\gamma$ non è un minorante di $S$. \qedhere
    \end{enumerate}
\end{proof}
\subsection{Esercizi}
\begin{exercise}
    \label{ex:2.1}
    Determinare $\sup$ e $\inf$, ed eventualmente massimo e minimo, dell'insieme
    \[
    A = \left\{ a_n = \frac{\cos(\pi n)}{n^2+1}, \ n\in\amsbb{N}\right\}
    \]
\end{exercise}
\begin{proof}[Soluzione]
    Per cercare di capire com'è fatto l'insieme, scriviamone i primi elementi:
    \[
    a_0 = \frac{\cos(0)}{0^2+1} = 1 \quad a_1 = \frac{\cos(\pi)}{1^2+1} = -\frac{1}{2}\quad a_2 = \frac{\cos(2\pi)}{2^2+1} = \frac{1}{5} \quad  a_3 = \frac{\cos(3\pi)}{3^2+1} = -\frac{1}{10} \ \dots
    \]
    Notiamo quindi che
     \[
     a_0, a_2 >0  \ \text{e} \ a_1, a_3<0
         \qquad \abs{a_0}>\abs{a_1}>\abs{a_2}>\abs{a_3}
    \]
    ossia sembrerebbe che per $n$ pari gli elementi dell'insieme siano positivi e che per $n$ dispari siano negativi, e che in valore assoluto questi decrescano al crescere di $n$. Vogliamo provare a dimostrare queste due proprietà.
    \begin{enumerate}[(i)]
        \item Notiamo che
        \[
        \cos(\pi n) = \begin{dcases}
            1\, & n = 2k, \ k\in \amsbb{N}\\
            -1\, & n=2k+1, \ k\in\amsbb{N}
        \end{dcases} = (-1)^n
        \]
        e di conseguenza la prima ipotesi sul comportamento dei termini dell'insieme è verificata.
        \item Vogliamo mostrare che
        \[
        \abs{a_n}>\abs{a_{n+1}} \ \text{per ogni} \ n\in\amsbb{N}
        \]
        Notiamo che grazie al punto (i) vale che
        \[
        a_n = \frac{\cos(\pi n)}{n^2+1} = \frac{(-1)^n}{n^2+1}
        \]
        ossia 
        \[
        \abs{a_n} = \frac{1}{n^2+1}
        \]
        Vogliamo quindi mostrare che 
        \[
        \frac{1}{n^2+1}>\frac{1}{(n+1)^2+1} \ \text{per ogni} \ n\in\amsbb{N}
        \]
        Consideriamo quindi la disequazione
        \[
        \frac{1}{n^2+1}>\frac{1}{(n+1)^2+1} \iff \frac{1}{n^2+1}-\frac{1}{(n+1)^2+1}>0
        \]
        di cui cerchiamo le soluzioni in $\amsbb{N}$. Abbiamo che
        \[
        \frac{1}{n^2+1}-\frac{1}{(n+1)^2+1} = \frac{(n+1)^2+1-(n^2+1)}{(n^2+1)((n+1)^2+1)} = \frac{2n+1}{(n^2+1)((n+1)^2+1)}
        \]
        e quindi la disequazione risulta essere
        \[
        \frac{\overbrace{2n+1}^{>0 \ \forall n\in\amsbb{N}}}{\underbrace{(n^2+1)}_{>0 \ \forall n\in\amsbb{N}}\underbrace{((n+1)^2+1)}_{>0 \ \forall n\in\amsbb{N}}}>0
        \]
        che è soddisfatta per ogni $n\in\amsbb{N}$; di conseguenza $\abs{a_n}>\abs{a_{n+1}}$ per ogni $n\in\amsbb{N}$.
    \end{enumerate}
    In conseguenza del punto (ii) concludiamo immediatamente che
    \[
    \abs{a_n}< \abs{a_{n-1}}<\dots < \abs{a_0} = 1
    \]
    ossia $1>a_n>-1$ per ogni $n\in\amsbb{N}$; quindi $A\subseteq [-1,1]$ è limitato superiormente e inferiormente, ed ammette quindi estremo superiore ed inferiore per il teorema \ref{th:2.1} e il corollario \ref{cor:2.1}.\\
    Notiamo che possiamo restringere l'intervallo in cui $A$ è incluso notando che
    \[
    a_1 = -\frac{1}{2}<a_n \ \text{per ogni} \ n\in\amsbb{N}
    \]
    Infatti, chiaramente $a_1 <a_{2k}$ per ogni $k\in\amsbb{N}$, mentre vale che
    \[
    \abs{a_1} \ge \abs{a_{2k+1}} \iff a_1 \le a_{2k+1} \ \text{per ogni} \ k\in\amsbb{N}
    \]
    ossia $A\subseteq \left[-\frac{1}{2}, 1\right]$; poiché $1=a_0\in A$ e $-\frac{1}{2}=a_1\in A$, vale che
    \[
    1 = \sup A = \max A \qquad -\frac{1}{2} = \inf A = \min A
    \]
\end{proof}
\begin{exercise}
    \label{ex:2.2}
    Determinare $\sup$ e $\inf$, ed eventualmente massimo e minimo, dell'insieme
    \[
    A = \left\{ a_n = \frac{n^2-5}{n^2+2}, \ n\in\amsbb{N}\right\}
    \]
\end{exercise}
\begin{proof}[Soluzione]
    Anche in questo caso per cercare di capire com'è fatto l'insieme scriviamone i primi elementi:
    \[
    a_0 = \frac{0^2-5}{0^2+2} = -\frac{5}{2} \quad a_1 = \frac{1^2-5}{1^2+2} = -\frac{4}{3} \quad a_3 = \frac{3^2-5}{3^2+2} = \frac{4}{11} \quad a_4 = \frac{4^2-5}{4^2+2} = \frac{11}{18} \ \dots
    \]
    In questo caso sembrerebbe che $a_{n+1}>a_n$ per ogni $n\in\amsbb{N}$; lo possiamo dimostrare come prima, ossia considerando la disequazione
    \[
    \frac{(n+1)^2-5}{(n+1)^2+2}>\frac{n^2-5}{n^2+2} \iff \frac{(n+1)^2-5}{(n+1)^2+2}-\frac{n^2-5}{n^2+2}>0
    \]
    e cercandone le soluzioni in $n\in\amsbb{N}$. Vale che
    \[
    \begin{split}
        &\frac{(n+1)^2-5}{(n+1)^2+2}-\frac{n^2-5}{n^2+2} = \frac{((n+1)^2-5)(n^2+2)-(n^2-5)((n+1)^2+2)}{(n^2+2)((n+1)^2+2)} = \\
        & = \frac{n^2(n+1)^2+2(n+1)^2-5n^2-10-n^2(n+1)^2-2n^2+5(n+1)^2+10}{(n^2+2)((n+1)^2+2)} = \\
        & = \frac{2n^2+4n+2-7n^2+5n^2+10n+5}{(n^2+2)((n+1)^2+2)} = \frac{14n+7}{(n^2+2)((n+1)^2+2)}
    \end{split}
    \]
    e quindi la disequazione risulta essere
    \[
    \frac{\overbrace{14n+7}^{>0 \ \forall n\in\amsbb{N}}}{\underbrace{(n^2+2)}_{>0 \ \forall n\in\amsbb{N}}\underbrace{((n+1)^2+2)}_{>0 \ \forall n\in \amsbb{N}}}>0
    \]
    che è soddisfatta per ogni $n\in\amsbb{N}$. Di conseguenza sappiamo che $A \subseteq \left[-\frac{5}{2}, +\infty\right)$, ed $A$ è quindi limitato inferiormente. Per quanto riguarda eventuali maggioranti invece, possiamo notare come gli elementi $a_n$ dell'insieme si avvicinino ad $1$ al crescere di $n$. Ragionando in maniera non rigorosa, al crescere di $n$ il contributo dei termini $-5$ a numeratore e $+2$ a denominatore diminuisce, e pertanto dominerebbero i termini di $n^2$; inoltre, poiché a numeratore abbiamo $n^2-5$ e a denominatore abbiamo invece $n^2+2$, il denominatore sarà sempre più grande del denominatore, e quindi i termini saranno sempre minori di $1$. Vogliamo quindi mostrare che
    \[
    \frac{n^2-5}{n^2+2}<1 \ \text{per ogni} \ n\in\amsbb{N}
    \]
    Notiamo che la disequazione è equivalente a
    \[
    \frac{n^2-5}{n^2+2}-1 <0 
    \]
    Consideriamo quindi
    \[
    \frac{n^2-5}{n^2+2}-1 = \frac{n^2-5-n^2-2}{n^2+2} = -\frac{7}{\underbrace{n^2+2}_{>0 \ \forall n\in\amsbb{N}}}
    \]
    che effettivamente è minore di 0 per ogni $n\in\amsbb{N}$. Quindi abbiamo che $A\subseteq\left[-\frac{5}{2}, 1\right)$, e di conseguenza $A$ è anche limitato superiormente; per il teorema \ref{th:2.1} e per il corollario \ref{cor:2.1} $A$ ammette di conseguenza estremo superiore ed inferiore.\\
    Ragionando come nell'esercizio \ref{ex:2.1} si può mostrare che $-\frac{5}{2} = a_0\in A$ è estremo inferiore e minimo di $A$; per quanto riguarda l'estremo superiore, ragionando sempre in modo non formale, al crescere di $n$ la frazione assomiglia sempre più ad un oggetto del tipo $\frac{n^2}{n^2}$, ossia si avvicina sempre più ad $1$. La nostra ipotesi è quindi che $1$ sia l'estremo superiore di $A$. Come facciamo a dimostrarlo? Sfruttiamo la seguente caratterizzazione dell'estremo superiore:
    \begin{tcolorbox}
        \begin{theorem}[Caratterizzazione dell'estremo superiore]
            \label{th:2.2}
            Sia $S\subseteq \amsbb{R}$, $S\ne \varnothing$ un insieme limitato superiormente; $\alpha\in \amsbb{R}$ maggiornate di $S$ è l'estremo superiore di $S$ se per ogni $\varepsilon>0$ esiste $s_{\varepsilon}\in A$ tale che $s_\varepsilon>\alpha-\varepsilon$.
        \end{theorem}
    \end{tcolorbox}
    Fissiamo quindi un generico $\varepsilon>0$; vogliamo mostrare che esiste un elemento $a_\varepsilon\in A$ tale che $a_\varepsilon>1-\varepsilon$. Poiché gli elementi di $A$ sono in corrispondenza biunivoca con i numeri naturali, $a_\varepsilon = a_{n_\varepsilon}$ per un qualche $n_\varepsilon \in \amsbb{N}$; dobbiamo quindi risolvere il seguente problema: dato $\varepsilon>0$, esiste $n_\varepsilon\in\amsbb{N}$ tale che
    \[
    1-\varepsilon<\frac{n_\varepsilon^2-5}{n_\varepsilon^2+2} \ \text{?}
    \]
    Consideriamo quindi la disequazione
    \[
    1-\varepsilon < \frac{n_\varepsilon^2-5}{n_\varepsilon^2+2} \iff 1-\varepsilon - \frac{n_\varepsilon^2-5}{n_\varepsilon^2+2} <0
    \]
    Possiamo scrivere
    \[
    1-\varepsilon-\frac{n_\varepsilon^2-5}{n_\varepsilon^2+2} = \frac{(1-\varepsilon)(n_\varepsilon^2+2)-n_\varepsilon^2+5}{n_\varepsilon^2+2} = \frac{-\varepsilon n_\varepsilon^2+7-2\varepsilon}{n_\varepsilon^2+2}
    \]
    e la disequazione è
    \[
    \frac{-\varepsilon n_\varepsilon^2 + 7 -2\varepsilon}{\underbrace{n_\varepsilon^2+2}_{>0 \ \forall n_\varepsilon \in \amsbb{N}}}<0
    \]
    Notiamo che se $7-2\varepsilon<0$, ossia se $\varepsilon>\frac{7}{2}$, il numeratore è sempre negativo, e quindi la disuguaglianza è verificata per ogni $n_\varepsilon\in\amsbb{N}$; quindi
    \[
    a_0> 1-\varepsilon
    \]
    Se invece $\varepsilon\ge\frac{7}{2}$ bisogna scegliere $n_\varepsilon$ in maniera più oculata. Notiamo che il polinomio $-\varepsilon n_\varepsilon^2 +7-2\varepsilon$ ammette due radici, eventualmente coincidenti nel caso $\varepsilon=\frac{7}{2}$,
    \[
    r_1 = -\sqrt{\frac{7}{\varepsilon}-2} \qquad r_2 = +\sqrt{\frac{7}{\varepsilon}-2}
    \]
    e che possiamo scrivere
    \[
    -\varepsilon n_\varepsilon^2 +7-2\varepsilon = -\varepsilon(n_\varepsilon-r_1)(n_\varepsilon-r_2)
    \]
    Studiamo quindi il segno del polinomio
    \begin{center}
        \begin{tikzpicture}
            \tikzset{h style/.style = {fill=black!30}}
            \tkzTabInit[lgt=4,espcl=2,deltacl=0]
            { /.8, $(n_\varepsilon-r_1)$ /.8, $(n_\varepsilon-r_2)$ /.8, $-\varepsilon(n_\varepsilon-r_1)(n_\varepsilon-r_2)$ /.8}
            {,$r_1$, $r_2$, } % four main references
            \tkzTabLine {,-,z,+, t, +, } % seven denotations
            \tkzTabLine {,-,t,-, z, +, }
            \tkzTabLine {,h,z,+,z, h, }
            \path (M13) -- (M14) node[black,midway]{-};
            \path (M33) -- (M34) node[black,midway]{-};
        \end{tikzpicture}
    \end{center}
    ossia $-\varepsilon n_\varepsilon^2+7-2\varepsilon$ è negativo se $n_\varepsilon <r_1$ o $n_\varepsilon> r_2$; poiché i numeri naturali sono contenuti nei numeri reali positivi, l'unica possibilità è che
    \[
    n_\varepsilon > \sqrt{\frac{7}{\varepsilon}-2}
    \]
    Quindi se prendiamo un numero naturale $n_\varepsilon > \sqrt{\frac{7}{\varepsilon}-2}$ vale che $1-\varepsilon <a_{n_\varepsilon}$; ad esempio, possiamo considerare
    \[
    n_\varepsilon = \left\lfloor\sqrt{\frac{7}{\varepsilon}-2}\right\rfloor+1
    \]
    ove $\amsbb{R}\ni x \mapsto \lfloor x \rfloor$ è la cosiddetta \emph{floor function}, che restituisce la parte intera di un numero reale ("approssimando verso $-\infty$")\footnote{Per i numeri reali positivi è analogo a fare il casting \texttt{n=(int)x} di una variabile \texttt{float} su C/C\texttt{++}. Come esempio, se consideriamo $\varepsilon=0.1$ abbiamo che $n_\varepsilon = \lfloor\sqrt{68}\rfloor+1 = 9$; $a_9 = \frac{76}{83}\simeq0.91>0.9$.}. Per concludere quindi
    \[
    -\frac{5}{2} = \inf A = \min A \qquad 1 = \sup A
    \]
    e $A$ non ammette massimo, poiché non esiste $n\in\amsbb{N}$ tale che $a_n = 1$.
\end{proof}
\begin{exercise}
    \label{ex:2.3}
    Determinare $\sup$ e $\inf$, ed eventualmente massimo e minimo, dell'insieme
    \[
    A = \left\{ x\in\amsbb{R} \colon 2^{\frac{x+1}{x-1}}>4^x\right\}
    \]
\end{exercise}
\begin{proof}[Soluzione]
    Notiamo che la funzione $\amsbb{R}\ni x \mapsto 2^{\frac{x+1}{x-1}}$ è la composizione di un'esponenziale con una funzione razionale definita su $\amsbb{R}\setminus \{1\}$; pertanto sicuramente $1\notin A$. Inoltre, ricordiamo che
    \begin{tcolorbox}
        Dato $a\in\amsbb{R}$ con $a>1$, la funzione $\amsbb{R}\ni x \mapsto a^x$ è monotona strettamente crescente, ossia
        \[
        x > y \implies f(x) > f(y)
        \]
    \end{tcolorbox}
    ossia equivalentemente $f(x)<f(y) \implies x <y$. Di conseguenza,
    \[
    2^{\frac{x+1}{x-1}}>2^{2x} \iff \frac{x+1}{x-1}> 2x
    \]
    Per determinare $A$ è quindi necessario studiare la disequazione
    \[
    \frac{x+1}{x-1}>2x \iff \frac{x+1}{x-1}-2x >0
    \]
    che possiamo scrivere come 
    \begin{equation}
        \label{eq:2.1}
        \frac{x+1-2x^2+2x}{x-1}>0; \qquad \frac{-2x^2+3x+1}{x-1}
    \end{equation}
    Le radici del polinomio $-2x^2+3x+1$ sono
    \[
    r_1 = \frac{3-\sqrt{17}}{4} \qquad r_2 = \frac{3+\sqrt{17}}{4}
    \]
    tramite cui possiamo scrivere $-2x^2+3x+1 = -2(x-r_1)(x-r_2)$; notiamo che $r_1<0$ e $r_2>1$. Studiamo quindi il segno della funzione razionale in (\ref{eq:2.1}):
    \begin{center}
        \begin{tikzpicture}
            \tikzset{h style/.style = {fill=black!30}}
            \tkzTabInit[lgt=4,espcl=2,deltacl=0]
            { /.8, $(x-r_1)$ /.8, $(x-r_2)$ /.8, ${-2x^2+3x+1}$ /.8,  $(x-1)$ /.8, $\frac{-2x^2+3x+1}{x-1}$ /.8}
            {,$r_1$, $1$, $r_2$, } % four main references
            \tkzTabLine {,-,z,+, t, +,t, +, } % seven denotations
            \tkzTabLine {,-,t,-, t, -, z, +, }
            \tkzTabLine {,-,z,+,t, +,z, -,  }
            \tkzTabLine{, -, t, -, z, +, t, +, }
            \tkzTabLine{, h, z, -, t, h, z, -, }
            \path (N35) -- (N36) node[black,midway,inner sep=2pt,draw,circle,fill=white]{};
            \path (M15) -- (M16) node[black,midway]{+};
            \path (M35) -- (M36) node[black,midway]{+};
        \end{tikzpicture}
    \end{center}
    Le soluzioni della disequazione in (\ref{eq:2.1}) sono quindi
    \[
    x<\frac{3-\sqrt{17}}{4} \quad \lor \quad  1 < x < \frac{3+\sqrt{17}}{4}
    \]
    ossia possiamo scrivere l'insieme $A$ come
    \[
    A = \left(-\infty, \frac{3-\sqrt{17}}{4}\right)\cup \left(1, \frac{3+\sqrt{17}}{4}\right)
    \]
    Osserviamo quindi che $A$ non è limitato inferiormente; pertanto convenzionalmente diciamo che $\inf A = -\infty$. $A$ è invece limitato superiormente, e vale che $\sup A = \frac{3+\sqrt{17}}{4}$; poiché $\sup A \notin A$, $A$ non ammette massimo.
\end{proof}
\begin{exercise}
    \label{ex:2.4}
    Determinare $\sup$ e $\inf$, ed eventualmente massimo e minimo, dell'insieme
    \[
    A = \left\{ x\in\amsbb{R} \colon \sin(x)+3\cos(x)+1 \le 0 \right\}
    \]
\end{exercise}
\begin{proof}[Soluzione]
    Il metodo più immediato per risolvere il problema è utilizzare le formule parametriche
    \begin{equation}
            \label{eq:2.2}
            \sin(x) = \frac{2t}{1+t^2} \qquad \cos(x) = \frac{1-t^2}{1+t^2} \qquad t = \tan\left(\frac{x}{2}\right)
    \end{equation}
    Notiamo che, mentre $\amsbb{R}\ni x \mapsto \sin(x)$ e $\amsbb{R}\ni x \mapsto \cos(x)$ sono definite su tutto $\amsbb{R}$, la funzione $\amsbb{R}\ni x \mapsto \tan\left(\frac{x}{2}\right)$ è definita su $\amsbb{R} \setminus \left\{\pi +2k\pi, \ k\in\amsbb{Z}\right\}$; pertanto se riscriviamo la funzione che descrive l'insieme $A$ usando (\ref{eq:2.2}), escluderemo di default i punti del tipo $\{\pi + 2k\pi, \ k\in\amsbb{Z}\}$, che bisognerà controllare manualmente. Procediamo quindi con la risoluzione dell'esercizio:
    \begin{enumerate}[(i)]
        \item iniziamo a verificare i punti dell'insieme $\{\pi + 2k\pi, k\in\amsbb{Z}\}$; avremo
        \[
        \sin(\pi + 2k\pi) + 3\cos(\pi+2k\pi)+1 = -3+1 = -2 <0
        \]
        e quindi
        \[
        \left\{\pi + 2k\pi, k\in\amsbb{Z}\right\}\subseteq A
        \]
        \item cerchiamo le soluzioni
        di
        \begin{equation}
            \label{eq:2.3}
            \frac{2t}{1+t^2}+3\frac{1-t^2}{1+t^2}+1\le 0, \qquad t= \tan\left(\frac{x}{2}\right)
        \end{equation}
        in $\amsbb{R}\setminus \{\pi+2k\pi, \ k\in\amsbb{Z}\}$. Possiamo riscrivere la disequazione in (\ref{eq:2.3}) come
        \[
        \frac{2t+3-3t^2+1+t^2}{1+t^2}\le 0; \qquad \frac{-2t^2+2t+4}{1+t^2}\le 0; \qquad \frac{-2(t^2-t-2)}{1+t^2}\le 0
        \]
        e infine come
        \[
        \frac{-2(t-2)(t+1)}{\underbrace{1+t^2}_{>0 \ \forall t\in\amsbb{R}}}\le 0
        \]
        Studiamo come prima il segno della funzione razionale:
        \begin{center}
            \begin{tikzpicture}
                \tikzset{h style/.style = {fill=black!30}}
                \tkzTabInit[lgt=4,espcl=2,deltacl=0]
                { /.8, $(t+1)$ /.8, $(t-2)$ /.8, ${-2t^2+2t+4}$ /.8,   $\frac{-2t^2+2t+4}{1+t^2}$ /.8}
                {,$-1$, $2$, } % four main references
                \tkzTabLine {,-,z,+, t, +, } % seven denotations
                \tkzTabLine {,-,t,-, z, +, }
                \tkzTabLine{, -, z, +, z, -,  }
                \tkzTabLine{, h, z, +, z, h,  }
                \path (M14) -- (M15) node[black,midway]{-};
                \path (M34) -- (M35) node[black,midway]{-};
            \end{tikzpicture}
        \end{center}
        Le soluzioni di (\ref{eq:2.3}) sono quindi
        \[
        t\le-1 \quad \lor \quad t\ge 2
        \]
        e di conseguenza l'insieme $A$ conterrà gli insiemi
        \[
        \underbrace{\left\{x\in \amsbb{R} \colon \tan\left(\frac{x}{2}\right)\le -1\right\}}_{\stepcounter{equation}\mbox{(\theequation)}} \nonumber\cup \underbrace{\left\{x\in\amsbb{R}\colon \tan\left(\frac{x}{2}\right)\ge 2\right\}}_{\stepcounter{equation}\mbox{(\theequation)}}\nonumber
        \]
        \addtocounter{equation}{-2}\refstepcounter{equation}\label{eq:2.4}
        \addtocounter{equation}{0}\refstepcounter{equation}\label{eq:2.5}
        ove per brevità notazionale abbiamo trascurato le condizioni di esistenza della tangente.
        Le soluzioni di (\ref{eq:2.4}) sono
        \[
        \left\{x\in\amsbb{R}\colon \frac{\pi}{2}+k\pi<\frac{x}{2}\le\frac{3}{4}\pi + k\pi, \ k\in\amsbb{Z}\right\}
        \]
        mentre le soluzioni di (\ref{eq:2.5}) sono
        \[
        \left\{x\in\amsbb{R}\colon \arctan(2)+k\pi \le \frac{x}{2}<\frac{\pi}{2}+k\pi, \ k\in\amsbb{Z} \right\}
        \]
        Riscrivendo i due insiemi in termini di disuguaglianze per $x$ abbiamo
        \[
        \left\{x\in\amsbb{R}\colon \pi+2k\pi<x\le\frac{3}{2}\pi + 2k\pi \right\} \quad \left\{x\in\amsbb{R}\colon 2\arctan(2)+2k\pi\le x <\pi + 2k\pi \right\}
        \]
    \end{enumerate}
    con $k$ che varia in $\amsbb{Z}$. $A$ risulta quindi essere
    \[
    \begin{split}
        \left\{\pi + 2k\pi, \ k\in \amsbb{Z}\right\} &\cup \left\{x\in\amsbb{R}\colon \pi+2k\pi<x\le\frac{3}{2}\pi + 2k\pi, \ k\in\amsbb{Z}\right\} \cup\\
        & \cup\left\{x\in\amsbb{R}\colon 2\arctan(2)+2k\pi\le x <\pi + 2k\pi, \ k\in\amsbb{Z} \right\}
    \end{split}
    \]
    ossia
    \[
    A = \left\{x\in\amsbb{R}\colon 2\arctan(2)+2k\pi \le x \le \frac{3}{2}\pi + 2k\pi, \ k\in\amsbb{Z}\right\}
    \]
    Notiamo quindi che $A$ è illimitato\footnote{Questo lo si poteva notare già dal punto (i): infatti l'insieme $\{\pi+2k\pi, \ k\in\amsbb{Z}\}$ è illimitato, e poiché $A \supseteq \{\pi +2k\pi, k \in \amsbb{Z}\}$ anche $A$ è illimitato.}; pertanto per convenzione
    \[
    \sup A = +\infty \qquad \inf A = -\infty
    \]
\end{proof}
\newpage
\section{Lezione 3}
\subsection{Ripasso: i numeri complessi}
\begin{definition}
    \label{def:3.1}
    Un \emph{numero complesso} $z$ è una coppia \textbf{ordinata} $(a,b)\in\amsbb{R}^2$. Diremo che $z_1 =(a_1, b_1)$ e $z_2 = (a_2, b_2)$ sono uguali se $a_1 = a_2$ e $b_1 = b_2$.
\end{definition}
\begin{remark}
    Sembra un punto secondario, ma la definizione di uguaglianza riveste un'importanza cruciale. La cosa è apparente se consideriamo altri insiemi numerici. Ad esempio, i numeri interi $\amsbb{Z}$ si definiscono a partire da coppie ordinate $(n, m)\in\amsbb{N}^2$, e in questo caso la definizione di uguaglianza cambia: diremo infatti che $x_1 = (n_1, m_1)$ e $x_2=(n_2, m_2)$ sono uguali se
    \[
    n_1 +m_2 = n_2 + m_1
    \]
\end{remark}
\begin{definition}
    \label{def:3.2}
    Su $\amsbb{R}^2$ definiamo due operazioni: la \emph{somma}
    \[
    \begin{split}
        +\colon \qquad  \amsbb{R}^2 \times \amsbb{R}^2 \qquad &\to \qquad \quad \amsbb{R}^2 \\
        \left((a_1, b_1), (a_2, b_2)\right)&\mapsto (a_1+a_2, b_1 + b_2)
    \end{split}
    \]
    e il \emph{prodotto}
    \[
    \begin{split}
        \cdot\colon \qquad  \amsbb{R}^2 \times \amsbb{R}^2 \qquad &\to \qquad \qquad \amsbb{R}^2 \\
        \left((a_1, b_1), (a_2, b_2)\right)&\mapsto (a_1a_2-b_1b_2, a_1b_2 + a_2b_1)
    \end{split}
    \]
\end{definition}
\begin{theorem}
    \label{th:3.1}
    $(\amsbb{R}^2, +, \cdot)$ è un campo che indichiamo con $\amsbb{C}$, e chiamiamo \emph{campo dei numeri complessi}. $(0,0)$ è l'elemento neutro della somma e lo indichiamo con $0$, mentre $(1,0)$ è l'elemento neutro del prodotto e lo indichiamo con $1$.
\end{theorem}
\begin{remark}
    Notiamo che l'insieme 
    \[
    S = \left\{ (a,0)\in \amsbb{R}^2, \ a\in\amsbb{R}\right\}
    \]
    è stabile sotto la somma e il prodotto: infatti
    \[
    (a,0)+(b,0) = (a+b, 0) \qquad (a,0)\cdot(b,0) = (ab, 0)
    \]
    Si può inoltre mostrare che la restrizione ad $S$ della somma e del prodotto così come definiti nella definizione \ref{def:3.2} soddisfano gli assiomi di campo; definendo pertanto una funzione
    \[
    \begin{split}
        \phi\colon \amsbb{R} &\to \quad S\\
        a &\mapsto(a,0)
    \end{split}
    \]
    otteniamo un omomorfismo di campi che è anche biiettivo; possiamo pertanto identificare $\amsbb{R}$ e $S$, ossia scrivere $\amsbb{R}\subseteq \amsbb{C}$ e dire che $\amsbb{R}$ è un sottocampo di $\amsbb{C}$.
\end{remark}
\begin{definition}
    \label{def:3.3}
    Definiamo l'\emph{unità immaginaria} $i=(0,1)$, e notiamo che $i^2 = (0,1)\cdot(0,1) = (-1,0)$.
\end{definition}
\begin{remark}
    Notiamo che, dato $b\in\amsbb{R}$ e identificandolo con $(b,0)\in\amsbb{C}$ vale che
    \[
    bi = (b,0)\cdot(0,1) = (0,b) = (0,1)\cdot(b,0) = ib
    \]
    Poiché la coppia $(a,b)\in\amsbb{C}$ può essere scritta come $(a,0) + (0,b)$, modulo identificazione di $(a,0)$ con $a$ possiamo scrivere
    \[
    (a,b) = a+ib
    \]
    Se indichiamo la coppia $(a,b)$ con $z$, il numero \textbf{reale} $a$ è detto \emph{parte reale} di $z$, $a=\text{Re}(z)$, mentre il numero \textbf{reale} $b$ è detto \emph{parte immaginaria} di $z$, $b=\text{Im}(z)$. La scrittura $z=a+ib$ viene detta \emph{rappresentazione algebrica} o \emph{cartesiana} di $z$; questo perché se pensiamo a $\amsbb{C}$ come alle coppie ordinate di $\amsbb{R}^2$, la scrittura $z=a+ib$ consente immediatamente di determinare le coordinate della coppia nel piano:
    \begin{center}
        \begin{tikzpicture}[scale=1]
            % \tzhelplines(5,5)
            \tzaxes(-.2,-.2)(5,5){$\text{Re}$}[b]{$\text{Im}$}[l]
            \tzdot*(3.5, 3.8){$z=a+ib$}
            %\txnode
            %\tzfn[dashed,thick]"line"{\x}[-.2:5]{$y=x$}[l]
            %\tzplotcurve[blue,thick]"curve"(.5,4.3)(1,4.2)(2.5,4.1)(4,.5){$y=g(x)$}[45]; % [ar] also works in version 2.0
            % intersection and projection
            %\tzXpoint{line}{curve}(X){$z$}
            \tzproj(3.5, 3.8){$a$}{$b$}
        \end{tikzpicture}
    \end{center}
    Un punto nel piano può essere descritto, oltre che tramite le coordinate rispetto ad un sistema di assi cartesiani (\emph{coordinate cartesiane}), anche con la distanza dall'origine del sistema di assi cartesiani $r$ e con l'angolo che la retta congiungente il punto e l'origine forma con una delle due rette del sistema di assi cartesiani $\theta$ (\emph{coordinate polari})\footnote{Notiamo che essendo $r$ una distanza deve essere un numero positivo o al più nullo, mentre essendo $\theta$ un angolo deve essere un numero compreso tra $-\pi$ e $\pi$.}, ossia
    \begin{center}
        \begin{tikzpicture}[scale=1]
            % \tzhelplines(5,5)
            \tzaxes(-.2,-.2)(5,5){$\text{Re}$}[b]{$\text{Im}$}[l]
            \tzdot*(3.5, 3.8){$z=a+ib$}
            \tzline[dashed, thick](0,0)(3.5, 3.8){$r$}[midway, a]
            \tzarc(0,0)(0:47.35:2.5){$\theta$}[midway, r]
            %\txnode
            %\tzfn[dashed,thick]"line"{\x}[-.2:5]{$y=x$}[l]
            %\tzplotcurve[blue,thick]"curve"(.5,4.3)(1,4.2)(2.5,4.1)(4,.5){$y=g(x)$}[45]; % [ar] also works in version 2.0
            % intersection and projection
            %\tzXpoint{line}{curve}(X){$z$}
            %\tzproj(3.5, 3.8){$a$}{$b$}
        \end{tikzpicture}
    \end{center}
\end{remark}
Si può passare agilmente da un sistema di coordinate ad un altro: infatti, conoscendo la coppia $(r,\theta)\in[0, +\infty)\times (-\pi, \pi]$ che descrive un dato numero complesso $z$ si può ottenere la coppia $(a,b)\in\amsbb{R}^2$ considerando
\begin{equation}
    \label{eq:3.1}
    a=r\cos(\theta) \qquad b= r\sin(\theta)
\end{equation}
Allo stesso modo, si può determinare la coppia $(r,\theta)\in[0, +\infty)\times (-\pi, \pi]$ a partire dalla coppia $(a,b)\in\amsbb{R}$:
\begin{equation}
    \label{eq:3.2}
    \underbrace{\abs{z}}_{\text{modulo di} \ z} = r = \sqrt{a^2+b^2} \qquad \underbrace{\text{Arg}(z)}_{\text{argomento di} \ z} = \theta \simeq \arctan\left(\frac{b}{a}\right)
\end{equation}
Notiamo che nell'equazione in (\ref{eq:3.2}) per determinare l'argomento di $z$ non vi è il segno di uguaglianza; infatti, vale che
\begin{enumerate}[(i)]
    \item se $a=0$ il rapporto $\frac{b}{a}$ non è definito. In questo caso, se $b\ne 0$ si pone 
    \[
    \text{Arg}(z) = \pm \frac{\pi}{2}
    \]
    ove $\pm$ indica il segno di $b$ (se $b>0$ avremo $+\frac{\pi}{2}$, mentre se $b<0$ avremo $-\frac{\pi}{2}$), mentre se $b=0$ $\text{Arg}(z)$ non è ben definito;
    \item poiché l'argomento di $\arctan$ nella seconda equazione in (\ref{eq:3.2}) è il rapporto di $b$ e $a$, l'arctangente non riesce a distinguere fra \uppercase\expandafter{\romannumeral 1\relax} e \uppercase\expandafter{\romannumeral 3\relax} quadrante e fra \uppercase\expandafter{\romannumeral 2\relax} e \uppercase\expandafter{\romannumeral 4\relax} quadrante. Di conseguenza, dato che l'immagine di $\amsbb{R}\ni x \mapsto \arctan(x)$ è $\left(-\frac{\pi}{2}, \frac{\pi}{2}\right)$, un'applicazione bovina della formula restituirà sempre un angolo che descriverà un numero nel \uppercase\expandafter{\romannumeral 1\relax} o nel \uppercase\expandafter{\romannumeral 4\relax} quadrante. Si può però operare la seguente correzione:
    \[
    \text{Arg}(z) = \begin{dcases}
        \arctan\left(\frac{b}{a}\right)\, & a>0, \ b\ge 0\\
        \arctan\left(\frac{b}{a}\right)+\pi\, & a<0, \ b\ge 0\\
        \arctan\left(\frac{b}{a}\right)-\pi\, & a<0, \ b< 0\\
        \arctan\left(\frac{b}{a}\right)\, & a>0, \ b< 0
        \end{dcases}
    \]
\end{enumerate}
Usando la formula (\ref{eq:3.1}) è possibile scrivere $z=a+ib$ come
\[
z=r\cos(\theta) + ir\sin(\theta) = r(\cos(\theta)+i\sin(\theta))
\]
\begin{theorem}[Formula di Eulero]
    \label{th:3.2}
    Per ogni $\theta\in\amsbb{R}$ vale che
    \[
    e^{i\theta} = \cos(\theta) + i\sin(\theta)
    \]
\end{theorem}
Alla luce del teorema \ref{th:3.2} possiamo quindi scrivere, dato un numero complesso $z\in\amsbb{C}$
\begin{equation}
    \label{eq:3.3}
    z=re^{i\theta}
\end{equation}
ove $(r,\theta)$ sono dati o sono calcolati tramite (\ref{eq:3.2}). Questa notazione è detta \emph{rappresentazione polare} o \emph{esponenziale} di $z$.
\begin{definition}
    \label{def:3.}
    Dato un numero complesso $z=a+bi$, il suo \emph{coniugato} è il numero
    \[
    \overline{z} = a-ib
    \]
\end{definition}
\begin{remark}
    Notiamo che
    \[
    z\overline{z} = (a+ib)(a-ib) = a^2+b^2
    \]
    e quindi $r=\sqrt{z\overline{z}}$. In particolare, se $z=a$ vale che $r = \sqrt{z\overline{z}} = \sqrt{z^2}=\sqrt{a^2}=\abs{a}$; per questo $r$ viene detto modulo di $z$.\\
    Cosa accade alla rappresentazione polare di $\overline{z}$? Notiamo che
    \[
    \abs{\overline{z}} = \sqrt{a^2+(-b)^2} = \sqrt{a^2+b^2} = \abs{z}
    \]
    e che
    \[
    \text{Arg}(\overline{z}) = \begin{dcases}
        \arctan\left(\frac{-b}{a}\right)\, & a>0, \ b\le 0\\
        \arctan\left(\frac{-b}{a}\right)+\pi\, & a<0, \ b\le 0\\
        \arctan\left(\frac{-b}{a}\right)-\pi\, & a<0, \ b> 0\\
        \arctan\left(\frac{-b}{a}\right)\, & a>0, \ b> 0
        \end{dcases} = \begin{dcases}
        -\arctan\left(\frac{b}{a}\right)\, & a>0, \ b\le 0\\
        -\arctan\left(\frac{b}{a}\right)-\pi+2\pi\, & a<0, \ b\le 0\\
        -\arctan\left(\frac{b}{a}\right)-2\pi+\pi\, & a<0, \ b> 0\\
        -\arctan\left(\frac{b}{a}\right)\, & a>0, \ b> 0
        \end{dcases}
    \]
    ossia
    \[
    \text{Arg}(\overline{z}) = -\text{Arg}(z)
    \]
    Infine, si può facilmente mostrare che
    \[
    \overline{z_1 z_2} = \overline{z_1} \cdot \overline{z_2}\qquad \overline{z_1 + z_2 } = \overline{z_1}+\overline{z_2} \ \text{per ogni} \ z_1, z_2\in\amsbb{C}
    \]
\end{remark}
\subsection{Esercizi: esercizi di base sui numeri complessi}
\begin{exercise}
    \label{ex:3.1}
    Calcolare $i^i$.
\end{exercise}
\begin{proof}[Soluzione]
    In generale, sappiamo che quando abbiamo a che fare con elevamenti a potenza conviene maneggiare gli esponenziali; scriviamo quindi $i$ in rappresentazione polare. Ricordiamo che $i=(0,1)$; pertanto
    \[
    \abs{i} = \sqrt{0^2 + 1^2} = 1 \qquad \text{Arg}(z) \overset{\text{Punto (i) osservazione }}{=} \frac{\pi}{2}
    \]
    ossia $i=e^{i\frac{\pi}{2}}$. Per calcolare $i^i$, ricordiamo il seguente risultato:
    \begin{tcolorbox}
        \begin{theorem}
            \label{th:3.3}
            Siano $z, z_1, z_2\in\amsbb{C}$; vale che 
            \[
            e^{z}\in\amsbb{C}\setminus\{0\} \qquad e^{z_1}e^{z_2} = e^{z_1+z_2} \qquad \frac{1}{e^{z}} = e^{-z} \qquad e^{z+i2k\pi} = e^{z} \ \forall k\in\amsbb{Z}
            \]
            Inoltre, per ogni numero intero $n\in\amsbb{Z}$, $z^n$ è ben definito e unico, mentre se consideriamo numeri $z_1, z_2\in\amsbb{C}$ tali che $\mathrm{Arg}(z_1), \mathrm{Arg}(z_2)\in(-\pi, \pi]$ e che $\mathrm{Arg}(z_1)+\mathrm{Arg}(z_2)\in(-\pi, \pi]$ definiamo
            \begin{equation}
                \label{eq:3.4}
                (z_1)^{z_2} = e^{z_2\left(\log\abs{z_1}+i\mathrm{Arg}(z_1)\right)}
            \end{equation}
            \begin{equation}
                \label{eq:3.5}
                (z_1 z_2)^z = z_1^z z_2^z
            \end{equation}
            Se $\mathrm{Arg}(z_1)+\mathrm{Arg}(z_2)\notin(-\pi, \pi]$, avremo che
            \begin{equation}
                \label{eq:3.6}
                (z_1 z_2)^z = z_1^z z_2^z e^{z(i2k\pi)}, \ k\in\amsbb{Z} \ \text{tale che} \ \mathrm{Arg}(z_1)+\mathrm{Arg}(z_2) + 2k\pi \in (-\pi, \pi]
            \end{equation}
        \end{theorem}
    \end{tcolorbox}
    Nel nostro caso, vale l'equazione (\ref{eq:3.4}): quindi
    \[
    i^i = e^{i\left(\log\abs{i} + i \mathrm{Arg}(i)\right)} = e^{i\left(\log(1) + i \frac{\pi}{2}\right)} = e^{-\frac{\pi}{2}}
    \]
\end{proof}
\begin{exercise}
    \label{ex:3.2}
    Calcolare
    \[
    \mathrm{Re}\left(\frac{(1-i)^2-(1+2i)^2}{(2+3i)^2+(1+i)^3}\right)
    \]
\end{exercise}
\begin{proof}[Soluzione]
    In questo caso il metodo più conveniente è effettuare i conti:
    \[
    \frac{(1-i)^2-(1+2i)^2}{(2+3i)^2+(1+i)^2} = \frac{(1-1-2i)-(1-4+4i)}{(4-9+12i)+(1-i+3i-3)} = \frac{3-6i}{-7+14i} = -\frac{3}{7}\frac{1-2i}{1-2i} = -\frac{3}{7}
    \]
    Quindi
    \[
    \mathrm{Re}\left(\frac{(1-i)^2-(1+2i)^2}{(2+3i)^2+(1+i)^3}\right) = \frac{(1-i)^2-(1+2i)^2}{(2+3i)^2+(1+i)^3} = -\frac{3}{7}
    \]
\end{proof}
\subsection{Ripasso: polinomi a coefficienti in \texorpdfstring{$\mathbb{C}$}{C}}
\begin{theorem}[Teorema fondamentale dell'algebra]
    \label{th:3.4}
    Dato un polinomio $P(x)$ a coefficienti in $\amsbb{C}$ di grado $n\ge 1$, questo ammette sempre una radice in $\amsbb{C}$. 
\end{theorem}
\begin{corollary}
    \label{cor:3.1}
    Un polinomio $P(x)$ a coefficienti complessi di grado $n$ ammette $n$ radici in $\amsbb{C}$, contate con la loro molteplicità algebrica.
\end{corollary}
\begin{remark}
    Poiché abbiamo visto che $\amsbb{R}$ è un sottocampo di $\amsbb{C}$, possiamo considerare un polinomio a coefficienti reali come un polinomio a coefficienti complessi, e cercarne le radici in $\amsbb{C}$. In particolare, se $\alpha\in\amsbb{C}$ è una radice del polinomio $P(x)$ a coefficienti reali, allora $\overline{\alpha}$ è anch'esso radice di $P(x)$: infatti, notiamo che se 
    \[
    P(x) = a_0 + \sum_{k=1}^n a_k x^k
    \]
    allora $0 = P(\alpha) = a_0 + \sum_{k=1}^n a_k\alpha^k$; ma
    \[
    0 = \overline{0} = \overline{P(\alpha)} = \overline{a_0}+\sum_{k=1}^n \overline{a_k}\overline{\alpha}^k = a_0 + \sum_{k=1}^n a_k \overline{\alpha}^k
    \]
    dato che $\{a_0, \dots, a_n\}\subset\amsbb{R}$.
\end{remark}
\begin{example}
    Consideriamo ad esempio il polinomio $P(x) = x^4+2x^2+1$; ispezionando il polinomio si vede che $z_1 =i$ è radice di $P$, e per l'osservazione precedente anche $z_1 = -i$ è soluzione di $P$. Quindi per la proposizione \ref{prop:1.1} e per la definizione \ref{def:1.5} vale che
    \[
    P(x) = Q(x)(x-i)(x+i) = Q(x)(x^2+1)
    \]
    Il polinomio $Q(x)$ può essere ottenuto effettuando la divisione fra polinomi:
    \[
        \polylongdiv[style=D]{x^4+2x^2+1}{x^2+1}
    \]
    Quindi
    \[
    P(x) = (x^2+1)^2
    \]
    e $P(x)$, come polinomio a coefficienti complessi, ammette due radici, $i$ e $-i$, ciascuna con molteplicità due.
\end{example}
\subsection{Esercizi: polinomi ed equazioni in \texorpdfstring{$\mathbb{C}$}{C}}
\begin{exercise}
    \label{ex:3.3}
    Scomporre il polinomio $P(x) = x^9+a^9$, $a\in\amsbb{R}$, in fattori irriducibili (cfr. teorema \ref{th:1.2}).
\end{exercise}
\begin{proof}[Soluzione]
    Supponiamo $a\ne 0$, altrimenti il polinomio è già scomposto. Possiamo scrivere in maniera semplice
    \[
    \begin{split}
         P(x) & = x^9 + a^9 = (x^3)^3 + (a^3)^3 = (x^3+a^3)(x^6-a^3x^3+a^6) \\
         & = \underbrace{(x+a)(x^2-ax+a^2)}_{\text{irriducibili}}(x^6-a^3x^3+a^6)
    \end{split}
    \]
    Il problema è scomporre il polinomio $Q(x) = (x^6-a^3x^3+a^6)$ in fattori irriducibili. Per farlo, conviene considerare $Q(x)$ come un polinomio a coefficienti complessi, cercare le 6 radici complesse di $Q$ e, con opportune manipolazioni algebriche, ottenere i polinomi a coefficienti reali. Per cercare le radici di $Q$, effettuiamo la sostituzione $x^3 = t$, e riscriviamo $Q(x)$ come 
    \[
    Q(t) = t^2 -a^3t +a^6
    \]
    Le sue radici sono date da
    \[
    t_{1,2} = \frac{a^3\pm \sqrt{a^6-4a^6}}{2} = \frac{a^3\pm \abs{a}^3\sqrt{1-4}}{2} = a^3\left(\frac{1}{2}\pm i\frac{\sqrt{3}}{2}\right)
    \]
    Per trovare le radici di $Q(x)$, dobbiamo risolvere
    \begin{equation}
        \label{eq:3.7}
        x^3 = \underbrace{a^3\left(\frac{1}{2}+i\frac{\sqrt{3}}{2}\right)}_{z_1} \qquad 
        x^3 = \underbrace{a^3\left(\frac{1}{2}-i\frac{\sqrt{3}}{2}\right)}_{z_2}
    \end{equation}
    Come prima, quando si ha a che fare con elevamenti a potenza conviene usare la notazione polare:
    \[
    \abs{z_1} = \sqrt{a^6\frac{1}{4} + a^6 \frac{3}{4}} = \abs{a}^3 \qquad \abs{z_2} = \sqrt{a^6\frac{1}{4} + a^6 \frac{3}{4}} = \abs{a}^3 
    \]
    \[
    \mathrm{Arg}(z_1) =\begin{dcases}
        \frac{\pi}{3}\, & a>0 \\
         \frac{\pi}{3}+\pi\, & a<0
    \end{dcases}\qquad 
    \mathrm{Arg}(z_2) = \begin{dcases}
         -\frac{\pi}{3}\, & a>0\\
         -\frac{\pi}{3}+\pi\, & a<0
    \end{dcases}
    \]
    Quindi, riscrivendo la (\ref{eq:3.7}), dobbiamo cercare le radici di
    \[
    \underbrace{x^3 = \abs{a}^3 e^{i\frac{\pi}{3}}}_{\stepcounter{equation}\mbox{(\theequation)}} \nonumber \qquad \underbrace{x^3 = \abs{a}^3 e^{-i\frac{\pi}{3}}}_{\stepcounter{equation}\mbox{(\theequation)}} \nonumber
    \]
    \addtocounter{equation}{-2}\refstepcounter{equation}\label{eq:3.8}
    \addtocounter{equation}{0}\refstepcounter{equation}\label{eq:3.9}
    se $a>0$ e di 
    \[
    \underbrace{x^3 = \abs{a}^3 e^{i\frac{4}{3}\pi}}_{\stepcounter{equation}\mbox{(\theequation)}} \nonumber \qquad \underbrace{x^3 = \abs{a}^3 e^{i\frac{2}{3}\pi}}_{\stepcounter{equation}\mbox{(\theequation)}} \nonumber
    \]
    \addtocounter{equation}{-2}\refstepcounter{equation}\label{eq:3.10}
    \addtocounter{equation}{0}\refstepcounter{equation}\label{eq:3.11}
    se $a<0$.
    \begin{tcolorbox}
        Ricordiamo che le soluzioni di
        \[
        w = z^{\frac{1}{n}}, \ n\in\amsbb{N}
        \]
        in generale sono date da
        \[
        w_k = \abs{z}^{\frac{1}{n}} e^{i\frac{\mathrm{Arg}(z)}{n}+i2\frac{k}{n}\pi}, \ k\in\amsbb{Z} 
        \]
        Notiamo che è sufficiente considerare $k\in\{0, 1, \dots, n-2, n-2\}$.
    \end{tcolorbox}
    \begin{enumerate}[(i)]
        \item Caso $a>0$: le soluzioni di (\ref{eq:3.8}) sono
    \[
    x_1 = {a} e^{i\frac{\pi}{9}} \qquad x_2 ={a}e^{i\frac{\pi}{9}+i\frac{2}{3}\pi} = {a} e^{i\frac{7}{9}\pi} \qquad x_3 = {a}e^{i\frac{\pi}{9}+i\frac{4}{3}\pi} = {a} e^{-i\frac{5}{9}\pi}
    \]
    mentre le soluzioni di (\ref{eq:3.9}) sono
    \[
    x_4 = {a} e^{-i\frac{\pi}{9}} \qquad x_5 ={a}e^{-i\frac{\pi}{9}+i\frac{2}{3}\pi} = {a}e^{i\frac{5}{9}\pi} \qquad x_6 = {a}e^{-i\frac{\pi}{9}+i\frac{4}{3}\pi} = {a} e^{-i\frac{7}{9}\pi}
    \]
    Notiamo che $x_4 = \overline{x_1}$, $x_6 = \overline{x_2}$ e $x_3 = \overline{x_5}$; possiamo scrivere
    \[
    P(x) = (x+a)(x^2-ax+a^2)\underbrace{(x-x_1)(x-\overline{x_1})(x-x_2)(x-\overline{x_2})(x-x_5)(x-\overline{x_5})}_{Q(x)}
    \]
    Ora,
    \[
    \begin{split}
        (x-x_1)(x-\overline{x_1}) &= x^2 + x_1 \overline{x_1}-x(x_1 + \overline{x_1}) = x^2 + \abs{x_1}^2 -2\mathrm{Re}(x_1)x = \\
        & = \underbrace{ x^2 + a^2 -2{a}x\cos\left(\frac{\pi}{9}\right)}_{\text{irriducibile}}
    \end{split}
    \]
    Lo stesso vale per gli altri prodotti di monomi; abbiamo quindi
    \[
    Q(x) = (x^2 -2{a}x\cos\left(\frac{\pi}{9}\right)+a^2)(x^2 -2{a}x\cos\left(\frac{5}{9}\pi\right) + a^2)(x^2 -2{a}x\cos\left(\frac{7}{9}\pi\right) + a^2)
    \]
    \item Caso $a<0$: le soluzioni di (\ref{eq:3.10}) sono
    \[
    x_1 = \abs{a} e^{i\frac{4}{9}\pi} = a e^{-i\frac{5}{9}\pi} \qquad x_2 =\abs{a}e^{i\frac{4}{9}\pi+i\frac{2}{3}\pi} = a e^{i\frac{\pi}{9}} \qquad x_3 = \abs{a}e^{i\frac{4}{9}\pi+i\frac{4}{3}\pi} = {a} e^{i\frac{7}{9}\pi}
    \]
    mentre le soluzioni di (\ref{eq:3.11}) sono
    \[
    x_4 = \abs{a} e^{i\frac{2}{9}\pi} = ae^{-i\frac{7}{9}\pi} \qquad x_5 =\abs{a}e^{i\frac{2}{9}+i\frac{2}{3}\pi} = {a}e^{-i\frac{\pi}{9}} \qquad x_6 = \abs{a}e^{i\frac{2}{9}\pi+i\frac{4}{3}\pi} = {a} e^{i\frac{5}{9}\pi}
    \]
    Notiamo che in questo caso $x_5 = \overline{x_2}$, $x_1 = \overline{x_6}$ e $x_4 = \overline{x_3}$; possiamo anche in questo caso scrivere
    \[
    P(x) = (x+a)(x^2-ax+a^2)\underbrace{(x-x_1)(x-\overline{x_1})(x-x_2)(x-\overline{x_2})(x-x_5)(x-\overline{x_5})}_{Q(x)}
    \]
    Ora,
    \[
    \begin{split}
        (x-x_1)(x-\overline{x_1}) &= x^2 + x_1 \overline{x_1}-x(x_1 + \overline{x_1}) = x^2 + \abs{x_1}^2 -2\mathrm{Re}(x_1)x = \\
        & ={ x^2 + a^2 -2{a}x\cos\left(\frac{\pi}{9}\right)}
    \end{split}
    \]
    Lo stesso vale per gli altri prodotti di monomi; abbiamo quindi, anche nel caso $a<0$,
    \[
    Q(x) = (x^2 -2{a}x\cos\left(\frac{\pi}{9}\right)+a^2)(x^2 -2{a}x\cos\left(\frac{5}{9}\pi\right) + a^2)(x^2 -2{a}x\cos\left(\frac{7}{9}\pi\right) + a^2)
    \]
    \end{enumerate}
\end{proof}
\begin{exercise}
    \label{ex:3.4}
    Risolvere l'equazione $z^2+i\overline{z} = 1$.
\end{exercise}
\begin{proof}[Soluzione]
    Ricordiamo (cfr. definizione \ref{def:3.1}) che due numeri complessi $z_1, z_2 \in\amsbb{C}$ sono uguali se
    \[
    \mathrm{Re}(z_1) = \mathrm{Re}(z_2) \qquad \mathrm{Im}(z_1) = \mathrm{Im}(z_2) 
    \]
    Di conseguenza, nel nostro caso l'equazione $z^2+i\overline{z}=1$ è equivalente al sistema
    \begin{equation}
        \label{eq:3.12}
        \begin{dcases}
        \mathrm{Re}(z^2+i\overline{z})= 1&\\
        \mathrm{Im}(z^2+i\overline{z})= 0&  
    \end{dcases}
    \end{equation}
    Per scrivere i membri di sinistra delle due equazioni, conviene rappresentare $z$ in forma cartesiana, $z=a+ib$, ed espandere i prodotti:
    \[
    z^2 + i \overline{z} = (a+ib)^2 + i(a-ib) = a^2-b^2 + i2ab +ia+b = a^2-b^2+b + i(a+2ab)
    \]
    Il sistema in (\ref{eq:3.12}) risulta quindi essere
    \begin{equation}
        \label{eq:3.13}
        \begin{dcases}
        a^2-b^2+b= 1&\\
        a(1+2b)= 0
    \end{dcases}
    \end{equation}
    Le soluzioni della seconda equazione in (\ref{eq:3.13}) sono $a=0$ e $b=-\frac{1}{2}$; consideriamo quindi i due casi:
    \begin{enumerate}[(i)]
        \item $a=0$: la prima equazione diventa $-b^2 +b-1=0$, che non ammette soluzioni reali in quanto $\Delta = 1-4<0$; quindi non esistono soluzioni dell'equazione con parte reale nulla;
        \item $b=-\frac{1}{2}$: la prima equazione diventa $a^2 =\frac{7}{4}$, che ha come soluzioni
        \[
        a_1 = \frac{\sqrt{7}}{2} \qquad a_2 = -\frac{\sqrt{7}}{2}
        \]
    \end{enumerate}
    Le soluzioni dell'equazione sono quindi
    \[
    \left\{\frac{\sqrt{7}}{2}-i\frac{1}{2}, -\frac{\sqrt{7}}{2}-i\frac{1}{2}\right\}
    \]
\end{proof}
\begin{exercise}
    \label{ex:3.5}
    Determinare l'insieme
    \[
    \left\{z\in\amsbb{C} \colon 5z^2+5\overline{z}^2-6z\overline{z}+8i(\overline{z}-z) = 4\right\}
    \]
\end{exercise}
\begin{proof}[Soluzione]
    Anche in questo caso conviene riscrivere l'equazione che descrive l'insieme esprimendo $z$ in forma cartesiana, $z=a+ib$:
    \[
    \begin{split}
    &5(a+ib)^2+5(a-ib)^2-6(a^2+b^2)+8i(a-ib-a-ib )= \\
    & = 5(a^2-b^2+i2ab)+5(a^2-b^2-i2ab)-6(a^2+b^2)+16b = 4a^2 -16b^2 +16 b 
    \end{split}
    \]
    L'equazione è quindi
    \[
    4a^2-16b^2+16b -4 = 0 \iff a^2-4b^2+4b-1 = 0
    \]
    L'equazione può essere riscritta come 
    \[
    (a+2b-1)(a-2b+1)=0
    \]
    le cui soluzioni se rappresentate graficamente sono
    \begin{center}
        \begin{tikzpicture}[scale=1]
            % \tzhelplines(5,5)
            \tzaxes(-3,-1.3)(3,3){$\text{Re}$}[b]{$\text{Im}$}[l]
            \tzticks{ -2, -1, 1, 2}{-1, 1, 2}
            %\tzdot*(3.5, 3.8){$z=a+ib$}
            %\txnode
            \tzfn[blue,thick]"line"{-.5*\x+.5}[-3:3]{$\mathrm{Im}(z)=-\frac{1}{2}\mathrm{Re}(z)+\frac{1}{2}$}[br]
            \tzfn[red,thick]"line"{.5*\x+.5}[-3:3]{$\mathrm{Im}(z)=\frac{1}{2}\mathrm{Re}(z)+\frac{1}{2}$}[ar]
            %\tzplotcurve[blue,thick]"curve"(.5,4.3)(1,4.2)(2.5,4.1)(4,.5){$y=g(x)$}[45]; % [ar] also works in version 2.0
            % intersection and projection
            %\tzXpoint{line}{curve}(X){$z$}
            %\tzproj(3.5, 3.8){$a$}{$b$}
        \end{tikzpicture}
    \end{center}
    L'insieme descritto è quindi costituito da due rette intersecantisi (un cono nel piano complesso).
\end{proof}
\begin{exercise}
    \label{ex:3.6}
    Determinare l'insieme
    \[
    \left\{z\in\amsbb{C}\colon \abs{z-1}\le \abs{z+i}\right\}\cap\left\{ z\in\amsbb{C}\colon \abs{z-2i}\ge 1\right\}
    \]
\end{exercise}
\begin{proof}[Soluzione]
    Anche in questo caso scriviamo $z$ in forma algebrica come $z=a+ib$; consideriamo quindi $\abs{z-1}\le \abs{z+i}$:
    \[
    \abs{z-1} = \abs{(a-1)+ib} = \sqrt{(a-1)^2 + b^2} \qquad \abs{z+i} = \abs{a+i(b+1)} = \sqrt{a^2+(b+1)^2}
    \] 
    e quindi la disequazione risulta essere
    \[
    \sqrt{(a-1)^2+b^2}\le \sqrt{a^2+(b+1)^2}
    \]
    \begin{tcolorbox}
        Ricordiamo che la funzione $\sqrt{\cdot}\colon \amsbb{R}^+ \to \amsbb{R}^+$ è monotona strettamente crescente, ossia
        \[
        x > y \implies \sqrt{x}>\sqrt{y}
        \]
        Di conseguenza, se $\sqrt{x}\le \sqrt{y}$, allora $x\le y$.
    \end{tcolorbox}
    Quindi la procedente disequazione è equivalente a
    \[
    (a-1)^2+b^2\le a^2+(b+1)^2 \iff -2a \le 2b \iff b \ge -a
    \]
    Di conseguenza il primo insieme è il semipiano generato dalla retta $\mathrm{Im}(z) = -\mathrm{Re}(z)$, ossia
    \begin{center}
        \begin{tikzpicture}
        \begin{axis}[y=7mm, ymin=-4.5, samples=101]
        \addplot [name path=A, blue] coordinates { (\Xmin, -\Xmin) (\Xmax,-\Xmax) };
        \path [name path=B] (\Xmax,\Ymax)
        node[below left, font=\footnotesize, text=black] {$\mathrm{Im}(z)\ge -\mathrm{Re}(z)$}
                                  -- (\Xmin,\Ymax);
        \addplot [blue!30] fill between [of=A and B];
        \end{axis}
        \end{tikzpicture}
    \end{center}
    Consideriamo ora il secondo insieme, e operiamo come prima: la disequazione risulta essere
    \[
    \abs{a+i(b-2)}\ge 1 \iff \sqrt{a^2+(b-2)^2}\ge 1 \iff a^2 +(b-2)^2\ge 1
    \]
    Ricordiamo che $(a-a_0)^2+(b-b_0)^2 \le 1$ descrive un cerchio di raggio $1$ e centro $(a_0, b_0)$; pertanto la disequazione descrive la regione di piano esterna ad una circonferenza di raggio $1$ e cento $(0,2)\in\amsbb{C}$, circonferenza inclusa, ossia
    \begin{center}
        \begin{tikzpicture}
        \pgfsetlayers{pre main,main}
        \begin{axis}[y=7mm,       % <--
             ymin=-4.5,   % <--
             samples=201]

        \addplot [name path=A, domain=-1:1, red] {+sqrt(1-x^2)+2};
        \addplot [name path=B, domain=-1:1, red] {-sqrt(1-x^2)+2};

        \pgfonlayer{pre main}
        % coloring of plane
        \fill[red!30]  (\Xmin,\Ymax) -| (\Xmax,\Ymin)
        node[above left, text=black, font = \footnotesize] {$\mathrm{Re}(z)^2 + (\mathrm{Im}(z)-2)^2 \ge 1$} 
                              -| cycle;
        % coloring of circle
        \addplot [white] fill between [of=A and B];
        \endpgfonlayer
        \end{axis}
    \end{tikzpicture}
    \end{center}
    La retta e la circonferenza sembrerebbero non intersecarsi; per verificare che sia effettivamente così, consideriamo il sistema
    \[
    \begin{dcases}
        b=-a\\
        a^2+(b-2)^2 = 1
    \end{dcases}
    \]
    La seconda equazione diventa $a^2+(a+2)^2 = 2a^2 +4a+3=0$ che non ammette soluzioni reali, in quanto $\Delta = 16-24<0$. A questo punto, l'intersezione dei due insiemi risulta essere
    \begin{center}
        \begin{tikzpicture}
        \begin{axis}[y=7mm, ymin=-4.5, samples=201]
        \addplot [name path=A, teal] coordinates { (\Xmin, -\Xmin) (\Xmax,-\Xmax) };
        \path [name path=B] (\Xmax,\Ymax)--(\Xmin,\Ymax);
        

        \addplot [name path=C, domain=-1:1, teal] {+sqrt(1-x^2)+2};
        \addplot [name path=D, domain=-1:1, teal] {-sqrt(1-x^2)+2};
        \addplot [teal!30] fill between [of=A and B];
        \addplot [white] fill between [of=C and D];
        %\endpgfonlayer
        \end{axis}
        \end{tikzpicture}
    \end{center}
    Di conseguenza l'insieme è un semipiano meno un disco aperto.
\end{proof}
\begin{exercise}
    \label{ex:3.7}
    Determinare l'insieme
    \[
    \left\{z\in\amsbb{C}\colon \abs{\frac{z-1}{z+1}}\le 1\right\}
    \]
\end{exercise}
\begin{proof}[Soluzione]
    Innanzitutto, notiamo che necessariamente $z\ne -1$; inoltre, possiamo scrivere
    \[
    \abs{\frac{z-1}{z+1}} \le 1
    \]
    Poiché $\abs{z+1}\ge 0$ e l'uguaglianza vale solo se $z+1=0$, caso escluso, possiamo moltiplicare ambo i membri della disuguaglianza per $\abs{z+1}$, ottenendo così
    \[
    \abs{z-1}\le \abs{z+1}
    \]
    Scriviamo ora $z$ in forma algebrica come $z=a+ib$, ottenendo
    \[
    \begin{split}
        \abs{(a-1)+ib}\le \abs{(a+1)+ib}& \iff \sqrt{(a-1)^2+b^2}\le \sqrt{(a+1)^2+b^2} \\
        &\iff (a-1)^2+b^2 \le (a+1)^2+b^2\\
        & \iff -2a \le 2a \iff a\ge 0
    \end{split}
    \]
    L'insieme è quindi
    \begin{center}
        \begin{tikzpicture}
        \begin{axis}[y=7mm, ymin=-4.5, samples=201]
        \addplot [name path=A, red] coordinates { (0,\Ymin) (0,\Ymax) };
        \path [name path=B] (\Xmax,\Ymin) -- (\Xmax,\Ymax)
    node[above left, font=\footnotesize, text=black] {$\mathrm{Re}(z) \ge 0$};
\addplot [red!30] fill between [of=A and B];
\end{axis}
    \end{tikzpicture}
    \end{center}
    ossia un sempiano.
\end{proof}
\begin{exercise}
    \label{ex:3.8}
    Sia $P(z)$ un polinomio a coefficienti reali tale che 
    \begin{enumerate}[(i)]
        \item $a_0=0$, ossia $P(z) = \sum_{k=1}^n a_k z^k$;
        \item $z_0 = a+ib$, $b\ne 0$ è radice di $P$ con molteplicità 2.
    \end{enumerate}
    Cosa possiamo concludere di $P$?
\end{exercise}
\begin{proof}[Soluzione]
    Sappiamo che $P$ ha coefficienti reali; di conseguenza se $z_0$ è radice di $P$ lo è anche $\overline{z_0}$. Inoltre, poiché $z_0$ ha molteplicità algebrica 2, anche $\overline{z_0}$ ha molteplicità algebrica 2. Di conseguenza,
    \[
    P(z) = (z-z_0)^2(z-\overline{z_0})^2 Q(z)
    \]
    Inoltre, poiché $a_0=0$, vale che
    \[
    P(z) = a_1 z + a_2 z^2 + \dots + a_{n-1}z^{n-1}+a_n z^n = z(a_1 + a_2 z + \dots + a_{n-1}z^{n-2} + a_n z^{n-1})
    \]
    ossia $P(z)$ è divisibile per $z$. Poiché $\mathrm{Im}(z_0)\ne 0$ sicuramente $(z-z_0)^2$ e $(z-\overline{z_0})^2$ non sono divisibili per $z$; di conseguenza $Q(z)$ è divisibile per $z$, i.e. per la definizione \ref{def:1.5} $Q(z) = zT(z)$;
    quindi
    \[
    P(z) = z(z-z_0)^2(z-\overline{z_0})^2 T(z)
    \]
    ossia $\deg(P)\ge 5$.
\end{proof}
\newpage

\section{Lezione 4}
\subsection{Ripasso: successioni}
\begin{definition}
    \label{def:4.1}
    Una \emph{successione} è una funzione $s\colon \mathbb{N}\to\mathbb{R}$. Denoteremo con $s_n$ il numero reale $s(n)$, $n\in\mathbb{N}$, e indicheremo la mappa $s\colon \mathbb{N}\to \mathbb{R}$ con $(s_n)_n$. L'immagine della successione verrà indicata con $\left\{s_n, \ n\in\mathbb{N}\right\}\subseteq \mathbb{R}$.
\end{definition}
\begin{remark}
    Data una successione $(s_n)_n$, se la sua immagine $\{s_n, \ n\in\amsbb{N}\}$ è limitata come sottinsieme di $\amsbb{R}$, diremo che $(s_n)_n$ è limitata.
\end{remark}
\begin{definition}
    \label{def:4.2}
    Diremo che $(s_n)_n$ \emph{converge a $s\in\mathbb{R}$} se per ogni $\varepsilon>0$ esiste $n_\varepsilon\in\amsbb{N}$ tale che
    \[
    \abs{s_n-s}<\varepsilon \ \text{per ogni}\ n>n_\varepsilon
    \]
    In caso contrario, diremo che $(s_n)_n$ diverge. Se $s_n$ converge a $s$, scriveremo
    \[
    \lim_{n\to\infty} s_n = s \quad \text{o} \quad s_n \to s
    \]
\end{definition}
\begin{remark}
    Nel caso in cui $(s_n)_n$ diverge, possiamo isolare due casi particolari:
    \begin{enumerate}[(i)]
        \item \emph{$(s_n)_n$ tende a $+\infty$}:
        \begin{equation}
            \label{eq:4.1}
            \text{per ogni} \ M \in\amsbb{N} \ \text{esiste} \ n_M\in \amsbb{N} \ \colon s_n >M \ \text{per ogni} \ n>n_M
        \end{equation}
        \item \emph{$(s_n)_n$ tende a $-\infty$}:
        \begin{equation}
            \label{eq:4.2}
            \text{per ogni} \ M \in\amsbb{N} \ \text{esiste} \ n_M\in \amsbb{N} \ \colon s_n <-M \ \text{per ogni} \ n>n_M
        \end{equation}
    \end{enumerate}
\end{remark}
\begin{theorem}
    \label{th:4.1}
    Data una successione $(s_n)_n$, consideriamo il sottoinsieme illimitato dei naturali $\{n_k\}_{k\in\amsbb{N}}\subseteq \amsbb{N}$, con $n_0<n_1<\dots < n_k < \dots$. Questo insieme è in biiezione con $\amsbb{N}$, ossia esiste una funzione iniettiva e suriettiva $\amsbb{N}\ni k \mapsto n_k$; possiamo quindi considerare la composizione $\amsbb{N}\ni k \mapsto s_{n_k}\in\amsbb{R}$; questa è detta essere una \emph{sottosuccessione} di $(s_n)_n$, e viene indicata con $(s_{n_k})_k$.\\
    $(s_n)_n$ converge a $s\in\amsbb{R}$ se e solo se ogni sua sottosuccessione $(s_{n_k})_k$ converge a $s$ 
\end{theorem}
\begin{proof}
    Mostriamo le due implicazioni:
    \begin{enumerate}[(i)]
        \item data una successione $(s_n)_n$, questa è una sottosuccessione di se stessa: infatti, se consideriamo l'insieme $\{n_k\}_{k\in\amsbb{N}}$ con $n_k = k$, allora $(s_{n_k})_k = (s_n)_n$. Di conseguenza, se ogni sottosuccessione di $(s_n)_n$ converge a $s$, anche la sottosuccessione $(s_{n_k})_{k}$ sopra descritta converge a $s$; ma poiché questa coincide con $(s_n)_n$, abbiamo l'asserto;
        \item supponiamo ora che $(s_n)_n$ converga ad $s$; per la definizione \ref{def:4.2}, per ogni scelta di $\varepsilon>0$ esiste $n_\varepsilon\in\amsbb{N}$ tale che $\abs{s_n-s}<\varepsilon$ per ogni $n\in\amsbb{N}$.\\
        Sia $(s_{n_k})_k$ una sottosuccessione di $(s_n)_n$; vale quindi che $n_0 < n_1 < \dots < n_k < \dots$. Per quanto detto antecedentemente, fissato un $\varepsilon>0$ esiste $n_\varepsilon$ tale che
        \[
        \abs{s_{n_k} - s}<\varepsilon \ \text{se} \ n_k > n_\varepsilon
        \]
        Poiché l'insieme $\{n_k\}_{k\in\amsbb{N}}$ è illimitato ed è crescente in $k$, esiste sicuramente un $k_\varepsilon$ tale che
        \[
        n_{k_\varepsilon}\ge n_\varepsilon \quad \text{e} \quad n_k > n_{k_\varepsilon} \ \text{per ogni} \ k>k_\varepsilon
        \]
        Di conseguenza, dato un qualsiasi $\varepsilon>0$, esiste $k_\varepsilon\in\amsbb{N}$ tale che
        \[
        \abs{s_{n_k}-s}<\varepsilon \ \text{per ogni} \ k>k_\varepsilon
        \]
        ossia $(s_{n_k})_k$ converge ad $s$.
    \end{enumerate}
\end{proof}
\subsection{Esercizi: convergenza di successioni, aritmetica dei limiti}
\begin{exercise}
    \label{ex:4.1}
    Determinare se la successione $(s_n)_n$ con
    \[
    s_n = (-1)^n \frac{n}{n+1}
    \]
    converge o meno.
\end{exercise}
\begin{proof}[Soluzione]
    Per cercare di stimare il comportamento della successione, scriviamo esplicitamente i primi termini:
    \[
    s_0 = 0 \quad s_1 = -\frac{1}{2} \quad s_2 = \frac{2}{3} \quad s_3 = -\frac{3}{4} \quad s_4 = \frac{4}{5} \ \dots
    \]
    Notiamo quindi che sembrerebbero esistere due tendenze: $s_n$ tende a $1$ per $n$ pari e $s_n$ tende a $-1$ per $n$ dispari. Se riusciamo a mostrare che questo accade effettivamente, grazie al teorema \ref{th:4.1} potremmo affermare che la successione $(s_n)_n$ non converge.\\
    Consideriamo quindi i sottoinsiemi di $\amsbb{N}$
    \[
    \left\{n_k = 2k, \ k\in\amsbb{N}\right\} \qquad \left\{n_l = 2l+1, \ l\in\amsbb{N}\right\}
    \]
    e le relative sottosuccessioni $(s_{n_k})_k$ e $(s_{n_l})_l$. I termini della successione $(s_{n_k})_k$ sono dati da
        \[
        s_{n_k} = (-1)^{2k} \frac{2k}{2k+1} = \frac{2k}{2k+1} = \frac{2k+1}{2k+1}-\frac{1}{2k+1} = 1-\frac{1}{2k+1}
        \]
    mentre i termini della successione $(s_{n_l})_l$ sono dati da
    \[
    s_{n_l} = (-1)^{2l+1}\frac{2l+1}{2l+2} = -\frac{2l+1}{2l+2} = -\frac{2l+2}{2l+2} + \frac{1}{2l+2} = -1 + \frac{1}{2l+2}
    \]
    Ricordiamo che
    \begin{tcolorbox}
        \begin{theorem}
            \label{th:4.2}
            Date due successioni $(a_n)_n$ e $(b_n)_n$ tali che $a_n \to a$ e $b_n \to b$, allora
            \[
            a_n \pm b_n \to a \pm b
            \]
        \end{theorem}
    \end{tcolorbox}
    Pertanto se la successione $\left(b_k = \frac{1}{2k+1}\right)_k$ converge a qualche numero reale $b$, allora $s_{n_k} \to 1-b$. In particolare, vale che $b_k \to 0$: infatti, fissato $\varepsilon>0$, se prendiamo
    \begin{equation}
        \label{eq:4.3}
        k_\varepsilon = \max\left\{ \left\lfloor\frac{1}{2\varepsilon}-\frac{1}{2} \right\rfloor+1, 0\right\}
    \end{equation}
    e consideriamo $k>k_\varepsilon$, vale che $k>\frac{1}{2\varepsilon}-\frac{1}{2}$; ma allora $2k+1 >\varepsilon$, ossia $\frac{1}{2k+1}<\varepsilon$. Di conseguenza, per ogni $\epsilon>0$ esiste $k_\varepsilon$, dato da (\ref{eq:4.3}), tale che $b_k<\varepsilon$ per ogni $k>k_\varepsilon$; quindi $b_k\to 0$, e per il teorema \ref{th:4.2} $s_{n_k}\to 1$. Allo stesso modo si può dimostrare che $s_{n_l}\to -1$.\\
    Esistono quindi due sottosuccessioni di $(s_n)_n$ che convergono a due limiti distinti; pertanto per il teorema \ref{th:4.1} $(s_n)_n$ non converge.\\
    Notiamo infine che 
    \[
    \abs{s_n} = \abs{(-1)^n \frac{n}{n+1}} = \frac{n}{n+1}\le 1 \ \text{per ogni} \ n\in\amsbb{N}
    \]
    pertanto $(s_n)$ non può divergere a $\pm \infty$.
\end{proof}
\begin{exercise}
    \label{ex:4.2}
    Determinare se la successione $(s_n)_n$ con
    \[
    s_n = n^2\cos\left(\frac{\pi}{2}(2n+1)\right)-\frac{1}{n}\sin\left(\frac{\pi}{2}(2n+1)\right)
    \]
    converge o meno.
\end{exercise}
\begin{proof}[Soluzione]
    In questo caso, la successione $s_n$ è data dalla differenza di due successioni 
    \[
    {a_n = n^2 \cos\left(\frac{\pi}{2}(2n+1)\right)}\qquad {b_n = \frac{1}{n}\sin\left(\frac{\pi}{2}(2n+1)\right)}
    \]
    Per applicare il teorema \ref{th:4.2}, verifichiamo se $(a_n)_n$ e $(b_n)_n$ convergono separatamente ad $a$ e $b$.
    \begin{enumerate}[(i)]
        \item Notiamo che 
        \[
        \cos\left(\frac{\pi}{2}(2n+1)\right) = 0 \ \text{per ogni} \ n\in\amsbb{N}
        \]
        Quindi $a_n = 0$ per ogni $n\in\amsbb{N}$, e di conseguenza $a_n \to 0$.
        \item Nel caso di $b_n$ invece
        \[
            \sin\left(\frac{\pi}{2}(2n+1)\right) = \sin\left(n\pi + \frac{\pi}{2}\right) = (-1)\sin\left((n-1)\pi + \frac{\pi}{2}\right) = \dots = (-1)^n
        \]
        e quindi $b_n = (-1)^n \frac{1}{n}$. Procediamo come prima per mostrare che $b_n\to 0$: fissiamo $\varepsilon>0$, e cerchiamo di trovare $n_\varepsilon$ tale che $\abs{b_n}<\varepsilon $ se $n>n_\varepsilon$.
        \[
        \varepsilon> \abs{(-1)^n\frac{1}{n}} = \frac{1}{n} \iff n > \frac{1}{\varepsilon}
        \]
        Se fissiamo $n_\varepsilon = \left \lfloor \frac{1}{\varepsilon} \right \rfloor+1$ e consideriamo $n>n_\varepsilon$ vale che 
        \[
        n>\frac{1}{\varepsilon} \iff \varepsilon>\frac{1}{n} = \abs{b_n}
        \]
    \end{enumerate}
    Di conseguenza $a_n \to 0 $ e $b_n \to 0$, e quindi $s_n = a_n - b_n \to 0$.
\end{proof}
\begin{exercise}
    \label{ex:4.3}
    Determinare se la successione $(s_n)_n$ con
    \[
    s_n = \sqrt{n+1}-\sqrt{n}
    \]
    converge o meno.
\end{exercise}
\begin{proof}[Soluzione]
    In questo caso il teorema \ref{th:4.2} non aiuta, in quanto le successioni $a_n = \sqrt{n+1}$ e $b_n = \sqrt{n}$ divergono entrambe a $+\infty$. Possiamo però riscrivere $s_n$ tramite manipolazioni algebriche:
    \[
    \begin{split}
        s_n & = \sqrt{n+1}-\sqrt{n} = \frac{\sqrt{n+1}-\sqrt{n}}{\sqrt{n+1}+\sqrt{n}}(\sqrt{n+1}+\sqrt{n}) = \frac{(\sqrt{n+1})^2 - (\sqrt{n})^2}{\sqrt{n+1}+\sqrt{n}} = \\
        & = \frac{1}{\sqrt{n+1}+\sqrt{n}} = \frac{1}{\sqrt{n}\left(1+\sqrt{1+\frac{1}{n}}\right)} = \frac{1}{\sqrt{n}}\frac{1}{1+\sqrt{1+\frac{1}{n}}}
    \end{split}
    \]
    A questo punto, ricordiamo i seguenti risultati:
    \begin{tcolorbox}
        \begin{theorem}
            \label{th:4.3}
            Date due successioni $(a_n)_n$ e $(b_n)_n$ tali che $a_n \to a$ e $b_n\to b$, allora
            \[
            a_nb_n \to ab
            \]
        \end{theorem}
    \end{tcolorbox}
    \begin{tcolorbox}
        \begin{theorem}
            \label{th:4.4}
            Date due successioni $(a_n)_n$ e $(b_n)$ tali che $a_n \to a$ e $b_n\to b\ne 0$, con $b_n \ne 0$ per ogni $n\in\amsbb{N}$, vale che
            \[
            \frac{a_n}{b_n}\to \frac{a}{b}
            \]
        \end{theorem}
    \end{tcolorbox}
    Consideriamo quindi
    \[
    a_n = \frac{1}{\sqrt{n}} \qquad b_n=\frac{1}{1+\sqrt{1+\frac{1}{n}}}
    \]
    Ragionando come prima si dimostra agilmente che $a_n\to 0$, mentre per $b_n$ si può dimostrare che 
    \[
    1+\sqrt{1+\frac{1}{n}}\to 2
    \]
    e quindi $b_n \to \frac{1}{2}$ per il teorema \ref{th:4.4}. Di conseguenza grazie al teorema \ref{th:4.3} possiamo dire che $s_n \to 0$.
\end{proof}
\begin{remark}
    Quando $s_n \to s$, ci possono essere due casi particolari:
    \[
    s_n \to s^+ \ \text{se} \ s_n>s \ \text{per ogni} \ n\in\amsbb{N}
    \]
    e
    \[
    s_n \to s^- \ \text{se} \ s_n<s \ \text{per ogni} \ n\in\amsbb{N}
    \]
    Ci chiediamo se nel caso dell'esercizio \ref{ex:4.3} vale che $s_n \to 0^\pm$, ossia bisogna verificare se
    \[
    \sqrt{n+1}-\sqrt{n}>0 \ \text{oppure} \ \sqrt{n+1}-\sqrt{n}<0 \ \text{per ogni} \ n\in\amsbb{N}
    \]
    Avevamo già ricordato che $\sqrt{\cdot}\colon\amsbb{R}^+ \to \amsbb{R}^+$ è monotona strettamente crescente; quindi, dato che $n+1>n$, $\sqrt{n+1}>\sqrt{n}$, e quindi $s_n \to 0^+$.
\end{remark}
\begin{exercise}
    \label{ex:4.4}
    Determinare se la successione $(s_n)_n$ con
    \[
    s_n = \frac{\sqrt[3]{n}\sin(n)}{n+1}
    \]
    converge o meno.
\end{exercise}
\begin{proof}[Soluzione]
    Anche in questo caso proviamo a calcolare i primi termini della successione:
    \[
    s_0 = 0 \quad s_1 = \frac{\sin(1)}{2} \quad s_2 = \frac{\sqrt[3]{2}\sin(2)}{3} \quad s_3 = \frac{\sqrt[3]{3}\sin(3)}{4}\ \dots
    \]
    Ricordiamo che $\abs{\sin(x)}\le 1$ per ogni $x\in\amsbb{R}$; di conseguenza, vale che
    \[
    \abs{s_n} = \abs{\frac{\sqrt[3]{n}\sin(n)}{n+1}}\le \underbrace{\frac{\sqrt[3]{n}}{n+1}}_{a_n}
    \]
    Consideriamo la successione $(a_n)_n$; vale che
    \[
    a_n = \frac{\sqrt[3]{n}}{n+1} = \sqrt[3]{n}\frac{1}{n\left(1+\frac{1}{n}\right)} = \frac{1}{n^{\frac{2}{3}}}\frac{1}{1+\frac{1}{n}}
    \]
    Come nei casi precedenti, usando la definizione \ref{def:4.2} e il teorema \ref{th:4.4} si dimostra che
    \[
    \frac{1}{n^{\frac{2}{3}}}\to 0 \qquad \frac{1}{1+\frac{1}{n}}\to 1
    \]
    e quindi per il teorema \ref{th:4.3} vale che $a_n \to 0$. Possiamo quindi applicare il teorema del confronto:
    \begin{tcolorbox}
        \begin{theorem}
            \label{th:4.5}
            Date tre successioni $(a_n)_n$, $(b_n)_n$ e $(c_n)_n$ tali che $a_n \le b_n \le c_n$ per ogni $n\in\amsbb{N}$ e che $a_n \to l$, $c_n \to l$, allora $b_n \to l$.
        \end{theorem}
    \end{tcolorbox}
    Infatti, $0\le \abs{s_n}\le a_n$, e $a_n \to 0$; di conseguenza $\abs{s_n}\to 0$. Cosa possiamo dire di $(s_n)_n$? 
    \begin{tcolorbox}
        \begin{theorem}
            \label{th:4.6}
            Data una successione $(a_n)_n$, $a_n \to 0$ se e solo se $\abs{a_n}\to 0$.
        \end{theorem}
        \begin{proof}
            \begin{enumerate}[(i)]
                \item Supponiamo che $a_n \to 0$; per la definizione \ref{def:4.2} sappiamo che per ogni $\varepsilon>0$ esiste $n_\varepsilon$ tale che
                \[
                \varepsilon>\abs{a_n} = \abs{\abs{a_n}} \ \text{per ogni} \ n>n_\varepsilon
                \]
                ossia $\abs{a_n}\to 0$.
                \item Supponiamo ora che $\abs{a_n}\to 0$. Chiaramente vale che $\abs{a_n}\ge a_n$ per ogni $n\in\amsbb{N}$; di conseguenza, per il teorema del confronto a due, sappiamo che, se $a_n$ ammette limite $l$, allora $l\le 0$. Allo stesso modo, $\abs{a_n}\ge -a_n$, ossia $a_n \ge -\abs{a_n}$, e sempre grazie al teorema del confronto a due possiamo dire che se $a_n$ ammette limite $l$, vale che $l\ge 0$; di conseguenza $0\le l \le 0$, e quindi $l=0$, se esiste.\\
                A questo punto mostriamo che $a_n \to 0$: fissato $\varepsilon>0$, sappiamo che esiste $n_\varepsilon\in\amsbb{N}$ tale che $\abs{a_n}<\varepsilon$ per ogni $n>n_\varepsilon$, dato che $\abs{a_n}\to 0$. Di conseguenza
                \[
                \abs{a_n - 0} = \abs{a_n}<\varepsilon
                \]
                se $n>n_\varepsilon$, e quindi $a_n \to 0$.
            \end{enumerate}
        \end{proof}
    \end{tcolorbox}
    Per il teorema \ref{th:4.6} abbiamo quindi che $s_n \to 0$.
\end{proof}
\begin{remark}
    Il teorema \ref{th:4.6} funziona solamente nel caso in cui $\abs{a_n} \to 0$: infatti, se consideriamo $a_n = (-1)^n$, vale che $\abs{a_n}\to 1$, ma chiaramente $a_n \not\to 1$.
\end{remark}
\begin{exercise}
    \label{ex:4.5}
    Determinare se la successione $(s_n)_n$ con
    \[
    s_n = \frac{n\sin(n)}{\sqrt{n^2+1}}
    \]
    converge o meno.
\end{exercise}
\begin{proof}[Soluzione]
    In questo caso non riusciamo a determinare il limite della successione $\abs{s_n}$, in quanto 
    \[
    0\le \abs{s_n}\le \underbrace{\frac{n}{\sqrt{n^2+1}}}_{\text{converge a 1}} \ \text{per ogni} \ n\in\amsbb{N}
    \]
    e non possiamo quindi sfruttare il teorema \ref{th:4.6}. Facciamo però la seguente osservazione: sappiamo che $\sin(x)> 0$ se $x\in(2k\pi, \pi + 2k\pi)$, $k\in\amsbb{N}$, mentre $\sin(x)<0$ se $x\in(\pi + 2k\pi, 2\pi + 2k\pi)$, $k\in\amsbb{N}$. Questi intervalli sono sufficientemente ampi per contenere degli interi; consideriamo quindi i seguenti sottoinsiemi illimitati di $\amsbb{N}$:
    \[
    \left\{ n_k, \ n_k \in(2k\pi, \pi + 2k\pi) \cap \amsbb{N}, k \in\amsbb{N}\right\} \qquad \left\{ n_l, \ n_l \in(\pi + 2k\pi, 2\pi + 2k\pi) \cap \amsbb{N}, l \in\amsbb{N}\right\} 
    \]
    e le sottosuccessioni $(s_{n_l})_l$ e $(s_{n_k})_k$; per la proprietà di $\amsbb{R}\ni x \mapsto \sin(x)$ enunciata prima vale che $\sin(n_k)>0$ per ogni $k\in\amsbb{N}$, e $\sin(n_l)<0$ per ogni $l\in\amsbb{N}$. Quindi
    \[
    s_{n_k}>0 \ \text{per ogni} \ k\in\amsbb{N} \qquad s_{n_l}<0 \ \text{per ogni} \ l\in\amsbb{N}
    \]
    Di conseguenza per il teorema del confronto a due vale che, se i limiti esistono,
    \[
    \lim_{k\to\infty} s_{n_k}\ge 0 \qquad \lim_{l\to\infty} s_{n_l} \le 0
    \]
    e quindi se il limite $l$ di $(s_n)_n$ esiste questo deve necessariamente essere $0$.\\
    Supponiamo quindi che $s_n \to 0$; se consideriamo la successione $(a_n)_n$ con
    \[
    a_n = \frac{n}{\sqrt{n^2+1}}
    \]
    vale che $a_n \to 1$, e di conseguenza per il teorema \ref{th:4.4}
    \[
    \sin(n) = \frac{s_n}{a_n} \to 0
    \]
    ove la successione $\frac{s_n}{a_n}$ è definita su $\amsbb{N}\setminus \{0\}$. Per il teorema \ref{th:4.1}, ogni sottosuccessione di $(\sin(n))_n$ converge quindi a 0; in particolare vale per $\sin(n+1)$ e $\sin(n-1)$; ma
    \[
    \begin{split}
        \sin(n+1)-\sin(n-1) & = \sin(n)\cos(1) + \cos(n)\sin(1) - \sin(n)\cos(1)+\cos(n)\sin(1) = \\
        & = 2\cos(n)\sin(1)
    \end{split}
    \]
    Ma questo è impossibile: infatti sappiamo che $\cos^2(n) + \sin^2(n) = 1$ per ogni $n\in\amsbb{N}$, e chiaramente se $\sin(n)\to 0 $ e $\cos(n) \to 0$ abbiamo un assurdo. Quindi $(s_n)_n$ non converge a 0 e di conseguenza non converge.
\end{proof}
\begin{exercise}
    \label{ex:4.6}
    Determinare se la successione $(s_n)_n$ con
    \[
    s_n = n \log\left(1+\frac{3}{n}\right)
    \]
    converge o meno.
\end{exercise}
\begin{proof}[Soluzione]
    Possiamo sfruttare le proprietà del logaritmo per scrivere
    \[
    s_n = n\log\left(1+\frac{3}{n}\right) =\log\left(\left(1+\frac{3}{n}\right)^n\right)
    \]
    Ricordiamo che
    \begin{tcolorbox}
        \begin{definition}
            \label{def:4.3}
            Il numero di Nepero $e$ è definito essere
            \[
            e = \lim_{n\to\infty}\left(1+\frac{1}{n}\right)^n
            \]
        \end{definition}
    \end{tcolorbox}
    e l'argomento del logaritmo nel nostro caso ci assomiglia molto; infatti,
    \[
    e = \lim_{n\to\infty} \left(1+\frac{1}{n}\right)^n = \lim_{n\to\infty} \left(1+\frac{3}{3n}\right)^\frac{3n}{3}
    \]
    Se definiamo $m= 3n$ abbiamo che $m\to \infty$ se $n\to\infty$, e
    \[
    e = \lim_{m\to\infty}\left(1+\frac{3}{m}\right)^\frac{m}{3}
    \]
    Manipolando l'argomento del logaritmo otteniamo
    \[
    s_n = \log\left(\left(1+\frac{3}{n}\right)^n\right) = \log\left(\left(1+\frac{3}{n}\right)^{n\frac{3}{3}}\right) = 3\log\left(\left(1+\frac{3}{n}\right)^{\frac{n}{3}}\right)
    \]
    Quindi
    \[
    \lim_{n\to \infty} s_n = \lim_{n\to\infty} 3 \log\left(\left(1+\frac{3}{n}\right)^{\frac{n}{3}}\right) = 3 \log(e) = 3
    \]
\end{proof}
\begin{remark}
    Nell'ultimo passaggio del precedente esercizio abbiamo in qualche modo scambiato limite e logaritmo; ci è consentito farlo?\\
    Innanzitutto, notiamo che è sufficiente mostrare che se $a_n \to 1$, allora $\log(a_n) \to 0$: infatti se $a_n \to a$, $a>0$, allora $\frac{a}{a_n}\to 1$, e 
    \[
    \log\left(\frac{a}{a_n}\right)\to 0 \iff \log(a)-\log(a_n) \to 0 \iff \log(a_n)\to \log(a)
    \]
    Consideriamo quindi $a_n \to 1$; per la definizione \ref{def:4.2} vale che per ogni $\varepsilon>0$ esiste $n_\varepsilon$ tale che
    \[
    1-\varepsilon < a_n < 1+\varepsilon \ \text{per ogni} \ n>n_\varepsilon
    \]
    Poiché $\log(\cdot) \colon \amsbb{R}^+\to\amsbb{R}$ è monotona strettamente crescente, vale che
    \[
    \log(1-\varepsilon) < \log(a_n) <\log(1+\varepsilon) \ \text{per ogni} \ n>n_\varepsilon
    \]
    Fissiamo ora $\delta>0$, e definiamo $\varepsilon_1 = 1-e^{-\delta}$; dato che $\delta>0$ vale che $\varepsilon_1>0$, e per quanto detto prima esiste $n_1$ tale che
    \[
    \log(1-\varepsilon_1) = \log(e^{-\delta}) = -\delta < \log(a_n) \ \text{per ogni} \ n>n_1
    \]
    Allo stesso modo, se definiamo $\varepsilon_2 = e^{\delta}-1$, $\varepsilon_2>0$ ed esiste $n_2$ tale che
    \[
    \log(a_n)<\log(1+\varepsilon_2) = \log(e^\delta) = \delta \ \text{per ogni} \ n>n_2
    \]
    Se definiamo $n_\delta = \max\{n_1, n_2\}$ abbiamo che
    \[
    -\delta < \log(a_n) < \delta \iff \abs{\log(a_n)}<\delta \ \text{per ogni} \ n>n_\delta
    \]
    ossia $\log(a_n)\to 0$. Abbiamo essenzialmente dimostrato che $\log(\cdot)\colon \amsbb{R}^+\to \amsbb{R}$ è continua.
\end{remark}
\begin{exercise}
    \label{ex:4.7}
    Determinare se la successione $(s_n)_n$ con
    \[
    s_n = \frac{1}{\sqrt{n^2}}+\frac{1}{\sqrt{n^2+1}} + \frac{1}{\sqrt{n^2+2}} + \dots + \frac{1}{\sqrt{n^2+2n}}, \qquad n\ge 1
    \]
    converge o meno.
\end{exercise}
\begin{proof}[Soluzione]
    Questa è una successione particolare: infatti al crescere di $n$ i termini che sommati danno $s_n$ decrescono e tendono a $0$, però il loro numero aumenta; di conseguenza non possiamo appellarci al teorema \ref{th:4.2}.\\
    Osserviamo però la cosa seguente: sappiamo che $\sqrt{\cdot}\colon \amsbb{R}^+ \to \amsbb{R}^+$ è monotona crescente, e di conseguenza
    \[
    \sqrt{n^2 + m} \ge \sqrt{n^2}\ \text{per ogni} \ m\in\amsbb{N}
    \]
    ossia
    \[
    \frac{1}{\sqrt{n^2+m}}\le \frac{1}{\sqrt{n^2}} \ \text{per ogni} \ m\in\amsbb{N}, \ n\ge 1
    \]
    Allo stesso modo, $n^2 + 2n \ge n^2 +2n -m$ per ogni $m\in\amsbb{N}$, e quindi $\sqrt{n^2+2n}\ge \sqrt{n^2+2n-m}$; ragionando allo stesso modo di prima possiamo scrivere
    \[
    \frac{1}{\sqrt{n^2+2n}}\le \frac{1}{\sqrt{n^2+2n-m}} \ \text{per ogni} \ m\in\amsbb{N}, \ m>n^2+2n, n \ge 1 
    \]
    Di conseguenza possiamo stimare dall'alto nel modo seguente:
    \[
    \underbrace{\frac{1}{\sqrt{n^2}}}_{\le \frac{1}{\sqrt{n^2}}}+\overbrace{\frac{1}{\sqrt{n^2+1}}}^{\le \frac{1}{\sqrt{n^2}}} + \underbrace{\frac{1}{\sqrt{n^2+2}}}_{\le \frac{1}{\sqrt{n^2}}} + \dots + \overbrace{\frac{1}{\sqrt{n^2+2n}}}^{\le \frac{1}{\sqrt{n^2}}}
    \]
    ossia
    \[
    s_n \le \underbrace{\frac{1}{\sqrt{n^2}} + \frac{1}{\sqrt{n^2}} + \dots + \frac{1}{\sqrt{n^2}}}_{2n+1 \ \text{termini}} = \frac{2n+1}{\sqrt{n^2}}
    \]
    Allo stesso modo, possiamo stimare dal basso:
    \[
    \underbrace{\frac{1}{\sqrt{n^2}}}_{\ge \frac{1}{\sqrt{n^2+2n}}}+\overbrace{\frac{1}{\sqrt{n^2+1}}}^{\ge \frac{1}{\sqrt{n^2+2n}}} + \underbrace{\frac{1}{\sqrt{n^2+2}}}_{\ge \frac{1}{\sqrt{n^2+2n}}} + \dots + \overbrace{\frac{1}{\sqrt{n^2+2n}}}^{\ge \frac{1}{\sqrt{n^2+2n}}}
    \]
    ossia
    \[
    s_n \ge \underbrace{\frac{1}{\sqrt{n^2+2n}} + \frac{1}{\sqrt{n^2+2n}} + \dots + \frac{1}{\sqrt{n^2+2n}}}_{2n+1 \ \text{termini}} = \frac{2n+1}{\sqrt{n^2+2n}}
    \]
    Quindi abbiamo
    \[
    \frac{2n+1}{\sqrt{n^2+2n}} \le s_n \le \frac{2n+1}{\sqrt{n^2}} \ \text{per ogni} \ n\ge 1
    \]
    Consideriamo ora le successioni
    \[
    a_n = \frac{2n+1}{\sqrt{n^2+2n}} \qquad b_n = \frac{2n+1}{\sqrt{n^2}}
    \]
    e verifichiamo se convergono. Vale che
    \[
    a_n = \frac{2n+1}{\sqrt{n^2+2n}} = \frac{2n\left(1+\frac{1}{2n}\right)}{n\sqrt{1+\frac{2}{n}}} = 2 \frac{\overbrace{1+\frac{1}{2n}}^{\to 1}}{\underbrace{\sqrt{1+\frac{2}{n}}}_{\to 1}} \overset{\text{Th. \ref{th:4.4}}}{\longrightarrow} 2
    \]
    e allo stesso modo
    \[
    b_n = \frac{2n+1}{\sqrt{n^2}} = \frac{2n\left(1+\frac{1}{2n}\right)}{n} = 2 \overbrace{\left(1+\frac{1}{2n}\right)}^{\to 1}{\longrightarrow} 2
    \]
    Possiamo quindi applicare il teorema \ref{th:4.5}, e concludere che $s_n \to 2$.
\end{proof}
\begin{exercise}
    \label{ex:4.8}
    Determinare se la successione $(s_n)_n$ con
    \[
    s_n = \begin{dcases}
        1\, & n=0\\
        1+\sqrt{s_{n-1}}\, & n\ge 1
    \end{dcases}
    \]
    converge o meno.
\end{exercise}
\begin{proof}[Soluzione]
    Scriviamo i primi termini della successione per cercare di capirne l'andamento:
    \[
    s_0 = 1 \quad s_1 = 2 \quad s_2 = 1+\sqrt{2} \quad s_3 = 1+ \sqrt{1+\sqrt{2}} \ \dots
    \]
    Sembrerebbe che la successione sia monotona crescente; vale in generale che
    \[
    s_{n-1}\le s_n \ \text{per ogni} \ n\in\amsbb{N} \text{?}
    \]
    Notiamo che nel nostro caso vale che per $n\ge 1$
    \[
    s_n = 1 +\sqrt{s_{n-1}} \iff s_n-1 = \sqrt{s_{n-1}} \iff (s_n-1)^2 = s_{n-1}
    \]
    Vogliamo capire quindi se $(s_n-1)^2 \le s_n$ per ogni $n\in\amsbb{N}$; la disequazione risulta essere
    \[
    s_n^2 -3s_n +1 \le 0
    \]
    il cui segno è dato da
    \begin{center}
            \begin{tikzpicture}
                \tikzset{h style/.style = {fill=black!30}}
                \tkzTabInit[lgt=4,espcl=2,deltacl=0]
                { /.8, $(s_n-\frac{3-\sqrt{5}}{2})$ /.8, $(s_n-\frac{3+\sqrt{5}}{2})$ /.8, ${s_n^2 -3s_n+1}$ /.8}
                {,$\frac{3-\sqrt{5}}{2}$, $\frac{3+\sqrt{5}}{2}$, } % four main references
                \tkzTabLine {,-,z,+, t, +, } % seven denotations
                \tkzTabLine {,-,t,-, z, +, }
                \tkzTabLine{, +, z, h, z, +,  }
                \path (M23) -- (M24) node[black,midway]{$-$};
            \end{tikzpicture}
        \end{center}
        Di conseguenza la successione $(s_n)_n$ è monotona crescente finché $s_n\in\left[\frac{3-\sqrt{5}}{2}, \frac{3+\sqrt{5}}{2}\right]$. Ora, $s_0 = 1 > \frac{3-\sqrt{5}}{2}$, e quindi la successione è effettivamente monotona decrescente se $s_n\le \frac{3+\sqrt{5}}{2}$ per ogni $n\in\amsbb{N}$. Per dimostrarlo, procediamo per induzione:
        \begin{enumerate}[(i)]
            \item caso base $n=0$: banalmente verificato dato che $s_0 = 1 < \frac{3+\sqrt{5}}{2}$.
            \item supponiamo che $s_k\le \frac{3+\sqrt{5}}{2}$ per ogni $0\le k \le n-1$, e mostriamo che vale per $n$: sappiamo che 
            \[
            (s_n-1)^2 = s_{n-1} \le \frac{3+\sqrt{5}}{2}
            \]
            Per la monotonia della radice quadrata vale che
            \[
            \abs{s_n-1}\le \sqrt{\frac{3+\sqrt{5}}{2}}
            \]
            E poiché $s_n = 1+\sqrt{s_{n-1}}$ abbiamo che $s_{n}-1\ge 0$, quindi possiamo scrivere
            \[
            s_n-1\le \sqrt{\frac{3+\sqrt{5}}{2}} \iff s_n\le 1+\sqrt{\frac{3+\sqrt{5}}{2}}
            \]
            Vale che
            \[
            1+\sqrt{\frac{3+\sqrt{5}}{2}}\le \frac{3+\sqrt{5}}{2}?
            \]
            Per farlo, capiamo quando vale la disuguaglianza $1+\sqrt{t}\le t$, che possiamo riscrivere, per $t\ge 1$ (come nel caso che ci interessa), come $t\le (t-1)^2$, ossia $t^2-3t+1\ge 0$:
            \begin{center}
            \begin{tikzpicture}
                \tikzset{h style/.style = {fill=black!30}}
                \tkzTabInit[lgt=4,espcl=2,deltacl=0]
                { /.8, $(t-\frac{3-\sqrt{5}}{2})$ /.8, $(t-\frac{3+\sqrt{5}}{2})$ /.8, ${t^2 -3t+1}$ /.8}
                {,$\frac{3-\sqrt{5}}{2}$, $\frac{3+\sqrt{5}}{2}$, } % four main references
                \tkzTabLine {,-,z,+, t, +, } % seven denotations
                \tkzTabLine {,-,t,-, z, +, }
                \tkzTabLine{, h, z, -, z, h,  }
                \path (M13) -- (M14) node[black,midway]{$+$};
                \path (M33) -- (M34) node[black,midway]{$+$};
            \end{tikzpicture}
        \end{center}
        La disuguaglianza $1+\sqrt{t}\le t$ vale quindi per $t\ge \frac{3+\sqrt{5}}{2}$; di conseguenza abbiamo che
        \[
        s_n \le 1+\sqrt{\frac{3+\sqrt{5}}{2}}\le \frac{3+\sqrt{5}}{2}
        \]
        \end{enumerate}
        Abbiamo quindi che
        \[
        s_n\in \left[\frac{3-\sqrt{5}}{2}, \frac{3+\sqrt{5}}{2}\right]\ \text{per ogni} \ n\in\amsbb{N}
        \]
        e di conseguenza $(s_n)_n$ è monotona crescente ed è anche limitata. Ricordiamo il seguente teorema:
        \begin{tcolorbox}
            \begin{theorem}
                \label{th:4.7}
                Sia $(s_n)_n$ una successione monotona; questa converge se e solo se è limitata.
            \end{theorem}
        \end{tcolorbox}
        Di conseguenza sappiamo che $\lim_{n\to\infty} s_n$ esiste, ed è uguale ad $s\in\amsbb{R}$. Per determinarne il valore, notiamo che
        \[
        s = \lim_{n\to\infty} s_n = \lim_{n\to\infty} \left(1+\sqrt{s_{n-1}}\right) = 1+\lim_{n\to\infty}\sqrt{s_{n-1}} = 1+\sqrt{s}
        \]
        Risolvendo l'equazione $s=1+\sqrt{s}$ possiamo quindi trovare il valore del limite: in particolare risulta che l'equazione ha due soluzioni
        \[
        \xi_1 = \frac{3+ \sqrt{5}}{2} \qquad \xi_2 = \frac{3-\sqrt{5}}{2}
        \]
        ma la seconda è da scartare in quanto $s_0>\xi_1$ e $(s_n)_n$ è monotona crescente; quindi
        \[
        \lim_{n\to\infty} s_n = \frac{3+\sqrt{5}}{2}
        \]
\end{proof}
\begin{remark}
    Questo problema ammette un'interessante rappresentazione grafica:
    \begin{center}
        \begin{tikzpicture}[scale=1.8, font=\footnotesize]
            % \tzhelplines(5,5)
            \tzaxes(-.2,-.2)(3,3){$x$}[b]{$y$}[l]
            %\tzdot*(3.5, 3.8){$z=a+ib$}
            %\tzline[dashed, thick](0,0)(3.5, 3.8){$r$}[midway, a]
            %\tzarc(0,0)(0:47.35:2.5){$\theta$}[midway, r]
            %\txnode
            \tzfn[blue, thick]"curve"{1+sqrt(\x)}[0:3]{$y=1+\sqrt{x}$}[b]
            \tzfn[dashed,thick]"line"{\x}[0:3]{$y=x$}[l]
            \tzdot*[red](0, 1){$s_0$}[l]
            \tzfn[red, dashed]{1}[0:1]
            \tzvfnat[red, dashed]{1}[1:2]
            \tzprojy[draw=red](1, 2){$s_1$}
            \tzdot*[red](0, 2)
            \tzfn[red, dashed]{2}[1:2]
            \tzvfnat[red, dashed]{2}[2:2.4142]
            \tzprojy[draw=red](2, 2.4142){$s_2$}
            \tzdot*[red](0, 2.4142)
            \tzfn[red, dashed]{{1+sqrt(2)}}[2:{1+sqrt(2)}]
            \tzvfnat[red, dashed]{2.4142}[2.4142:2.5537]
            \tzprojy[draw=red](2.4142, 2.5537){$s_3$}
            \tzdot*[red](0, 2.5537)
            %\tzplotcurve[blue,thick]"curve"(.5,4.3)(1,4.2)(2.5,4.1)(4,.5){$y=g(x)$}[45]; % [ar] also works in version 2.0
            % intersection and projection
            \tzXpoint{line}{curve}(Y)
            \tzprojy[draw=red](Y){$s$}[al]
            \tzdot*[red](0, 2.6180)
            %\tzproj(3.5, 3.8){$a$}{$b$}
        \end{tikzpicture}
    \end{center}
    ossia la successione tende verso la coordinata $y$ (equivalentemente $x$) del \emph{punto fisso} di $f(x) = 1+\sqrt{x}$.
\end{remark}
\newpage
\section{Lezione 5}
\subsection{Ripasso: limiti di funzioni}
\begin{definition}
    \label{def:5.1}
    Data una funzione $f\colon I\subseteq \amsbb{R}\to\amsbb{R}$ e dato $y$ un punto di accumulazione\footnote{Un punto di accumulazione per $I\subseteq \amsbb{R}$ è un punto $y\in\amsbb{R}$ tale che per ogni $\delta>0$ esiste $x_\delta\in I$ tale che $x_\delta \in (y-\delta, y+ \delta)$. In questo modo possiamo dare senso al tendere ad $y0$ tramite elementi di $I$.} di $I$, diremo che
    \[
    \lim_{x\to y}f(x) = q
    \]
    se per ogni $\varepsilon>0$ esiste $\delta_\varepsilon>0$ tale che $\abs{f(x)-q}\varepsilon$ se $\abs{x-y}<\delta_{\varepsilon}$.
\end{definition}
\begin{theorem}
    \label{th:5.1}
    Data una funzione $f\colon I \to\amsbb{R}$, vale che
    \[
    \lim_{x\to y} f(x) = q
    \]
    se e solo se \textbf{per ogni} successione $(x_n)_n$ con $\{x_n, \ n\in\amsbb{N}\}\subseteq I$, $x_n\ne y$ per ogni $n\in\amsbb{N}$ e $x_n \to y$ vale che la successione $(f(x_n))_n$ converge a $q$.
\end{theorem}
\begin{definition}
    \label{def:5.2}
    Data una funzione $f\colon(a,b)\to \amsbb{R}$ e $y\in[a,b)$, diremo che
    \[
    \lim_{x\to y^+} f(x) = q
    \]
    se per ogni successione $(x_n)_n$ tale che $\{x_n, \ n\in\amsbb{N}\}\subseteq (y,b)$ e $x_n \to y$ la successione $(f(x_n))_n $ tende a $q$.\\
    Dato $y\in(a,b]$, diremo invece che
    \[
    \lim_{x\to y^-} f(x) = q
    \]
    se per ogni successione $(x_n)_n$ tale che $\{x_n, \ n\in\amsbb{N}\}\subseteq (a,y)$ e $x_n \to y$ la successione $(f(x_n))_n $ tende a $q$.\\
\end{definition}
\begin{theorem}
    \label{th:5.2}
    Dato $y\in(a,b)$, $\lim_{x\to y} f(x) =q$ se e solo se 
    \[
    \lim_{x\to y^+}f(x) = \lim_{x\to y^-}f(x) = q
    \]
\end{theorem}
\subsection{Esercizi: limiti di funzioni}
\begin{exercise}
    \label{ex:5.1}
    Calcolare, se esiste
    \[
    \lim_{x\to 1} \left(\frac{1}{2}\right)^{\frac{1}{x-1}}
    \]
\end{exercise}
\begin{proof}[Soluzione]
    Notiamo che per $x>1$ e $x$ prossimo a $1$ la funzione $\frac{1}{x-1}$ assume valori molto grandi e positivi, e di conseguenza $\left(\frac{1}{2}\right)^{\frac{1}{x-1}}$ tende ad essere molto piccolo; invece per $x<1$ e $x$ prossimo a $1$ la funzione $\frac{1}{x-1}$ assume valori molto grandi (in modulo) e negativi, e di conseguenza $\left(\frac{1}{2}\right)^{\frac{1}{x-1}}$ tende ad essere molto grande. Questo ragionamento euristico ci porta a dire che il limite non esiste. \\
    Per dimostrare che effettivamente è così, possiamo sfruttare il teorema \ref{th:5.1}: consideriamo quindi due successioni,
    \[
    a_n = 1+\frac{1}{n} \qquad b_n = 1-\frac{1}{n}
    \]
    notiamo che $a_n \to 1$ e $b_n \to 1$, e che $\{a_n, \ n\in\amsbb{N}\}\subseteq (1,+\infty)$ e $\{b_n, \ n\in\amsbb{N}\}\subseteq (-\infty, 1)$; calcoliamo quindi
    \[
    \lim_{n\to\infty} \left(\frac{1}{2}\right)^{\frac{1}{a_n-1}} = \lim_{n\to\infty} \left(\frac{1}{2}\right)^{\frac{1}{\frac{1}{n}}} = \lim_{n\to\infty}\left(\frac{1}{2}\right)^n = 0
    \]
    e 
    \[
    \lim_{n\to\infty} \left(\frac{1}{2}\right)^{\frac{1}{b_n-1}} = \lim_{n\to\infty} \left(\frac{1}{2}\right)^{\frac{1}{-\frac{1}{n}}} = \lim_{n\to\infty}\left(\frac{1}{2}\right)^{-n} = +\infty
    \]
    Abbiamo trovato quindi due successioni $(a_n)_n$ e $(b_n)_n$ con $a_n\to 1$ e $b_n\to 1$ tali che
    \[
    \lim_{n\to\infty}\left(\frac{1}{2}\right)^{\frac{1}{a_n-1}} \ne 
    \lim_{n\to\infty}\left(\frac{1}{2}\right)^{\frac{1}{b_n-1}}
    \]
    e di conseguenza per il teorema \ref{th:5.1} il limite non esiste.
\end{proof}
\begin{exercise}
    \label{ex:5.2}
    Calcolare
    \[
    \lim_{x\to 0}\frac{e^{3x^4}-\cos(x^2)}{\log(1+\sin(x^\alpha))}
    \]
    al variare di $\alpha\in\amsbb{R}$.
\end{exercise}
\begin{proof}[Soluzione]
    Per calcolare il limite, cerchiamo di ricondurci ai limiti notevoli
    \begin{tcolorbox}
        \[
        \underbrace{\lim_{x\to 0} \frac{e^x-1}{x} = 1}_{\stepcounter{equation}\mbox{(\theequation)}} \quad \underbrace{\lim_{x\to 0} \frac{1-\cos(x)}{x^2} = \frac{1}{2}}_{\stepcounter{equation}\mbox{(\theequation)}} \quad \underbrace{\lim_{x\to 0} \frac{\log(1+x)}{x} = 1}_{\stepcounter{equation}\mbox{(\theequation)}}
        \quad \underbrace{\lim_{x\to 0}  \frac{\sin(x)}{x} = 1}_{\stepcounter{equation}\mbox{(\theequation)}}
        \]
        \addtocounter{equation}{-3}\refstepcounter{equation}\label{eq:5.1}
        \addtocounter{equation}{0}\refstepcounter{equation}\label{eq:5.2}
        \addtocounter{equation}{0}\refstepcounter{equation}\label{eq:5.3}
        \addtocounter{equation}{0}\refstepcounter{equation}\label{eq:5.4}
    \end{tcolorbox}
    Per procedere, conviene separare due casi: $\alpha\ge 0$ e $\alpha<0$.
    \begin{enumerate}[(i)]
        \item se $\alpha=0$, vale che $x^0=1$ per ogni $x\in\amsbb{R}$, e quindi $\log(1+\sin(1))> 0$; bisogna quindi calcolare
        \[
        \lim_{x\to 0}e^{3x^4}-\cos(x^2)
        \]
        Notiamo che, grazie al teorema \ref{th:5.1}, le regole aritmetiche per i limiti di successioni (teoremi \ref{th:4.2}, \ref{th:4.3} e \ref{th:4.4}) valgono anche per i limiti di funzioni. Di conseguenza possiamo calcolare separatamente 
        \[
        \lim_{x\to0}e^{3x^4} = 1 \qquad \lim_{x\to 0}\cos(x^2) = 1
        \]
        ove i risultati sono dovuti alla continuità delle funzioni esponenziale, coseno ed elevamento a potenza.\\
        Quindi se $\alpha=0$
        \[
        \lim_{x\to 0}\frac{e^{3x^4}-\cos(x^2)}{\log(1+\sin(x^\alpha))} = 0
        \]
        \item Supponiamo quindi $\alpha>0$; notiamo che se $\alpha\notin \amsbb{N}$ la funzione è definita solamente sui numeri reali positivi, e di conseguenza il limite sarà in realtà un limite destro. Per ricondurci ai limiti (\ref{eq:5.1}) e (\ref{eq:5.2}), aggiungiamo e sottraiamo 1 a numeratore:
        \[
        \frac{e^{3x^4}-1+1-\cos(x^2)}{\log(1+\sin(x^\alpha))} = \frac{e^{3x^4}-1}{\log(1+\sin(x^\alpha))}+\frac{1-\cos(x^2)}{\log(1+\sin(x^\alpha))}
        \]
        e moltiplichiamo in entrambi i casi numeratore e denominatore per $\sin(x^\alpha)$:
        \[
        \frac{e^{3x^4}-1}{\log(1+\sin(x^\alpha))}+\frac{1-\cos(x^2)}{\log(1+\sin(x^\alpha))} = \frac{(e^{3x^4}-1)\sin(x^\alpha)}{\log(1+\sin(x^\alpha))\sin(x^\alpha)}+\frac{(1-\cos(x^2))\sin(x^\alpha)}{\log(1+\sin(x^\alpha))\sin(x^\alpha)}
        \]
        Poi nella prima frazione moltiplichiamo e dividiamo per $3x^4$, e nella seconda per $x^4$:
        \[
        \begin{split}
           & \frac{(e^{3x^4}-1)\sin(x^\alpha)}{\log(1+\sin(x^\alpha))\sin(x^\alpha)}+\frac{(1-\cos(x^2))\sin(x^\alpha)}{\log(1+\sin(x^\alpha))\sin(x^\alpha)} = \\
            & = \frac{(e^{3x^4}-1)\sin(x^\alpha)3x^4}{\log(1+\sin(x^\alpha))\sin(x^\alpha)3x^4}+\frac{(1-\cos(x^2))\sin(x^\alpha)x^4}{\log(1+\sin(x^\alpha))\sin(x^\alpha)x^4}
        \end{split}
        \]
        e infine moltiplichiamo e dividiamo per $x^\alpha$ ambo le frazioni:
        \[
        \begin{split}
            & \frac{(e^{3x^4}-1)\sin(x^\alpha)3x^4}{\log(1+\sin(x^\alpha))\sin(x^\alpha)3x^4}+\frac{(1-\cos(x^2))\sin(x^\alpha)x^4}{\log(1+\sin(x^\alpha))\sin(x^\alpha)x^4} = \\
            & \frac{(e^{3x^4}-1)\sin(x^\alpha)3x^{4}x^\alpha}{\log(1+\sin(x^\alpha))\sin(x^\alpha)3x^4 x^\alpha}+\frac{(1-\cos(x^2))\sin(x^\alpha)x^4x^\alpha}{\log(1+\sin(x^\alpha))\sin(x^\alpha)x^4x^\alpha}
        \end{split}
        \]
        Notiamo ora che se $\alpha>0$, per $x\to 0$ la funzione $\sin(x^\alpha)\to 0$; di conseguenza possiamo considerare
        \[
        \lim_{x\to 0} \frac{\log(1+\sin(x^\alpha))}{\sin(x^\alpha)} \overset{y = \sin(x^\alpha)}{=} \lim_{y\to 0} \frac{\log(1+y)}{y} \overset{(\ref{eq:5.3})}{=} 1
        \]
        allo stesso modo $3x^4\to 0$, $x^6\to 0$ e $x^\alpha\to0$per $x\to 0$, e quindi
        \[
        \lim_{x\to 0} \frac{e^{3x^4}-1}{3x^4} \overset{y = 3x^4}{=} \lim_{y\to 0} \frac{e^y-1}{y}\overset{(\ref{eq:5.1})}{=} 1
        \]
        \[
        \lim_{x\to 0} \frac{1-\cos(x^2)}{x^4} \overset{y = x^2}{=} \lim_{y\to 0} \frac{1-\cos(y)}{y^2} \overset{(\ref{eq:5.2})}{=} \frac{1}{2}
        \]
        \[
        \lim_{x\to 0} \frac{\sin(x^\alpha)}{x^\alpha} \overset{y = x^\alpha}{=} \lim_{y\to 0} \frac{\sin(y)}{y} \overset{(\ref{eq:5.4})}{=} 1
        \]
        Quindi
        \[
        \begin{split}
            & \frac{(e^{3x^4}-1)\sin(x^\alpha)3x^4x^\alpha}{\log(1+\sin(x^\alpha))\sin(x^\alpha)3x^4x^\alpha}+\frac{(1-\cos(x^2))\sin(x^\alpha)x^4x^\alpha}{\log(1+\sin(x^\alpha))\sin(x^\alpha)x^4x^\alpha} = \\
            & = \frac{e^{3x^4}-1}{3x^4}\frac{\sin(x^\alpha)}{\log(1+\sin(x^\alpha))}\frac{x^\alpha}{\sin(x^\alpha)}3x^{4-\alpha}+\frac{1-\cos(x^2)}{x^4}\frac{\sin(x^\alpha)}{\log(1+\sin(x^\alpha)}\frac{x^\alpha}{\sin(x^\alpha)}x^{4-\alpha}
        \end{split}
        \]
        Per applicare il teorema \ref{th:4.3} per i limiti dobbiamo solamente determinare il valore di
        \[
        \lim_{x\to 0} 3x^{4-\alpha} \qquad \lim_{x\to 0} x^{4-\alpha}
        \]
        Chiaramente se $4-\alpha>0$ le due funzioni tendono a 0 per la continuità dell'elevamento a potenza, e di conseguenza sempre per il teorema \ref{th:4.3} e \ref{th:4.2} abbiamo
        \[
        \begin{split}
            &\lim_{x\to 0} \frac{e^{3x^4}-\cos(x^2)}{\log(1+\sin(x^\alpha))} = \lim_{x\to 0} \overbrace{\frac{e^{3x^4}-1}{3x^4}}^{\to 1}\overbrace{\frac{\sin(x^\alpha)}{\log(1+\sin(x^\alpha))}}^{\to 1}\overbrace{\frac{x^\alpha}{\sin(x^\alpha)}}^{\to 1} \overbrace{3x^{4-\alpha}}^{\to 0} + \\
            & + \lim_{x\to 0}\underbrace{ \frac{1-\cos(x^2)}{x^4}}_{\to \frac{1}{2}}\underbrace{\frac{\sin(x^\alpha)}{\log(1+\sin(x^\alpha)}}_{\to1}\underbrace{\frac{x^\alpha}{\sin(x^\alpha)}}_{\to 1}\underbrace{x^{4-\alpha}}_{\to 0} = 0
        \end{split}
        \]
        se invece $4-\alpha=0$ le funzioni tendono a 3 e 1 rispettivamente, e di conseguenza
        \[
        \begin{split}
            &\lim_{x\to 0} \frac{e^{3x^4}-\cos(x^2)}{\log(1+\sin(x^\alpha))} = \lim_{x\to 0} \overbrace{\frac{e^{3x^4}-1}{3x^4}}^{\to 1}\overbrace{\frac{\sin(x^\alpha)}{\log(1+\sin(x^\alpha))}}^{\to 1}\overbrace{\frac{x^\alpha}{\sin(x^\alpha)}}^{\to 1} \overbrace{3x^{4-\alpha}}^{\to 3} + \\
            & + \lim_{x\to 0}\underbrace{ \frac{1-\cos(x^2)}{x^4}}_{\to \frac{1}{2}}\underbrace{\frac{\sin(x^\alpha)}{\log(1+\sin(x^\alpha)}}_{\to1}\underbrace{\frac{x^\alpha}{\sin(x^\alpha)}}_{\to 1}\underbrace{x^{4-\alpha}}_{\to 1} = 3 + \frac{1}{2} = \frac{7}{2}
        \end{split}
        \]
        Infine, se $4-\alpha<0$ dobbiamo considerare separatamente il caso $4-\alpha\in\amsbb{R}\setminus \amsbb{Z}$ e $4-\alpha\in \amsbb{Z}$. Nel primo caso la funzione è definita su $\amsbb{R}^+$ e
        \[
        \lim_{x\to 0} \frac{e^{3x^4}-\cos(x^2)}{\log(1+\sin(x^\alpha))} = \lim_{x\to 0^+}\frac{e^{3x^4}-\cos(x^2)}{\log(1+\sin(x^\alpha))}
        \]
        In questo caso ($4-\alpha<0$, $4-\alpha\in\amsbb{R}\setminus \amsbb{Z}$)
        \[
        \lim_{x\to 0^+} 3x^{4-\alpha} = \lim_{x\to 0^+} x^{4-\alpha} = +\infty
        \]
        e di conseguenza
        \[
        \lim_{x\to 0} \frac{e^{3x^4}-\cos(x^2)}{\log(1+\sin(x^\alpha))} = +\infty
        \]
        Altrimenti, ossia nel caso $4-\alpha\in\amsbb{Z}$, se $\abs{4-\alpha}$ è pari vale che
        \[
        \lim_{x\to 0} 3x^{4-\alpha} = \lim_{x\to 0} x^{4-\alpha} = +\infty
        \]
        altrimenti se $\abs{4-\alpha}$ è dispari le funzioni $3x^{4-\alpha}$ e $x^{4-\alpha}$ non hanno limite per $x\to 0$. Quindi possiamo concludere il caso $\alpha\ge 0$ dicendo che
        \[
        \lim_{x\to 0} \frac{e^{3x^4}-\cos(x^2)}{\log(1+\sin(x^\alpha))} = \begin{dcases}
            0\, & 0\le \alpha<4\\
            \frac{7}{2}\, & \alpha=4 \\
            +\infty\, & \alpha>4, \ \alpha\in\amsbb{R}\setminus \amsbb{N} \lor \alpha\in\amsbb{N}, \ \alpha \ \text{pari} \\
            \text{non esiste}\, & \alpha>4, \ \alpha\in\amsbb{N}, \ \alpha \ \text{dispari} 
        \end{dcases}
        \]
        \item nel caso $\alpha<0$, la funzione ha \emph{molti} problemi (vi invito ad utilizzare \href{https://www.desmos.com/calculator}{Desmos} per capire meglio la gravità della situazione).
        Infatti,
        \begin{enumerate}[(a)]
            \item la funzione non è definita se $\sin(x^\alpha) = 0$, ossia se 
            \[
            x^{\alpha} = k\pi, \ k\in\amsbb{Z} \iff x_k = \sqrt[\abs{\alpha}]{\left(\frac{1}{k\pi}\right)}
            \]
            (eventualmente con $k\in\amsbb{N}$ se $\alpha$ è un numero reale non intero);
            \item la funzione non è definita se $\sin(x^\alpha)=-1$, ossia se
            \[
            x^\alpha = \frac{3}{2}\pi + 2k\pi, \ k\in\amsbb{Z} \iff \hat{x}_k =\sqrt[\abs{\alpha}]{\frac{1}{\frac{3}{2}\pi + 2k\pi}}
            \]
            (anche in questo caso con $k\in\amsbb{N}$ se $\alpha\in\amsbb{R}\setminus \amsbb{Z}$).
        \end{enumerate}
        ossia man mano che ci avviciniamo all'origine troviamo sempre più punti in cui la funzione non è definita. Concentriamoci ora su $\{x\in\amsbb{R}\colon x>0\}$; vale che
        \begin{enumerate}[(a)]
            \item $e^{3x^4}-\cos(x^2)>0$ per ogni $x>0$: infatti $e^{3x^4}>1$ se $x>0$, e $\abs{\cos(x^2)}\le 1$; di conseguenza $-\cos(x)\ge -1$, e 
            \[
            e^{3x^4}-\cos(x)\ge e^{3x^4}-1 >0
            \]
            \item fissato $k\in\amsbb{N}$,
            \[
            2k\pi + \pi  < 2k\pi + \frac{3}{2}\pi < 2k\pi + 2\pi \iff \frac{1}{2k\pi + \pi}> \frac{1}{2k\pi + \frac{3}{2}\pi}>\frac{1}{2k\pi + 2\pi}
            \]
            e per la monotonia di $\sqrt[\abs{\alpha}]{\cdot}\colon \amsbb{R}^+\to\amsbb{R}^+$ vale che
            \[
            \sqrt[\abs{\alpha}]{\frac{1}{2k\pi + \pi}}> \sqrt[\abs{\alpha}]{\frac{1}{2k\pi + \frac{3}{2}\pi}}>\sqrt[\abs{\alpha}]{\frac{1}{2k\pi + 2\pi}}
            \]
            ossia
            \[
            x_{2k+1}>\hat{x}_{k}>x_{2k+2}
            \]
            \item $\sin(x^\alpha)$ è positivo per $x\in (x_{2k+1}, x_{2k})$ e negativo per $x\in (x_{2k+2}, x_{2k})$; quindi
            \[
            \frac{1}{\log(1+\sin(x^\alpha))}>0 \ \text{per} \ x\in(x_{2k+1}, x_{2k}) \  \forall k\in\amsbb{N}\setminus\{0\}
            \]
            e
            \[
            \frac{1}{\log(1+\sin(x^\alpha))}<0 \ \text{per} \ x\in(x_{2k+2}, x_{2k+1}) \  \forall k\in\amsbb{N}
            \]
            Questo significa che
            \[
            \lim_{x\to x_{2k+1}^+} \frac{e^{3x^4}-\cos(x^2)}{\log(1+\sin(x^\alpha))} = +\infty \qquad \lim_{x\to x_{2k+2}^-} \frac{e^{3x^4}-\cos(x^2)}{\log(1+\sin(x^\alpha))} = +\infty
            \]
            e 
            \[
            \lim_{x\to x_{2k+1}^-} \frac{e^{3x^4}-\cos(x^2)}{\log(1+\sin(x^\alpha))} = -\infty \qquad \lim_{x\to x_{2k+2}^+} \frac{e^{3x^4}-\cos(x^2)}{\log(1+\sin(x^\alpha))} = -\infty
            \]
            in quanto il denominatore tende a $0^\pm$; inoltre, dato che per $x\to \hat{x}_k$ il denominatore tende a $-\infty$ abbiamo che
            \[
            \lim_{x\to \hat{x}_k} \frac{e^{3x^4}-\cos(x^2)}{\log(1+\sin(x^\alpha))} = 0^-
            \]
        \end{enumerate}
        \begin{center}
        \begin{tikzpicture}[xscale=18, yscale = .25, font=\tiny]
            \tzaxes(0, -8)(0.7,15){$x$}[b]{$y$}[l]
            \tzticks{0.3183/$\frac{1}{(2k+1)\pi}$}[br]
            \tzticks{ 0.1063/$\frac{1}{(2k+3)\pi}$}[bl]
            \tzticks{0.1592/$\frac{1}{(2k+2)\pi}$}[ar]
            \tzfn[blue, thick, samples = 151]"curve"{exp(3*\x^4-cos(deg(\x^2)))/(ln(1+sin(deg(1/\x))))}[0.3215:0.6]{$\frac{e^{3x^4}-\cos(x^2)}{\log(1+\sin(x^{-1}))}$}[a]
            \tzfn[blue, thick, samples = 101]"curve"{exp(3*\x^4-cos(deg(\x^2)))/(ln(1+sin(deg(1/\x))))}[0.1603:0.3140]
            \tzfn[blue, thick, samples = 59]"curve"{exp(3*\x^4-cos(deg(\x^2)))/(ln(1+sin(deg(1/\x))))}[0.1065:0.1582]
            \tzvfnat[dashed]{0.3183}[-8:15]
            \tzvfnat[dashed]{0.1591}[-8:15]
            \tzvfnat[dashed]{0.1063}[-8:15]
            \tzdot*[red](0.2122, 0)
            \tznode[red](0.2122, 0){$2k\pi + \frac{3}{2}\pi$}[below, red]
        \end{tikzpicture}
    \end{center}
        Abbiamo tutti gli strumenti per dimostrare che la funzione non ammette limite per $x\to 0$ se $\alpha<0$; per farlo, notiamo che la funzione
        \[
        (x_{2k+2}, \hat{x}_k) \ni x\mapsto \frac{e^{3x^4}-\cos(x^2)}{\log(1+\sin(x^\alpha))}
        \]
        è continua, essendo composizione di funzioni continue; vale pertanto il teorema dei valori intermedi
        \begin{tcolorbox}
            \begin{theorem}
                \label{th:5.3}
                Sia $f\colon [a,b]\to \amsbb{R}$ una funzione continua tale che $f(a)<f(b)$ (vale un risultato analogo se $f(a)>f(b)$). Se $c\in(f(a), f(b))$ allora esiste $x\in(a,b)$ tale che $f(x)=c$.
            \end{theorem}
        \end{tcolorbox}
        Quindi per ogni $y\in(-\infty, 0)$ esiste $\xi_k\in(x_{2k+2}, \hat{x}_k)$ tale che
        \[
        \frac{e^{3\xi_k^4}-\cos(\xi_k^2)}{\log(1+\sin(\xi_k^\alpha))} = y
        \]
        Fissiamo quindi $y_1$ e $y_2$ in $(-\infty, 0$, con $y_1 \ne y_2$. Possiamo quindi associare ad ogni intervallo $(x_{2k+2}, \hat{x}_k)$ uno $\xi_k$ e uno $\zeta_k$ tali che la funzione se valutata in $\xi_k$ dia sempre $y_1$ e se valutata in $\zeta_k$ dia sempre $y_2$. Notiamo che, poiché
        \[
        \sqrt[\abs{\alpha}]{\frac{1}{(2k+2)\pi}} = x_{2k+1}<\xi_k<\hat{x}_k =\sqrt[\alpha]{\frac{1}{\frac{3}{2}\pi + 2k\pi}}
        \]
        e sia $x_{2k+1}$, sia $\hat{x}_k$ tendono a 0 per $k\to \infty$ per il teorema \ref{th:4.5} vale che $\xi_k \to 0$ per $k\to\infty$, e lo stesso vale per $\zeta_k$. Abbiamo quindi due successioni, $(\xi_k)_k$ e $(\zeta_k)_k$, tali che
        \[
        \frac{e^{3\xi_k^4}-\cos(\xi_k^2)}{\log(1+\sin(\xi_k^\alpha))} \to y_1 \qquad \frac{e^{3\zeta_k^4}-\cos(\zeta_k^2)}{\log(1+\sin(\zeta_k^\alpha))} \to y_2
        \]
        essendo le due successioni costanti. Di conseguenza per la definizione \ref{def:5.2} il limite 
        \[
        \lim_{x\to 0^+} \frac{e^{3x^4}-\cos(x^2)}{\log(1+\sin(x^\alpha))}
        \]
        non esiste se $\alpha<0$, e di conseguenza per il teorema \ref{th:5.2} non esiste nemmeno
        \[
        \lim_{x\to 0} \frac{e^{3x^4}-\cos(x^2)}{\log(1+\sin(x^\alpha))}
        \]
        se $\alpha\in\amsbb{Z}$.
    \end{enumerate}
\end{proof}
\subsection{Ripasso: continuità}
\begin{definition}
    \label{def:5.3}
    Data una funzione $f\colon I \to \amsbb{R}$, diremo che $f$ è \emph{continua} in $x_0\in I$ se
    \[
    \lim_{x\to x_0} f(x) = f(x_0)
    \]
\end{definition}
\begin{remark}
    Notiamo che affinché la definizione abbia senso, la funzione $f$ deve essere definita in $x_0$.\\
    $f\colon I \to \amsbb{R}$ si dirà continua su $I$ se $f$ è continua in ogni punto di $I$.
\end{remark}
\begin{example}
    Vogliamo mostrare che $\exp(\cdot)\colon \amsbb{R}\to\amsbb{R}^+$ è continua in $\amsbb{R}$. Notiamo che è sufficiente dimostrare la continuità in $x=0$; infatti, dato $x\ne 0$, se consideriamo una successione $(x_n)_n$ con $x_n\to x$, definendo la successione $(\hat{x}_n = x_n -x)_n$ abbiamo che $\hat{x}_n\to 0$ e, se $\exp$ è continua in 0, per il teorema \ref{th:5.1}
    \[
    1=\lim_{n\to\infty} e^{\hat{x}_n}=\lim_{n\to\infty}e^{x_n-x} \iff \lim_{n\to\infty} e^{x_n} = e^x
    \]
    Consideriamo quindi, sempre per il teorema \ref{th:5.1}, una generica successione $(x_n)_n$ tale che $x_n\to 0$; per la definizione \ref{def:4.2} sappiamo che per ogni $\varepsilon>0$ esiste $n_\varepsilon$ tale che
    \[
    \abs{x_n}<\varepsilon \iff -\varepsilon<x_n < \varepsilon \ \text{se} \ n>n_\varepsilon
    \]
    Fissiamo ora $\delta>0$, e consideriamo $\varepsilon_1 = \log(1+\delta)$; notiamo che $\varepsilon_1>0$ in quanto $\delta>0$, e di conseguenza per quanto detto prima esiste $n_{\varepsilon_1}$ tale che
    \[
    -\varepsilon_1 < x_n < \varepsilon_1 \ \text{se} \ n>n_{\varepsilon_1}
    \]
    In particolare, poiché $\exp(\cdot)\colon \amsbb{R}\to \amsbb{R}^+$ è monotona strettamente crescente, vale che
    \begin{equation}
        \label{eq:5.6}
        e^{x_n}<e^{\varepsilon_1} = 1+\delta \ \text{se} \ n>n_{\varepsilon_1}
    \end{equation}
    Allo stesso modo, se fissiamo $\varepsilon_2 = -\log(1-\delta)$; anche in questo caso $\varepsilon_2>0$, in quanto $\log(1-\delta)<0$ per $\delta>0$; anche in questo caso quindi possiamo trovare $n_{\varepsilon_2}$ tale che
    \[
    -\varepsilon_2 < x_n < \varepsilon_2 \ \text{se} \ n>n_{\varepsilon_2}
    \]
    Sempre sfruttando la monotonia di $\exp(\cdot)$ possiamo quindi scrivere
    \begin{equation}
        \label{eq:5.7}
        e^{x_n}>e^{-\varepsilon_2} = e^{-(-\log(1-\delta))} = 1-\delta \ \text{se} \ n>n_{\varepsilon_2}
    \end{equation}
    Definiamo ora $n_\delta = \max\{n_{\varepsilon_1}, n_{\varepsilon_2}\}$; per (\ref{eq:5.6}) e (\ref{eq:5.7}) vale che, se prendiamo $n>n_\delta$,
    \[
    1-\delta < e^{x_n} < 1+\delta
    \]
    ossia dato un generico $\delta>0$ abbiamo trovato $n_\delta$ tale che $\abs{e^{x_n}-1}<\delta$ se $n>n_\delta$; quindi $e^{x_n}\to 1$. Poiché la successione che abbiamo considerato è generica abbiamo che
    \[
    \lim_{x\to 0} e^x = 1 = e^0
    \]
    e quindi $\exp(\cdot)\colon \amsbb{R}\to \amsbb{R}^+$ è continua in 0.
\end{example}
\subsection{Esercizi: continuità}
\begin{exercise}
    \label{ex:5.3}
    Determinare per quali valori dei parametri $\alpha, \beta\in\amsbb{R}$ la funzione
    \[
    f(x) = \begin{dcases}
        x^\alpha\sin^2(x)\, & 0<x<1\\
        0\, & x=0\\
        \abs{x}^\beta \arctan(x)\, & -1<x<0
    \end{dcases}
    \]
\end{exercise}
\begin{proof}[Soluzione]
    Notiamo innanzitutto che $(0,1)\ni x \mapsto x^\alpha \sin^2(x)$ è continua, essendo composizione e prodotto di funzioni continue. Allo stesso modo, $(-1, 0)\ni x \mapsto \abs{x}^\beta \arctan(x)$ è continua, per lo stesso motivo. L'unico punto problematico può quindi essere $x=0$. Per la definizione \ref{def:5.3} e per il teorema \ref{th:5.2}, $f$ è continua in $x=0$ se
    \[
    \underbrace{\lim_{x\to 0^+} f(x)}_{\stepcounter{equation}\mbox{(\theequation)}} = \underbrace{\lim_{x\to 0^-} f(x)}_{\stepcounter{equation}\mbox{(\theequation)}} = f(0) = 0
    \]
    \addtocounter{equation}{-2}\refstepcounter{equation}\label{eq:5.8}
    \addtocounter{equation}{0}\refstepcounter{equation}\label{eq:5.9}
    Calcoliamo quindi i due limiti, sfruttando il teorema \ref{th:4.3}:
    \begin{enumerate}[(i)]
        \item per quanto riguarda (\ref{eq:5.8}), sappiamo che in $(0,1)$ la funzione è definita da $x^\alpha\sin^2(x)$; quindi
        \[
        \lim_{x\to 0^+}f(x) = \lim_{x\to 0^+} x^\alpha \sin^2(x) = \lim_{x\to 0^+} x^{\alpha +2} \frac{\sin^2(x)}{x^2} = \begin{dcases}
            0\, & \alpha>-2\\
            1\, & \alpha = -2\\
            +\infty\, & \alpha<-2
        \end{dcases}
        \]
        \item per (\ref{eq:5.9}) vale invece che, essendo $f$ definita come $\abs{x}^\beta \arctan(x)$ in $(-1,0)$,
        \[
        \lim_{x\to 0^-} f(x) = \lim_{x\to 0^-} \abs{x}^\beta \arctan(x) = \lim_{x\to 0^-} -(-x)^{\beta+1} \frac{\arctan(x)}{x}= \begin{dcases}
            0\, & \beta>-1\\
            -1\, & \beta=-1\\
            -\infty\, & \beta<-1
        \end{dcases}
        \]
    \end{enumerate}
    Di conseguenza la funzione $f$ è continua in $x=0$ (e quindi è continua sul suo insieme di definizione) se $\alpha>-1$ e $\beta>-1$.
\end{proof}
\subsection{Ripasso: simboli di Landau (\texorpdfstring{$o$}{o}-piccolo)}
\begin{definition}
    \label{def:5.4}
    Date $f\colon I \to \amsbb{R}$ e $g\colon I \to \amsbb{R}$ e $x_0\in\amsbb{R}$ un punto di accumulazione per $I$ (o eventualmente $+\infty$), diremo che $f$ \emph{è un $o$-piccolo di $g$ per $x\to x_0$}, scritto $f=o(g)$, se
    \[
    \lim_{x\to x_0} \frac{f(x)}{g(x)} = 0
    \]
\end{definition}
\subsection{Esercizi: limiti con i simboli di Landau}
\begin{exercise}
    \label{ex:5.4}
    Calcolare, se esiste,
    \[
    \lim_{x\to +\infty} \left(\sqrt[4]{1+   \arctan\left(\frac{5}{x^2}\right)}-\cos\left(\frac{3}{x}\right)\right)x^2
    \]
\end{exercise}
\begin{proof}[Soluzione]
    Notiamo che se definiamo $y = \frac{5}{x^2}$, abbiamo che $y\to 0$ in quanto $x\to +\infty$; di conseguenza, ricordando che, per $\xi\to 0$,
    \begin{tcolorbox}
    \[
    \underbrace{\arctan(\xi) = \xi+o(\xi)}_{\stepcounter{equation}\mbox{(\theequation)}} \qquad \underbrace{\sqrt[4]{1+\xi} = 1+\frac{\xi}{4}+o(\xi)}_{\stepcounter{equation}\mbox{(\theequation)}} \qquad \underbrace{\cos(\xi) = 1-\frac{\xi^2}{2}+o(\xi^2)}_{\stepcounter{equation}\mbox{(\theequation)}}
    \]
    \addtocounter{equation}{-3}\refstepcounter{equation}\label{eq:5.10}
    \addtocounter{equation}{0}\refstepcounter{equation}\label{eq:5.11}
    \addtocounter{equation}{0}\refstepcounter{equation}\label{eq:5.12}
    \end{tcolorbox}
    possiamo scrivere
    \[
    \begin{split}
        \sqrt[4]{1+\arctan\left(\frac{5}{x^2}\right)} & = \sqrt[4]{1+\arctan(y)} \overset{(\ref{eq:5.11})}{=} 1+\frac{\arctan(y)}{4}+o(\arctan(y)) \overset{(\ref{eq:5.10})}{=} \\
        & = 1 + \frac{y+o(y)}{4} + o(y+o(y)) = 1 + \frac{y}{4}+o(y)
    \end{split} 
    \]
    ove abbiamo usato il fatto che $\arctan(y)\to 0$ per $y\to 0$ e le proprietà dei simboli di Landau
    \[
    o(y+o(y)) = o(y) \qquad o(y)+o(y) = o(y)
    \]
    Riscrivendo in funzione di $x$ abbiamo che per $x\to +\infty$
    \[
    \sqrt[4]{1+\arctan\left(\frac{5}{x^2}\right)} = 1 + \frac{5}{4}\frac{1}{x^2} + o\left(\frac{1}{x^2}\right)
    \]
    Allo stesso modo, definendo $y=\frac{3}{y}$ abbiamo
    \[
    \cos\left(\frac{3}{x}\right) = \cos(y) \overset{(\ref{eq:5.12})}{=} 1-\frac{y^2}{2} + o(y^2)
    \]
    che riscritta in funzione di $x$ diventa
    \[
    \cos\left(\frac{3}{x}\right) = 1-\frac{9}{2}\frac{1}{x^2} + o\left(\frac{1}{x^2}\right) \ \text{per} \ x\to+\infty 
    \]
    Il limite diventa quindi
    \[
    \begin{split}
        & \lim_{x\to+\infty} \left(\sqrt[4]{1+   \arctan\left(\frac{5}{x^2}\right)}-\cos\left(\frac{3}{x}\right)\right)x^2 = \\
        & =  \lim_{x\to+\infty}\left(1+\frac{5}{4}\frac{1}{x^2} + o\left(\frac{1}{x^2}\right)-1+\frac{9}{2}\frac{1}{x^2}+o\left(\frac{1}{x^2}\right)\right)x^2 = \\
        & = \lim_{x\to+\infty}\left(\frac{5}{4}\frac{1}{x^2}+\frac{9}{2}\frac{1}{x^2} + o\left(\frac{1}{x^2}\right)\right)x^2 = \lim_{x\to+\infty} \frac{23}{4}+o\left(\frac{1}{x^2}\right)x^2
    \end{split}
    \]
    Ricordiamo che, per la definizione \ref{def:5.4}, una funzione $f$ è $o\left(\frac{1}{x^2}\right)$ per $x\to+\infty$ se
    \[
    \lim_{x\to+\infty}\frac{f(x)}{\frac{1}{x^2}} = \lim_{x\to+\infty}x^2 f(x) = 0
    \]
    Quindi 
    \[
    \lim_{x\to+\infty} o\left(\frac{1}{x^2}\right)x^2 = 0
    \]
    e di conseguenza
    \[
    \lim_{x\to +\infty} \left(\sqrt[4]{1+   \arctan\left(\frac{5}{x^2}\right)}-\cos\left(\frac{3}{x}\right)\right)x^2 = \frac{23}{4}
    \]
\end{proof}
\begin{exercise}
    \label{ex:5.5}
    Calcolare, se esiste,
    \[
    \lim_{x\to 0} \frac{1-e^{\cos(x)-1}}{\sqrt{\cos(\log(1+\sin(x)))}-1}
    \]
\end{exercise}
\begin{proof}[Soluzione]
    Per svolgere l'esercizio, consideriamo i seguenti sviluppi in termini di simboli di Landau per $\xi \to 0$: 
    \begin{tcolorbox}
    \[
    \!\!\!
    \underbrace{e^\xi= 1+\xi+o(\xi)}_{\stepcounter{equation}\mbox{(\theequation)}} \quad \underbrace{\log(1+\xi) = \xi + o(\xi)}_{\stepcounter{equation}\mbox{(\theequation)}} \quad \underbrace{\sin(\xi) = \xi + o(\xi)}_{\stepcounter{equation}\mbox{(\theequation)}} \quad \underbrace{\sqrt{1+\xi} = 1 + \frac{\xi}{2}+o(\xi)}_{\stepcounter{equation}\mbox{(\theequation)}}
    \]
    \addtocounter{equation}{-4}\refstepcounter{equation}\label{eq:5.13}
    \addtocounter{equation}{0}\refstepcounter{equation}\label{eq:5.14}
    \addtocounter{equation}{0}\refstepcounter{equation}\label{eq:5.15}
    \addtocounter{equation}{0}\refstepcounter{equation}\label{eq:5.16}
    \end{tcolorbox}
    Consideriamo separatamente numeratore e denominatore:
    \begin{enumerate}[(i)]
        \item Per il numeratore, notiamo che se definiamo $y= \cos(x)-1$ vale che $y\to 0$ per $x\to 0$; quindi
        \[
        \begin{split}
            e^{\cos(x)-1} & = e^y \overset{(\ref{eq:5.13})}{=} 1+y+o(y) = 1 + (\cos(x)-1) + o(\cos(x)-1) = \\
            & = \cos(x)+o(\cos(x)-1) \overset{(\ref{eq:5.12})}{=} 1-\frac{x^2}{2}+o(x^2) + o\left(1-\frac{x^2}{2}-1\right) = \\
            & = 1-\frac{x^2}{2}+o(x^2)
        \end{split}
        \]
        e di conseguenza per $x\to 0$ il numeratore è
        \[
        1-e^{\cos(x)-1} = 1-\left(1-\frac{x^2}{2}+o(x^2)\right) = \frac{x^2}{2} + o(x^2)
        \]
        \item Notiamo che $y=\sin(x)\to0$ per $x\to 0$, che $z = \log(1+y)\to 0$ per $y\to 0$ e che $\xi = \cos(z)-1\to 0$ per $z\to 0$; quindi
        \[
        \begin{split}
            & \sqrt{\cos(\log(1+\sin(x)))}-1 = \sqrt{1+(\cos(z)-1)}-1 = \sqrt{1+\xi} -1\overset{(\ref{eq:5.16})}{=}\\
            & = 1+\frac{\xi}{2} + o(\xi) -1 = \frac{\cos(z)-1}{2} + o(\cos(z)-1) \overset{(\ref{eq:5.12})}{=} \\
            & = \frac{1}{2}\left(1-\frac{z^2}{2}+o(z^2)-1\right)+o\left(\frac{z^2}{2}+o(z^2)\right) = -\frac{z^2}{4}+o(z^2) =\\
            & = -\frac{\log^2(1+y)}{4}+o(\log^2(1+y)) \overset{(\ref{eq:5.14})}{=} -\frac{(y+o(y))^2}{4} +o((y+o(y))^2) =  \\
            & = -\frac{y^2 +2yo(y)+o(y)^2}{4}+o(y^2 +2yo(y)+o(y)^2) = -\frac{y^2}{4}+o(y^2) = \\
            & = -\frac{\sin^2(x)}{4}+o(\sin^2(x)) \overset{(\ref{eq:5.15})}{=} -\frac{(x+o(x))^2}{4} + o((x+o(x))^2) = -\frac{x^2}{4}+o(x^2)
        \end{split}
        \]
    \end{enumerate}
    Quindi per $x\to 0$
    \[
    \frac{1-e^{\cos(x)-1}}{\sqrt{\cos(\log(1+\sin(x)))}-1} = \frac{\frac{x^2}{2}+o(x^2)}{-\frac{x^2}{4}+o(x^2)} = -\frac{\frac{x^2}{2}}{\frac{x^2}{4}}\frac{\left(1+2\frac{o(x^2)}{x^2}\right)}{\left(1-4\frac{o(x^2)}{x^2}\right)} = -2\frac{\left(1+2\frac{o(x^2)}{x^2}\right)}{\left(1-4\frac{o(x^2)}{x^2}\right)}
    \]
    Come prima, per definizione di $o$-piccolo, $\lim_{x\to 0} \frac{o(x^2)}{x^2} = 0$; di conseguenza, sfruttando i teoremi \ref{th:4.2}, \ref{th:4.3} e \ref{th:4.4} vale che
    \[
    \lim_{x\to 0} \frac{1-e^{\cos(x)-1}}{\sqrt{\cos(\log(1+\sin(x)))}-1} = \lim_{x\to 0}-2\frac{\left(1+2\frac{o(x^2)}{x^2}\right)}{\left(1-4\frac{o(x^2)}{x^2}\right)} = -2
    \]
\end{proof}
\newpage
\section{Lezione 6}
\subsection{Ripasso: derivabilità e differenziabilità, interpretazione geometrica della derivata}
\begin{definition}
    \label{def:6.1}
    Sia $f\colon(a,b)\to\amsbb{R}$, e sia $x\in(a,b)$; definiamo il \emph{rapporto incrementale di $f$ in $x$} come la funzione $\phi_x \colon (a,b) \setminus \{x\} \to \amsbb{R}$,
    \[
    \phi_x(t) = \frac{f(t)-f(x)}{t-x}
    \]
    Diremo che $f$ è \emph{derivabile in $x$} se esiste finito
    \begin{equation}
        \label{eq:6.1}
        \lim_{t\to x} \phi_x(t) = \lim_{t\to x} \frac{f(t)-f(x)}{t-x} = f'(x)
    \end{equation}
\end{definition}
\begin{remark}
    Detto $I\subseteq (a,b)$ l'insieme dei punti in cui $f$ è derivabile, possiamo definire una funzione $ I\ni x \mapsto f'(x)\in \amsbb{R}$, detta \emph{derivata prima} di $f$, che si indica con il simbolo $f'$.
\end{remark}
\begin{definition}
    \label{def:6.2}
    Dati $f\colon(a,b)\to\amsbb{R}$ e $x\in(a,b)$, diremo che $f$ è \emph{differenziabile in $x$} se esiste un'applicazione lineare $A_x\colon\amsbb{R}\to \amsbb{R}$ tale che
    \begin{equation}
        \label{eq:6.2}
        \lim_{\abs{h}\to0}\frac{\abs{f(x+h)-f(x)-A_x h}}{\abs{h}}=0
    \end{equation}
\end{definition}
\begin{remark}
    Ricordiamo che vale il teorema \ref{th:4.6}, e che quindi $\abs{h}\to 0$ se e solo se $h\to0$ e $\abs{f(h)}\to0$ se e solo se $f(h)\to0$. Possiamo quindi riscrivere la (\ref{eq:6.1}) come
    \[
    \lim_{h\to 0} \frac{f(x+h)-f(x)-A_x h}{h}=0
    \]
    Per la definizione \ref{def:5.4} vale quindi che
    \[
    f(x+h)-f(x)-A_xh = o(h) \iff f(x+h) = f(x) + A_x h+o(h)
    \]
    ossia $f$ è differenziabile in $x$ se in un intorno di $h$ possiamo approssimare $f$ con un'applicazione lineare $A_x$ commettendo un errore contenuto ($o(h)$).
\end{remark}
\begin{remark}
    Dai corsi di geometria sappiamo che, fissata una base di $\amsbb{R}$, esiste una corrispondenza biunivoca fra applicazioni lineari $A_x\in L(\amsbb{R}, \amsbb{R})$ e matrici $1\times 1$ a coefficienti reali, ossia numeri reali: in particolare, data l'applicazione lineare $A_x\colon \amsbb{R}\to\amsbb{R}$, esiste $m_{A_x}\in\amsbb{R}$ tale che
    \[
    A_x(h) = m_{A_x}h \ \text{per ogni} \ h\in\amsbb{R}
    \]
\end{remark}
\begin{theorem}
    \label{th:6.1}
    Data $f\colon (a,b)\to\amsbb{R}$, $f$ è differenziabile in $x$ se e solo se $f$ è derivabile in $x$.
\end{theorem}
\begin{proof}
    Dimostriamo le due implicazioni.
    \begin{enumerate}[(i)]
        \item Supponiamo che $f$ sia differenziabile in $x$, e mostriamo che $f$ è derivabile in $x$. Sappiamo che
        \[
        f(x+h)=f(x)+A_x h +o(h) = f(x)+m_{A_x}h +o(h)
        \]
        e di conseguenza
        \[
        \lim_{h\to 0} \frac{f(x+h)-f(x)}{h} = \lim_{h\to 0} \frac{f(x)+m_{A_x}h+o(h)-f(x)}{h} = m_{A_x}+\lim_{h\to 0} \frac{o(h)}{h} = m_{A_x} 
        \]
        Notiamo che il limite in (\ref{eq:6.1}) può essere scritto come
        \[
        \lim_{t\to x} \frac{f(t)-f(x)}{t-x} = \lim_{t\to x} \frac{f(t-x+x)-f(x)}{t-x}\overset{h=t-x}{=} \lim_{h\to 0} \frac{f(x+h)-f(x)}{h}
        \]
        in quanto $h\to 0$ se $t\to x$. Abbiamo quindi che il limite (\ref{eq:6.1}) esiste finito, e che $f'(x) = m_{A_x}$.
        \item Supponiamo ora che $f$ sia derivabile in $x$ e mostriamo che $f$ è differenziabile in x. Per definizione vale che
        \[
        \lim_{t\to x}\frac{f(t)-f(x)}{t-x} = \lim_{h\to 0}\frac{f(x+h)-f(x)}{h} = f'(x)
        \]
        ossia
        \[
        \lim_{h\to 0}\frac{f(x+h)-f(x)}{h}-f'(x) = 0
        \]
        Possiamo riscrivere il precedente limite come
        \[
        0=\lim_{h\to 0}\frac{f(x+h)-f(x)}{h}-f'(x) = \lim_{h\to 0} \frac{f(x+h)-f(x)-f'(x)h}{h}
        \]
        Per il teorema \ref{th:4.6} sappiamo che 
        \[
        \lim_{h\to 0}\abs{\frac{f(x+h)-f(x)-f'(x)h}{h}} = \lim_{h\to 0} \frac{\abs{f(x+h)-f(x)-f'(x)h}}{\abs{h}}=0
        \]
        e di conseguenza l'applicazione lineare $\amsbb{R}\ni h \mapsto f'(x)h$ indotta da $f'(x)\in\amsbb{R}$ soddisfa la definizione \ref{def:6.2}.
    \end{enumerate}
\end{proof}
\begin{remark}
    Questo teorema ci dice che per funzioni $f\colon \amsbb{R}\to \amsbb{R}$ differenziabilità e derivabilità sono due concetti equivalenti. Non è così per funzioni $f\colon \amsbb{R}^n\to \amsbb{R}$! Ci sono esempi\footnote{Ad esempio, la funzione $\amsbb{R}^2\ni (x,y)\mapsto \frac{x^2y}{x^2+y^2}$ per $(x,y)\ne (0,0)$ e $f(0,0)=0$ è derivabile (direzionalmente) in $(0,0)$ ma non è differenziabile.} di funzioni derivabili ma non differenziabili.
\end{remark}
\begin{theorem}
    \label{th:6.2}
    Siano $f,g\colon (a,b)\to \amsbb{R}$ due funzioni differenziabili in $x\in(a,b)$; allora
    \begin{enumerate}[(i)]
        \item $(f+g)'(x) = f'(x) + g'(x)$;
        \item $(\lambda f)'(x) = \lambda f'(x)$;
        \item (Regola di Leibniz)
        \begin{equation}
            \label{eq:6.3}
            (fg)'(x) = f'(x)g(x) + f(x)g'(x)
        \end{equation}
    \end{enumerate}
    Inoltre, se $g\colon (c,d) \to \amsbb{R}$ è tale per cui $f((a,b))\subseteq (c,d)$ e $g$ è differenziabile in $f(x)\in(c,d)$, allora $g\circ f \colon (a,b) \to \amsbb{R}$ è differenziabile in $x$ e vale la regola della catena
    \begin{equation}
        \label{eq:6.4}
        (g\circ f)'(x) = g'(f(x))f'(x)
    \end{equation}
\end{theorem}
\begin{example}
    Consideriamo la funzione $f\colon\amsbb{R}\to \amsbb{R}$,
    \[
    f(x) = \begin{dcases}
        x^2\sin\left(\frac{1}{x}\right)\, & x\ne 0\\
        0\, & x=0
    \end{dcases}
    \]
    e cerchiamo di determinare l'insieme $I\subseteq \amsbb{R}$ in cui è differenziabile. In $\amsbb{R}\setminus\{0\}$ la funzione è prodotto e composizione di funzioni differenziabili, e quindi per il teorema \ref{th:6.2} vale che $f$ è differenziabile in $\amsbb{R}\setminus \{0\}$, e la sua derivata prima in $\amsbb{R}\setminus\{0\}$ è data da
    \[
    \begin{split}
        f'(x) & = \frac{d}{dx}\left(x^2 \sin\left(\frac{1}{x}\right)\right) \overset{(\ref{eq:6.3})}{=} 2x\sin\left(\frac{1}{x}\right) + x^2 \frac{d}{dx}\left(\sin\left(\frac{1}{x}\right)\right) \overset{(\ref{eq:6.4})}{=} \\
        & = 2x\sin\left(\frac{1}{x}\right)+x^2 \cos\left(\frac{1}{x}\right)\left(-\frac{1}{x^2}\right) = 2x\sin\left(\frac{1}{x}\right)-\cos\left(\frac{1}{x}\right)
    \end{split}
    \]
    Per la derivabilità in $0$, bisogna considerare invece il rapporto incrementale (\ref{eq:6.1}),
    \[
    \phi_0(t) = \frac{f(t)-f(0)}{t-0} = \frac{1}{t}t^2\sin\left(\frac{1}{t}\right) = t\sin\left(\frac{1}{t}\right)
    \]
    e calcolare $\lim_{t\to 0} \phi_0(t)$. Notiamo che, poiché $\abs{\sin(x)}\le 1$ per ogni $x\in\amsbb{R}$,
    \[
    0\le \abs{\phi_0(t)}= \abs{t\sin\left(\frac{1}{t}\right)} \le \abs{t}
    \]
    e quindi per il teorema \ref{th:4.5} vale che $\abs{\phi_0(t)}\to 0$ per $t\to 0$, e per il teorema \ref{th:4.6} vale che
    \[
    \lim_{t\to 0} \phi_0(t) = 0 = f'(0)
    \]
    Quindi $f$ è derivabile su tutto $\amsbb{R}$, e
    \[
    f'(x) = \begin{dcases}
        2x\sin\left(\frac{1}{x}\right)-\cos\left(\frac{1}{x}\right)\, & x\ne 0\\
        0\, & x=0
    \end{dcases}
    \]
    Notiamo che $f'$ è continua in $\amsbb{R}\setminus\{0\}$, essendo composizione e prodotto di funzioni continue,  ma non è continua in $0$: infatti, considerando le due successioni
    \[
    a_n = \frac{1}{2n\pi} \qquad b_n = \frac{1}{(2n+1)\pi}
    \]
    che tendono a $0$ per $n\to\infty$ vale che
    \[
    \lim_{n\to\infty} f'(a_n) = \frac{1}{2n\pi}\sin(2n\pi))-\cos(2n\pi) = \lim_{n\to\infty} -\cos(2n\pi) = -1
    \]
    e che
    \[
    \lim_{n\to\infty} f'(b_n) = \frac{1}{(2n+1)\pi}\sin((2n+1)\pi))-\cos((2n+1)\pi) = \lim_{n\to\infty} -\cos((2n+1)\pi) = 1
    \]
    Per il teorema \ref{th:5.1} quindi $\lim_{t\to 0} f'(t)$ non esiste, e quindi $f'$ non può essere continua in 0.
\end{example}
Data una funzione $f\colon(a,b)\to\amsbb{R}$, sappiamo che $f$ è derivabile in $x_0\in(a,b)$ se 
\[
f(x_0+h) -f(x_0) = f'(x_0)h+o(h)
\]
Consideriamo quindi $F\colon h\mapsto f(x_0+h)-f(x_0)$; sappiamo che
\[
F(h) = f'(x_0)h +o(h)
\]
ossia $h\mapsto f'(x_0)h$ è l'applicazione lineare che meglio approssima $F$ in prossimità dell'origine.
\begin{center}
    \begin{minipage}{.45\linewidth}
        \centering
        \begin{tikzpicture}[scale=1.3, font=\footnotesize]
            \tzaxes(-1.2,-1.2)(4,2.4){$x$}[b]
            \tzfn[blue]"curve"{cos(3*deg(\x))+sin(.9*deg(\x))}[-.5:3.3]{$f(x)$}[b]
            \tzproj(1.5, 0.7649){$x_0$}{$f(x_0)$}
        \end{tikzpicture}
    \end{minipage}
    \hspace{1pt}
    \begin{minipage}{.45\linewidth}
        \centering
        \begin{tikzpicture}[scale=1.3, font=\footnotesize]
            \tzaxes(-2.7,-1.9)(2.5,1.7){$h$}[b]
            \tzfn[red]"curve"{cos(3*deg(\x+1.5))+sin(.9*deg(\x+1.5))-0.7649}[-2:1.8]{$F(h)$}[b]
            \tzfn[teal]"line"{3.1297*\x}[-.5:.5]{$y=f'(x_0)h$}[r]
        \end{tikzpicture}
    \end{minipage}
\end{center}
Notiamo che il grafico, riportato a sinistra, di $F$, altro non è se non il grafico di $f(x)$ traslato di $(-x_0, -f(x_0))$: infatti, se consideriamo la coppia $(h, F(h))$ e la trasliamo di $(x_0, f(x_0))$ otteniamo
\[
(h, F(h)) \longrightarrow (h+x_0, F(h)+f(x_0)) = (h+x_0, f(h+x_0))
\]
Possiamo sempre scrivere $h=x-x_0$ per un qualche $x\in\amsbb{R}$; di conseguenza il punto $(h, F(h))$ viene mandato dalla traslazione in $(x,f(x))$. Poiché $h$ scorre su tutto $\amsbb{R}$, così fa $x$, e quindi abbiamo
\[
\{(h, F(h)), \ h\in\amsbb{R}\}\longrightarrow \{(x, f(x)), \ x\in\amsbb{R}\}
\]
Consideriamo invece la retta $\{(h, f'(x_0)h), \ h\in\amsbb{R}\}$: la traslazione rigida per il vettore $(x_0, f(x_0))$ agisce come
\[
(h, f'(x_0)h)\longrightarrow (h+x_0, f'(x_0)h+f(x_0))
\]
e scrivendo $h=x-x_0$ come fatto precedentemente otteniamo
\[
(h, f'(x_0)h)\longrightarrow (x, f'(x_0)(x-x_0) + f(x_0))
\]
Quindi
\[
\{(h, f'(x_0)h), \ h\in\amsbb{R}\} \longrightarrow \underbrace{\{(x, f'(x_0)(x-x_0)+f(x_0)), \ x\in\amsbb{R}\}}_{\text{retta passante per }(x_0, f(x_0)) \ \text{con coeff. angolare} \ f'(x_0)}
\]
ossia la retta che meglio approssima la funzione $F$ in un intorno dell'origine viene trasformata nella retta che meglio approssima il grafico della funzione $f$ in un intorno del punto $(x_0, f(x_0))$. La derivata prima $f'(x_0)$ è il coefficiente angolare di questa retta; ora, dalla definizione \ref{def:6.1} e dall'equazione (\ref{eq:6.1}) sappiamo che
\[
f'(x_0) = \lim_{t\to x_0} \frac{f(t)-f(x_0)}{t-x_0}
\]
ossia per calcolare $f'(x_0)$ stiamo considerando il coefficiente angolare della secante passante per i punti $(x_0, f(x_0))$ e $(t, f(t))$ e ne valutiamo il comportamento nel limite in cui i due punti coincidono; possiamo cioè immaginarci di tendere alla retta tangente al grafico di $f(x)$ nel punto $(x_0, f(x_0))$. Alla luce della precedente discussione, nel caso in cui $f$ sia differenziabile in $x_0$ possiamo quindi definire la retta tangente come 
\[
\{(x, f'(x_0)(x-x_0)+f(x_0)), \ x\in\amsbb{R}\}
\]
\begin{example}
    Consideriamo la funzione $f(x) = \log(\arctan(x^2))$; vogliamo determinarne l'equazione della retta tangente al grafico di $f(x)$ nel punto $\left(1, \log\left(\frac{\pi}{4}\right)\right)$. Per quanto detto precedentemente, sappiamo che questa è definita da
    \[
    \left\{\left(x, f'(1)(x-1)+\log\left(\frac{\pi}{4}\right)\right), \ x\in\amsbb{R}\right\}
    \]
    Dobbiamo quindi calcolare $f'(1)$:
    \[
    \begin{split}
        f'(1) & = \frac{d}{dx}f(x)\bigg|_{x=1} = \frac{d}{dx}\left(\log(\arctan(x^2))\right)\bigg|_{x=1}\overset{(\ref{eq:6.4})}{=}\left(\frac{1}{\arctan(x^2)}\frac{1}{1+x^2}2x\right)\bigg|_{x=1} =\\
        & = \frac{1}{\arctan(1)} = \frac{4}{\pi}
    \end{split}
    \]
    L'equazione della retta tangente al grafico di $f$ in $\left(1, \log\left(\frac{\pi}{4}\right)\right)$ è quindi data da
    \[
    y = \frac{4}{\pi}(x-1)+\log\left(\frac{\pi}{4}\right) = \frac{4}{\pi}x+\log\left(\frac{\pi}{4}\right) - \frac{4}{\pi} 
    \]
    \begin{center}
        \begin{tikzpicture}[scale=1.5, font=\footnotesize]
            \tzaxes(0,-2)(4,2){$x$}[b]
            \tzfn[blue]"curve"{ln(rad(atan(\x*\x))))}[0.4:3.5]{$\log(\arctan(x))$}[b]
            \tzfn[red]"curve"{1.2732*\x -0.2416-1.2732}[0.4:2]{$\frac{4}{\pi}(x-1)+\log\left(\frac{\pi}{4}\right)$}[ar]
            \tzprojx(1, -0.2416){$1$}[a]
            \tzprojy(1, -0.2416){$\log\left(\frac{\pi}{4}\right)$}[l]
        \end{tikzpicture}
    \end{center}
    Se avessimo invece voluto sapere l'equazione della retta tangente al grafico di $f^{-1}(x)$ nel punto $\left(\log\left(\frac{\pi}{4}\right), 1\right)$?\\
    Notiamo innanzitutto che $f\colon\amsbb{R}\setminus\{0\} \to \amsbb{R}$ è pari; pertanto questa non è iniettiva su $\amsbb{R}\setminus\{0\}$. Possiamo però restringerci a $f\colon \amsbb{R}^+\to \amsbb{R}$; in questo caso notiamo che $\log(\cdot)\colon \amsbb{R}^+\to \amsbb{R}$ è monotona strettamente crescente, e lo stesso vale per $\arctan(\cdot)\colon \amsbb{R}^+\to\amsbb{R}$; pertanto
    \[
    f(x)<f(y) \iff x<y \ \text{su} \ \amsbb{R}^+
    \]
    e l'iniettività segue. Per quanto riguarda la suriettività, notiamo che
    \[
    \lim_{x\to +\infty} f(x) = \log\left(\frac{\pi}{2}\right) \qquad \lim_{x\to 0^+} f(x) = -\infty
    \]
    e che $f$ è continua, essendo composizione di funzioni continue; vale pertanto il teorema dei valori intermedi \ref{th:5.3}: di conseguenza $f\colon \amsbb{R}^+ \to \left(-\infty, \log\left(\frac{\pi}{2}\right)\right)$ è iniettiva e suriettiva, e pertanto invertibile. \\
    A questo punto, la retta tangente al grafico di $f^{-1}$ in $\left(\log\left(\frac{\pi}{4}, 1\right)\right)$ sarà data da
    \[
    \left\{\left(x, (f^{-1})'\left(\log\left(\frac{\pi}{4}\right)\right)\left(x-\log\left(\frac{\pi}{4}\right)\right)+1\right), \ x\in\amsbb{R}\right\}
    \]
    Per calcolare $(f^{-1})'\left(\log\left(\frac{\pi}{4}\right)\right)$ abbiamo diverse possibilità:
    \begin{enumerate}[(i)]
        \item Ci ricordiamo che vale la regola della catena (\ref{eq:6.4}), ossia se $f$ è derivabile in $x$ e $f^{-1}$ è derivabile in $f(x)$ vale che
        \[
        \frac{d}{dx}(x) = \frac{d}{dx}(f^{-1}(f(x)) = (f^{-1})'(f(x)) f'(x) 
        \]
        Notiamo che se $f'(x) = 0$ non possiamo usare questo metodo per calcolare $(f^{-1})'(f(x))$; se invece $f'(x)\ne 0$ vale che
        \[
        (f^{-1})'(f(x)) = \frac{1}{f'(x)} \iff (f^{-1})'(y) = \frac{1}{f'(f^{-1}(y))}
        \]
        Nel nostro caso vale che $f'(1) \ne 0$, e quindi
        \[
        (f^{-1})'\left(\log\left(\frac{\pi}{4}\right)\right) = \frac{1}{f'(1)} = \frac{\pi}{4}
        \]
        e quindi l'equazione della retta tangente al grafico di $f^{-1}$ in $\left(\log\left(\frac{\pi}{4}\right), 1\right)$ è data da
        \[
        y=\frac{\pi}{4}x+1 - \frac{\pi}{4}\log\left(\frac{\pi}{4}\right)
        \]
        \item Osserviamo le equazioni delle due rette:
        \[
        y=\frac{4}{\pi}x + \log\left(\frac{\pi}{4}\right)-\frac{4}{\pi} \qquad y=\frac{\pi}{4}x+1 - \frac{\pi}{4}\log\left(\frac{\pi}{4}\right)
        \]
        Notiamo che se nella prima scambiamo $x$ e $y$ ottenendo
        \[
        x=\frac{4}{\pi}y+\log\left(\frac{\pi}{4}\right)-\frac{4}{\pi}
        \]
        ed esplicitiamo la $y$ in funzione di $x$ otteniamo
        \[
        x- \log\left(\frac{\pi}{4}\right)+\frac{4}{\pi} = \frac{4}{\pi}y \iff y = \frac{\pi}{4}x-\frac{\pi}{4}\log\left(\frac{\pi}{4}\right)+1
        \]
        cioè la seconda equazione.\\
        Perché funziona? Supponiamo di avere una funzione $f\colon D_f\to I_f$ invertibile; la sua funzione $f^{-1}\colon D_{f^{-1}}\to I_{f^{-1}}$ avrà grafico
        \[
        \text{Graph}(f^{-1}) = \left\{(x, f^{-1}(x))\in\amsbb{R}^2, \ x\in D_{f^{-1}}\right\}
        \]
        Ricordiamo che nel caso di funzioni invertibili $D_{f^{-1}} = I_f$; quindi per ogni $x\in D_{f^{-1}}$ esiste un unico $t\in D_f$ tale che $x=f(t)$; quindi
        \[
        \text{Graph}(f^{-1}) = \left\{(f(t), f^{-1}(f(t))), \ t\in D_f\right\} = \left\{(f(t), t), \ t\in D_f\right\} 
        \]
        ossia possiamo disegnare il grafico di $f^{-1}$ tracciando il grafico di $f$ invertendo gli assi. Supponiamo ora che la retta
        \[
        r = \{(t, r(t)), \ x\in\amsbb{R}\}
        \]
        sia tangente al grafico di $f$ in $(t_0, f(t_0))$; allora necessariamente la retta
        \[
        r'=\{(r(t), t), \ t\in\amsbb{R} \}
        \]
        sarà tangente a
        \[
        \left\{(f(t), t), \ t\in D_f\right\} = \left\{(x, f^{-1}(x)), \ x\in D_{f^{-1}}\right\} = \text{Graph}(f^{-1})
        \]
        in $(f(t_0), t_0)$.
        \item Talvolta possiamo calcolare direttamente $f^{-1}(x)$; in questo caso ad esempio la funzione inversa $f^{-1}\colon \left(-\infty, \log\left(\frac{\pi}{2}\right)\right)\to\amsbb{R}^+$ è data da
        \[
        f^{-1}(x) = \sqrt{\tan(e^{x})}
        \]
        A questo punto è sufficiente derivare $f^{-1}$ per calcolare la derivata prima in $\log\left(\frac{\pi}{4}\right)$:
        \[
        \frac{d}{dx}f^{-1}(x))\bigg|_{x=\log\left(\frac{\pi}{4}\right)} = \frac{1}{2}\frac{1}{\sqrt{\tan(e^x)}}\frac{1}{\cos^2(e^x)}e^x\bigg|_{x=\log\left(\frac{\pi}{4}\right)} = \frac{1}{2}\frac{4}{2}\frac{\pi}{4} = \frac{\pi}{4}
        \]
        e procedere come prima.
    \end{enumerate}
\end{example}
\subsection{Esercizi: derivabilità e differenziabilità}
\begin{exercise}
    \label{ex:6.1}
    Data la funzione
    \[
    f(x) = \begin{dcases}
        \alpha(\beta x + \alpha)^3\, & x\le 0\\
        2e^{\beta x} - \pi \cos\left(\frac{2\alpha}{\pi} x\right)\, & x>0
    \end{dcases}
    \]
    determinare i valori di $\alpha, \beta \in\amsbb{R}$ tali che $f$ sia differenziabile in 0.
\end{exercise}
\begin{proof}[Soluzione]
    Ricordiamo il seguente risultato
    \begin{tcolorbox}
        \begin{theorem}
            \label{th:6.3}
            Se $f\colon (a,b)\to\amsbb{R}$ è differenziabile in $x\in(a,b)$, allora $f$ è continua in $x$.
        \end{theorem}
    \end{tcolorbox}
    che è equivalente a
    \begin{tcolorbox}
        Se $f\colon (a,b)\to\amsbb{R}$ non è continua in $x\in(a,b)$, allora $f$ non è differenziabile in $x$.
    \end{tcolorbox}
    Cerchiamo allora per quali valori dei parametri la funzione è continua in $x=0$, ricordando che vale il teorema \ref{th:5.2}: dobbiamo quindi verificare che
    \[
    \lim_{x\to 0^+} f(x) = \lim_{x\to 0^-} f(x) = f(0) = \alpha^4
    \]
    Sicuramente per la continuità delle funzioni che contribuiscono alla definizione di $f$ in $(-\infty, 0]$ vale che
    \[
    \lim_{x\to 0^-}f(x) = \alpha^4
    \]
    Consideriamo quindi $\lim_{x\to 0^+} f(x)$: poiché le funzioni che definiscono $f$ in $(0, +\infty)$ sono la restrizione a questo intervallo illimitato di funzioni continue su tutto $\amsbb{R}$, vale che
    \[
    \lim_{x\to 0^+} f(x) = \lim_{x\to 0^+} 2e^{\beta x} - \pi \cos\left(\frac{2\alpha}{\pi} x\right) = 2e^0-\pi \cos(0) = 2-\pi
    \]
    Per avere una funzione continua dobbiamo quindi trovare i valori del parametro $\alpha$ tali che
    \[
    \alpha^4 = 2-\pi
    \]
    Questa equazione però non ha soluzioni in $\amsbb{R}$, in quanto $2-\pi<0$. Pertanto non esiste scelta dei parametri $\alpha,\beta$ tale da rendere $f$ differenziabile in $0$.
\end{proof}
\begin{remark}
    L'unico legame fra continuità e differenziabilità è quello descritto nel teorema \ref{th:6.3}. Il fatto che una funzione sia continua non implica nulla sul fatto che una funzione sia derivabil: esistono infatti esempi di funzioni continue su tutto $\amsbb{R}$ non derivabili in nessun punto (\cite[Teorema 7.18]{rudin1976principles}).
\end{remark}
\begin{exercise}
    \label{ex:6.2}
    Data la funzione
    \[
    f(x) = \begin{dcases}
        \alpha\sin(\pi e^{\beta(x-e)})+\beta\, & x\le e\\
        (x-e)^3-\alpha e^{-\alpha(x-e)}\, & x>e
    \end{dcases}
    \]
    determinare i valori di $\alpha, \beta\in\amsbb{R}$ tali che $f$ sia differenziabile in $e$.
\end{exercise}
\begin{proof}[Soluzione]
    Come prima, alla luce del teorema \ref{th:6.3} verifichiamo per quali valori dei parametri $f$ è continua in $e$, e grazie al teorema \ref{th:5.2} possiamo vedere se
    \[
    \lim_{x\to e^+} f(x) = \lim_{x\to e^-} f(x) = f(e) = \beta
    \]
    Poiché $f\colon (-\infty, e]\to \amsbb{R}$ è data dalla restrizione di $\alpha\in(\pi e^{\beta(x-e)})+\beta$, che è composizione di funzioni continue su tutto $\amsbb{R}$, vale che
    \[
    \lim_{x\to e^-}f(x) = f(e) = \beta
    \]
    Per quanto riguarda $f\colon (e, +\infty)$, anche in questo caso $f$ è data dalla restrizione all'intervallo di funzioni continue su tutto $\amsbb{R}$; quindi
    \[
    \lim_{x\to e^+}f(x) = \lim_{x\to e^+} (x-e)^3-\alpha e^{-\alpha (x-e)} = -\alpha
    \]
    Quindi affinché $f$ sia continua in $e$ deve valere che $\beta = -\alpha$.\\
    Per quanto riguarda invece la differenziabilità, in questo caso possiamo operare in due modi diversi:
    \begin{enumerate}[(i)]
        \item Possiamo usare la definizione \ref{def:6.1}, in particolare l'equazione (\ref{eq:6.1}), e il teorema \ref{th:5.2} per dire che $f$ è differenziabile in $e$ se
        \[
        \underbrace{\lim_{t\to e^+}\frac{f(t)-f(e)}{t-e}}_{\stepcounter{equation}\mbox{(\theequation)}} = \underbrace{\lim_{t\to e^-}\frac{f(t)-f(e)}{t-e}}_{\stepcounter{equation}\mbox{(\theequation)}}
        \]
        \addtocounter{equation}{-2}\refstepcounter{equation}\label{eq:6.5}
        \addtocounter{equation}{0}\refstepcounter{equation}\label{eq:6.6}
        ed entrambi i limiti esistono finiti. Consideriamo il limite (\ref{eq:6.5}): ricordando che necessariamente $\alpha=-\beta$,
        \[
        \begin{split}
            &\lim_{x\to e^+} \frac{f(x)-f(e)}{x-e} = \lim_{x\to e^+} \frac{(x-e)^3 - \alpha e^{-\alpha(x-e)}-\beta}{x-e} \overset{t=x-e}{=} \\
            & = \lim_{t\to 0^+} \frac{t^3-\alpha e^{-\alpha t}-\beta}{t} \overset{(\ref{eq:5.13})}{=} \lim_{t\to 0^+} \frac{t^3-\alpha(1-\alpha t + o(t))+\alpha}{t} = \\
            & = \lim_{t\to 0^+} t^2 +\alpha^2 + \frac{o(t)}{t} = \alpha^2
        \end{split}
        \]
        Allo stesso modo, se consideriamo il limite (\ref{eq:6.6}) abbiamo che
        \[
        \begin{split}
            & \lim_{x\to e^-} \frac{f(x)-f(e)}{x-e} = \lim_{x\to e^-}\frac{\alpha\sin(\pi e^{\beta(x-e)})+\beta-\beta}{x-e} = \lim_{x\to e^-} \frac{\alpha\sin(\pi e^{\beta(x-e)}+\pi -\pi)}{x-e} = \\
            & = \lim_{x\to e^-} \frac{-\alpha\sin(\pi e^{\beta(x-e)}-\pi)}{x-e}= \overset{t=e^{\beta(x-e)}-1}{=} \lim_{t\to 0^-} \frac{-\alpha\sin(\pi t)}{\frac{1}{\beta}\log(1+t)} \overset{(\ref{eq:5.15})+(\ref{eq:5.14})}{=} \\
            & = \lim_{t\to 0^-} -\alpha\frac{-\alpha\pi t +o(t)}{t+o(t)} = -\alpha\lim_{t\to 0^-} \frac{-\alpha \pi t}{t}\frac{1+\frac{o(t)}{t}}{1+\frac{o(t)}{t}} = \pi \alpha^2 \lim_{t\to 0^-} \frac{1+\frac{o(t)}{t}}{1+\frac{o(t)}{t}} = \pi \alpha^2
        \end{split}
        \]
        ove abbiamo usato il teorema \ref{th:4.4} e la definizione \ref{def:5.4}. Quindi $f$ è differenziabile in $e$ se
        \[
        \begin{dcases}
            \alpha^2 = \pi \alpha^2 \\
            \beta = -\alpha 
        \end{dcases} \iff \alpha = \beta = 0
        \]
        \item Notiamo che $f$ è derivabile separatamente nei due tratti $(-\infty, e)$ e $(e, +\infty)$. Possiamo quindi provare a derivare $f$ separatamente nei due tratti, ottenendo una funzione
        \[
        f'(x) = \begin{dcases}
            \alpha \cos(\pi e^{\beta(x-e)})\pi \beta e^{\beta(x-e)}\, & x<e\\
            3(x-e)^2 + \alpha^2 e^{-\alpha(x-e)}\, & x>e
        \end{dcases}
        \]
        Resta da capire cosa succede nel punto $x=e$. Se consideriamo il limite sinistro del rapporto incrementale abbiamo che
        \[
        \lim_{t\to e^-} \frac{f(t)-f(e)}{t-e} = \lim_{t\to e^-}\frac{(\alpha\sin(\pi e^{\beta(t-e)})+\beta)-(\alpha\sin(\pi e^{\beta(e-e)})+\beta)}{t-e}
        \]
        Notiamo che la funzione $\alpha\sin(\pi e^{\beta(x-e)})+\beta$ è derivabile su tutto $\amsbb{R}$, e la sua derivata è continua; pertanto
        \[
        \begin{split}
            &\lim_{t\to e^-}\frac{(\alpha\sin(\pi e^{\beta(t-e)})+\beta)-(\alpha\sin(\pi e^{\beta(e-e)})+\beta)}{t-e} = \frac{d}{dt}\left(\alpha\sin(\pi e^{\beta(t-e)})+\beta\right)\bigg|_{t=e} = \\
            & = \lim_{t\to e^-} \frac{d}{dt}\left(\alpha\sin(\pi e^{\beta(t-e)})+\beta\right)
        \end{split}
        \]
        Quindi vale che
        \[
        \lim_{t\to e^-} \frac{f(t)-f(e)}{t-e} = \lim_{t\to e^-} f'(t)
        \]
        Dato che abbiamo imposto che $f$ sia continua in $e$ con $\beta = -\alpha$, vale che 
        \[
        f(e) = \lim_{x\to e^+} (x-e)^3-\alpha e^{-\alpha(x-e)} = (e-e)^3-\alpha e^{-\alpha(e-e)}
        \]
        Procedendo come prima si può quindi mostrare che
        \[
        \lim_{t\to e^+} \frac{f(t)-f(e)}{t-e} = \lim_{t\to e^+}f'(t)
        \]
        Se calcoliamo i due limiti, per la continuità delle funzioni abbiamo
        \[
        \lim_{t\to e^-}f'(t) = \lim_{t\to e^-} \alpha \cos(\pi e^{\beta(t-e)})\pi \beta e^{\beta(t-e)} = \alpha \cos(\pi) \pi \beta = \pi\alpha^2
        \]
        e 
        \[
        \lim_{t\to e^+} f'(t) = \lim_{t\to e^+} 3(t-e)^2 + \alpha^2 e^{-\alpha(t-e)} = \alpha^2
        \]
        Ci siamo quindi ricondotti allo stesso sistema
        \[
        \begin{dcases}
            \alpha^2 = \pi \alpha^2\\
            \beta = - \alpha
        \end{dcases} \iff \alpha=\beta = 0
        \]
    \end{enumerate}
\end{proof}
\subsection{Ripasso: teorema di de \texorpdfstring{l'H{\^o}pital}{l'Hôpital}}
\begin{theorem}
    \label{th:6.4}
    Siano $f,g\colon(a,b)\to\amsbb{R}$ due funzioni definite sull'intervallo $(a,b)$ con $-\infty \le a < b \le +\infty$ tali che
    \begin{enumerate}[(i)]
        \item $f,g$ siano differenziabili su $(a,b)$;
        \item $g'(x)\ne 0$ per ogni $x\in(a,b)$;
        \item $f, g \to 0$ o $f,g\to \pm \infty$ per $x\to a$ o per $x\to b$. 
    \end{enumerate}
    Allora se
    \[
    \lim_{x\to a} \frac{f'(x)}{g'(x)} = l \ \text{o} \ \lim_{x\to b} \frac{f'(b)}{g'(b)} = l
    \]
    con $l$ che può essere anche $\pm \infty$, allora
    \[
    \lim_{x\to a} \frac{f(x)}{g(x)} = l \ \text{o} \ \lim_{x\to b}\frac{f(x)}{g(x)} = l
    \]
\end{theorem}
\subsection{Esercizi: teorema di de \texorpdfstring{l'H{\^o}pital}{l'Hôpital}}
\begin{exercise}
    \label{ex:6.3}
    Calcolare, se esiste,
    \[
    \lim_{x\to +\infty} x\left(\arctan(\log(x))-\arctan(x)\right)
    \]
\end{exercise}
\begin{proof}[Soluzione]
    Notiamo che per continuità vale che
    \[
    \lim_{x\to+\infty} \arctan(\log(x))-\arctan(x)) = 0
    \]
    e di conseguenza 
    \[
    \lim_{x\to +\infty} x\left(\arctan(\log(x))-\arctan(x)\right) = [0\cdot \infty] \ \text{F.I.}
    \]
    Notiamo che possiamo ricondurci ad una forma indeterminata $[\frac{0}{0}]$ scrivendo
    \[
    x\left(\arctan(\log(x))-\arctan(x)\right)  = \frac{\arctan(\log(x))-\arctan(x)}{\frac{1}{x}}
    \]
    Possiamo quindi provare ad applicare il teorema di de l'H{\^o}pital \ref{th:6.4}, verificando le altre ipotesi:
    \begin{enumerate}[(i)]
        \item sia $f=\arctan(\log(x))-\arctan(x)$, sia $g=\frac{1}{x}$ sono differenziabili su $(0, +\infty)$;
        \item la derivata prima di $g$ è
        \[
        g'(x) = -\frac{1}{x^2}\ne 0 \ \text{per ogni} \ x\in(0,+\infty)
        \]
    \end{enumerate}
    Se calcoliamo 
    \[
    \begin{split}
        \lim_{x\to +\infty} \frac{f'(x)}{g'(x)} & = \lim_{x\to +\infty}\left(\frac{1}{1+\log^2(x)}\frac{1}{x}-\frac{1}{1+x^2}\right)\frac{1}{-\frac{1}{x^2}} = \\
        & = \lim_{x\to+\infty} \left(\frac{x^2}{1+x^2}-\frac{x}{1+\log^2(x)}\right)
    \end{split}
    \]
    Notiamo che
    \[
    \lim_{x\to +\infty} \frac{x}{1+\log^2(x)} = \left[\frac{\infty}{\infty}\right] \ \text{F.I.}
    \]
    e quindi possiamo provare ad applicare nuovamente il teorema di de l'H{\^o}pital: 
    \begin{enumerate}[(i)]
        \item $u=x$ e $v=1+\log^2(x)$ sono differenziabili in $(0,+\infty)$;
        \item $v'(x) = \frac{2\log(x)}{x}= 0$ se $x=1$; possiamo però risolvere il problema restringendoci all'intervallo $(1+\infty)$, in quanto stiamo considerando il limite per $x\to+\infty$. Poiché $u$ e $v$ sono differenziabili in $(0,+\infty)$ lo sono anche in $(1, +\infty)$, e quindi tutti i requisiti del teorema \ref{th:6.4} sono soddisfatti.
    \end{enumerate}
    Calcoliamo quindi
    \[
    \lim_{x\to+\infty} \frac{u'(x)}{v'(x)} = \lim_{x\to +\infty} \frac{x}{2\log(x)}
    \]
    A questo punto possiamo di nuovo applicare il teorema di de l'H{\^o}pital, oppure possiamo appellarci alle gerarchie degli infiniti, per concludere che
    \[
    \lim_{x\to+\infty}\frac{u'(x)}{v'(x)} = +\infty \overset{\text{d.H.}}{\implies} \lim_{x\to+\infty} \frac{u(x)}{v(x)} = \lim_{x\to+\infty} \frac{x}{1+\log^2(x)} = +\infty
    \]
    Di conseguenza, poiché
    \[
    \lim_{x\to+\infty}\frac{x^2}{1+x^2} = 1
    \]
    abbiamo che
    \[
    -\infty = \lim_{x\to +\infty} \left(\frac{x^2}{1+x^2}-\frac{x}{1+\log^2(x)}\right) = \lim_{x\to+\infty} \frac{f'(x)}{g'(x)} \overset{\text{d.H.}}{\implies} \lim_{x\to+\infty} \frac{f(x)}{g(x)} = -\infty
    \]
    Concludiamo quindi che
    \[
    \lim_{x\to +\infty} x\left(\arctan(\log(x))-\arctan(x)\right) = -\infty
    \]
\end{proof}
\begin{remark}
    Mostriamo che date due funzioni $f$, $g$ tali che 
    \[
    \lim_{x\to+\infty} f(x) = l \qquad \lim_{x\to+\infty} g(x) = +\infty
    \]
    allora $f-g\to -\infty$ per $x\to+\infty$.
    \begin{enumerate}[(i)]
        \item Supponiamo che $g(x)\to +\infty$ per $x\to+\infty$; allora per ogni $M\in\amsbb{N}$ esiste $x_M\in\amsbb{R}$ tale che $g(x)>M$ se $x>x_M$. Se moltiplichiamo ambo i membri della disuguaglianza $g(x)>M$ per $-1$ otteniamo che, fissato $M\in\amsbb{N}$, esiste $x_M$ tale che
        \[
        -g(x)<-M \ \text{per ogni} \ x>x_M
        \]
        ossia $-g(x)\to-\infty$ per $x\to+\infty$.
        \item Se $f(x)\to l$ per $x\to+\infty$, vuol dire che per ogni $\varepsilon>0$ esiste $x_1\in\amsbb{R}$ tale che
        \[
        l-\varepsilon < f(x) <l+\varepsilon \ \text{per ogni} \ x> x_\varepsilon
        \]
        Fissiamo ora $M\in\amsbb{N}$, e consideriamo il numero reale $M+l+\varepsilon$; per la proprietà archimedea dei numeri reali esiste $N\in\amsbb{N}$ tale che $N>M+l+\varepsilon$. A questo punto per il punto (i) sappiamo che esiste $x_2\in\amsbb{R}$ tale che
        \[
        -g(x)<-N\ \text{per ogni} \ x>x_2
        \]
        Se consideriamo $x_M = \max\{x_1, x_2\}$ vale che
        \[
        f(x)<l+\varepsilon \ \text{e} \ -g(x)<-N \ \text{per ogni} \ x> x_M
        \]
        quindi
        \[
        f(x)-g(x)<l+\varepsilon -N < -M-l-\varepsilon+l+\varepsilon
        \]
        Ossia fissato $M\in\amsbb{N}$ abbiamo trovato $x_M$ tale che
        \[
        f(x)-g(x)<-M \ \text{per ogni} \ x>x_M
        \]
    \end{enumerate}
\end{remark}
\subsection{Ripasso: sviluppo di Taylor}
\begin{theorem}
    \label{th:6.5}
    Data $f\colon (a,b)\to \amsbb{R}$ una funzione differenziabile $n$ volte in $x_0\in(a,b)$, esiste un polinomio $P(x)$ di grado $n$ tale che
    \[
    f(x) = P(x)+o\left((x-x_0)^n\right)
    \]
    Tale polinomio viene detto \emph{polinomio di Taylor} ed è dato da
    \begin{equation}
        \label{eq:6.7}
        P(x) = \sum_{k=0}^n \frac{1}{k!} f^{(k)}(x_0) (x- x_0)^k \quad f^{(k)}(x_0) = \frac{d^k}{dx^k}f(x)\bigg|_{x=x_0} \quad f^{(0)}(x_0) = f(x_0)
    \end{equation}
\end{theorem}
\subsection{Esercizi: sviluppi di Taylor}
\begin{exercise}
    \label{ex:6.4}
    Calcolare lo sviluppo di Taylor centrato in $0$ di 
    \[
    f(x) = \sinh(x)
    \]
    fino al quarto ordine.
\end{exercise}
\begin{proof}[Soluzione]
    Dall'equazione (\ref{eq:6.7}) sappiamo che nel caso di $\sinh(x)$ il polinomio di Taylor centrato in $0$ è dato da
    \[
    P(x) = \sum_{k=0}^n \frac{1}{k!} \sinh^{(k)}(0)x^k + o(x^k)
    \]
    Dobbiamo quindi calcolare $\sinh^{(k)}(0)$ per $k=0, \dots, 4$. Ricordiamo che 
    \[
    \sinh(x) = \frac{e^x-e^{-x}}{2}
    \]
    e quindi
    \[
    \sinh(0) = 0 \quad \frac{d}{dx}\sinh(x)\bigg|_{x=0} = \frac{d}{dx}\left(\frac{e^x-e^{-x}}{2}\right)\bigg|_{x=0} = \frac{e^x+e^{-x}}{2}\bigg|_{x=0} = \cosh(0) = 1
    \]
    Notiamo che
    \[
    \frac{d}{dx}\sinh(x) = \cosh(x) \qquad \frac{d}{dx}\cosh(x) = \sinh(x)
    \]
    Di conseguenza
    \[
    \sinh(0) = 0 \quad \sinh'(0) = 1 \quad \sinh^{(2)}(0) = 0 \quad \sinh^{(3)}(0) = 1 \quad \sinh^{(4)}(0) = 0
    \]
    e il polinomio di Taylor di $\sinh(x)$ di ordine 4 è dato da
    \[
    P(x) = 0+x+0\frac{1}{2}x^2 + \frac{1}{6}x^3 + 0\frac{1}{24}x^4
    \]
    Concludiamo che
    \[
    \sinh(x) = x+\frac{x^3}{6} + o(x^4)
    \]
\end{proof}
\begin{exercise}
    \label{ex:6.5}
    Calcolare lo sviluppo di Taylor centrato in $0$ di 
    \[
    f(x) = \log(1+\sinh(x))-x
    \]
    fino al quarto ordine.
\end{exercise}
\begin{proof}[Soluzione]
    Notiamo che $\sinh(0) = 0$; di conseguenza, se effettuiamo la sostituzione $\xi = \sinh(x)$, la funzione $f$ sarà data da
    \[
    f(\xi) = \log(1+\xi)-x(\xi)
    \]
    Possiamo quindi considerare lo sviluppo di Taylor di $\log(1+\xi)$ centrato in $0$: questo sarà dato fino al quarto ordine da
    \[
    \begin{split}
        P(\xi) & = \log(1+0) + (\log(1+\xi))'\Big|_{\xi=0} \xi + (\log(1+\xi))^{(2)}\Big|_{\xi=0}\frac{\xi^2}{2} + (\log(1+\xi))^{(3)}\Big|_{\xi=0}\frac{\xi^3}{6} + \\
        & + (\log(1+\xi))^{(4)}\Big|_{\xi=0}\frac{\xi^4}{24} + o(\xi^4) = 0 + \frac{1}{1+\xi}\bigg|_{\xi=0} \xi - \frac{1}{(1+\xi)^2}\bigg|_{\xi=0}\frac{\xi^2}{2} + \\
        & + \frac{2}{(1+\xi)^3}\bigg|_{\xi=0}\frac{\xi^3}{6}-\frac{6}{(1+\xi)^4}\bigg|_{\xi=0}\frac{\xi^4}{24} +o(\xi^4) = \\
        & = \xi -\frac{\xi^2}{2} + \frac{\xi^3}{3}-\frac{\xi^4}{4}+o(\xi^4)
    \end{split}
    \]
    Quindi
    \[
    \log(1+\sinh(x))= \sinh(x)-\frac{\sinh^2(x)}{2}+\frac{\sinh^3(x)}{3}-\frac{\sinh^4(x)}{4}+o(\sinh^4(x))
    \]
    Possiamo ora usare l'esercizio \ref{ex:6.4} per sviluppare $\sinh(x)$ fino al quarto ordine all'interno dell'espressione precedente
    \[
    \begin{split}
        \log(1+\sinh(x)) & = x + \frac{x^3}{6}+o(x^4)-\frac{1}{2}\left(x+\frac{x^3}{6}+o(x^4)\right)^2 + \frac{1}{3}\left(x+\frac{x^3}{6}+o(x^4)\right)^3-\\
        & -\frac{1}{4}\left(x+\frac{x^3}{6}+o(x^4)\right)^4 + o\left(\left(x+\frac{x^3}{6}+o(x^4)\right)^4\right)
    \end{split}
    \]
    Espandendo le potenze di trinomio otteniamo, sfruttando le proprietà dei simboli di Landau,
    \[
    \left(x+\frac{x^3}{6}+o(x^4)\right)^2 = x^2 + \frac{x^6}{36}+o(x^8) + \frac{x^4}{3} + o(x^5) + o(x^7) = x^2 + \frac{x^4}{3} + o(x^4)
    \]
    \[
    \begin{split}
        \left(x+\frac{x^3}{6}+o(x^4)\right)^3 & = (x+o(x^4))^3 + \frac{x^9}{6^3} + \frac{x^6}{12}(x+o(x^4))+\frac{x^3}{2}(x+o(x^4))^2 = \\
        & = x^3 + o(x^{12}) + o(x^9)+o(x^6) + \frac{x^9}{6^3} +o(x^7) +o(x^{10}) + \frac{x^5}{2} + \\
        & + o(x^{11}) + o(x^8) = x^3 + o(x^4)
    \end{split}
    \]
    e infine 
    \[
        \left(x+\frac{x^3}{6}+o(x^4)\right)^4= x^4 +o(x^4)
    \]
    Quindi
    \[
    \begin{split}
        \log(1+\sinh(x))& = x +\frac{x^3}{6}+o(x^4) -\frac{x^2}{2}-\frac{x^4}{6} +o(x^4)+\frac{x^3}{3}+o(x^4)-\frac{x^4}{4}+o(x^4) = \\
        & = x -\frac{x^2}{2} +\frac{x^3}{2}-\frac{5}{12}x^4 +o(x^4)
    \end{split} 
    \]
    Quindi
    \[
    \begin{split}
        f(x) & = \log(1+\sinh(x))-x = x -\frac{x^2}{2} +\frac{x^3}{2}-\frac{5}{12}x^4 +o(x^4) -x = \\
        & = -\frac{x^2}{2} +\frac{x^3}{2}-\frac{5}{12}x^4 +o(x^4)
    \end{split}
    \]
\end{proof}
\begin{exercise}
    \label{ex:6.6}
    Calcolare il limite
    \[
    \lim_{x\to 0}\frac{e^x\arctan(x)-x\sqrt[3]{1+3x}}{x^3}
    \]
\end{exercise}
\begin{proof}[Soluzione]
    In questo caso utilizziamo lo sviluppo di Taylor per calcolare il limite, in quanto ci consentono di stimare il comportamento di una funzione in un intorno di un punto (in questo caso di 0) con polinomi, che sono semplici da maneggiare.\\
    Poiché a denominatore abbiamo il monomio $x^3$, a numeratore andremo a sviluppare con Taylor fino al terzo ordine: infatti, gli ordini superiori a 3 una volta divisi per $x^3$ tenderanno a $0$ per $x\to 0$, e quindi non influiranno sul limite. Ci ricordiamo quindi che 
    \[
    e^x = 1 + x + \frac{x^2}{2}+\frac{x^3}{6}+o(x^3)
    \]
    e calcoliamo gli altri sviluppi:
    \[
    \begin{split}
        \arctan(x) & = \arctan(0)+\frac{1}{1+x^2}\bigg|_{x=0}x -\frac{2x}{(1+x^2)^2}\bigg|_{x=0}\frac{x^2}{2}-\\
        & - \left(\frac{2}{(1+x^2)^2}-2\frac{(2x)^2}{(1+x^2)^3}\right)\bigg|_{x=0}\frac{x^3}{6}+o(x^3) = \\
        & = x-\frac{x^3}{3}+o(x^3)
    \end{split}
    \]
    \[
    \begin{split}
        \sqrt[3]{1+3x} & = \sqrt[3]{1+0}+\frac{1}{3}3(1+3x)^{-\frac{2}{3}}\bigg|_{x=0}x -\frac{2}{3}3(1+3x)^{-\frac{5}{3}}\bigg|_{x=0}\frac{x^2}{2}+\\
        & + 2\frac{5}{3}3(1+3x)^{-\frac{8}{3}}\frac{x^3}{6}+o(x^3) = \\
        & = 1 + x-x^2 + \frac{5}{3}x^3+o(x^3)
    \end{split}
    \]
    A questo punto possiamo calcolare lo sviluppo di Taylor del numeratore semplicemente effettuando i prodotti:
    \[
    \begin{split}
        e^x\arctan(x)-x\sqrt[3]{1+3x} & = \left(1+x+\frac{x^2}{2}+\frac{x^3}{6}+o(x^3)\right)\left(x-\frac{x^3}{3}+o(x^3)\right)-\\
        & - x\left(1+x-x^2+\frac{5}{3}x^3+o(x^3)\right) = \\
        & = x-\frac{x^3}{3}+x^2 + \frac{x^3}{2} + o(x^3)-x-x^2+x^3+o(x^3) =\\
        & = \frac{7}{6}x^3+o(x^3)
    \end{split}
    \]
    Di conseguenza
    \[
    \lim_{x\to 0} \frac{e^x\arctan(x)-x\sqrt[3]{1+3x}}{x^3} = \lim_{x\to 0} \frac{1}{x^3}\left(\frac{7}{6}x^3 +o(x^3)\right) = \frac{7}{6} +\lim_{x\to 0}\frac{o(x^3)}{x^3} = \frac{7}{6}
    \]
\end{proof}
\newpage

\section{Lezione 7}
\subsection{Ripasso: teorema di de \texorpdfstring{l'H{\^o}pital}{l'Hôpital}}
\begin{theorem}
    \label{th:6.4}
    Siano $f,g\colon(a,b)\to\amsbb{R}$ due funzioni definite sull'intervallo $(a,b)$ con $-\infty \le a < b \le +\infty$ tali che
    \begin{enumerate}[(i)]
        \item $f,g$ siano differenziabili su $(a,b)$;
        \item $g'(x)\ne 0$ per ogni $x\in(a,b)$;
        \item $f, g \to 0$ o $f,g\to \pm \infty$ per $x\to a$ o per $x\to b$. 
    \end{enumerate}
    Allora se
    \[
    \lim_{x\to a} \frac{f'(x)}{g'(x)} = l \ \text{o} \ \lim_{x\to b} \frac{f'(b)}{g'(b)} = l
    \]
    con $l$ che può essere anche $\pm \infty$, allora
    \[
    \lim_{x\to a} \frac{f(x)}{g(x)} = l \ \text{o} \ \lim_{x\to b}\frac{f(x)}{g(x)} = l
    \]
\end{theorem}
\subsection{Esercizi: teorema di de \texorpdfstring{l'H{\^o}pital}{l'Hôpital}}
\begin{exercise}
    \label{ex:6.3}
    Calcolare, se esiste,
    \[
    \lim_{x\to +\infty} x\left(\arctan(\log(x))-\arctan(x)\right)
    \]
\end{exercise}
\begin{proof}[Soluzione]
    Notiamo che per continuità vale che
    \[
    \lim_{x\to+\infty} \arctan(\log(x))-\arctan(x)) = 0
    \]
    e di conseguenza 
    \[
    \lim_{x\to +\infty} x\left(\arctan(\log(x))-\arctan(x)\right) = [0\cdot \infty] \ \text{F.I.}
    \]
    Notiamo che possiamo ricondurci ad una forma indeterminata $[\frac{0}{0}]$ scrivendo
    \[
    x\left(\arctan(\log(x))-\arctan(x)\right)  = \frac{\arctan(\log(x))-\arctan(x)}{\frac{1}{x}}
    \]
    Possiamo quindi provare ad applicare il teorema di de l'H{\^o}pital \ref{th:6.4}, verificando le altre ipotesi:
    \begin{enumerate}[(i)]
        \item sia $f=\arctan(\log(x))-\arctan(x)$, sia $g=\frac{1}{x}$ sono differenziabili su $(0, +\infty)$;
        \item la derivata prima di $g$ è
        \[
        g'(x) = -\frac{1}{x^2}\ne 0 \ \text{per ogni} \ x\in(0,+\infty)
        \]
    \end{enumerate}
    Se calcoliamo 
    \[
    \begin{split}
        \lim_{x\to +\infty} \frac{f'(x)}{g'(x)} & = \lim_{x\to +\infty}\left(\frac{1}{1+\log^2(x)}\frac{1}{x}-\frac{1}{1+x^2}\right)\frac{1}{-\frac{1}{x^2}} = \\
        & = \lim_{x\to+\infty} \left(\frac{x^2}{1+x^2}-\frac{x}{1+\log^2(x)}\right)
    \end{split}
    \]
    Notiamo che
    \[
    \lim_{x\to +\infty} \frac{x}{1+\log^2(x)} = \left[\frac{\infty}{\infty}\right] \ \text{F.I.}
    \]
    e quindi possiamo provare ad applicare nuovamente il teorema di de l'H{\^o}pital: 
    \begin{enumerate}[(i)]
        \item $u=x$ e $v=1+\log^2(x)$ sono differenziabili in $(0,+\infty)$;
        \item $v'(x) = \frac{2\log(x)}{x}= 0$ se $x=1$; possiamo però risolvere il problema restringendoci all'intervallo $(1+\infty)$, in quanto stiamo considerando il limite per $x\to+\infty$. Poiché $u$ e $v$ sono differenziabili in $(0,+\infty)$ lo sono anche in $(1, +\infty)$, e quindi tutti i requisiti del teorema \ref{th:6.4} sono soddisfatti.
    \end{enumerate}
    Calcoliamo quindi
    \[
    \lim_{x\to+\infty} \frac{u'(x)}{v'(x)} = \lim_{x\to +\infty} \frac{x}{2\log(x)}
    \]
    A questo punto possiamo di nuovo applicare il teorema di de l'H{\^o}pital, oppure possiamo appellarci alle gerarchie degli infiniti, per concludere che
    \[
    \lim_{x\to+\infty}\frac{u'(x)}{v'(x)} = +\infty \overset{\text{d.H.}}{\implies} \lim_{x\to+\infty} \frac{u(x)}{v(x)} = \lim_{x\to+\infty} \frac{x}{1+\log^2(x)} = +\infty
    \]
    Di conseguenza, poiché
    \[
    \lim_{x\to+\infty}\frac{x^2}{1+x^2} = 1
    \]
    abbiamo che
    \[
    -\infty = \lim_{x\to +\infty} \left(\frac{x^2}{1+x^2}-\frac{x}{1+\log^2(x)}\right) = \lim_{x\to+\infty} \frac{f'(x)}{g'(x)} \overset{\text{d.H.}}{\implies} \lim_{x\to+\infty} \frac{f(x)}{g(x)} = -\infty
    \]
    Concludiamo quindi che
    \[
    \lim_{x\to +\infty} x\left(\arctan(\log(x))-\arctan(x)\right) = -\infty
    \]
\end{proof}
\begin{remark}
    Mostriamo che date due funzioni $f$, $g$ tali che 
    \[
    \lim_{x\to+\infty} f(x) = l \qquad \lim_{x\to+\infty} g(x) = +\infty
    \]
    allora $f-g\to -\infty$ per $x\to+\infty$.
    \begin{enumerate}[(i)]
        \item Supponiamo che $g(x)\to +\infty$ per $x\to+\infty$; allora per ogni $M\in\amsbb{N}$ esiste $x_M\in\amsbb{R}$ tale che $g(x)>M$ se $x>x_M$. Se moltiplichiamo ambo i membri della disuguaglianza $g(x)>M$ per $-1$ otteniamo che, fissato $M\in\amsbb{N}$, esiste $x_M$ tale che
        \[
        -g(x)<-M \ \text{per ogni} \ x>x_M
        \]
        ossia $-g(x)\to-\infty$ per $x\to+\infty$.
        \item Se $f(x)\to l$ per $x\to+\infty$, vuol dire che per ogni $\varepsilon>0$ esiste $x_1\in\amsbb{R}$ tale che
        \[
        l-\varepsilon < f(x) <l+\varepsilon \ \text{per ogni} \ x> x_\varepsilon
        \]
        Fissiamo ora $M\in\amsbb{N}$, e consideriamo il numero reale $M+l+\varepsilon$; per la proprietà archimedea dei numeri reali esiste $N\in\amsbb{N}$ tale che $N>M+l+\varepsilon$. A questo punto per il punto (i) sappiamo che esiste $x_2\in\amsbb{R}$ tale che
        \[
        -g(x)<-N\ \text{per ogni} \ x>x_2
        \]
        Se consideriamo $x_M = \max\{x_1, x_2\}$ vale che
        \[
        f(x)<l+\varepsilon \ \text{e} \ -g(x)<-N \ \text{per ogni} \ x> x_M
        \]
        quindi
        \[
        f(x)-g(x)<l+\varepsilon -N < -M-l-\varepsilon+l+\varepsilon
        \]
        Ossia fissato $M\in\amsbb{N}$ abbiamo trovato $x_M$ tale che
        \[
        f(x)-g(x)<-M \ \text{per ogni} \ x>x_M
        \]
    \end{enumerate}
\end{remark}
\subsection{Ripasso: sviluppo di Taylor}
\begin{theorem}
    \label{th:6.5}
    Data $f\colon (a,b)\to \amsbb{R}$ una funzione differenziabile $n$ volte in $x_0\in(a,b)$, esiste un polinomio $P(x)$ di grado $n$ tale che
    \[
    f(x) = P(x)+o\left((x-x_0)^n\right)
    \]
    Tale polinomio viene detto \emph{polinomio di Taylor} ed è dato da
    \begin{equation}
        \label{eq:6.7}
        P(x) = \sum_{k=0}^n \frac{1}{k!} f^{(k)}(x_0) (x- x_0)^k \quad f^{(k)}(x_0) = \frac{d^k}{dx^k}f(x)\bigg|_{x=x_0} \quad f^{(0)}(x_0) = f(x_0)
    \end{equation}
\end{theorem}
\subsection{Esercizi: sviluppi di Taylor}
\begin{exercise}
    \label{ex:6.4}
    Calcolare lo sviluppo di Taylor centrato in $0$ di 
    \[
    f(x) = \sinh(x)
    \]
    fino al quarto ordine.
\end{exercise}
\begin{proof}[Soluzione]
    Dall'equazione (\ref{eq:6.7}) sappiamo che nel caso di $\sinh(x)$ il polinomio di Taylor centrato in $0$ è dato da
    \[
    P(x) = \sum_{k=0}^n \frac{1}{k!} \sinh^{(k)}(0)x^k + o(x^k)
    \]
    Dobbiamo quindi calcolare $\sinh^{(k)}(0)$ per $k=0, \dots, 4$. Ricordiamo che 
    \[
    \sinh(x) = \frac{e^x-e^{-x}}{2}
    \]
    e quindi
    \[
    \sinh(0) = 0 \quad \frac{d}{dx}\sinh(x)\bigg|_{x=0} = \frac{d}{dx}\left(\frac{e^x-e^{-x}}{2}\right)\bigg|_{x=0} = \frac{e^x+e^{-x}}{2}\bigg|_{x=0} = \cosh(0) = 1
    \]
    Notiamo che
    \[
    \frac{d}{dx}\sinh(x) = \cosh(x) \qquad \frac{d}{dx}\cosh(x) = \sinh(x)
    \]
    Di conseguenza
    \[
    \sinh(0) = 0 \quad \sinh'(0) = 1 \quad \sinh^{(2)}(0) = 0 \quad \sinh^{(3)}(0) = 1 \quad \sinh^{(4)}(0) = 0
    \]
    e il polinomio di Taylor di $\sinh(x)$ di ordine 4 è dato da
    \[
    P(x) = 0+x+0\frac{1}{2}x^2 + \frac{1}{6}x^3 + 0\frac{1}{24}x^4
    \]
    Concludiamo che
    \[
    \sinh(x) = x+\frac{x^3}{6} + o(x^4)
    \]
\end{proof}
\begin{exercise}
    \label{ex:6.5}
    Calcolare lo sviluppo di Taylor centrato in $0$ di 
    \[
    f(x) = \log(1+\sinh(x))-x
    \]
    fino al quarto ordine.
\end{exercise}
\begin{proof}[Soluzione]
    Notiamo che $\sinh(0) = 0$; di conseguenza, se effettuiamo la sostituzione $\xi = \sinh(x)$, la funzione $f$ sarà data da
    \[
    f(\xi) = \log(1+\xi)-x(\xi)
    \]
    Possiamo quindi considerare lo sviluppo di Taylor di $\log(1+\xi)$ centrato in $0$: questo sarà dato fino al quarto ordine da
    \[
    \begin{split}
        P(\xi) & = \log(1+0) + (\log(1+\xi))'\Big|_{\xi=0} \xi + (\log(1+\xi))^{(2)}\Big|_{\xi=0}\frac{\xi^2}{2} + (\log(1+\xi))^{(3)}\Big|_{\xi=0}\frac{\xi^3}{6} + \\
        & + (\log(1+\xi))^{(4)}\Big|_{\xi=0}\frac{\xi^4}{24} + o(\xi^4) = 0 + \frac{1}{1+\xi}\bigg|_{\xi=0} \xi - \frac{1}{(1+\xi)^2}\bigg|_{\xi=0}\frac{\xi^2}{2} + \\
        & + \frac{2}{(1+\xi)^3}\bigg|_{\xi=0}\frac{\xi^3}{6}-\frac{6}{(1+\xi)^4}\bigg|_{\xi=0}\frac{\xi^4}{24} +o(\xi^4) = \\
        & = \xi -\frac{\xi^2}{2} + \frac{\xi^3}{3}-\frac{\xi^4}{4}+o(\xi^4)
    \end{split}
    \]
    Quindi
    \[
    \log(1+\sinh(x))= \sinh(x)-\frac{\sinh^2(x)}{2}+\frac{\sinh^3(x)}{3}-\frac{\sinh^4(x)}{4}+o(\sinh^4(x))
    \]
    Possiamo ora usare l'esercizio \ref{ex:6.4} per sviluppare $\sinh(x)$ fino al quarto ordine all'interno dell'espressione precedente
    \[
    \begin{split}
        \log(1+\sinh(x)) & = x + \frac{x^3}{6}+o(x^4)-\frac{1}{2}\left(x+\frac{x^3}{6}+o(x^4)\right)^2 + \frac{1}{3}\left(x+\frac{x^3}{6}+o(x^4)\right)^3-\\
        & -\frac{1}{4}\left(x+\frac{x^3}{6}+o(x^4)\right)^4 + o\left(\left(x+\frac{x^3}{6}+o(x^4)\right)^4\right)
    \end{split}
    \]
    Espandendo le potenze di trinomio otteniamo, sfruttando le proprietà dei simboli di Landau,
    \[
    \left(x+\frac{x^3}{6}+o(x^4)\right)^2 = x^2 + \frac{x^6}{36}+o(x^8) + \frac{x^4}{3} + o(x^5) + o(x^7) = x^2 + \frac{x^4}{3} + o(x^4)
    \]
    \[
    \begin{split}
        \left(x+\frac{x^3}{6}+o(x^4)\right)^3 & = (x+o(x^4))^3 + \frac{x^9}{6^3} + \frac{x^6}{12}(x+o(x^4))+\frac{x^3}{2}(x+o(x^4))^2 = \\
        & = x^3 + o(x^{12}) + o(x^9)+o(x^6) + \frac{x^9}{6^3} +o(x^7) +o(x^{10}) + \frac{x^5}{2} + \\
        & + o(x^{11}) + o(x^8) = x^3 + o(x^4)
    \end{split}
    \]
    e infine 
    \[
        \left(x+\frac{x^3}{6}+o(x^4)\right)^4= x^4 +o(x^4)
    \]
    Quindi
    \[
    \begin{split}
        \log(1+\sinh(x))& = x +\frac{x^3}{6}+o(x^4) -\frac{x^2}{2}-\frac{x^4}{6} +o(x^4)+\frac{x^3}{3}+o(x^4)-\frac{x^4}{4}+o(x^4) = \\
        & = x -\frac{x^2}{2} +\frac{x^3}{2}-\frac{5}{12}x^4 +o(x^4)
    \end{split} 
    \]
    Quindi
    \[
    \begin{split}
        f(x) & = \log(1+\sinh(x))-x = x -\frac{x^2}{2} +\frac{x^3}{2}-\frac{5}{12}x^4 +o(x^4) -x = \\
        & = -\frac{x^2}{2} +\frac{x^3}{2}-\frac{5}{12}x^4 +o(x^4)
    \end{split}
    \]
\end{proof}
\begin{exercise}
    \label{ex:6.6}
    Calcolare il limite
    \[
    \lim_{x\to 0}\frac{e^x\arctan(x)-x\sqrt[3]{1+3x}}{x^3}
    \]
\end{exercise}
\begin{proof}[Soluzione]
    In questo caso utilizziamo lo sviluppo di Taylor per calcolare il limite, in quanto ci consentono di stimare il comportamento di una funzione in un intorno di un punto (in questo caso di 0) con polinomi, che sono semplici da maneggiare.\\
    Poiché a denominatore abbiamo il monomio $x^3$, a numeratore andremo a sviluppare con Taylor fino al terzo ordine: infatti, gli ordini superiori a 3 una volta divisi per $x^3$ tenderanno a $0$ per $x\to 0$, e quindi non influiranno sul limite. Ci ricordiamo quindi che 
    \[
    e^x = 1 + x + \frac{x^2}{2}+\frac{x^3}{6}+o(x^3)
    \]
    e calcoliamo gli altri sviluppi:
    \[
    \begin{split}
        \arctan(x) & = \arctan(0)+\frac{1}{1+x^2}\bigg|_{x=0}x -\frac{2x}{(1+x^2)^2}\bigg|_{x=0}\frac{x^2}{2}-\\
        & - \left(\frac{2}{(1+x^2)^2}-2\frac{(2x)^2}{(1+x^2)^3}\right)\bigg|_{x=0}\frac{x^3}{6}+o(x^3) = \\
        & = x-\frac{x^3}{3}+o(x^3)
    \end{split}
    \]
    \[
    \begin{split}
        \sqrt[3]{1+3x} & = \sqrt[3]{1+0}+\frac{1}{3}3(1+3x)^{-\frac{2}{3}}\bigg|_{x=0}x -\frac{2}{3}3(1+3x)^{-\frac{5}{3}}\bigg|_{x=0}\frac{x^2}{2}+\\
        & + 2\frac{5}{3}3(1+3x)^{-\frac{8}{3}}\frac{x^3}{6}+o(x^3) = \\
        & = 1 + x-x^2 + \frac{5}{3}x^3+o(x^3)
    \end{split}
    \]
    A questo punto possiamo calcolare lo sviluppo di Taylor del numeratore semplicemente effettuando i prodotti:
    \[
    \begin{split}
        e^x\arctan(x)-x\sqrt[3]{1+3x} & = \left(1+x+\frac{x^2}{2}+\frac{x^3}{6}+o(x^3)\right)\left(x-\frac{x^3}{3}+o(x^3)\right)-\\
        & - x\left(1+x-x^2+\frac{5}{3}x^3+o(x^3)\right) = \\
        & = x-\frac{x^3}{3}+x^2 + \frac{x^3}{2} + o(x^3)-x-x^2+x^3+o(x^3) =\\
        & = \frac{7}{6}x^3+o(x^3)
    \end{split}
    \]
    Di conseguenza
    \[
    \lim_{x\to 0} \frac{e^x\arctan(x)-x\sqrt[3]{1+3x}}{x^3} = \lim_{x\to 0} \frac{1}{x^3}\left(\frac{7}{6}x^3 +o(x^3)\right) = \frac{7}{6} +\lim_{x\to 0}\frac{o(x^3)}{x^3} = \frac{7}{6}
    \]
\end{proof}
\begin{exercise}
    \label{ex:6.7}
    Determinare il valore dei parametri $\alpha, \beta\in\mathbb{R}$ tali che
    \[
    \lim_{x\to +\infty} \left(\sqrt[x]{x} + \frac{\sqrt[x]{x}}{x}\right)^x+\alpha x + \beta = 0
    \]
\end{exercise}
\begin{proof}[Soluzione]
    Innanzitutto, notiamo che la funzione di cui stiamo cercando di calcolare il limite è ben definita se $x\in [0, +\infty)$; inoltre, la possiamo riscrivere come
    \[
    \left(\sqrt[x]{x} + \frac{\sqrt[x]{x}}{x}\right)^x + \alpha x + \beta = x \left(1+\frac{1}{x}\right)^x+\alpha x + \beta 
    \]
    Sappiamo che (cfr. Definizione \ref{def:4.3})
    \[
    \lim_{x\to +\infty} \left(1+\frac{1}{x}\right)^ x = e
    \]
    e di conseguenza ci si potrebbe aspettare che per $\alpha=-e$, $\beta = 0$ si ottenga il risultato desiderato. \emph{Questo è tuttavia falso}: infatti per $\alpha=-e$ otteniamo una forma indeterminata $\infty - \infty$, e bisogna quindi agire diversamente.\\
    Possiamo effettuare i seguenti passaggi:
    \[
    x \left(1+\frac{1}{x}\right)^x+\alpha x + \beta = x e^{x\log\left(1+\frac{1}{x}\right)}+\alpha x + \beta 
    \]
    Notiamo che effettuando la sostituzione $t=\frac{1}{x}$ abbiamo che $t\to 0^+$ per $x\to+\infty$; possiamo quindi considerare lo sviluppo di Taylor di $\log(1+t)$ in un intorno di $t=0$:
    \[
    \log(1+t) = t-\frac{t^2}{2}+\frac{t^3}{3} + o(t^3) = \frac{1}{x}-\frac{1}{2x^2}+\frac{1}{3x^3} + o\left(\frac{1}{x^3}\right)
    \]
    e di conseguenza abbiamo che
    \[
    \begin{split}
        & x e^{x\log\left(1+\frac{1}{x}\right)}+\alpha x + \beta  = x e^{x\left(\frac{1}{x}-\frac{1}{2x^2}+\frac{1}{3x^3}+o\left(\frac{1}{x^3}\right)\right)}+\alpha x + \beta =  \\
        & = xe^{\left(1-\frac{1}{2x}+\frac{1}{3x^2}+o\left(\frac{1}{x^2}\right)\right)}+\alpha x + \beta = ex e^{\left(-\frac{1}{2x}+\frac{1}{3x^2} + o\left(\frac{1}{x^2}\right)\right)} + \alpha x+\beta
    \end{split}
    \]
    per $x\to +\infty$. Notiamo che $\left(-\frac{1}{2x}+\frac{1}{3x^2} + o\left(\frac{1}{x^2}\right)\right)\to 0$ per $x\to+\infty$, e quindi effettuando la sostituzione $\xi = \left(-\frac{1}{2x}+\frac{1}{3x^2} + o\left(\frac{1}{x^2}\right)\right)$ possiamo usare lo sviluppo di Taylor dell'esponenziale per scrivere
    \[
    \begin{split}
        & ex e^{\xi} + \alpha x+\beta = ex\left(1+\xi + \frac{\xi^2}{2}+o(\xi^2)\right) +\alpha x + \beta = \\
        & = (e+\alpha)x + \beta +ex\left(-\frac{1}{2x}+\frac{1}{3x^2}+o\left(\frac{1}{x^2}\right) + \frac{1}{2}\left(\frac{1}{4x^2} +o\left(\frac{1}{x^2}\right)\right)+o\left(\frac{1}{x^2}\right)\right) = \\
        & = (e+\alpha)x + \beta  -\frac{e}{2}+ \frac{11e}{24x}+ex o\left(\frac{1}{x^2}\right)
    \end{split}
    \]
    Quindi per $x\to +\infty$ abbiamo che
    \[
    \lim_{x\to +\infty} \left(\sqrt[x]{x} + \frac{\sqrt[x]{x}}{x}\right)^x+\alpha x + \beta = \lim_{x\to+\infty}  (e+\alpha)x + \beta  -\frac{e}{2}+ \frac{11e}{24x}+ex o\left(\frac{1}{x^2}\right) 
    \]
    Notiamo che, come abbiamo ipotizzato inizialmente, se $\alpha = -e$ eliminiamo la divergenza, e di conseguenza possiamo sperare che il limite sia finito; in questo caso avremo
    \[
    \lim_{x\to +\infty} \left(\sqrt[x]{x} + \frac{\sqrt[x]{x}}{x}\right)^x+\alpha x + \beta = \lim_{x\to+\infty} = \lim_{x\to+\infty}  \beta  -\frac{e}{2}+ \frac{11e}{24x}+ex o\left(\frac{1}{x^2}\right) = \beta -\frac{e}{2}
    \]
    Quindi affinché il limite sia $0$ è necessario che $\beta = \frac{e}{2}$.
\end{proof}
\begin{exercise}
    \label{ex:6.8}
    Calcolare, data la funzione $f\colon \amsbb{R}\to \amsbb{R}$,
    \[
    f(x) = 2x-x^3\cos(2x)+\abs{x}x^5 e^{-x} \cos(x)
    \]
    il valore di $f^{(5)}(0)$.
\end{exercise}
\begin{proof}[Soluzione]
    Dal teorema \ref{th:6.5} sappiamo che il polinomio di Taylor di $f$ di centro $x_0=0$ e di ordine $n$ è dato da 
    \[
    P(x) = \sum_{k=0}^n \frac{1}{k!}f^{(k)}(0) x^k +o(x^n)
    \]
    Pertanto se riusciamo a determinare il coefficiente $c_5$ del monomio $x^5$ del polinomio di Taylor di $f$, abbiamo che
    \[
    f^{(5)}(0) = 5!c_5
    \]
    Possiamo determinare agilmente il polinomio di Taylor di $f$ usando gli sviluppi di Taylor delle funzioni elementari:
    \begin{tcolorbox}
    \[
    \underbrace{e^x = \sum_{k=0}^n \frac{x^k}{k!}+o(x^n)}_{\stepcounter{equation}\mbox{(\theequation)}} \qquad \underbrace{\cos(x)=\sum_{k=0}^{\left\lfloor \frac{n}{2}\right\rfloor} (-1)^{k}\frac{x^{2k}}{(2k)!}+o(x^{2\left\lfloor \frac{n}{2}\right\rfloor+1})}_{\stepcounter{equation}\mbox{(\theequation)}}
    \]
    \addtocounter{equation}{-2}\refstepcounter{equation}\label{eq:6.8}
    \addtocounter{equation}{0}\refstepcounter{equation}\label{eq:6.9}
    \end{tcolorbox}
    Sostituendo $(x)t=2x$, $u(x)=-x$ abbiamo che $t, u \to 0$ per $x\to 0$; quindi usando gli sviluppi \eqref{eq:6.8} e \eqref{eq:6.9} vale che, in un intorno di $0$,
    \[
    \begin{split}
        f(x) & = 2x - x^3\left(1-\frac{t^2(x)}{2}+o(t^2(x))\right)+\abs{x}x^5\left(1+u(x) + o(u(x)\right)\left(1-\frac{x^2}{2}+o(x^2)\right) = \\
        & = 2x - x^3\left(1-\frac{(2x)^2}{2}+o(x^2)\right)+\underbrace{\abs{x}x^5\left(1-x + o(x)\right)\left(1-\frac{x^2}{2}+o(x^2)\right)}_{\abs{x}x^5 + o(x^5)} = \\
        & = 2x-x^3 + 2x^5+o(x^5) +\abs{x}x^5+o(x^5)
    \end{split}
    \]
    Notiamo che $\abs{x}x^5$ è un o-piccolo di $x^5$ per $x\to 0$: infatti,
    \[
    \lim_{x\to 0} \frac{\abs{x}x^5}{x^5} = \lim_{x\to 0}\abs{x} = 0
    \]
    e quindi
    \[
    f(x)  = 2x-x^3 + 2x^5+o(x^5)
    \]
    Pertanto sappiamo che $c_5 = 2$, e di conseguenza
    \[
    f^{(5)}(0)= 5! c_5 = 240
    \]
\end{proof}
\newpage
\section{Lezione 8}
\section{Lezione 10}
\subsection{Ripasso e approfondimento: teoria dell'integrazione secondo Riemann}
\begin{definition}
    \label{def:8.1}
    Sia $f\colon[a,b]\to\amsbb{R}$ una funzione \textbf{limitata definita su} $\mathbf{[a,b]}$. Una \emph{partizione} $\mathscr{P}$ di $[a,b]$ è una collezione di punti $\{x_0, \dots, x_n\}\subset [a,b]$ tale che
    \[
    a=x_0 < x_1 < \dots < x_{n-1}<x_n= b
    \]
    Data una partizione $\mathscr{P}$ di $[a,b]$ e la funzione $f$, definiamo le quantità
    \begin{equation}
        \label{eq:8.1}
        \Delta_i = x_i - x_{i-1} \qquad M_i = \sup_{x\in[x_i, x_{i-1}]} f(x) \qquad m_i = \inf_{x\in[x_i, x_{i-1}]} f(x)
    \end{equation}
    con cui definiamo la somma di Darboux superiore $U(\mathscr{P}, f)$ e inferiore $L(\mathscr{P},f)$
    \[
    \underbrace{U(\mathscr{P},f) = \sum_{i=1}^n M_i \Delta_i}_{\stepcounter{equation}\mbox{(\theequation)}} \qquad \underbrace{L(\mathscr{P},f) = \sum_{i=1}^n m_i \Delta_i}_{\stepcounter{equation}\mbox{(\theequation)}}
    \]
    \addtocounter{equation}{-2}\refstepcounter{equation}\label{eq:8.2}
    \addtocounter{equation}{0}\refstepcounter{equation}\label{eq:8.3}
\end{definition}
\begin{remark}
    Notiamo che somma superiore ed inferiore ammettono un'interessante interpretazione grafica. Nei grafici successivi riportiamo, data una generica funzione $f$ e una partizione $\mathscr{P}= \{x_0, \dots, x_6\}$ di $[a,b]$, la somma superiore $U(\mathscr{P},f)$ e inferiore $L(\mathscr{P},f)$:  
    \begin{center}
    \begin{minipage}{.45\linewidth}
        \centering
        \begin{tikzpicture}[scale=1.4, font=\footnotesize]
            \tzaxes(-1.2,-0.5)(3.3,2.7){$x$}[b]
            \tzticks{-.5/$a$, 0.6/$x_2$, 1.4/$x_4$, 2.6/$b$}[b]
            \tzticks*{-.5/$a$, 0.2/$x_1$, 0.6/$x_2$, 1.2/$x_3$, 1.4/$x_4$, 2.1/$x_5$, 2.6/$b$}
            \tzticks{0.2/$x_1$, 1.2/$x_3$, 2.1/$x_5$}[a]
            \tzfn[black]"curve"{cos(3*deg(\x))+sin(.9*deg(\x))+0.5}[-.5:2.6]{$f(x)$}[br]
            \tzpolygon*[blue,auto,dashed,text=black]
             (-0.5,0)(0.2,0)(0.2,1.5452)(-0.5,1.5452);
            \tzpolygon*[blue,auto,dashed,text=black]
             (0.2,0)(0.6,0)(0.6,1.5043)(0.2,1.5043);
            \tzpolygon*[blue,auto,dashed,text=black]
             (0.6,0)(1.2,0)(1.2,0.7869)(0.6,0.7869);
            \tzpolygon*[blue,auto,dashed,text=black]
             (1.2,0)(1.4,0)(1.4,0.9618)(1.2,0.9618);
            \tzpolygon*[blue,auto,dashed,text=black]
             (1.4,0)(2.1,0)(2.1,2.455)(1.4,2.455);
            \tzpolygon*[blue,auto,dashed,text=black]
             (2.1,0)(2.6,0)(2.6,2.4493)(2.1,2.4493);
        \end{tikzpicture}
    \end{minipage}
    \hspace{1pt}
    \begin{minipage}{.45\linewidth}
        \centering
        \begin{tikzpicture}[scale=1.4, font=\footnotesize]
            \tzaxes(-1.2,-0.5)(3.3,2.7){$x$}[b]
            \tzticks{-.5/$a$, 0.6/$x_2$, 1.4/$x_4$, 2.6/$b$}[b]
            \tzticks*{-.5/$a$, 0.2/$x_1$, 0.6/$x_2$, 1.2/$x_3$, 1.4/$x_4$, 2.1/$x_5$, 2.6/$b$}
            \tzticks{0.2/$x_1$, 1.2/$x_3$, 2.1/$x_5$}[a]
            \tzfn[black]"curve"{cos(3*deg(\x))+sin(.9*deg(\x))+0.5}[-.5:2.6]{$f(x)$}[br]
            \tzpolygon*[red,auto,dashed,text=black]
             (-0.5,0)(0.2,0)(0.2,0.1358)(-0.5,0.1358);
            \tzpolygon*[red,auto,dashed,text=black]
             (0.2,0)(0.6,0)(0.6,0.7869)(0.2,0.7869);
            \tzpolygon*[red,auto,dashed,text=black]
             (0.6,0)(1.2,0)(1.2,0.2922)(0.6,0.2922);
            \tzpolygon*[red,auto,dashed,text=black]
             (1.2,0)(1.4,0)(1.4,0.4852)(1.2,0.4852);
            \tzpolygon*[red,auto,dashed,text=black]
             (1.4,0)(2.1,0)(2.1,0.9618)(1.4,0.9618);
            \tzpolygon*[red,auto,dashed,text=black]
             (2.1,0)(2.6,0)(2.6,1.2724)(2.1,1.2724);
        \end{tikzpicture}
    \end{minipage}
\end{center}
Come si può osservare, $U(\mathscr{P}, f)$ approssima l'area sottesa dal grafico di $f$ dall'alto, mentre $L(\mathscr{P}, f)$ la approssima dal basso. Notiamo inoltre come queste due quantità dipendano sia dalla funzione $f$ considerata, sia dalla scelta della partizione $\mathscr{P}$, e che
\begin{equation}
    \label{eq:8.4}
    \left(\sup_{x\in[a,b]}f(x)\right)(b-a)\ge U(\mathscr{P}, f) \ge L(\mathscr{P},f) \ge \left(\inf_{x\in[a,b]}f(x)\right)(b-a) \ \text{per ogni} \ \mathscr{P}
\end{equation}
\end{remark}
\begin{definition}
    \label{def:8.2}
    Chiamiamo \emph{integrale inferiore} di $f$ su $[a,b]$ la quantità
    \[
    \bint_a^b f = \sup_{\mathscr{P}} L(\mathscr{P},f)
    \]
    ossia l'estremo superiore, preso su tutte le possibili partizioni $\mathscr{P}$ di $[a,b]$, dell'insieme delle somme inferiori. Analogamente, chiamiamo \emph{integrale superiore} di $f$ su $[a,b]$ la quantità
    \[
    \aint_a^b f = \inf_{\mathscr{P}} U(\mathscr{P},f)
    \]
    ossia l'estremo inferiore, preso su tutte le possibili partizioni $\mathscr{P}$ di $[a,b]$, dell'insieme delle somme superiori. Notiamo che queste due quantità sono sempre ben definite per una funzione limitata, in quanto vale la (\ref{eq:8.4}) (e ci ricordiamo il teorema \ref{th:2.1} e il corollario \ref{cor:2.1}). Diremo che $f$ è \emph{Riemann-integrabile} su $[a,b]$ ($f\in\mathscr{R}([a,b])$) se
    \[
    \bint_a^b f = \aint_a^b f
    \]
    è in tal caso il valore comune verrà denotato con
    \[
    \int_a^b f = \int_a^b f(x)\, dx 
    \]
    e verrà detto \emph{integrale di $f$ su $[a,b]$}.
\end{definition}
\begin{remark}
    Notiamo che in generale vale che
    \[
    \bint_a^b f \le \aint_a^b f
    \]
    Per mostrare la validità della disuguaglianza, procediamo come di seguito:
    \begin{enumerate}[(i)]
        \item Data una partizione $\mathscr{P}$ di $[a,b]$, un suo \emph{raffinamento} $\mathscr{Q}$ è una partizione tale che $\mathscr{P}\subseteq \mathscr{Q}$. Date due partizioni $\mathscr{P}_1$ e $\mathscr{P}_2$, il loro raffinamento comune sarà dato da $\mathscr{P}_1 \cup \mathscr{P}_2$.
        \item Se $\mathscr{Q}$ è un raffinamento di $\mathscr{P}$, allora vale che
        \[
        L(\mathscr{P}, f) \le L(\mathscr{Q},f) \qquad U(\mathscr{Q},f)\le U(\mathscr{P},f)
        \]
        Per mostrare che valgono queste due disuguaglianze, consideriamo il caso base in cui $\mathscr{Q} = \mathscr{P} \cup \{\overline{x}\}$, $x_{i-1} < \overline{x}<x_i$ per qualche $i$. Definiamo
        \[
        \overline{m}_1 = \inf_{x\in[x_{i-1}, \overline{x}]}f(x) \qquad \overline{m}_2 = \inf_{x\in[\overline{x}, x_i]}f(x)
        \]
        e notiamo che
        \[
        \overline{m}_1 \ge m_i \qquad \overline{m}_2 \ge m_i
        \]
        A questo punto vale che
        \[
        \begin{split}
            L(\mathscr{Q}, f)-L(\mathscr{P},f) &= \sum_{k=1}^{i-1}m_k \Delta_k + \overline{m}_1 (\overline{x}-x_{i-1}) + \overline{m}_2 (x_i - \overline{x}) + \sum_{k=i+1}^n m_k \Delta_k - \\
            & -\sum_{k=1}^n m_k \Delta_k = \overline{m}_1 (\overline{x}-x_{i-1}) + \overline{m}_2 (x_i - \overline{x})-m_i(x_{i}-x_{i-1})\ge \\
            & \ge m_i(\overline{x}-x_{i-1}) + m_i (x_i -\overline{x})-m_i(x_i-x_{i-1}) = 0
        \end{split}
        \]
        Il risultato si dimostra analogamente per $U(\mathscr{Q}, f) \le U(\mathscr{P},f)$.
        \item Per il punto precedente vale che, date due generiche partizioni $\mathscr{P}_1$ e $\mathscr{P}_2$ e indicando con $\mathscr{Q}$ il loro raffinamento comune,
        \[
        L(\mathscr{P}_1, f)\le L(\mathscr{Q}, f) \le U(\mathscr{Q}, f)\le U(\mathscr{P}_2, f)
        \]
        ossia
       \begin{equation}
           \label{eq:8.5}
           L(\mathscr{P}_1, f)\le U(\mathscr{P}_2, f) \ \text{per ogni} \ \mathscr{P}_1, \mathscr{P}_2
       \end{equation}
       Fissiamo a questo punto $\mathscr{P}_1$; vale che
       \[
       L(\mathscr{P}_1, f)\le U(\mathscr{P}_2, f) \ \forall \mathscr{P}_2 \implies L(\mathscr{P}_1, f)\le \aint_a^b f
       \]
       Poiché la partizione $\mathscr{P}_1$ che avevamo fissato era generica, allo stesso modo vale che
       \[
       L(\mathscr{P}_1, f)\le \aint_a^b f \ \forall \mathscr{P}_1 \implies \bint_a^b f \le \aint_a^b f
       \]
    \end{enumerate}
\end{remark}
\begin{theorem}[Criterio di Riemann]
    \label{th:8.1}
    Una funzione $f\colon [a,b]\to \amsbb{R}$ limitata su $[a,b]$ è Riemann-integrabile se e solo se per ogni $\varepsilon>0$ esiste una partizione $\mathscr{P}_\varepsilon$ di $[a,b]$ tale che
    \[
    U(\mathscr{P}_\varepsilon, f) - L(\mathscr{P}_\varepsilon, f)< \varepsilon
    \]
\end{theorem}
\begin{proof}
    Mostriamo le due implicazioni.
    \begin{enumerate}[(i)]
        \item Supponiamo che $f\in\mathscr{R}([a,b])$. Notiamo che per ogni partizione $\mathscr{P}$ di $[a,b]$ vale che
        \[
        L(\mathscr{P},f) \le \bint_a^b f \le \aint_a^b f \le U(\mathscr{P}, f)
        \]
        Per definizione (cfr. definizione \ref{def:2.2}) di estremo superiore, per ogni $\varepsilon>0$ esiste una partizione $\mathscr{P}_1$ tale che
        \[
        L(\mathscr{P}_1, f) > \bint_a^b f -\frac{\varepsilon}{2} \iff L(\mathscr{P}_1, f) + \frac{\varepsilon}{2}> \bint_a^b f
        \]
        e allo stesso modo per definizione di estremo inferiore per ogni $\varepsilon>0$ esiste una partizione $\mathscr{P}_2$ tale che
        \[
        U(\mathscr{P}_2, f) < \aint_a^b f +\frac{\varepsilon}{2} \iff U(\mathscr{P}_2, f) - \frac{\varepsilon}{2}<\aint_a^b f
        \]
        Definiamo $\mathscr{P}_\varepsilon = \mathscr{P}_1\cup \mathscr{P}_2$; vale che
        \[
        U(\mathscr{P}_\varepsilon, f) \le U(\mathscr{P}_2, f) < \underbrace{\bint_a^bf +\frac{\varepsilon}{2} = \aint_a^b f -\frac{\varepsilon}{2}+\varepsilon}_{\text{Definizione \ref{def:8.2}}} < L(\mathscr{P}_1, f)+\varepsilon \le L(\mathscr{P}_\varepsilon, f) + \varepsilon
        \]
        ossia abbiamo trovato, per ogni $\varepsilon>0$, una partizione $\mathscr{P}_\varepsilon$ tale che
        \[
        U(\mathscr{P}_\varepsilon, f) < L(\mathscr{P}_\varepsilon, f) + \varepsilon \iff U(\mathscr{P}_\varepsilon, f) - L(\mathscr{P}_\varepsilon, f) <  \varepsilon
        \]
        \item Supponiamo che per ogni $\varepsilon>0$ esista una partizione $\mathscr{P}_\varepsilon$ tale che
        \[
        U(\mathscr{P}_\varepsilon, f) - L(\mathscr{P}_\varepsilon, f)< \varepsilon
        \]
        Sappiamo che $U(\mathscr{P}_\varepsilon, f)\ge \aint_a^b f$ e che $L(\mathscr{P}_\varepsilon, f)\le \bint_a^b f$; quindi
        \[
        \varepsilon>U(\mathscr{P}_\varepsilon, f) - L(\mathscr{P}_\varepsilon, f) \ge \aint_a^b f - L(\mathscr{P}_\varepsilon, f) \ge \underbrace{\aint_a^b f - \bint_a^b f\ge 0}_{\text{Osservazione precedente}}
        \]
        ossia per ogni $\varepsilon>0$,
        \[
        0\le \aint_a^b f - \bint_a^b f <\varepsilon
        \]
        Ma questo significa che
        \[
        0\le \aint_a^b f - \bint_a^b f\le 0 \iff \aint_a^b f = \bint_a^b f
        \]
        e quindi $f$ soddisfa la definizione \ref{def:8.2}.\qedhere
    \end{enumerate}
\end{proof}
\begin{corollary}
    \label{cor:8.1}
    Data una funzione $f\colon[a,b]\to \amsbb{R}$, $f\in\mathscr{R}([a,b])$ se soddisfa una delle seguenti proprietà:
    \begin{enumerate}[(i)]
        \item $f$ è continua su $[a,b]$;
        \item $f$ è monotona su $[a,b]$;
        \item $f$ è limitata su $[a,b]$ ed ha al più un numero finito di discontinuità.
    \end{enumerate}
    Inoltre, se $f,g\in\mathscr{R}([a,b])$ vale che
    \begin{enumerate}[(i)]
        \item $f+g\in\mathscr{R}([a,b])$ e 
        \[
        \int_a^b f+g = \int_a^b f + \int_a^b g
        \]
        \item $\lambda f\in\mathscr{R}([a,b])$ per ogni $\lambda\in\amsbb{R}$, e
        \[
        \int_a^b \lambda f = \lambda \int_a^b f
        \]
        \item $fg\in\mathscr{R}([a,b])$;
        \item $\abs{f}\in\mathscr{R}([a,b])$ e
        \[
        \abs{\int_a^b f} \le \int_a^b \abs{f}
        \]
        \item Se $a<c<b$, allora $f\in\mathscr{R}([a,c])$ e $f\in\mathscr{R}([c,b])$ e
        \[
        \int_a^b f = \int_a^c f + \int_c^b f
        \]
    \end{enumerate}
\end{corollary}
\begin{corollary}
    \label{cor:8.2}
    Sia $f\colon [a,b]\to \amsbb{R}$ una funzione Riemann-integrabile su $[a,b]$, e sia $\widehat{f}\colon[a,b]\to \amsbb{R}$ definita da
    \[
    \widehat{f}(x) = \begin{dcases}
        f(x)\, & x\ne \overline{x}\\
        k\ne f(\overline{x})\, & x=\overline{x}
    \end{dcases}
    \]
    per qualche $\overline{x}\in[a,b]$. Allora $\widehat{f}\in \mathscr{R}([a,b])$ e 
    \[
    \int_a^b f = \int_a^b \widehat{f}
    \]
\end{corollary}
\begin{proof}
    Consideriamo come primo caso $f\equiv 0$ su $[a,b]$, e quindi
    \[
    \widehat{f}(x) = \begin{dcases}
        0\, & x\ne \overline{x}\\
        M\, & x=\overline{x}
    \end{dcases}
    \]
    con $M>0$ per semplicità. Data la forma di $f$, sappiamo che se consideriamo la partizione $\mathscr{P}=\{x_0 = a, x_1 = b\}$ vale che
    \[
    U(\mathscr{P}, f) -L(\mathscr{P}, f) <\varepsilon
    \]
    per ogni possibile scelta di $\varepsilon>0$. \\
    Consideriamo ora un $\varepsilon>0$ fissato; vogliamo utilizzare il criterio di Riemann \ref{th:8.1} per dimostrare che $\widehat{f}\in\mathscr{R}([a,b])$. A questo scopo, definiamo la partizione
    \[
    \mathscr{P}_\varepsilon = \left\{x_0, \overline{x}-\frac{\varepsilon}{4M}, \overline{x}+\frac{\varepsilon}{4M}, x_1\right\}
    \]
    e calcoliamo $U(\mathscr{P}_\varepsilon, \widehat{f})$ e $L(\mathscr{P}_\varepsilon, \widehat{f})$: vale che
    \[
    U(\mathscr{P}_\varepsilon, \widehat{f}) = \sup_{x\in \left[\overline{x}-\frac{\varepsilon}{4M}, \overline{x}+\frac{\varepsilon}{4M}\right]} \widehat{f}(x) \frac{\varepsilon}{2M} = M \frac{\varepsilon}{2M} = \frac{\varepsilon}{2}<\varepsilon
    \]
    e che
    \[
    L(\mathscr{P}_\varepsilon, \widehat{f}) = 0 
    \]
    Di conseguenza 
    \[
    U(\mathscr{P}_\varepsilon, \widehat{f}) - L(\mathscr{P}_\varepsilon, \widehat{f})<\varepsilon 
    \]
    e quindi $\widehat{f}\in\mathscr{R}([a,b])$. Per dimostrare che
    \[
    \int_a^b \widehat{f} = \int_a^b f = 0
    \]
    notiamo che, seguendo il ragionamento visto prima, per ogni $\varepsilon>0$ riusciamo a trovare una partizione $\mathscr{P}_\varepsilon$ tale che 
    \[
    U(\mathscr{P}_\varepsilon, \widehat{f})<\varepsilon
    \]
    Di conseguenza, poiché
    \[
    0= L(\mathscr{P}_\varepsilon, \widehat{f}) \le \bint_a^b \widehat{f} = \aint_a^b \widehat{f} \le U(\mathscr{P}_\varepsilon, \widehat{f})<\varepsilon
    \]
    possiamo dire che
    \[
    0\le \int_a^b \widehat{f} < \varepsilon \ \text{per ogni} \ \varepsilon>0
    \]
    ossia $\int_a^b \widehat{f} = 0$. Analogo risultato vale se $\widehat{f}(\overline{x})<0$.\\
    Per quanto riguarda invece il nostro asserto originario, notiamo che data una generica funzione $f\in\mathscr{R}([a,b])$ possiamo scrivere
    \[
    \widehat{f} = f + \underbrace{\widehat{f}-f}_{g}
    \]
    $g(x)$ è una funzione con le proprietà sopra descritte; pertanto $g\in\mathscr{R}([a,b])$ e $\int_a^b g = 0$. Usando il corollario \ref{cor:8.1} abbiamo quindi che $f+g = \widehat{f}\in\mathscr{R}([a,b])$ e che
    \[
    \int_a^b f = \int_a^b f + \int_a^b g = \int_a^b(f+g) = \int_a^b \widehat{f}
    \]
\end{proof}
\begin{remark}
    Il corollario \ref{cor:8.2} ci dice che l'integrazione secondo Riemann non dipende dal valore di una funzione in un numero $n$ arbitrario (ma finito!) di punti.
\end{remark}
\begin{theorem}
    \label{th:8.2}
    Sia $f\colon\mathscr{R}([a,b])$, e $x\in[a,b]$; se definiamo
    \begin{equation}
        \label{eq:8.6}
        F(x) = \int_a^xf(t)\, dt
    \end{equation}
    $F$ è continua in $[a,b]$; inoltre, se $f$ è continua in $x_0\in[a,b]$ allora $F$ è differenziabile in $x_0$ e $F'(x_0) = f(x_0)$.
\end{theorem}
\begin{proof}
    Notiamo che $F$ è ben definita per il corollario \ref{cor:8.1}, punto (v).\\
    Fissiamo $x\in[a,b]$; vogliamo mostrare che per ogni $\varepsilon>0$ esiste $\delta_\varepsilon>0$ tale che
    \[
    \abs{F(x)-F(y)}<\varepsilon \ \text{se} \ \abs{x-y}<\delta_\varepsilon
    \]
    Possiamo supporre $y<x$; allora per il corollario \ref{cor:8.1} vale che
    \[
    \abs{F(x)-F(y)} = \abs{\int_a^x f - \int_a^y f} \overset{\text{(v)}}{=} \abs{\int_x^y f} \overset{\text{(iv)}}{\le }\int_x^y \abs{f}
    \]
    Ricordiamo che $f$ è una funzione limitata, ossia
    \[
    \abs{f(x)} \le M \ \text{per ogni} \ x\in[a,b]
    \]
    Quindi
    \[
    \abs{F(x)-F(y)} {\le }\int_x^y \abs{f} \le M (x-y)
    \]
    Quindi se prendiamo $\delta_\varepsilon = \frac{\varepsilon}{2M}$ abbiamo che
    \[
    \abs{F(x)-F(y)}<\varepsilon \ \text{se} \ \abs{x-y}<\delta_\varepsilon
    \]
    E di conseguenza $F$ è continua in $x$; poiché $x\in[a,b]$ è generico, $F$ è continua in $[a,b]$.\\
    Supponiamo ora che $f$ sia continua in $x_0\in[a,b]$; allora per ogni $\varepsilon>0$ esiste $\delta_\varepsilon>0$ tale che
    \begin{equation}
        \label{eq:8.7}
        \abs{f(x_0)-f(t)}<\varepsilon \ \text{se} \ \abs{x_0-t}<\delta_\varepsilon
    \end{equation}
    Vogliamo mostrare che $F$ è differenziabile in $x_0$, ossia (cfr. definizione \ref{def:6.1}) che 
    \[
    \lim_{t\to x_0} \frac{F(t)-F(x_0)}{t-x_0} = f(x_0)
    \]
    Fissiamo quindi $\varepsilon>0$, e consideriamo
    \[
    \begin{split}
        & \abs{\frac{F(t)-F(x_0)}{t-x_0}-f(x_0)} = \abs{\frac{1}{t-x_0}\int_{x_0}^tf -\frac{t-x_0}{t-x_0}f(x_0)} = \abs{\frac{1}{t-x_0} \int_{x_0}^t f - \frac{f(x_0)}{t-x_0}\int_{x_0}^t} = \\
        & = \abs{\frac{1}{t-x_0}\int_{x_0}^t (f-f(x_0))}\le \frac{1}{\abs{t-x_0}}\int_{x_0}^t\abs{f(u)-f(x_0)}\, du \overset{(\ref{eq:8.7})}{<}\frac{1}{\abs{t-x_0}}\int_{x_0}^t \varepsilon = \varepsilon
    \end{split}
    \]
    se $\abs{t-x_0}<\delta_\varepsilon$, in quanto in questo caso $\abs{u-x_0}<\delta_\varepsilon$ per ogni $u\in[x_0,t]$. Quindi dato $\varepsilon>0$ abbiamo trovato $\delta_\varepsilon$ tale che
    \[
    \abs{\frac{F(t)-F(x_0)}{t-x_0}-f(x_0)}<\varepsilon \ \text{se} \ \abs{t-x_0}<\delta_\varepsilon
    \]
    ossia vale, come richiesto, 
    \[
    \lim_{t\to x_0} \frac{F(t)-F(x_0)}{t-x_0} = f(x_0)
    \]
\end{proof}
\begin{remark}
    La funzione $F$ definita nell'equazione (\ref{eq:8.6}) è detta \emph{funzione integrale} di $f$.
\end{remark}
\begin{theorem}[Teorema fondamentale del calcolo]
    \label{th:8.3}
    Sia $f\in\mathscr{R}([a,b])$; se esiste una funzione $F\colon[a,b]\to \amsbb{R}$ tale che $F'=f$, allora
    \[
    \int_a^b f(x)\, dx = F(b)-F(a)
    \]
\end{theorem}
\begin{proof}
    Per il criterio di Riemann \ref{th:8.1} sappiamo che per ogni $\varepsilon>0$ esiste una partizione $\mathscr{P}_\varepsilon$ tale che 
    \[
    U(\mathscr{P}_\varepsilon, f) - L(\mathscr{P}_\varepsilon, f)<\varepsilon
    \]
    Tale partizione è data da $\mathscr{P}_\varepsilon = \{x_0, x_1, \dots, x_i, \dots, x_n\}$. Consideriamo ora $F\Big|_{[x_{i-1}, x_i]}$; poiché $F$ è differenziabile su (un aperto contenente) $[a,b]$, lo è in $(x_{i-1}, x_i)$, ed è inoltre continua in $[x_{i-1}, x_i]$. Pertanto possiamo applicare il teorema di Lagrange \ref{th:7.2}: esiste $t_i\in(x_{i-1}, x_i)$ tale che
    \[
    F(x_i) -F(x_{i-1}) = f(t_i) (x_i - x_{i-1})
    \]
    Notiamo quindi che
    \[
    \sum_{i=1}^n F(x_i) -F(x_{i-1}) = F(b)-F(a)
    \]
    Consideriamo
    \[
    \sum_{i=1}^nf(t_i) \Delta_i
    \]
    Notiamo che $m_i \le f(t_i) \le M_i $, e di conseguenza
    \[
    L(\mathscr{P}_\varepsilon, f) \le F(b)-F(a) \le U(\mathscr{P}_\varepsilon, f)
    \]
    Ricordando che 
    \[
    L(\mathscr{P}, f) \le \int_a^b f \le U(\mathscr{P}, f) \ \text{per ogni partizione}\ \mathscr{P}
    \]
    possiamo scrivere
    \[
    \left(F(b)-F(a)\right)-\int_a^bf \le \left(F(b)-F(a)\right)-L(\mathscr{P}_\varepsilon, f) \le U(\mathscr{P}_\varepsilon, f)-L(\mathscr{P}_\varepsilon, f) < \varepsilon
    \]
    e
    \[
    \int_a^b f -\left(F(b)-F(a)\right)\le U(\mathscr{P}_\varepsilon, f) - \left(F(b)-F(a)\right) \le U(\mathscr{P}_\varepsilon, f) - L(\mathscr{P}_\varepsilon, f) <\varepsilon
    \]
    ossia
    \[
    \abs{\left(F(b)-F(a)\right)-\int_a^bf}<\varepsilon
    \]
    per ogni $\varepsilon>0$; ciò implica che
    \[
    \abs{\left(F(b)-F(a)\right)-\int_a^bf} = 0 \iff F(b)-F(a)=\int_a^bf
    \]
\end{proof}
\begin{remark}
    Una funzione $F$ avente le proprietà descritte dall'enunciato del teorema \ref{th:8.3} è detta \emph{primitiva} di $f$. Chiaramente, se $f$ ammette una primitiva ne ammette in realtà infinite, in quanto $(F+c)' = f$ per ogni $c\in\amsbb{R}$. Notiamo che il teorema \ref{th:8.2} dice che ogni funzione continua ammette una primitiva; questa non è però necessariamente scrivibile in termini di funzioni elementari.\\ 
    Il teorema fondamentale del calcolo \ref{th:8.3} è la ragione per cui si ricercano le primitive delle funzioni per calcolare gli integrali.
\end{remark}
\subsection{Esercizi: ricerca delle primitive}
\begin{exercise}
    \label{ex:8.1}
    Data la funzione $\amsbb{R}\setminus\{0\}\ni x \mapsto \frac{1}{x}$, determinarne una primitiva.
\end{exercise}
\begin{proof}[Soluzione]
    Ricordiamo che una primitiva di $f$ è una funzione $F\colon \amsbb{R}\setminus\{0\}\to \amsbb{R}$ tale che $F'(x) = f(x)$ per ogni $x\in\amsbb{R}\setminus\{0\}$.\\
    Sappiamo che se consideriamo $\log(\cdot)\colon (0, +\infty)\to \amsbb{R}$, questa è tale per cui
    \[
    \frac{d}{dx}\log(x) = \frac{1}{x}
    \]
    Tuttavia, questa non è una primitiva di $f$, poiché è definita su di un dominio diverso. Possiamo però considerare la funzione
    \[
    \log(-\cdot)\colon (-\infty, 0) \to \amsbb{R}
    \]
    che è tale che
    \[
    \frac{d}{dx}\log(-x) \overset{(\ref{eq:6.4})}{=} \frac{1}{-x}\frac{d}{dx}(-x) = -\frac{1}{-x} = \frac{1}{x}
    \]
    Pertanto abbiamo che la funzione $\log\abs{\cdot}\colon \amsbb{R}\setminus\{0\}\to \amsbb{R}$ è una primitiva di $f$.
\end{proof}
\begin{remark}
    In generale, trovare delle primitive non è facile; possiamo aiutarci scrivendo una primitiva \emph{formalmente} come
    \[
    \int f(x)\, dx
    \]
    e usando formalmente i teoremi validi nella teoria dell'integrazione.
\end{remark}
\begin{exercise}
    \label{ex:8.2}
    Data la funzione $f\colon(-\infty, 1] \to \amsbb{R}$, $f(x) = x\sqrt{1-x}$, determinarne una primitiva.
\end{exercise}
\begin{proof}[Soluzione]
    Scriviamo la primitiva di $f$ come
    \[
    \int x\sqrt{1-x}\, dx
    \]
    e ricordiamo il seguente risultato:
    \begin{tcolorbox}
        \begin{theorem}[Cambiamento di variabile]
            \label{th:8.4}
            Sia $f\colon[a,b]\to\amsbb{R}$ una funzione integrabile su $[a,b]$, e sia $\varphi\colon [c,d]\to [a,b]$ una funzione differenziabile su $[c,d]$ con $\varphi'\in\mathscr{R}([c,d])$ e tale che $\varphi([c,d])\subseteq [a,b]$; allora $f\circ \varphi\in\mathscr{R}([c,d])$ e
            \[
            \int_c^d (f\circ \varphi)(y) \varphi'(y)\, dy = \int_{\varphi(c)}^{\varphi(d)} f(x)\, dx
            \]
        \end{theorem}
    \end{tcolorbox}
    Vorremmo quindi applicarlo formalmente alla scrittura precedente. Per farlo, dobbiamo scegliere in maniera oculata la funzione $\varphi$, in modo tale che ci aiuti effettivamente a trovare una primitiva.\\
    Ricordiamo che
    \[
    \cos^2\theta + \sin^2 \theta = 1 \ \text{per ogni} \ \theta\in\amsbb{R}
    \]
    e notiamo che $\sin^2(\cdot)\colon \left[0, \frac{\pi}{2}\right]\to [0,1]$ è una funzione differenziabile e con derivata continua; inoltre è e biiettiva, e quindi se $x\in[0,1]$ lo possiamo scrivere come $\sin^2(\theta_x)$ per qualche $\theta_x\in\left[0, \frac{\pi}{2}\right]$. 
    Come nel caso precedente, restringiamo quindi $f$ all'intervallo $[0,1]$ e verifichiamo poi se riusciamo a ricondurre la primitiva trovata ad una funzione definita su tutto $(-\infty, 1]$.
    Per il teorema \ref{th:8.4}, applicato formalmente, abbiamo che
    \[
    \begin{split}
        \int x\sqrt{1-x}\, dx & = \int f(x)\, dx = \int (f\circ \sin^2)(\theta) \frac{d}{d\theta}\sin^2(\theta)\, d\theta = \\
        & = \int 2\sin^3(\theta)\cos(\theta)\sqrt{1-\sin^2(\theta)}\, d\theta = \int 2\sin^3(\theta)\cos^2(\theta)\, d\theta
    \end{split}
    \]
    ove $\sqrt{\cos^2(\theta)} = \cos(\theta)$ poiché su $\left[0, \frac{\pi}{2}\right]$ il coseno assume valori positivi.\\
    In generale, possiamo trovare le primitive di funzioni del tipo $\sin^n(\theta)\cos^m(\theta)$ sfruttando il fatto che
    \[
    \frac{d}{d\theta}\sin^n(\theta) = n\sin^{n-1}(\theta)\cos(\theta) \quad \frac{d}{d\theta}\cos^n(\theta) = -n\cos^{n-1}(\theta)\sin(\theta) \quad \sin^2(\theta) + \cos^2(\theta) = 1
    \]
    Nel nostro caso, abbiamo 
    \[
    \begin{split}
        & \int 2\sin^3(\theta)\cos^2(\theta)\, d\theta = 2\int \sin(\theta)(1-\cos^2(\theta))\cos^2(\theta)\, d\theta = \\
        & = 2\int \sin(\theta)\cos^2(\theta)\, d\theta - 2\int \sin(\theta)\cos^4(\theta)\, d\theta = \\
        & = -\frac{2}{3}\int \frac{d}{d\theta}(\cos^3(\theta))\, d\theta +\frac{2}{5} \int \frac{d}{d\theta}(\cos^5(\theta)) = -\frac{2}{3}\cos^3(\theta) + \frac{2}{5}\cos^5(\theta)
    \end{split}
    \]
    Per trovare una primitiva di $f(x)$, è ora necessario scrivere $\theta$ in funzione di $x$, sapendo che vale la biiezione
    \[
    [0,1]\ni x \mapsto \sin^2(\theta), \ \theta\in\left[0,\frac{\pi}{2}\right] 
    \]
    Possiamo quindi scrivere $\theta=\arcsin(\sqrt{x})$; a questo punto
    \[
    \begin{split}
        \int x\sqrt{1-x}\, dx & = -\frac{2}{3}\cos^3(\arcsin(\sqrt{x}))+\frac{2}{5}\cos^5(\arcsin{\sqrt{x}}) = \\
        & = -\frac{2}{3}(1-x)^{\frac{3}{2}}+\frac{2}{5}(1-x)^{\frac{5}{2}} = F(x)
    \end{split}
    \]
    Ricordiamo che abbiamo trovato la primitiva restringendo il dominio della funzione a $[0,1]$; quindi a priori la funzione $F$ ottenuta è definita su $[0,1]$. Tuttavia, notiamo che $F$ è in realtà la restrizione a $[0,1]$ di una funzione definita su $(-\infty, 1]$, e inoltre
    \[
    \frac{d}{dx}F(x) = -\sqrt{1-x}(-1)+(\sqrt{1-x})^3(-1) = \sqrt{1-x}\left(1-(\sqrt{1-x})^2\right) = x\sqrt{1-x}
    \]
    Di conseguenza la funzione $F$ che abbiamo trovato è definita su tutto $(-\infty, 1]$ e soddisfa $F'(x) = f(x)$ per ogni $x\in(-\infty, 1]$.
\end{proof}
\begin{remark}
    In questo caso la primitiva poteva essere ottenuta anche tramite un'integrazione per parti (cfr. teorema \ref{th:8.5}): infatti scrivendo
    \[
    F(x)=x \qquad g(x)=\sqrt{1-x}
    \]
    e notando che $G(x)=-\frac{2}{3}(1-x)^{\frac{3}{2}}$ abbiamo che
    \[
    \begin{split}
        &\int x\sqrt{1-x}\, dx = \int F(x)g(x)\, dx = F(x)G(x)-\int f(x)G(x)\, dx = -\frac{2}{3}(1-x)^{\frac{3}{2}}x+\\
        & +\frac{2}{3}\int (1-x)^{\frac{3}{2}} = -\frac{2}{3}(1-x)^{\frac{3}{2}}x-\frac{4}{15}(1-x)^{\frac{5}{2}} = \frac{2}{3}(1-x)^{\frac{3}{2}}(1-x-1) -\frac{4}{15}(1-x)^{\frac{5}{2}} = \\
        & = -\frac{2}{3}(1-x)^{\frac{3}{2}}  + \left(\frac{2}{3}-\frac{4}{15}\right)(1-x)^{\frac{5}{2}} = -\frac{2}{3}(1-x)^{\frac{3}{2}} + \frac{2}{5}(1-x)^{\frac{5}{2}}
    \end{split} 
    \]
    In questo caso avremmo quindi potuto ottenere l'integrale con altri mezzi, senza passare per la procedura di restrizione e sostituzione con $\sin^2(\theta)$. Ci sono casi in cui questo non è possibile: ad esempio se volessimo trovare una primitiva di $\sqrt{1-x^2}$ o di $(1-x^2)^{-\frac{3}{2}}$ l'integrazione per parti non è uno strumento utile per trovare una primitiva. In questo caso la sostituzione $x=\sin(\theta)$ fornisce invece la primitiva.\\
    La sostituzione fatta nell'esercizio precedente torna inoltre utile anche nel caso in cui l'integrazione per parti conduca ad un integrale, ma dopo molti passaggi, come ad esempio nel caso di $f(x)=x^5\sqrt{1-x}$.
\end{remark}
\begin{exercise}
    \label{ex:8.3}
    Data la funzione $\amsbb{R}\ni x \mapsto \frac{1}{\sqrt{x^2+a^2}}$, $a\ne 0$, determinarne una primitiva.
\end{exercise}
\begin{proof}[Soluzione]
    Anche in questo caso, scriviamo la sua primitiva come
    \[
    \int \frac{1}{\sqrt{x^2+a^2}}\, dx = \int \frac{1}{\sqrt{a^2\left(1+\frac{x^2}{a^2}\right)}}\, dx = \int \frac{1}{\abs{a}\sqrt{1+\frac{x^2}{a^2}}}\, dx = \frac{a}{\abs{a}}\int \frac{1}{\sqrt{1+\frac{x^2}{a^2}}}\frac{1}{a}\, dx 
    \]
    Possiamo considerare la funzione $x\mapsto \frac{1}{\sqrt{1+\frac{x^2}{a^2}}}$ come la composizione di due funzioni,
    \[
    \frac{1}{\sqrt{1+(\cdot)^2}} \circ \frac{1}{a} \colon \amsbb{R}\to \amsbb{R}
    \]
    e di conseguenza possiamo applicare il teorema \ref{th:8.4}, notando che 
    \[
    \frac{d}{dx}\left(\frac{1}{a}x\right) = \frac{1}{a}
    \]
    Quindi
    \[
    \int \frac{1}{\sqrt{x^2+a^2}}\, dx = \frac{a}{\abs{a}}\int \frac{1}{\sqrt{1+\frac{x^2}{a^2}}}\frac{1}{a}\, dx = \frac{a}{\abs{a}}\int \frac{1}{\sqrt{1+y^2}}\, dy
    \]
    Vogliamo riapplicare il teorema \ref{th:8.4}, come fatto nell'esercizio \ref{ex:8.2}. Dobbiamo quindi trovare una funzione $\varphi$ che ci aiuti a trovare una primitiva. A questo scopo, ricordiamo che
    \[
    \cosh^2(x)-\sinh^2(x) = 1 \ \text{per ogni} \ x\in\amsbb{R}
    \]
    e che $\sinh(\cdot)\colon \amsbb{R} \to \amsbb{R}$ è una funzione biiettiva, differenziabile e con derivata continua. Applichiamo quindi il teorema \ref{th:8.4}:
    \[
    \frac{a}{\abs{a}}\int \frac{1}{\sqrt{1+y^2}}\, dy = \frac{a}{\abs{a}}\int \frac{1}{\sqrt{1+\sinh^2(\xi)}}\cosh(\xi)\, d\xi = \frac{a}{\abs{a}}\int \frac{\cosh(\xi)}{\cosh(\xi)}\, d\xi = \frac{a}{\abs{a}}\xi
    \]
    dove abbiamo usato il fatto che il coseno iperbolico assume sempre valori positivi.\\
    Come nell'esercizio precedente, per trovare l'effettiva primitiva di $f(x)$, è necessario esprimere $\xi$ in funzione di $x$, sapendo che vale la biiezione
    \[
    \amsbb{R}\ni y \mapsto \sinh(\xi)\in\amsbb{R}
    \]
    Vale che
    \[
    y = \sinh(\xi) = \frac{1}{2}(e^{\xi}-e^{-\xi}) \iff 2y = e^\xi - e^{-\xi} \iff (e^\xi)^2 -2y e^{\xi}-1 = 0
    \]
    Sostituendo $t=e^\xi$ abbiamo
    \[
    t^2 -2yt -1 = 0 \iff t =y\pm\sqrt{y^2+1}
    \]
    Poiché $t=e^\xi>0$, vale che $t = y+\sqrt{y^2+1}$; di conseguenza
    \[
    e^\xi = y+\sqrt{y^2+1} \iff \xi = \log(y+\sqrt{y^2+1})
    \]
    Ricordando che $y=\frac{x}{a}$ abbiamo infine che
    \[
    \int \frac{1}{\sqrt{x^2+a^2}}\, dx = \frac{a}{\abs{a}}\xi = \frac{a}{\abs{a}}\log(y+\sqrt{y^2+1}) = \frac{a}{\abs{a}}\log\left(\frac{x}{a}+\sqrt{\frac{x^2}{a^2}+1}\right) = F(x)
    \]
    Notiamo che $F$ è definita su tutto $\amsbb{R}$, e che
    \[
    \begin{split}
        \frac{d}{dx}F(x) & = \frac{a}{\abs{a}}\frac{1}{\frac{x}{a}+\sqrt{\frac{x^2}{a^2}+1}}\left(\frac{1}{a}+\frac{x}{a^2\sqrt{\frac{x^2}{a^2}+1}}\right) = \frac{1}{\abs{a}}\frac{1}{\frac{x}{a}+\sqrt{\frac{x^2}{a^2}+1}}\frac{\sqrt{\frac{x^2}{a^2}+1}+\frac{x}{a}}{\sqrt{\frac{x^2}{a^2}+1}}\\
        & = \frac{1}{\abs{a}}\frac{1}{\sqrt{\frac{x^2}{a^2}+1}} = \frac{1}{\sqrt{x^2+a^2}}
    \end{split}
    \]
\end{proof}
\begin{exercise}
    \label{ex:8.4}
    Data la funzione $f\colon \amsbb{R}^+\ni x \mapsto \sqrt{x}\log(x)$ determinarne una primitiva.
\end{exercise}
\begin{proof}[Soluzione]
    Ricordiamo il seguente risultato:
    \begin{tcolorbox}
        \begin{theorem}[Integrazione per parti]
            \label{th:8.5}
            Siano $F,G\colon [a,b]\to \amsbb{R}$ due funzioni differenziabili in $[a,b]$ con $F'=f$ e $G' = g$, $f,g\in\mathscr{R}([a,b])$. Allora
            \[
            \int_a^b Fg\,dx = \left(F(b)G(b)-F(a)G(a)\right)-\int_a^b fG\, dx
            \]
        \end{theorem}
    \end{tcolorbox}
    In generale quando si ha a che fare con il logaritmo l'integrazione per parti torna molto utile perché sappiamo che
    \[
    \frac{d}{dx}\log(x) = \frac{1}{x}
    \]
    Quindi in questo caso $F(x) = \log(x)$ e $g(x)=\sqrt{x}$, e sappiamo che
    \[
    f(x) = \frac{1}{x} \qquad G(x) = \frac{2}{3}x^{\frac{3}{2}}
    \]
    Quindi applicando il teorema \ref{th:8.5} abbiamo che
    \[
    \begin{split}
        \int \sqrt{x}\log(x)\, dx & = \int F(x)g(x)\, dx = F(x)G(x)-\int f(x)G(x)\, dx = \\
        & = \frac{2}{3}\log(x)x^{\frac{3}{2}}-\int \frac{1}{x}\frac{2}{3}x^{\frac{3}{2}}\, dx = \frac{2}{3}x^{\frac{3}{2}}\log(x)-\frac{4}{9}x^{\frac{3}{2}} = F(x)
    \end{split}
    \]
    Notiamo che $F$ è definita su $\amsbb{R}^+$ come $f$, e che
    \[
    \frac{d}{dx}F(x) \overset{(\ref{eq:6.3})}{=} \sqrt{x}\log(x)+\frac{2}{3}\sqrt{x} -\frac{2}{3}\sqrt{x} = \sqrt{x}\log(x) 
    \]
\end{proof}
\begin{exercise}
    \label{ex:8.5}
    Data $f\colon \amsbb{R}\setminus\{-a, a\}\to \amsbb{R}$, $f(x) = \frac{1}{x^2-a^2}$, determinarne una primitiva.
\end{exercise}
\begin{proof}[Soluzione]
    Sappiamo trovare agilmente le primitive di funzioni del tipo $x\mapsto \frac{1}{x+a}$, 
    \[
    \int \frac{1}{x+a}\, dx = \int \frac{1}{y}\, dy = \log\abs{y} = \log\abs{x+a}
    \]
    Notiamo che possiamo scrivere
    \[
    f(x) = \frac{1}{x^2-a^2} = \frac{1}{(x+a)(x-a)}
    \]
    e quindi ci chiediamo se esistano due coefficienti $k_1, k_2\in\amsbb{R}$ tali che
    \[
    f(x) = \frac{1}{(x-a)(x+a)} = \frac{k_1}{x+a} + \frac{k_2}{x-a}
    \]
    Effettuando esplicitamente la somma di frazioni otteniamo
    \[
    \frac{k_1}{x+a} + \frac{k_2}{x-a} = \frac{k_1(x-a)+k_2(x+a)}{x^2-a^2} = \frac{(k_1+k_2)x+(k_2-k_1)a}{x^2-a^2}
    \]
    e affinché valga l'uguaglianza
    \[
    f(x) = \frac{k_1}{x+a}+\frac{k_2}{x-a}
    \]
    per l'indipendenza lineare dei polinomi di diverso grado i coefficienti $k_1, k_2$ devono soddisfare il sistema
    \[
    \begin{dcases}
        k_1 +k_2 = 0\\
        (k_2-k_1)a=1
    \end{dcases} \iff \begin{dcases}
        k_1 =- k_2\\
        (k_2-k_1)a=1
    \end{dcases} \iff k_2 = \frac{1}{2a}
    \]
    Quindi
    \[
    \begin{split}
        \int \frac{1}{x^2-a^2}\, dx & = \int \frac{1}{2a}\frac{1}{x-a}-\frac{1}{2a}\frac{1}{x+a}\, dx = \frac{1}{2a}\int \frac{1}{x-a}\, dx - \frac{1}{2a}\int \frac{1}{x+a}\, dx = \\
        & = \frac{1}{2a}\log\abs{\frac{x-a}{x+a}}=F(x)
    \end{split}
    \]
    Notiamo che $F$ è definita su tutto $\amsbb{R}\setminus\{-a,a\}$ e che 
    \[
    \frac{d}{dx}F(x) = \frac{1}{2a} \frac{1}{x-a}-\frac{1}{2a}\frac{1}{x+a} = \frac{1}{2a}\frac{x+a-x+a}{x^2-a^2} = \frac{1}{x^2-a^2}
    \]
\end{proof}
\begin{exercise}
    \label{ex:8.6}
    Data $f\colon \amsbb{R}\to\amsbb{R}$, $f(x)=\frac{2x-1}{x^2+2x+2}$, determinarne una primitiva.
\end{exercise}
\begin{proof}[Soluzione]
    $f$ è definita su tutto $\amsbb{R}$, infatti il discriminante del denominatore è $\Delta = 4-8 <0$ e di conseguenza è sempre positivo. Non possiamo quindi agire come nell'esercizio \ref{ex:8.5}; notiamo però che
    \[
    x^2+2x+2 = x^2 +2x+1 +1 = (x+1)^2+1
    \]
    e che 
    \[
    \frac{d}{dx}(x+1)^2 = 2x+2 = 2x-1+3
    \]
    Quindi possiamo scrivere
    \[
    \begin{split}
        \int \frac{2x-1}{x^2+2x+2}\, dx & = \int \frac{2x-1}{(x+1)^2+1}\, dx = \int \frac{\frac{d}{dx}(x+1)^2-3}{(x+1)^2+1}\, dx = \\
        & = \int \frac{1}{(x+1)^2+1}\frac{d}{dx}(x+1)^2\, dx -3\int \frac{1}{(x+1)^2+1}\, dx \overset{\text{Teorema \ref{th:8.4}}}{=} \\
        & = \int \frac{1}{y+1}\, dy -3\int \frac{1}{(x+1)^2+1} \frac{d}{dx}(x+1)\, dx \overset{\text{Teorema \ref{th:8.4}}}{=} \\
        & = \log\abs{y+1} -3 \int \frac{1}{z^2+1}\, dz = \log\abs{y+1}-3\arctan(z) = \\
        & = \log\abs{(x+1)^2+1} -3\arctan(x+1) = F(x)
    \end{split}
    \]
    Notiamo che $F$ è definita su tutto $\amsbb{R}$ come $f$, e che
    \[
    \frac{d}{dx}F(x) = \frac{2(x+1)}{(x+1)^2+1}-3\frac{1}{(x+1)^2+1} = \frac{2x-1}{x^2+2x+2}
    \]
\end{proof}
\begin{exercise}
    \label{ex:8.7}
    Data $f\colon \amsbb{R}\to \amsbb{R}$, $f(x) = \frac{1}{\cos(x)+2}$, determinarne una primitiva.
\end{exercise}
\begin{proof}[Soluzione]
    In questo caso conviene usare le formule parametriche per scrivere
    \[
    \cos(x)=\frac{1-t^2}{1+t^2} \qquad t = \tan\left(\frac{x}{2}\right)\iff x = 2\arctan(t) \ \text{per} \ x\in(-\pi, \pi)
    \]
    Notiamo però che se scriviamo effettuiamo la sostituzione sopra descritta, $f$ non sarà più definita su $\amsbb{R}$, bensì su un $\amsbb{R}\setminus\{\pi+2k\pi, k\in\amsbb{Z}\}$. Poiché conviene restringersi ad intervalli connessi, scegliamo $(-\pi, \pi)$ come intervallo di definizione di $f$. Notando che se consideriamo $\cos\colon (-\pi, \pi)\to \mathbb{R}$ vale che 
    \[
    \cos \circ (2\arctan(t)) = \frac{1-t^2}{1+t^2}
    \]
    come dalle formule parametriche e usando il teorema \ref{th:8.4} abbiamo che
    \[
    \begin{split}
        \int \frac{1}{\cos(x)+2}\, dx & = \int \frac{1+t^2}{1-t^2+2+2t^2}\frac{d}{dt}(2\arctan(t))\, dt = \\
        & = \int \frac{1+t^2}{t^2+3}\frac{2}{1+t^2}\, dt = \int \frac{2}{t^2+3}\, dt  =\\
        & = \frac{2}{3}\int \frac{1}{1+\left(\frac{t}{\sqrt{3}}\right)^2}\frac{\sqrt{3}}{\sqrt{3}}\, dt \overset{\text{Teorema \ref{th:8.4}}}{=} \frac{2}{\sqrt{3}}\int \frac{1}{1+y^2}\, dy = \frac{2}{\sqrt{3}}\arctan(y) =\\
        & = \frac{2}{\sqrt{3}}\arctan\left(\frac{t}{\sqrt{3}}\right) = \frac{2}{\sqrt{3}}\arctan\left(\frac{1}{\sqrt{3}}\tan\left(\frac{x}{2}\right)\right) = F(x)
    \end{split}
    \]
    Notiamo che in questo caso $F$ è definita in $\amsbb{R}\setminus\{\pi+2k\pi, k\in\amsbb{Z}\}$, e di conseguenza non è una primitiva di $f$, che è invece definita su tutto $\amsbb{R}$. Come possiamo fare a determinare un'effettiva primitiva di $f$?\\
    Innanzitutto, notiamo che $F$ ammette limite destro e sinistro finiti per $x\to \pi+2k\pi$: infatti,
    \[
    \begin{split}
        \lim_{x\to (\pi+2k\pi)^-} F(x) & = \lim_{x\to (\pi+2k\pi)^-}\frac{2}{\sqrt{3}}\arctan\left(\frac{1}{\sqrt{3}}\tan\left(\frac{x}{2}\right)\right) \overset{y=\frac{1}{\sqrt{3}}\tan\left(\frac{x}{2}\right)}{=} \\
        & = \lim_{y\to +\infty} \frac{2}{\sqrt{3}}\arctan(y) = \frac{\pi}{\sqrt{3}}
    \end{split}
    \]
    e
    \[
    \begin{split}
        \lim_{x\to (\pi+2k\pi)^+} F(x) & = \lim_{x\to (\pi+2k\pi)^+}\frac{2}{\sqrt{3}}\arctan\left(\frac{1}{\sqrt{3}}\tan\left(\frac{x}{2}\right)\right) \overset{y=\frac{1}{\sqrt{3}}\tan\left(\frac{x}{2}\right)}{=} \\
        & = \lim_{y\to -\infty} \frac{2}{\sqrt{3}}\arctan(y) = -\frac{\pi}{\sqrt{3}}
    \end{split}
    \]
    \begin{center}
        \begin{tikzpicture}[xscale=.6, yscale = 1, font=\tiny]
            \tzaxes(-9.7, -2.3)(9.7,2.3){$x$}[b]{$y$}[l]
            \tzticks{3.1415/$\pi$}[b]
            \tzticks{6.2832/$2\pi$}[b]
            \tzticks{9.4248/$3\pi$}[b]
            \tzticks{-3.1415/$-\pi$}[b]
            \tzticks{-6.2832/$-2\pi$}[b]
            \tzticks{-9.4248/$-3\pi$}[b]
            \tzfn[blue, thick, samples = 401]"curve"{2/1.7321*rad(atan(1/1.7321*tan(deg(\x/2))))}[-3.1413:3.1413]
            \tzfn[blue, thick, samples = 401]"curve"{2/1.7321*rad(atan(1/1.7321*tan(deg(\x/2))))}[3.1418:9.4246]{$F(x)$}[r]
            \tzfn[blue, thick, samples = 401]"curve"{2/1.7321*rad(atan(1/1.7321*tan(deg(\x/2))))}[-9.4246:-3.1418]
            \tzhfnat[dashed]{1.8138}[-9.4248:9.4248]{$\frac{\pi}{\sqrt{3}}$}[a]
            \tzhfnat[dashed]{-1.8138}[-9.4248:9.4248]{$-\frac{\pi}{\sqrt{3}}$}[b]
            %\tzdot*[red](0.2122, 0)
            %\tznode[red](0.2122, 0){$2k\pi + \frac{3}{2}\pi$}[below, red]
        \end{tikzpicture}
    \end{center}
    Poiché sono distinti, il limite
    \[
    \lim_{x\to (\pi +2k\pi)} F(x)
    \]
    non esiste, e quindi non è possibile estendere $F$ per continuità a tutto $\amsbb{R}$. Possiamo però agire nel modo seguente: innanzitutto definiamo
    \[
    \widehat{F}(x) = \begin{dcases}
        \frac{2}{\sqrt{3}}\arctan\left(\frac{1}{\sqrt{3}}\tan\left(\frac{x}{2}\right)\right)\, x\in(-\pi, \pi) \\
        \frac{\pi}{\sqrt{3}}\, & x=\pi
    \end{dcases}
    \]
    In questo modo abbiamo esteso $F$ a $(-\pi, \pi]$ preservando la continuità da sinistra. Notiamo poi che per ogni $x\in\amsbb{R}$, esiste un unico $k\in\amsbb{Z}$ tale che
    \[
    x\in(-\pi + 2k\pi, \pi + 2k\pi]
    \]
    in quanto
    \[
    \bigcup_{k\in\amsbb{Z}} (-\pi + 2k\pi, \pi +2k\pi] = \amsbb{R} \quad (-\pi+2k\pi, \pi+2k\pi]\cap (-\pi+2i\pi, \pi+2i\pi]=\varnothing \ \text{se} \ k\ne i
    \]
    Ci ricordiamo, poi, come scritto nell'osservazione successiva al teorema \ref{th:8.3}, che le primitive di una funzione non sono uniche, bensì sono definite a meno di una costante additiva. In questo caso, possiamo sfruttare questa proprietà a nostro vantaggio per definire un'ulteriore estensione di $F$ nel modo seguente:
    \[
    \tilde{F}(x) = \widehat{F}(x-2k\pi)+k\frac{2\pi}{\sqrt{3}}, \ k \ \text{tale che} \ x\in(-\pi +2k\pi, \pi+2k\pi]
    \]
    In questo modo $\tilde{F}$ è continua su tutto $\amsbb{R}$: infatti, $\tilde{F}$ è chiaramente continua su $\amsbb{R}\setminus\{\pi + 2k\pi, k\in\amsbb{Z}\}$, e vale che
    \[
    \lim_{x\to (\pi+2k\pi)^-}\tilde{F}(x) = \lim_{x\to (\pi +2k\pi)^-} \widehat{F}(x-2k\pi) + k\frac{2\pi}{\sqrt{3}} = \frac{(2k+1)\pi}{\sqrt{3}}
    \]
    e
    \[
    \begin{split}
        \lim_{x\to (\pi+2k\pi)^+}\tilde{F}(x) & = \lim_{x\to (\pi +2k\pi)^+} \widehat{F}(x-2(k+1)\pi) + (k+1)\frac{2\pi}{\sqrt{3}} = -\frac{\pi}{\sqrt{3}} +  (k+1)\frac{2\pi}{\sqrt{3}}= \\
        & = (2k+1)\frac{\pi}{\sqrt{3}}
    \end{split}
    \]
    Inoltre, $\tilde{F}$ è differenziabile su tutto $\amsbb{R}$: anche in questo caso $\tilde{F}$ è chiaramente differenziabile su $\amsbb{R}\setminus\{\pi +2k\pi, k\in\amsbb{Z}\}$, e vale che
    \[
    \begin{split}
        &\lim_{x\to (\pi+2k\pi)^+}\frac{\tilde{F}(x)-\tilde{F}(\pi+2k\pi)}{x-\pi -2k\pi}= \lim_{x\to (\pi+2k\pi)^+}\frac{\widehat{F}(x-2(k+1)\pi)+(k+1)\frac{2\pi}{\sqrt{3}}-\frac{\pi}{\sqrt{3}}-k\frac{2\pi}{\sqrt{3}}}{x-\pi -2k\pi} = \\
        & \overset{t=x-2(k+1)\pi}{=} \lim_{t\to-\pi^+}\frac{\widehat{F}(t)+\frac{\pi}{\sqrt{3}}}{t+\pi}
    \end{split}
    \]
    Per calcolare
    \[
    \lim_{t\to-\pi^+}\frac{\widehat{F}(t)+\frac{\pi}{\sqrt{3}}}{t+\pi}
    \]
    proviamo ad usare il teorema di de l'H{\^o}pital \ref{th:6.4}: sappiamo che $\widehat{F}+\frac{\pi}{\sqrt{3}}$ e $t+\pi$ sono differenziabili in $(-\pi, \pi)$, e che $(t+\pi)' = 1$ è diverso da $0$ per ogni $t\in(-\pi, \pi)$; inoltre $(\widehat{F}+\frac{\pi}{\sqrt{3}})'(t) = \frac{1}{\cos(t)+2}$ per ogni $t\in (-\pi, \pi)$, e 
    \[
    \lim_{t\to -\pi^+} \frac{1}{\cos(t)+2} \overset{\text{continuità}}{=} 1
    \]
    e quindi per il teorema di de l'H{\^o}pital vale che
    \[
    \lim_{x\to (\pi+2k\pi)^+}\frac{\tilde{F}(x)-\tilde{F}(\pi+2k\pi)}{x-\pi -2k\pi}=1
    \]
    Allo stesso modo si dimostra che
    \[
    \begin{split}
        &\lim_{x\to (\pi+2k\pi)^-}\frac{\tilde{F}(x)-\tilde{F}(\pi+2k\pi)}{x-\pi -2k\pi}= \lim_{x\to (\pi+2k\pi)^-}\frac{\widehat{F}(x-2k\pi)+k\frac{2\pi}{\sqrt{3}}-\frac{\pi}{\sqrt{3}}-k\frac{2\pi}{\sqrt{3}}}{x-\pi -2k\pi} = \\
        & \overset{t=x-2k\pi}{=} \lim_{t\to\pi^-}\frac{\widehat{F}(t)-\frac{\pi}{\sqrt{3}}}{t-\pi} = 1
    \end{split}
    \]
    Di conseguenza $\tilde{F}$ è derivabile anche in $\{\pi+2k\pi, k\in\amsbb{Z}\}$; quindi $\tilde{F}$ è definita su tutto $\amsbb{R}$, è differenziabile su tutto $\amsbb{R}$ e vale che $F'(x) = f(x)$ per ogni $x\in\amsbb{R}$.
    \begin{center}
        \begin{tikzpicture}[xscale=.6, yscale = 0.5, font=\tiny]
            \tzaxes(-9.7, -5.7)(9.7,5.7){$x$}[b]{$y$}[l]
            \tzticks{3.1415/$\pi$}[b]
            \tzticks{6.2832/$2\pi$}[b]
            \tzticks{9.4248/$3\pi$}[b]
            \tzticks{-3.1415/$-\pi$}[b]
            \tzticks{-6.2832/$-2\pi$}[b]
            \tzticks{-9.4248/$-3\pi$}[b]
            \tzfn[blue, thick, samples = 401]"curve"{2/1.7321*rad(atan(1/1.7321*tan(deg(\x/2))))}[-3.1413:3.1413]
            \tzfn[blue, thick, samples = 401]"curve"{2/1.7321*rad(atan(1/1.7321*tan(deg(\x/2))))+1.8138*2}[3.1418:9.4246]{$\tilde{F}(x)$}[r]
            \tzfn[blue, thick, samples = 401]"curve"{2/1.7321*rad(atan(1/1.7321*tan(deg(\x/2))))-2*1.8138}[-9.4246:-3.1418]
            %\tzhfnat[dashed]{1.8138}[-9.4248:9.4248]{$\frac{\pi}{\sqrt{3}}$}[a]
            %\tzhfnat[dashed]{-1.8138}[-9.4248:9.4248]{$-\frac{\pi}{\sqrt{3}}$}[b]
            \tzdot*[blue](3.115, 1.8138)
            \tzdot*[blue](-3.115, -1.8138)
            \tzdot*[blue](9.4248, 5.4414)
            \tzdot*[blue](-9.424, -5.4414)
            \tzprojy(-3.115, -1.8138){$-\frac{\pi}{\sqrt{3}}$}[r]
            \tzprojy(3.115, 1.8138){$\frac{\pi}{\sqrt{3}}$}[l]
            \tzprojy(-9.424, -5.4414){$-3\frac{\pi}{\sqrt{3}}$}[r]
            \tzprojy(9.4248, 5.4414){$3\frac{\pi}{\sqrt{3}}$}[bl]
            %\tznode[red](0.2122, 0){$2k\pi + \frac{3}{2}\pi$}[below, red]
        \end{tikzpicture}
    \end{center}
\end{proof}
\begin{remark}
    La primitiva $\tilde{F}$ trovata nel precedente esercizio è un esempio di primitiva non scrivibile in termini di funzioni elementari su tutto il dominio di definizione; ci sono molti altri esempi di questo tipo, come le primitive di $e^{-x^2}$ e di $\frac{\sin(x)}{x}$.
\end{remark}
\begin{exercise}
    \label{ex:8.8}
    Calcolare
    \[
    \int_{-\pi}^\pi \frac{\sin(x)}{\cos^4(x)+\cos^2(x)+2}\, dx
    \]
\end{exercise}
\begin{proof}[Soluzione]
    In questo caso possiamo procedere in due modi: 
    \begin{enumerate}[(i)]
        \item possiamo provare a risolvere direttamente l'integrale: notiamo che la funzione integranda
        \[
        f(x) = \frac{\sin(x)}{\cos^4(x)+\cos^2(x)+2}
        \]
        può essere scritta come
        \[
        f(x) = -\frac{1}{\cos^4(x)+\cos^2(x)+2}\frac{d}{dx}(\cos(x)) = (g \circ \cos(x))\frac{d}{dx}(\cos(x))
        \]
        ove $g\colon x \mapsto -\frac{1}{x^4+x^2+2}$. Possiamo quindi applicare il teorema \ref{th:8.4}, e scrivere
        \[
        \begin{split}
            &\int_{-\pi}^\pi  \frac{\sin(x)}{\cos^4(x)+\cos^2(x)+2}\, dx = \int_{-\pi}^\pi (g\circ \cos)(x) \frac{d}{dx}\sin(x)\, dx = \\
            & = \int_{\cos(-\pi)}^{\cos(\pi)}-\frac{1}{y^4+y^2+2}\, dy = -\int_{-1}^{-1} \frac{1}{y^4+y^2+2}\, dy = 0
        \end{split}
        \]   
        \item Oppure possiamo usare il seguente risultato:
        \begin{tcolorbox}
            \begin{proposition}
                \label{prop:8.1}
                Sia $f\colon\amsbb{R}\to \amsbb{R}$ una funzione dispari tale che $f\in\mathscr{R}([-a,a])$, $a>0$; allora
                \[
                \int_{-a}^a f(x)\, dx = 0
                \]
            \end{proposition}
            \begin{proof}
                Sappiamo, per il corollario \ref{cor:8.1} (v), che $f\in\mathscr{R}([-a,0])$ e $f\in\mathscr{R}([0,a])$, e che
                \vspace{-0.2cm}
                \[
                \int_{-a}^a f(x)\, dx = \int_{-a}^0 f(x)\, dx + \int_0^a f(x)\, dx
                \vspace{-0.2cm}
                \]
                Possiamo quindi considerare la funzione $\varphi\colon [-a,0]\ni x \mapsto -x\in[0,a]$, che è differenziabile e ammette derivata continua in $[-a,0]$, e applicare il teorema \ref{th:8.4} al primo integrale: avremo
                \vspace{-0.2cm}
                \[
                \int_{0}^a f(x)\, dx = \int_{\varphi(0)}^{\varphi(-a)}f(x)\, dx =  \int_{0}^{-a} (f \circ \varphi)(x) \varphi'(x)\, dx = -\int_{0}^{-a} f(-x)\, dx
                \vspace{-0.2cm}
                \]
                Poiché $f$ è dispari su $\amsbb{R}$ vale che $f(-x) = -f(x)$; di conseguenza
                \vspace{-0.2cm}
                \[
                \int_{0}^a f(x)\, dx = -\int_0^{-a} f(-x)\, dx = \int_0^{-a} f(x)\, dx = -\int_{-a}^0 f(x)\, dx
                 \vspace{-0.2cm}
                \]
                Quindi
                \[
                \int_{-a}^a f(x)\, dx = \int_{-a}^0 f(x)\, dx - \int_{-a}^0 f(x)\, dx = 0\qedhere
                \]
                \vspace{-0.5cm}
            \end{proof}
        \end{tcolorbox}
        Verifichiamo che la funzione integranda sia dispari:
        \[
        \frac{\sin(-x)}{\cos^4(-x)+\cos^2(-x)+2} = \frac{-\sin(x)}{\cos^4(x)+\cos^2(x)+2} = -\frac{\sin(x)}{\cos^4(x)+\cos^2(x)+2}
        \]
        Possiamo quindi applicare la proposizione \ref{prop:8.1}, e concludere che
        \[
        \int_{-\pi}^\pi  \frac{\sin(x)}{\cos^4(x)+\cos^2(x)+2}\, dx =0
        \]
    \end{enumerate}
\end{proof}
\newpage
\section{Lezione 10}
\subsection{Ripasso: integrazione di Riemann impropria}
\begin{definition}
    \label{def:10.1}
    Sia $f\colon[a,+\infty)\to \amsbb{R}$ una funzione tale che $f\in\mathscr{R}([a,x])$ per ogni $x>a$. Diremo che la quantità
    \[
    \int_a^{+\infty} f(x)\, dx
    \]
    esiste se esiste finito il limite
    \[
    \lim_{x\to +\infty} \int_a^x f(t)\, dt = \lim_{x\to+\infty} F(x)
    \]
\end{definition}
Allo stesso modo, 
\begin{definition}
    \label{def:10.2}
    Data una funzione $f\colon [a,b)\to\amsbb{R}$ con $f\in\mathscr{R}([a,x])$ per ogni $x\in(a,b)$, diremo che la quantità
    \[
    \int_a^b f(x)\, dx
    \]
    esiste se esiste finito
    \[
    \lim_{x\to b^-} \int_a^x f(x)\, dx = \lim_{x\to b^-} F(x)
    \]
\end{definition}
\begin{remark}
    Notiamo che in questo secondo caso possono presentarsi due casi: la funzione $f\colon[a,b)\to\amsbb{R}$ non è definita in $b$ ma ammette limite finito per $x\to b^-$ (ad esempio se $f$ fosse $[-1, 0)\ni x\mapsto \frac{\sin(x)}{x}$), oppure $f$ non è definita in $b$ e non ammette limite finito per $x\to b^-$ (ad esempio, $[1,0)\ni x\mapsto \frac{1}{x}$). Nel primo caso possiamo agire nel modo seguente: possiamo definire per continuità una funzione $\tilde{f}\colon[a,b]\to \amsbb{R}$ che è definita su tutto $[a,b]$ ed è tale che $\tilde{f}\in\mathscr{R}([a,b])$. In virtù del corollario \ref{cor:8.2}, poiché $\tilde{f}$ e $f$ differiscono solamente in un punto, possiamo dire che
    \[
    \int_a^b f(x)\, dx = \int_a^b \tilde{f}(x)\, dx
    \]
    Di conseguenza, nel primo caso il limite esiste sempre; di conseguenza, in generale quando si parla di integrazione di Riemann impropria si tende a considerare unicamente due casi:
    \begin{enumerate}[(i)]
        \item funzioni $f$ \textbf{definite su un intervallo illimitato};
        \item funzioni $f$ \textbf{definite su un intervallo limitato aventi immagine illimitata}.
    \end{enumerate}
\end{remark}
\begin{example}
    Consideriamo $f\colon[0,+\infty)\ni t\mapsto  \sin(t)\in\amsbb{R}$; ci chiediamo se esiste
    \[
    \int_0^{+\infty}\sin(t)\, dt
    \]
    Calcoliamo quindi
    \[
    F(x) = \int_0^{x}\sin(t)\, dt = -\cos(t)\bigg|_0^x = 1-\cos(x)
    \]
    Vogliamo determinare, se esiste,
    \[
    \lim_{x\to+\infty}F(x)
    \]
    Ci ricordiamo che vale il teorema \ref{th:5.1}: consideriamo quindi due successioni,
    \[
    (a_n = 2n\pi)_n \qquad (bn=\pi + 2n\pi)_n
    \]
    Entrambe divergono a $+\infty$ per $n\to \infty$, e possiamo considerare quindi
    \[
    \lim_{n\to\infty} F(a_n) = \lim_{n\to\infty}1-\cos(2n\pi) = 0 \qquad \lim_{n\to\infty} F(b_n) = \lim_{n\to\infty} 1-\cos(\pi+2n\pi) = 2
    \]
    Di conseguenza per il teorema \ref{th:5.1} sappiamo che
    \[
    \lim_{n\to\infty}F(x)
    \]
    non esiste.\\
    Questo esempio ammette un'interessante interpretazione grafica: se consideriamo infatti il grafico di $f(t) = \sin(t)$
    \begin{center}
        \begin{tikzpicture}
            \pgfplotsset{
                axis lines=middle,
                xlabel style=below,
                ylabel style=left
            }
            \begin{axis}[y=7mm, ymin=-1.2, ymax = 1.2, xmin = -.2, xmax = 13.2, xscale =1.3, yscale = 3, samples=201, xlabel={$x$}, ylabel = {$y$}, xtick={3.14159, 6.28318, 9.42477, 12.56637}, xticklabels={$\pi$, $2\pi$, $3\pi$, $4\pi$}]
        %\addplot [name path=A, teal] coordinates { (\Xmin, -\Xmin) (\Xmax,-\Xmax) };
        %\path [name path=B] (\Xmax,\Ymax)--(\Xmin,\Ymax);
        

            \addplot [name path=A, domain=0:3.14159, black] {sin(deg(x))};
            \path [name path=B] (0, 0)--(3.14159, 0);
            \addplot [red!30] fill between [of=A and B];
            \addplot [name path=C, domain=3.14159:6.28318, black] {sin(deg(x))};
            \path [name path=D] (3.14159, 0)--(6.28318, 0);
            \addplot [blue!30] fill between [of=C and D];
            \addplot [name path=E, domain=6.28318:9.42477, black] {sin(deg(x))};
            \path [name path=F] (6.28318, 0)--(9.42477,0);
            \addplot [red!30] fill between [of=E and F];
            \addplot [name path=G, domain=9.42477:12.56637, black] {sin(deg(x))};
            \path [name path=H] (9.42477, 0)--(12.56637, 0);
            \addplot [blue!30] fill between [of=G and H];
        %\addplot [white] fill between [of=C and D];
        %\endpgfonlayer
                \end{axis}
            \end{tikzpicture}
        \end{center}
        sappiamo che $F(x)$ rappresenta l'area sottesa dal grafico di $\sin(t)$ da $0$ a $x$. Per $x$ che varia da 0 a $\pi$ quest'area cresce fino a raggiungere il valore massimo di $2$ per $x=\pi$; dopodiché decresce sino a raggiungere il valore minimo di $0$ per $x=2\pi$, e il processo si ripete ogni periodo. Poiché l'area sottesa dal grafico oscilla in questo modo, non tende a stabilizzarsi verso un valore predefinito per $x\to +\infty$, e di conseguenza l'integrale improprio non esiste.
\end{example}
Per capire se un integrale improprio esiste o meno, possiamo appellarci ai seguenti risultati:
\begin{theorem}
    \label{th:10.1}
    Siano $f, g\colon[a,+\infty)\to \amsbb{R}$ due funzioni continue (in modo tale da soddisfare le richieste di integrabilità) e tali che esista $M>0$ tale che
    \[
    Mg(x)\ge f(x)\ge 0 \ \text{per ogni} \ x\in[a,+\infty)
    \]
    Allora:
    \begin{enumerate}[(i)]
        \item Se $\int_a^{+\infty}g(x)\, dx$ esiste allora $\int_a^{+\infty}g(x)\, dx$ esiste;
        \item Se $\int_a^{+\infty}f(x)\, dx$ diverge allora $\int_a^{+\infty}g(x)$ diverge.
    \end{enumerate}
\end{theorem}
\begin{theorem}
    \label{th:10.2}
    Siano $f,g\colon[a,+\infty)\to \amsbb{R}$ due funzioni continue tali che $f(x), g(x)\ge 0$ per ogni $x\in[a,+\infty)$ e tali che
    \[
    \lim_{x\to+\infty} \frac{f(x)}{g(x)} = l\in\amsbb{R}\setminus\{0\}
    \]
    Allora
    \[
    \int_a^{+\infty}f(x)\, dx \ \text{e} \ \int_a^{+\infty}g(x)\, dx
    \]
    hanno lo stesso comportamento.
\end{theorem}
Analoghi risultati valgono per nel caso di integrali impropri nel senso della definizione \ref{def:10.2}.
\subsection{Esercizi: integrazione di Riemann impropria}
\begin{exercise}
    \label{ex:10.1}
    Determinare se
    \[
    \int_0^1 \frac{1}{\sqrt{1-x^4}}\, dx
    \]
    esiste.
\end{exercise}
\begin{proof}[Soluzione]
    Notiamo che la funzione integranda $f\colon[0,1)\to\amsbb{R}$ può essere scritta come
    \[
    f(x) = \frac{1}{\sqrt{(1-x)(1+x)(1+x^2)}}
    \]
    e che $\lim_{x\to 1^-} f(x) = +\infty$. Pertanto l'integrale che stiamo considerando è un integrale di Riemann improprio; per determinare se questo esiste, conviene usare uno dei teoremi \ref{th:10.1} e \ref{th:10.2}.\\
    Notiamo che per la definizione \ref{def:10.2} vale che
    \[
    \begin{split}
        \int_0^1 \frac{1}{\sqrt{1-x^4}}\, dx &= \lim_{t\to1^-} \int_0^t \frac{1}{\sqrt{(1-x)(1+x)(1+x^2)}}\, dx \overset{\text{Teorema \ref{th:8.4}}}{=} \\
        &=\lim_{y\to0^+} -\int_1^y \frac{1}{\sqrt{y}\sqrt{2+y}\sqrt{1+(1+y)^2}}\, dy = \\
        & = \lim_{y\to 0^+}\int_y^1 \frac{1}{\sqrt{y}\sqrt{2+y}\sqrt{1+(1+y)^2}}\, dy
    \end{split}
    \]
    A questo punto possiamo provare a sviluppare con Taylor (cfr. teorema \ref{th:6.5}) parte del denominatore per provare a trovare una funzione $g(y)$ che soddisfi il teorema \ref{th:10.2}:
    \[
    \frac{1}{\sqrt{2+y}} = \frac{1}{\sqrt{2}}-\frac{1}{4\sqrt{2}}y +o(y) \qquad \frac{1}{\sqrt{1+(1+y)^2}} = \frac{1}{\sqrt{2}}-\frac{1}{2\sqrt{2}}y +o(y)
    \]
    La funzione integranda può essere quindi sviluppata come
    \[
    \begin{split}
        & \frac{1}{\sqrt{y}}\left(\frac{1}{\sqrt{2}}-\frac{1}{4\sqrt{2}}y+o(y)\right)\left(\frac{1}{\sqrt{2}}-\frac{1}{2\sqrt{2}}y+o(y)\right) = \\
        & = \frac{1}{\sqrt{y}}\left(\frac{1}{2}-\frac{3}{8}y+o(y)\right) = \frac{1}{2\sqrt{y}}-\frac{3}{8}\sqrt{y}+o(\sqrt{y})
    \end{split}
    \]
    Se consideriamo $g(y) = \frac{1}{2\sqrt{y}}$ abbiamo che
    \[
    \lim_{y\to 0^+} \frac{f(y)}{g(y)} = \lim_{y\to 0^+} (\frac{1}{2\sqrt{y}}-\frac{3}{8}\sqrt{y}+o(\sqrt{y}))2\sqrt{y} = \lim_{y\to 0^+} 1 -\frac{3}{4}y+o(y) = 1
    \]
    Notiamo che $f(y)$ e $g(y)$ sono continue in $[0,1)$ e positive; quindi per il teorema \ref{th:10.2} vale che
    \[
    \int_0^1 \frac{1}{\sqrt{1-x^4}}\, dx \ \text{e} \ \int_0^1 \frac{1}{2\sqrt{y}}\, dy
    \]
    hanno lo stesso comportamento. Sappiamo che $\int_0^1 \frac{1}{x^\alpha}\, dx$ esiste se $\alpha<1$; pertanto 
    \[
    \int_0^1 \frac{1}{\sqrt{1-x^4}}\, dx
    \]
    esiste.
\end{proof}
\begin{exercise}
    \label{ex:10.2}
    Determinare i valori di $\alpha, \beta>0$ tali che
    \[
    \int_0^\pi \frac{\abs{\cos(x)}^{2\alpha}}{\sin^\beta(x)}\, dx
    \]
    esiste.
\end{exercise}
\begin{proof}[Soluzione]
    Notiamo che in questo caso i punti problematici sono due: infatti, per ogni scelta di $\alpha, \beta>0$ vale che
    \[
    \lim_{x\to 0^+}\frac{\abs{\cos(x)}^{2\alpha}}{\sin^\beta(x)} = +\infty \qquad \lim_{x\to \pi^-} \frac{\abs{\cos(x)}^{2\alpha}}{\sin^\beta(x)} = +\infty
    \]
    Poiché sappiamo trattare funzioni con eventualmente un solo punto problematico, possiamo appellarci al punto (i) della seconda parte del corollario \ref{cor:8.1} e al teorema \ref{th:4.2} per scrivere
    \[
    \int_0^\pi \frac{\abs{\cos(x)}^{2\alpha}}{\sin^\beta(x)}\, dx = \int_0^{\frac{\pi}{2}}\frac{\abs{\cos(x)}^{2\alpha}}{\sin^\beta(x)}\, dx + \int_{\frac{\pi}{2}}^\pi \frac{\abs{\cos(x)}^{2\alpha}}{\sin^\beta(x)}\, dx
    \]
    Notiamo che, date le ipotesi del teorema \ref{th:4.2} questa scrittura è possibile solamente se esistono separatamente
    \[
    \underbrace{\lim_{y\to 0^+} \int_y^{\frac{\pi}{2}} \frac{\abs{\cos(x)}^{2\alpha}}{\sin^\beta(x)}\, dx}_{\stepcounter{equation}\mbox{(\theequation)}} \qquad \underbrace{\lim_{z\to \pi^-}\int_{\frac{\pi}{2}}^0 \frac{\abs{\cos(x)}^{2\alpha}}{\sin^\beta(x)}\, dx}_{\stepcounter{equation}\mbox{(\theequation)}}
    \]
    \addtocounter{equation}{-2}\refstepcounter{equation}\label{eq:10.1}
    \addtocounter{equation}{0}\refstepcounter{equation}\label{eq:10.2}
    Consideriamo quindi i due limiti:
    \begin{enumerate}[(i)]
        \item per quanto riguarda (\ref{eq:10.1}), operiamo come nella soluzione dell'esercizio \ref{ex:10.1}, ossia sviluppiamo con Taylor la funzione integranda in un intorno di $x_0=0$, notando che $\cos(x)\ge 0$ per ogni $x\in\left[0,\frac{\pi}{2}\right]$:
        \[
        \abs{\cos(x)}^{2\alpha} = (\cos(x))^{2\alpha} = (1+o(x))^{2\alpha} \qquad \sin^\beta(x) = \left(x + o(x)\right)^\beta = x^\beta\left(1+\frac{o(x)}{x}\right)^\beta
        \]
        Quindi abbiamo che
        \[
        \frac{\abs{\cos(x)}^{2\alpha}}{\sin^\beta(x)} = \frac{(1+o(x))^{2\alpha}}{x^\beta\left(1+\frac{o(x)}{x}\right)^\beta}
        \]
        Se consideriamo $g(x) = \frac{1}{x^\beta}$ vale che
        \[
        \lim_{x\to 0^+}\frac{f(x)}{g(x)} = \lim_{x\to 0^+} \frac{(1+o(x))^{2\alpha}}{x^\beta\left(1+\frac{o(x)}{x}\right)^\beta} x^\beta = \lim_{x\to 0^+} \frac{(1+o(x))^{2\alpha}}{\left(1+\frac{o(x)}{x}\right)^\beta} = 1
        \]
        Entrambe le funzioni sono continue e positive su $\left(0, \frac{\pi}{2}\right]$, e per il teorema \ref{th:10.2} abbiamo che
        \[
        \int_0^{\frac{\pi}{2}} \frac{\abs{\cos(x)}^{2\alpha}}{\sin^\beta(x)}\, dx \ \text{e} \ \int_0^{\frac{\pi}{2}}\frac{1}{x^\beta}\, dx
        \]
        hanno lo stesso comportamento. Come prima, sappiamo che $\int_0^{\frac{\pi}{2}} \frac{1}{x^\beta}\, dx$ esiste se $\beta<1$.
        \item Consideriamo ora il limite (\ref{eq:10.2}): notiamo innanzitutto che $\cos(x)\le0$ per ogni $x\in\left[\frac{\pi}{2}, \pi\right]$ e quindi
        \[
        \abs{\cos(x)}^{2\alpha} = (-\cos(x))^{2\alpha}
        \]
        Procedendo come prima e sviluppando numeratore e denominatore della funzione integranda in un intorno di $x_0=\pi$ abbiamo
        \[
        (-\cos(x))^{2\alpha} = (1+o(x-\pi))^{2\alpha}
        \]
        e
        \[
        \sin^\beta(x) = \left(-(x-\pi)+o((x-\pi)) \right)^\beta = (\pi-x)^\beta\left(1+\frac{o(\pi-x)}{\pi-x}\right)^\beta
        \]
        e se consideriamo $g(x) = \frac{1}{(\pi-x)^\beta}$ vale che
        \[
        \lim_{x\to \pi^-} \frac{f(x)}{g(x)} = \lim_{x\to \pi^-} \frac{(1+o(x-\pi))^{2\alpha}}{(\pi-x)^\beta \left(1+\frac{o(\pi-x)}{\pi-x}\right)^\beta} (\pi-x)^\beta = \lim_{x\to \pi^-} \frac{(1+o(x-\pi))^{2\alpha}}{\left(1+\frac{o(\pi-x)}{\pi-x}\right)^\beta} = 1 
        \]
        con $g(x), f(x)$ continue e positive in $\left[\frac{\pi}{2}, \pi\right)$. Quindi per il teorema \ref{th:10.2}
        \[
        \int_{\frac{\pi}{2}}^\pi \frac{\abs{\cos(x)}^{2\alpha}}{\sin^\beta(x)}\, dx \ \text{e} \ \int_{\frac{\pi}{2}}^\pi \frac{1}{(\pi-x)^\beta}\, dx 
        \]
        hanno lo stesso comportamento. Notiamo che
        \[
        \int_{\frac{\pi}{2}}^\pi \frac{1}{(\pi-x)^\beta}\, dx = \lim_{z\to \pi^-}\int_{\frac{\pi}{2}}^\pi \frac{1}{(\pi-x)^\beta}\, dx \overset{\text{Teorema \ref{th:8.4}}}{=} \lim_{y\to 0^+}\int_0^{\frac{\pi}{2}} \frac{1}{x^\beta}\, dx
        \]
        che esiste finito per $\beta<1$.
    \end{enumerate}
    Quindi entrambi i limiti (\ref{eq:10.1}) e (\ref{eq:10.2}) esistono finiti se $0<\beta<1$ e per ogni $\alpha>0$; quindi l'integrale
    \[
    \int_0^\pi \frac{\abs{\cos(x)}^{2\alpha}}{\sin^\beta(x)}\, dx
    \]
    esiste se $\alpha>0$, $0<\beta<1$.
\end{proof}
\begin{exercise}
    \label{ex:10.3}
    Calcolare, se esiste,
    \[
    \int_0^{+\infty}\frac{1}{e^x+e^{-x}}\, dx
    \]
\end{exercise}
\begin{proof}[Soluzione]
    Notiamo che $e^x+e^{-x}>0$ per ogni $x\in\amsbb{R}$; pertanto l'unica criticità dell'integrale improprio è il dominio di integrazione illimitato. Procediamo quindi secondo la definizione \ref{def:10.1}, ossia
    \[
    \int_0^{+\infty} \frac{1}{e^x+e^{-x}}\, dx = \lim_{t\to +\infty} \int_0^t \frac{1}{e^x + e^{-x}}\, dx 
    \]
    Notiamo che
    \[
    \int_0^t \frac{1}{e^x+e^{-x}}\, dx = \int_0^{t}\frac{e^x}{e^{2x}+1}\, dx = \int_0^t \frac{1}{(e^x)^2+1}e^x\, dx \overset{\text{Teorema \ref{th:8.4}}}{=} \int_1^{e^t} \frac{1}{y^2+1}\, dy
    \]
    Quindi
    \[
    \begin{split}
        \lim_{t\to +\infty} \int_0^t \frac{1}{e^x+e^{-x}}\, dx &= \lim_{t\to +\infty} \int_1^{e^t} \frac{1}{y^2+1}\, dy = \lim_{t\to +\infty} \left(\arctan(e^t)-\arctan(1)\right) = \\
        & = \frac{\pi}{2}-\frac{\pi}{4} = \frac{\pi}{4}
    \end{split}
    \]
\end{proof}
\begin{exercise}
    \label{ex:10.4}
    Dato $\alpha\in\amsbb{R}$, si consideri
    \[
    I_\alpha = \int_{-\infty}^{+\infty} \abs{x}^{-\alpha}\arctan\left(\frac{x}{x^4+1}\right)\, dx
    \]
    \begin{enumerate}[(i)]
        \item Determinare per quali valori di $\alpha\in\amsbb{R}$ l'integrale improprio esiste;
        \item per questi $\alpha$, calcolare $I_\alpha$.
    \end{enumerate}
\end{exercise}
\begin{proof}[Soluzione]
    Innanzitutto, detta $f(x)$ la funzione 
    \[
    f(x) = \abs{x}^{-\alpha}\arctan\left(\frac{x}{x^4+1}\right)
    \]
    notiamo che per $\alpha\le 0$ la funzione è ben definita su tutto $\amsbb{R}$, mentre per $\alpha>0$ la funzione è definita su $\amsbb{R}\setminus\{0\}$. In entrambi i casi, notiamo che 
    \[
    f(-x) = \abs{-x}^{-\alpha}\arctan\left(\frac{-x}{(-x)^4+1}\right) = -\abs{x}^{-\alpha} \arctan\left(\frac{x}{x^4+1}\right) = -f(x)
    \]
    ossia $f$ è dispari sul proprio dominio di definizione. Consideriamo quindi $I_\alpha$: questo integrale improprio consiste in realtà, come nel caso dell'esercizio \ref{ex:10.2}, nella somma di 4 integrali impropri ``di base'', ossia
    \[
    I_\alpha =  \underbrace{\int_{-\infty}^{-1} f(x)\, dx}_{\stepcounter{equation}\mbox{(\theequation)}} + \underbrace{\int_{-1}^0 f(x)\, dx}_{\stepcounter{equation}\mbox{(\theequation)}} + \underbrace{\int_0^1 f(x)\, dx}_{\stepcounter{equation}\mbox{(\theequation)}} + \underbrace{\int_1^{+\infty} f(x)\, dx}_{\stepcounter{equation}\mbox{(\theequation)}}
    \]
    \addtocounter{equation}{-4}\refstepcounter{equation}\label{eq:10.3}
    \addtocounter{equation}{0}\refstepcounter{equation}\label{eq:10.4}
    \addtocounter{equation}{0}\refstepcounter{equation}\label{eq:10.5}
    \addtocounter{equation}{0}\refstepcounter{equation}\label{eq:10.6}
    Quindi affinché $I_\alpha$ esista devono esistere questi singolarmente questi 4 integrali. Consideriamo ora i fatti seguenti:
    \begin{enumerate}[(i)]
        \item se consideriamo gli integrali (\ref{eq:10.3}) e (\ref{eq:10.6}), che esplicitamente sono rispettivamente
        \[
        \lim_{y\to-\infty} \int_y^1 f(x)\, dx \ \text{e} \ \lim_{y\to +\infty} \int_1^y f(x)\, dx
        \]
        notiamo che
        \[
        \begin{split}
            \lim_{y\to-\infty} \int_{y}^{-1} f(x)\, dx &= \lim_{y\to+\infty} \int_{-y}^{-(1)} f(x)\, dx \overset{\text{Teorema \ref{th:8.4}}}{=} \lim_{y\to +\infty} \int_y^{1} -f(-x)\, dx = \\
            & = \lim_{y\to+\infty} \int_1^y f(-x)\, dx = -\lim_{y\to+\infty} \int_1^y f(x)\, dx
        \end{split}
        \]
        Quindi se esiste (\ref{eq:10.6}) automaticamente esiste (\ref{eq:10.3}). Possiamo quindi studiare unicamente il limite (\ref{eq:10.6}). Un analogo risultato vale per i limiti (\ref{eq:10.4}) e (\ref{eq:10.5}).
        \item In virtù del punto (i), notiamo che se (\ref{eq:10.5}) e (\ref{eq:10.6}) esistono allora $I_\alpha$ esiste e vale che
        \[
        I_\alpha = 0
        \]
        Di conseguenza quando $I_\alpha$ esiste, questo vale 0.
    \end{enumerate}
    Determiniamo ora i valori di $\alpha\in\amsbb{R}$ tali che (\ref{eq:10.5}) e (\ref{eq:10.6}) esistono finiti.
    \begin{enumerate}[(i)]
        \item Consideriamo (\ref{eq:10.5}), ossia
        \[
        \lim_{y\to 0^+} \int_y^1 f(x)\, dx 
        \]
        Per determinare se esiste, procediamo come nel caso degli esercizi \ref{ex:10.1} e \ref{ex:10.2}, ossia proviamo ad applicare il teorema \ref{th:10.2} sfruttando l'espansione di Taylor di $f(x)$ in un intorno di $x_0=0$:
        \[
        \arctan\left(\frac{x}{x^4+1}\right) = x+o(x) = x\left(1+\frac{o(x)}{x}\right)
        \]
        quindi
        \[
        f(x) = \abs{x}^{-\alpha} \arctan\left(\frac{x}{x^4+1}\right) = ({x})^{-\alpha} x \left(1+\frac{o(x)}{x}\right) = \frac{1}{x^{\alpha-1}} \left(1+\frac{o(x)}{x}\right) 
        \]
        Di conseguenza, se consideriamo $g(x) = \frac{1}{x^{\alpha-1}}$ abbiamo che
        \[
        \lim_{x\to 0^+} \frac{f(x)}{g(x)} = \lim_{x\to 0^+}\frac{1}{x^{\alpha-1}}\left(1+\frac{o(x)}{x}\right)x^{\alpha-1} = \lim_{x\to 0^+} \left(1+\frac{o(x)}{x}\right) = 1
        \]
        e poiché $f$ e $g$ sono entrambe continue e positive in $(0,1]$ abbiamo che
        \[
        \int_0^1 f(x)\, dx \ \text{e} \ \int_0^1 \frac{1}{x^{\alpha-1}}\, dx
        \]
        hanno lo stesso comportamento alla luce del teorema \ref{th:10.2}. Poiché sappiamo che $\int_0^1 \frac{1}{x^{\alpha-1}}\, dx$ esiste se $\alpha<2$, lo stesso vale per (\ref{eq:10.5}).
        \item Se consideriamo (\ref{eq:10.6}), 
        \[
        \lim_{y\to +\infty} \int_1^y f(x)\, dx 
        \]
        per operare come prima bisogna stimare il comportamento di $\arctan\left(\frac{x}{x^4+1}\right)$ per grandi $x$; chiaramente 
        \[
        \lim_{x\to+\infty} \arctan\left(\frac{x}{x^4+1}\right)=0
        \]
        però per decidere se l'integrale esiste o meno dobbiamo stimare la velocità di decadimento della funzione. Per fare questo dobbiamo fare uno ``sviluppo all'$\infty$'' della funzione. A questo scopo, consideriamo i seguenti passaggi:
        \[
        \lim_{x\to +\infty} \arctan\left(\frac{x}{x^4+1}\right) = \lim_{x\to +\infty} \arctan\left(\frac{x}{x^4\left(1+\frac{1}{x^4}\right)}\right ) \overset{y=\frac{1}{x}}{=} \lim_{y\to 0^+} \arctan\left(\frac{y^3}{1+y^4}\right)
        \]
        Quindi il comportamento all'infinito di $\arctan\left(\frac{x}{x^4+1}\right)$ è analogo al comportamento in un intorno di $0$ di $\arctan\left(\frac{y^3}{y^4+1}\right)$. Consideriamone quindi lo sviluppo di Taylor in un intorno di $x_0=0$:
        \[
        \arctan\left(\frac{y^3}{y^4+1}\right) = y^3 + o(y^3)
        \]
        e di conseguenza 
        \[
        \arctan\left(\frac{x}{x^4+1}\right) = \frac{1}{x^3}+o\left(\frac{1}{x^3}\right) \ \text{per} \ x\to +\infty
        \]
        Quindi
        \[
        f(x) = \abs{x}^{-\alpha}\frac{1}{x^3}\left(1+x^3o\left(\frac{1}{x^3}\right)\right) = \frac{1}{x^{3+\alpha}}\left(1+x^3o\left(\frac{1}{x^3}\right)\right)
        \]
        e se definiamo $g(x) = \frac{1}{x^{3+\alpha}}$ abbiamo che
        \[
        \lim_{x\to +\infty} \frac{f(x)}{g(x)} = \lim_{x\to+\infty}\frac{1}{x^{3+\alpha}}\left(1+x^3o\left(\frac{1}{x^3}\right)\right) x^{3+\alpha} = \lim_{x\to+\infty} \left(1+x^3o\left(\frac{1}{x^3}\right)\right) = 1
        \]
        Dato che $f$ e $g$ sono continue e positive su $[1, +\infty)$, per il teorema \ref{th:10.2} abbiamo che
        \[
        \int_1^{+\infty} f(x)\, dx \ \text{e} \ \int_1^{+\infty} \frac{1}{x^{\alpha+3}}\, dx
        \]
        hanno lo stesso comportamento. Dato che $\int_1^{+\infty}\frac{1}{x^{\alpha+3}}\, dx$ esiste se $\alpha>-2$, lo stesso vale per (\ref{eq:10.6}).
    \end{enumerate}
    Possiamo quindi concludere dicendo che $I_\alpha$ esiste per $\alpha\in(-2,2)$ e che per $\alpha$ in questo intervallo $I_\alpha = 0$.
\end{proof}
\begin{exercise}
    \label{ex:10.5}
    Determinare per quali $\alpha, \beta\in\amsbb{R}$ l'integrale
    \[
    \int_1^{+\infty} \frac{\log(x)e^{-x^2}}{(x+1)^\alpha(x-1)^\beta}\, dx
    \]
    esiste.
\end{exercise}
\begin{proof}[Soluzione]
    Come nel caso dell'esercizio \ref{ex:10.4}, i problemi di questo integrale improprio sono due: la singolarità della funzione integranda
    \[
    f(x) = \frac{\log(x)e^{-x^2}}{(x+1)^{\alpha}(x-1)^\beta}
    \]
    in $x=1$ e l'illimitatezza del dominio di integrazione. \`E quindi necessario considerare separatamente i due casi
    \[
    I = \underbrace{\int_1^e f(x)\, dx}_{\stepcounter{equation}\mbox{(\theequation)}} + \underbrace{\int_e^{+\infty}f(x)\, dx}_{\stepcounter{equation}\mbox{(\theequation)}}
    \]
    \addtocounter{equation}{-2}\refstepcounter{equation}\label{eq:10.7}
    \addtocounter{equation}{0}\refstepcounter{equation}\label{eq:10.8}
    \begin{enumerate}[(i)]
        \item Consideriamo l'integrale improprio (\ref{eq:10.7}), che possiamo scrivere come
        \[
        \lim_{y\to 1^+} \int_y^e f(x)\, dx
        \]
        Per poter applicare il teorema \ref{th:10.2} dobbiamo, come nei casi precedenti, determinare il comportamento in un intorno di $x_0=0$ di $f(x)$ sfruttando l'espansione in serie di Taylor:
        \[
        \log(x)=\log(1+(x-1))= (x-1)+o(x-1) \qquad e^{-x^2} = \frac{1}{e} -\frac{2}{e}(x-1) +o(x-1)
        \]
        e
        \[
        \frac{1}{(x+1)^\alpha} = \frac{1}{2^\alpha } -\frac{\alpha}{2^{\alpha+1}}(x-1) +o(x-1)
        \]
        Quindi
        \[
        \begin{split}
            f(x) & = \frac{(x-1)^{1-\beta}}{2^{\alpha}e}\left(1+\frac{o(x-1)}{x-1}\right)\left(1-2(x-1)+o(x-1)\right)\times \\
            & \times \left(1-\frac{\alpha}{2}(x-1)+o(x-1)\right)
        \end{split}
        \]
        e definendo $g(x) = \frac{(x-1)^{1-\beta}}{2^{\alpha}e}$ abbiamo che
        \[
        \lim_{x\to 1^+} \frac{f(x)}{g(x)} = 1
        \]
        Dato che $f$ e $g$ sono positive e continue in $(1, e]$ per il teorema \ref{th:10.2} 
        \[
        \int_1^e \frac{\log(x)e^{-x^2}}{(x+1)^\alpha(x-1)^\beta}\, dx \ \text{e} \ \int_1^e \frac{(x-1)^{1-\beta}}{2^{\alpha}e}\, dx
        \]
        hanno lo stesso comportamento; effettuando passaggi simili a quelli dell'esercizio \ref{ex:10.2} possiamo quindi dire che (\ref{eq:10.7}) esiste se $\beta<2$.
        \item Per quanto riguarda l'integrale (\ref{eq:10.8}), notiamo innanzitutto che 
        \begin{equation}
            \label{eq:10.9}
            \frac{\log(x)e^{-x^2}}{(x+1)^\alpha(x-1)^\beta} \le \frac{xe^{-x^2}}{(x+1)^\alpha(x-1)^\beta} \ \text{per ogni} \ x\in [e,+\infty)
        \end{equation}
        e, se consideriamo $g(x) = x^{1-\beta-\alpha}e^{-x^2}$, abbiamo che
        \[
        \lim_{x\to +\infty} \frac{xe^{-x^2}}{(x+1)^\alpha (x-1)^\beta} \frac{1}{g(x)} = \lim_{x\to +\infty} \frac{x^\alpha}{(x+1)^\alpha} \frac{x^\beta}{(x-1)^\beta} = 1
        \]
        e di conseguenza $\frac{xe^{-x^2}}{(x+1)^\alpha (x-1)^\beta}$ e $g$ hanno lo stesso comportamento per il teorema \ref{th:10.2}. Determiniamo quindi per quali valori di $\alpha, \beta\in\amsbb{R}$ l'integrale
        \[
        \int_e^{+\infty} x^{1-\alpha-\beta} e^{-x^2}
        \]
        esiste. A questo scopo, consideriamo separatamente i casi $1-\alpha-\beta\le 0$ e $1-\alpha-\beta>0$. Nel caso $1-\alpha-\beta=0$ il risultato segue dal fatto che l'integrale $\int_0^{+\infty} e^{-x^2}\, dx$ esiste finito, mentre nel caso $1-\alpha-\beta<0$ possiamo ragionare come segue: la funzione
        \[
        [e,+\infty)\ni x \mapsto x^{1-\alpha-\beta}
        \]
        è monotona decrescente, e quindi
        \[
        x^{1-\alpha-\beta}e^{-x^2}\le e^{1-\alpha-\beta}e^{-x^2} \ \forall x\in[e, +\infty)
        \]
        Dato che $\int_0^{+\infty}e^{-x^2}\, dx$ esiste, grazie al teorema \ref{th:10.1} abbiamo che
        \[
        \int_e^{+\infty} x^{1-\alpha-\beta} e^{-x^2}\, dx
        \]
        esiste finito; per il teorema \ref{th:10.2} segue che
        \[
        \int_e^{+\infty} \frac{xe^{-x^2}}{(x+1)^\alpha(x-1)^\beta}\, dx
        \]
        esiste e quindi grazie al teorema \ref{th:10.1} e alla stima (\ref{eq:10.9}) possiamo concludere che
        \[
        \int_e^{+\infty}\frac{\log(x)e^{-x^2}}{(x+1)^\alpha(x-1)^\beta}\, dx
        \]
        esiste finito, senza vincoli su $\alpha,\beta\in\amsbb{R}$.\\
        Se invece $1-\alpha-\beta>0$, bisogna operare diversamente. Vogliamo mostrare che
        \[
        \int_e^{+\infty}x^{1-\alpha-\beta} e^{-x^2}\, dx 
        \]
        esiste. A questo scopo, restringiamoci inizialmente al caso $1-\alpha-\beta\in\amsbb{N}$ e procediamo per induzione:
        \begin{enumerate}[(a)]
            \item \emph{caso base $n=1$}: vale che
            \[
            \begin{split}
                \int_e^{+\infty} x e^{-x^2}\, dx & = \lim_{y\to +\infty} \int_e^y x e^{-x^2}\, dx = \lim_{y\to +\infty} -\frac{1}{2}\int_e^y e^{-x^2} \frac{d}{dx}(-x^2)\, dx \overset{\text{Teorema \ref{th:8.4}}}{=} \\
                & = \lim_{y\to+\infty} -\frac{1}{2}\int_{-e^2}^{-t^2} e^z\, dz = \frac{1}{2}e^{-e^2} - \lim_{t\to +\infty} \frac{e^{-t^2}}{2} = \frac{1}{2}e^{-e^{2}}
            \end{split} 
            \]
            \item \emph{passo induttivo}: supponiamo che valga per $n\ge 1$, e mostriamo che vale per $n+1$. Consideriamo quindi
            \[
            \begin{split}
                & \int_e^{+\infty} x^{n+1}e^{-x^2}\, dx = \lim_{y\to +\infty} -\frac{1}{2} \int_e^y x^n \frac{d}{dx}(e^{-x^2})\, dx \overset{\text{Teorema \ref{th:8.5}}}{=} \\
                & = \frac{1}{2}e^n e^{-e^2}-\lim_{y\to +\infty}\left(\frac{1}{2}y^ne^{-y^2}-\frac{n}{2}\int_e^{y} x^{n-1} e^{-x^2}\, dx\right)
            \end{split} 
            \]
            Notiamo che 
            \[
            \lim_{y\to+\infty} \frac{1}{2}y^n e^{-y^2} = 0 \ \text{per ogni} \ n\in\amsbb{N}
            \]
            e che
            \[
            \lim_{y\to+\infty}-\frac{n}{2}\int_e^y x^{n-1}e^{-x^2}\, dx  
            \]
            esiste finito per l'ipotesi induttiva\footnote{$x^n\ge x^{n-1}$ per ogni scelta di $x\in[e,+\infty)$ per $n\ge 1$; di conseguenza l'asserzione segue dal teorema \ref{th:10.1}.}. Quindi possiamo applicare il teorema \ref{th:4.2} e concludere che
            \[
            \int_e^{+\infty} x^{n+1} e^{-x^2}\, dx
            \]
            esiste finito.
        \end{enumerate}
        Quindi 
        \[
        \int_0^{+\infty} x^n e^{-x^2}\, dx 
        \]
        esiste finito per ogni $n\in\amsbb{N}$. Noi vogliamo mostrare che
        \[
        \int_e^{+\infty} x^{1-\alpha-\beta} e^{-x^2}\, dx
        \]
        converge per ogni scelta di $\alpha, \beta$ tali che $1-\alpha-\beta >0$. Ma questo segue dalla proprietà archimedea dei numeri reali e dal teorema \ref{th:10.1}: infatti, per ogni scelta di $\alpha,\beta$ tali che $1-\alpha-\beta>0$, esiste un numero naturale $n\in\amsbb{N}$ tale che
        \[
        n>1-\alpha-\beta
        \]
        e ne consegue che possiamo trovare $n\in\amsbb{N}$ tale che
        \[
        x^n e^{-x^2}\ge x^{1-\alpha-\beta} e^{-x^2} \ \text{per ogni} \ x\in[e,+\infty)
        \]
        Per il teorema \ref{th:10.1} vale quindi che
        \[
        \int_e^{+\infty} x^{1-\alpha-\beta} e^{-x^2}\, dx
        \]
        esiste finito per ogni scelta di $\alpha, \beta\in\amsbb{R}$; il teorema \ref{th:10.2} a questo punto ci dice che 
        \[
        \int_e^{+\infty} \frac{xe^{-x^2}}{(x+1)^\alpha (x-1)^\beta}\, dx
        \]
        esiste finito per ogni scelta di $\alpha, \beta\in\amsbb{R}$. Possiamo quindi concludere, grazie alla stima (\ref{eq:10.9}) e al teorema \ref{th:10.2}, che
        \[
        \int_e^{+\infty} \frac{\log(x)e^{-x^2}}{(x+1)^\alpha (x-1)^\beta}\, dx
        \]
        esiste finito per ogni scelta di $\alpha, \beta\in\amsbb{R}$.
    \end{enumerate}
    Quindi per il punto (i) sappiamo che (\ref{eq:10.7}) esiste se $\beta<2$, mentre per il punto (ii) l'integrale (\ref{eq:10.8}) esiste per ogni scelta di $\alpha,\beta\in\amsbb{R}$. Quindi l'integrale
    \[
    \int_1^{+\infty}  \frac{\log(x)e^{-x^2}}{(x+1)^\alpha (x-1)^\beta}\, dx
    \]
    esiste per ogni $\alpha\in\amsbb{R}$ e $\beta<2$.
\end{proof}
\begin{exercise}
    \label{ex:10.6}
    Determinare per quali valori del parametro $x\in\amsbb{R}$ l'integrale
    \[
    \Gamma(x) = \int_0^{+\infty} t^{x-1} e^{-t}\, dt
    \]
    esiste finito.
\end{exercise}
\begin{proof}[Soluzione]
    Concentriamoci, come fatto nell'esercizio \ref{ex:10.5}, sui naturali, procedendo per induzione:
    \begin{enumerate}[(i)]
        \item \emph{caso base $n=1$}: abbiamo che
        \[
        \Gamma(1) = \int_0^{+\infty} e^{-t}\, dt = \lim_{x\to +\infty} \int_0^x e^{-t}\, dt = 1-\lim_{x\to+\infty}e^{-x} = 1 
        \]
        \item \emph{passo induttivo}: supponiamo che $\Gamma(n)$ sia finito, e mostriamo che anche $\Gamma(n+1)$ lo è. Consideriamo
        \[
        \begin{split}
            \Gamma(n+1) & = \int_0^{+\infty} t^n e^{-t}\, dt = \lim_{x\to +\infty} -\int_0^x t^{n}\frac{d}{dt}(e^{-t})\, dt \overset{\text{Teorema \ref{th:8.5}}}{=} \\
            & = \lim_{x\to+\infty} n\int_0^x t^{n-1} e^{-t}\, dt = n \int_0^{+\infty} t^{n-1}e^{-t}\, dt = n\Gamma(n)
        \end{split}
        \]
        Di conseguenza $\Gamma(n+1)$ esiste finito.
    \end{enumerate}
    Consideriamo ora un generico $x\in\amsbb{R}$, con $x>1$; possiamo scrivere
    \[
    \lim_{y\to +\infty} \int_0^{y} t^{x-1} e^{-t}\, dt = \lim_{y\to+\infty} \int_{0^2}^{(\sqrt{y})^2} t^{x-1}e^{-t}\, dt = \lim_{y\to +\infty} 2\int_0^{\sqrt{y}} u^{2x-2} e^{-u^2} \, du 
    \]
    Nell'esercizio precedente avevamo già mostrato che, nel caso $x>1$, l'ultimo limite della catena di uguaglianze esiste; pertanto
    \[
    \Gamma(x) = \int_0^{+\infty} t^{x-1} e^{-t}\, dt
    \]
    esiste per $x>1$.\\
    Se $x<1$, abbiamo un potenziale problema in $x=0$; dobbiamo pertanto considerare
    \[
    \int_0^{+\infty} t^{x-1} e^{-t}\, dt = \underbrace{\int_0^1 t^{x-1}e^{-t}\, dt}_{\stepcounter{equation}\mbox{(\theequation)}} + \underbrace{\int_1^{+\infty}t^{x-1}e^{-t}\, dt}_{\stepcounter{equation}\mbox{(\theequation)}} 
    \]
    \addtocounter{equation}{-2}\refstepcounter{equation}\label{eq:10.10}
    \addtocounter{equation}{0}\refstepcounter{equation}\label{eq:10.11}
    Seguendo una procedura analoga a quella dell'esercizio \ref{ex:10.5}, è possibile dimostrare che l'integrale (\ref{eq:10.11}) esiste per ogni scelta di $x<1$. Concentriamoci quindi su (\ref{eq:10.10}): possiamo sviluppare con Taylor la funzione integranda per provare ad applicare il teorema \ref{th:10.2}. In particolare abbiamo che
    \[
    e^{-t} = 1-t +o(t)
    \]
    e quindi 
    \[
    \frac{e^{-t}}{t^{1-x}} = \frac{1}{t^{1-x}}(1-t+o(t))
    \]
    Se definiamo $g(t) = \frac{1}{t^{1-x}}$ abbiamo che
    \[
    \lim_{t\to 0^+} \frac{e^{-t}}{t^{1-x}}t^{1-x} = \lim_{t\to0^+} e^{-t} = 1
    \]
    e per il teorema \ref{th:10.2}
    \[
    \int_0^1 \frac{e^{-t}}{t^{1-x}}\, dt \ \text{e} \ \int_0^1 \frac{1}{t^{1-x}}\, dt
    \]
    hanno lo stesso comportamento. Sappiamo che $\int_0^1 \frac{1}{t^{1-x}}\, dt$ esiste se $x>0$, e pertanto anche (\ref{eq:10.10}) esiste se $x>0$. Di conseguenza, $\Gamma(x)$ è definita per $x>0$.
\end{proof}
\begin{remark}
    La funzione $\amsbb{R}^+\ni x \mapsto \Gamma(x)$ è la \emph{funzione Gamma di Eulero}, e come si può osservare da $\Gamma(n+1) = n \Gamma(n)$, questa costituisce una generalizzazione del fattoriale ai numeri reali positivi.
\end{remark}
\newpage
\section{Lezione 11}
\newpage
\section{Lezione 12}
\subsection{Ripasso ed esercizi: equazioni differenziali ordinarie del secondo ordine lineari a coefficienti costanti}
\begin{definition}
    \label{def:12.1}
    Un'equazione differenziale ordinaria del secondo ordine lineare a coefficienti costanti omogenea è un'equazione del tipo
    \[
    au''(t)+bu'(t)+cu(t) = 0 \qquad a,b,c\in\amsbb{R}
    \]
\end{definition}
Per risolvere un'equazione differenziale del tipo presentato nella definizione \ref{def:12.1}, operiamo nel modo seguente: consideriamo il \emph{polinomio caratteristico} associato all'equazione differenziale:
\[
P(\lambda) = a \lambda^2 +b\lambda^1 + c \lambda^0 = a\lambda^2 +b\lambda + c
\]
Poiché $P(\lambda)$ è un polinomio di grado due a coefficienti reali, abbiamo le seguenti casistiche:
\begin{enumerate}[(i)]
    \item $\Delta = b^2-4ac>0$: in questo caso $P(\lambda)$ ammette due radici reali distinte, $\lambda_1, \lambda_2\in\amsbb{R}$. In questo caso la soluzione generale dell'equazione differenziale è data da
    \[
    u(t) = c_1 e^{\lambda_1 t} + c_2 e^{i\lambda_2 t} \qquad c_1, c_2\in\amsbb{R}
    \]
    \item $\Delta = b^2-4ac = 0$: in questo caso $P(\lambda)$ ammette un'unica radice reale $\lambda_1\in\amsbb{R}$ con molteplicità algebrica 2. La soluzione è data da
    \[
    u(t) = (c_1 +c_2 t)e^{\lambda_1 t} \qquad c_1, c_2\in\amsbb{R}
    \]
    \item $\Delta= b^2-4ac<0$: in questo caso $P(\lambda)$ ammette due radici complesse $\omega_1, \omega_2\in\amsbb{C}$ tali che $\omega_2 = \overline{\omega_1}$, ossia
    \[
    \text{Re}(\omega_1) = \text{Re}(\omega_2) \qquad \text{Im}(\omega_1) = -\text{Im}(\omega_2)
    \]
    In questo caso la soluzione sarà data da
    \[
    u(t) = c_1 e^{\omega_1 t} + c_2 e^{\omega_2 t} = e^{\text{Re}(\omega_1)t}\left(c_1 e^{i\text{Im}(\omega_1)t} + c_2 e^{-i \text{Im}(\omega_1)t}\right) \qquad c_1, c_2\in\amsbb{C}
    \]
\end{enumerate}
\begin{remark}
    Le costanti $c_1$ e $c_2$ sono fissate tramite le condizioni iniziali del problema di Cauchy. Notiamo che nell'ultimo caso, l'equazione differenziale ha come soluzione una funzione $u(t)\colon I \to \amsbb{R}$; pertanto le due costanti $c_1, c_2\in\amsbb{C}$ devono essere tali per cui la soluzione avrà valori reali (in particolare sarà una combinazione lineare di $\sin(\text{Im}(\omega_1)t)$ e $\cos(\text{Im}(\omega_1)t)$).
\end{remark}
\begin{example}
    Consideriamo il problema di Cauchy
    \[
    \begin{dcases}
        u''(t)-4u'(t)+13u(t) = 0\\
        u(0) = 0\\
        u'(0)=3
    \end{dcases}
    \]
    Vogliamo determinarne la soluzione. \\
    Come suggerito, consideriamo il polinomio caratteristico 
    \[
    P(\lambda) = \lambda^2-4\lambda +3
    \]
    e calcoliamone il discriminante $\Delta = 16-4\times 13 <0$. Pertanto ci troviamo nel caso (iii): le sue radici saranno due numeri complessi $\omega_1, \omega_2$ dati da
    \[
    \omega_{1,2} = {2\pm \sqrt{4-13}} = 2\pm i3
    \]
    Di conseguenza la soluzione generale dell'equazione differenziale è data da
    \[
    u(t) = e^{2t}\left(c_1 e^{i3t} +c_2 e^{-i3t}\right)
    \]
    Per determinare le costanti $c_1, c_2\in\amsbb{C}$ dobbiamo imporre le condizioni iniziali:
    \[
    \begin{dcases}
        \begin{aligned}
            0 = u(0)&= e^0\left(c_1 e^0 +c_2 e^0\right) = c_1+c_2\\
            3 = u'(0)&= \left(2e^{2t}\left(c_1 e^{i3t}+c_2e^{-i3t}\right)+e^{2t}\left(i3c_1 e^{i3t} -i3c_2 e^{-i3t}\right)\right)\bigg|_{t=0} = \\
            & = 2(c_1+c_2)+i3(c_1-c_2)
        \end{aligned}
    \end{dcases}
    \]
    Dalla prima equazione sappiamo che $c_1+c_2 =0$, e quindi la seconda equazione si riduce a
    \[
    3 = i3(c_1-c_2) \iff 1=i(c_1-c_2) \iff (c_1-c_2) = -i
    \]
    e possiamo quindi riscrivere il sistema come
    \[
    \begin{dcases}
        c_1 + c_2 = 0\\
        c_1 - c_2 = -i 
    \end{dcases} \iff c_1 = -\frac{i}{2} \quad c_2 = \frac{i}{2}
    \]
    La soluzione sarà quindi
    \[
    u(t) = e^{2t}\left(\frac{i}{2}e^{-i3t} -\frac{i}{2}e^{i3t}\right) = e^{2t}\sin(3t)
    \]
\end{example}
\begin{definition}
    \label{def:12.2}
    Un'equazione differenziale ordinaria del secondo ordine lineare a coefficienti costanti non omogenea è un'equazione del tipo
    \[
    au''(t)+bu'(t)+cu(t) = f(t) \qquad a,b,c\in\amsbb{R}, \ f(t)\in\mathscr{C}(I)
    \]
\end{definition}
In questo caso, la soluzione di un'equazione differenziale del tipo indicato dalla definizione \ref{def:12.2} sarà 
\[
u(t) = u_0(t) + u_p(t)
\]
ove $u_0(t)$ è una soluzione dell'equazione differenziale omogenea (cfr. definizione \ref{def:12.1})
\[
au''(t)+bu'(t)+cu(t) = 0
\]
mentre $u_p(t)$ è una soluzione (detta \emph{soluzione particolare}) dell'equazione differenziale non omogenea.
\begin{remark}
    Possiamo trovare un'analogia con l'algebra lineare: sappiamo che dato un operatore lineare $A\colon V \to W$ che agisce sullo spazio vettoriale $V$ a valori nello spazio vettoriale $W$, se $w\in A(V)$ esiste un elemento $v_w\in\ V$ tale che
    \[
    A v_w = w
    \]
    Se $A$ non è iniettivo (ossia se $\ker A \ne \{0\}$) questo elemento $v_w$ non è unico: infatti
    \[
    A(v_w + v_0) = w 
    \]
    per ogni $v_0\in\ker A$.
\end{remark}
\begin{example}
    Dato il problema di Cauchy
    \[
    \begin{dcases}
        u''(t)-2u'(t)-3u(t) = \sin(t)\\
        u(0) = -\frac{1}{2}\\
        u'(0) = \frac{1}{5}
    \end{dcases}
    \]
    determinarne la soluzione sapendo che la soluzione particolare dell'equazione differenziale ha forma
    \[
    u_p(t) = a e^{it}+be^{-it}
    \]
    Dalla discussione successiva alla definizione \ref{def:12.2}, sappiamo che la soluzione $u(t)$ sarà data da
    \[
    u(t) = u_0(t) + u_p(t)
    \]
    \begin{enumerate}[(i)]
        \item Determiniamo per prima cosa la soluzione particolare $u_p(t)$ andando ad inserire la forma proposta nell'equazione differenziale: abbiamo
        \[
        u_p'(t) = iae^{it}-ibe^{-it} \qquad u_p''(t) = -ae^{it} - be^{-it}
        \]
        e quindi
        \[
        \begin{split}
            \sin(t) & = u_p''(t) - 2u'(t)-3u(t) = -ae^{it}-be^{-it} -2(iae^{it}-ibe^{-it})-3ae^{it}-3be^{-it} = \\
            & = e^{it}(-4a-2ia)+e^{-it}(-4b+2ib)
        \end{split}
        \]
        Ricordiamo che $\sin(t) = \frac{1}{2i}e^{it}-\frac{1}{2i}e^{-it}$, e dato che $e^{it}$ e $e^{-it}$ sono linearmente indipendenti su $I$ vale che la precedente equazione è equivalente al sistema
        \[
        \begin{dcases}
            -4a-2ia = \frac{1}{2i}\\
            -4b+2ib = -\frac{1}{2i}
        \end{dcases}  \iff a = \frac{1}{2i}\frac{1}{-4-2i} = \frac{1}{20}+\frac{i}{10}, \ b= -\frac{1}{2i}\frac{1}{-4+2i} = \frac{1}{20}-\frac{i}{10}
        \]
        La soluzione particolare è quindi data da
        \[
        u_p(t) = \frac{1}{20}\left(e^{it}+e^{-it}\right) + \frac{i}{10}\left(e^{it}-e^{-it}\right) = \frac{1}{10}\cos(t) -\frac{1}{5}\sin(t)
        \]
        \item Determiniamo ora la soluzione dell'equazione omogenea
        \[
        u''(t)-2u'(t)-3u(t) = 0
        \]
        Il polinomio caratteristico in questo caso è dato da
        \[
        P(\lambda) = \lambda^2 -2\lambda-3 = (\lambda-3)(\lambda+1)
        \]
        le cui radici sono $\lambda_1 = 3$ e $\lambda_2 = -1$. La soluzione sarà quindi data da
        \[
        u_0(t) = c_1 e^{3t}+c_2 e^{-t}
        \]
    \end{enumerate}
    Di conseguenza la soluzione dell'equazione differenziale è data da
    \[
    u(t) = c_1 e^{3t}+c_2 e^{-t} +\frac{1}{10}\cos(t)-\frac{1}{5}\sin(t)
    \]
    Per determinare la soluzione del problema di Cauchy, dobbiamo imporre le condizioni iniziali:
    \[
    \begin{dcases}
        -\frac{1}{2} = u(0) = c_1 +c_2 +\frac{1}{10}\\
        \frac{1}{5} = u'(0) = 3c_1 -c_2 -\frac{1}{5}
    \end{dcases} \iff \begin{dcases}
        c_1 + c_2 = -\frac{3}{5}\\
        3c_1 -c_2 = \frac{2}{5}
    \end{dcases}
    \]
    le cui soluzioni sono $c_1 = -\frac{1}{20}$ e $c_2 = -\frac{11}{20}$. Quindi la soluzione del problema di Cauchy è
    \[
    u(t) = -\frac{1}{20}e^{3t} -\frac{11}{20}e^{-t}+\frac{1}{10}\cos(t)-\frac{1}{5}\sin(t)
    \]
\end{example}
\subsection{Approfondimento: introduzione all'analisi numerica}
In generale, data un'equazione differenziale del tipo
\[
u'(t) = f(t,u(t))
\]
non è sempre detto sia possibile trovare esplicitamene una soluzione $u\colon I \to \amsbb{R}$ (anzi, in generale è normale \emph{non} avere una soluzione scrivibile in termini di funzioni matematiche elementari). Si è quindi inevitabilmente portati a considerare soluzioni approssimate della soluzione, attraverso i metodi studiati dall'\emph{analisi numerica}. \\
Supponiamo di considerare un problema di Cauchy
\[
\begin{dcases}
    u'(t) = f(t,u(t))\\
    u(t_0) = u_0
\end{dcases}
\]
con $f\colon I \times J \to \amsbb{R}$, $u(I)\subseteq J$ e $I=[t_0, t_f]$. Per approssimare la soluzione dell'equazione differenziale, consideriamo una partizione $\mathscr{P} = \{t_0, \dots, t_n = t_f\}$ di $I$ e definiamo $u_i = u(t_i)$. A questo punto, consideriamo, dato un tempo $t_i$, lo sviluppo di Taylor di $u(t_i+h)$, ove $h$ va inteso come una variabile che assume valori molto piccoli, in un intorno di $h=0$:
\[
u(t_i+h) = u(t_i) + u'(t_i)h + o(h)
\]
Sappiamo che vale l'equazione differenziale
\[
u'(t) = f(t,u(t))
\]
e di conseguenza possiamo scrivere
\[
u(t_i+h) = u(t_i)+ h f(t_i, u(t_i)) +o(h) = u_i +hf(t_i, u_i)+o(h)
\]
Abbiamo quindi trovato un modo di scrivere $u$ in prossimità di $t_i$ in funzione della distanza da $t_i$, del valore di $u_i$ e di $f$ calcolata in $t_i$ e $u_i$, modulo un errore $o(h)$ che viene detto \emph{errore di troncamento locale}. In particolare, se scegliamo una partizione uniforme
\[
\mathscr{P} = \left\{t_0, \dots, t_n\right\} \qquad t_i = t_0+idt, \ dt =\frac{t_f-t_0}{n}
\]
e se consideriamo $h=dt$ abbiamo che
\[
u_i + dt f(t_i, u_i) +o(dt) = u(t_i+dt) = u(t_{i+1}) = u_{i+1}
\]
Di conseguenza possiamo determinare il valore di $u$ al tempo $t_{i+1}$ conoscendo il valore di $u$ al tempo $t_i$, calcolando
\begin{equation}
    \label{eq:12.1}
    u_{i+1} = u_i + dtf(t_i, u_i)
\end{equation}
Il metodo iterativo descritto prende il nome di \emph{metodo di Eulero esplicito} (\emph{forward Euler method}), perché $u_{i+1}$ è descritto in forma esplicita utilizzando quantità note.
\begin{example}
    Consideriamo il problema di Cauchy
    \[
    \begin{dcases}
        u'(t) = -au(t)\\
        u(0) = 1
    \end{dcases}
    \]
    con $a>0$. La sua soluzione esatta è data da $u_e(t) = e^{-at}$. Se consideriamo un intervallo $[0, t_f]$ partizionato in modo uniforme in $\{t_0, \dots, t_n\}$ secondo il passo $dt$ e applichiamo il metodo di Eulero esplicito (\ref{eq:12.1}) per ottenere una soluzione approssimata otteniamo
    \[
    u_{i+1} = u_i-dt(au_i) = u_i(1-adt) = u_{i-1}(1-adt)^2 = \dots = u_0(1-adt)^{i+1} = (1-adt)^{i+1}
    \]
    Di seguito riportiamo per diversi valori di $dt$ la soluzione approssimata per $a=100$.
    \begin{center}
        \begin{tikzpicture}
            \pgfplotsset{
                scale only axis,
                compat = newest,
                axis lines=middle,
                height=7cm,
                width=0.85\textwidth,
                scaled x ticks=false
            }
            \begin{axis}[
                xmax = 0.23,
                ymax = 1.05,
                xtick={0, 0.02,...,0.2},
                ytick={0, 0.1, ..., 1},
                xlabel={$t$},
                ylabel={$u(t)$},
                xlabel style={below},
                ylabel style={left},
                legend cell align=left,
                legend style={at={(axis cs:0.12, 0.7)},anchor=west}, xticklabel style={/pgf/number format/fixed, /pgf/number format/precision=2}]
                \addplot[domain=0:0.2, color=red, samples = 401]{exp(-100*x)};
                \addlegendentry{\(y=e^{-100t}\)}
                \addplot+[
                    mark=square*,
                    mark size=2pt, dashed, black, mark options={solid}]table{Data/f_e_m004000.txt};
                \addlegendentry{$dt = 0.004$}
                \addplot+[
                    mark=triangle*,
                    mark size=2pt, dashed, blue, mark options={solid}]table{Data/f_e_m008000.txt};
                \addlegendentry{$dt = 0.008$}
            \end{axis}
        \end{tikzpicture}
    \end{center}
    Notiamo che otteniamo un buon accordo; in un'ottica di efficienza computazionale possiamo provare a ridurre un po' il passo $dt$, e vedere se l'approssimazione è comunque soddisfacente.
    \begin{center}
        \begin{tikzpicture}
            \pgfplotsset{
                scale only axis,
                compat = newest,
                axis lines=middle,
                height=7cm,
                width=0.85\textwidth,
                scaled y ticks=false,
                scaled x ticks=false
            }
            \begin{axis}[
                xmax = 0.23,
                ymax = 1.05,
                xtick={0, 0.02,...,0.2},
                ytick={-0.6, -0.4, ..., 1},
                xlabel={$t$},
                ylabel={$u(t)$},
                xlabel style={below},
                ylabel style={left},
                legend cell align=left,
                legend style={at={(axis cs:0.12, 0.6)},anchor=west}, yticklabel style={
        /pgf/number format/fixed, /pgf/number format/precision=2}, xticklabel style={
        /pgf/number format/fixed, /pgf/number format/precision=2}]
                \addplot[domain=0:0.2, color=red, samples = 401]{exp(-100*x)};
                \addlegendentry{\(y=e^{-100t}\)}
                \addplot+[
                    mark=square*,
                    mark size=2pt, dashed, black, mark options={solid}]table{Data/f_e_m015000.txt};
                \addlegendentry{$dt = 0.015$}
            \end{axis}
        \end{tikzpicture}
    \end{center}
    Il risultato è ancora più catastrofico se consideriamo $dt$ ancora più grandi:
    \begin{center}
        \begin{tikzpicture}
            \pgfplotsset{
                scale only axis,
                compat = newest,
                axis lines=middle,
                height=7cm,
                width=0.85\textwidth,
                scaled y ticks=false,
                scaled x ticks=false
            }
            \begin{axis}[
                xmax = 0.23,
                ymax = 1.3,
                xtick={0, 0.02,...,0.2},
                ytick={-1.2, -0.9, ..., 1.2},
                xlabel={$t$},
                ylabel={$u(t)$},
                xlabel style={below},
                ylabel style={left},
                legend cell align=left,
                legend style={at={(axis cs:0.18, 0.5)},anchor=west}, yticklabel style={
        /pgf/number format/fixed, /pgf/number format/precision=2}, xticklabel style={
        /pgf/number format/fixed, /pgf/number format/precision=2}]
                \addplot[domain=0:0.2, color=red, samples = 401]{exp(-100*x)};
                \addlegendentry{\(y=e^{-100t}\)}
                \addplot+[
                    mark=square*,
                    mark size=2pt, dashed, black, mark options={solid}]table{Data/f_e_m020100.txt};
                \addlegendentry{$dt = 0.0201$}
            \end{axis}
        \end{tikzpicture}
    \end{center}
    Perché succede? Notiamo che (\ref{eq:12.1}) restituisce
    \[
    u_i = (1-a dt)^i 
    \]
    e di conseguenza, se $(1-adt)<0$, ossia se $dt>\frac{1}{a}$, vale che
    \[
    u_i = (1-adt)^i = (-1)^i(adt-1)^i \ \text{con} \ adt-1>0
    \]
    Quindi vediamo che il comportamento oscillatorio nel primo dei due grafici problematici è dovuto alla particolare scelta di $dt$ operata. \\
    Nel secondo caso, notiamo che dato $dt=0.0201$ abbiamo che
    \[
    (1-adt)<-1
    \]
    In questo caso il problema è quindi duplice:
    \begin{enumerate}[(i)]
        \item il comportamento oscillatorio rimane, dato che $(1-adt)<0$;
        \item abbiamo inoltre un fenomeno di amplificazione degli errori: infatti,
        \[
        u_i = (1-adt)^{i-1}u_1 = (1-adt)^{i-1}\left( 1-adt +o(dt)\right) = (1-adt)^{i} + (1-adt)^{i-1}o(dt)
        \]
        e vediamo quindi che dato che $\abs{1-adt}>1$ l'errore $o(dt)$ viene amplificato ad ogni step.
    \end{enumerate}
    Capiamo quindi che per prevenire il comportamento oscillatorio dobbiamo richiedere
    \[
    (1-adt)>0 \iff dt<\frac{1}{a}
    \]
    e per prevenire l'amplificazione degli errori dobbiamo avere
    \[
    \abs{1-adt}<1  \iff dt<\frac{2}{a}
    \]
    Queste condizioni dicono che il metodo di Eulero esplicito è \emph{condizionatamente stabile}: abbiamo dei requisiti nella discretizzazione del problema che dobbiamo tenere a mente se vogliamo che l'approssimazione sia soddisfacente. Questa è una caratteristica generale dei metodi espliciti.
\end{example}
Esiste un'altra classe di metodi numerici che non presenta questi problemi, i metodi \emph{impliciti}. Consideriamo, come prima, il problema di Cauchy
\[
\begin{dcases}
    u'(t) = f(t,u(t))\\
    u(t_0) = u_0
\end{dcases}
\]
con $f\colon I \times J \to \amsbb{R}$, $u(I)\subseteq J$ e $I=[t_0, t_f]$. Per approssimare la soluzione dell'equazione differenziale, consideriamo al pari del caso precedente una partizione $\mathscr{P} = \{t_0, \dots, t_n = t_f\}$ di $I$ e definiamo $u_i = u(t_i)$. A questo punto, consideriamo i passaggi seguenti:
\[
u(t_i) = u(t_{i+1} - \underbrace{(t_{i+1}-t_{i})}_{h})
\]
e se $h$ è sufficientemente piccolo possiamo scrivere, sviluppando $u(t_{i+1}-h)$ in un intorno di $h=0$,
\[
u(t_i) = u(t_{i+1}-h) = u(t_{i+1})-u'(t_{i+1})h +o(h) = u(t_{i+1})-f(t_{i+1}, u_{i+1})h+o(h)
\]
Quindi anche in questo caso se la partizione è una partizione uniforme
\[
\mathscr{P} = \left\{t_0, \dots, t_n\right\} \qquad t_i = t_0+idt, \ dt =\frac{t_f-t_0}{n}
\]
abbiamo che $h=t_{i+1}-t_i = dt$ e 
\[
u(t_i) = u(t_i+1)-f(t_{i+1}, u_{i+1})dt +o(dt)
\]
e trascurando l'errore abbiamo
\begin{equation}
    \label{eq:12.2}
    u_i = u_{i+1}-f(t_{i+1}, u_{i+1})dt
\end{equation}
Questo metodo viene detto \emph{metodo di Eulero implicito}, poiché $u_{i+1}$ è descritto implicitamente (non conosciamo $u_{i+1}$, e quindi non conosciamo $f(t_{i+1}, u_{i+1})$!) dall'equazione
\[
u_i = u_{i+1}-f(t_{i+1}, u_{i+1})dt
\]
\begin{example}
    Consideriamo il problema di Cauchy dell'esempio precedente,
    \[
    \begin{dcases}
        u'(t) = -au(t)\\
        u(0) = 1
    \end{dcases}
    \]
    con $a>0$. Se consideriamo il medesimo intervallo $[0, t_f]$ partizionato in modo uniforme in $\{t_0, \dots, t_n\}$ secondo il passo $dt$ e applichiamo il metodo di Eulero implicito (\ref{eq:12.1}) otteniamo
    \[
    u_{i} = u_{i+1}-(-au_{i+1})dt = u_{i+1}\left(1+adt\right) \iff u_{i+1} = \frac{u_i}{(1+adt)} = \dots = \frac{1}{(1+adt)^{i+1}} 
    \]
    Riportiamo ora le approssimazioni della soluzione $u(t)$ per diverse scelte di $dt$.
    \begin{center}
    	\begin{tikzpicture}
    		\pgfplotsset{
    			scale only axis,
    			compat = newest,
    			axis lines=middle,
    			height=7cm,
    			width=0.85\textwidth,
    			scaled x ticks=false
    		}
    		\begin{axis}[
    			xmax = 0.23,
    			ymax = 1.05,
    			xtick={0, 0.02,...,0.2},
    			ytick={0, 0.1, ..., 1},
    			xlabel={$t$},
    			ylabel={$u(t)$},
    			xlabel style={below},
    			ylabel style={left},
    			legend cell align=left,
    			legend style={at={(axis cs:0.12, 0.7)},anchor=west}, xticklabel style={
    				/pgf/number format/fixed, /pgf/number format/precision=2}]
    			\addplot[domain=0:0.2, color=red, samples = 401]{exp(-100*x)};
    			\addlegendentry{\(y=e^{-100t}\)}
    			\addplot+[
    			mark=square*,
    			mark size=2pt, dashed, black, mark options={solid}]table{Data/b_e_m004000.txt};
    			\addlegendentry{$dt = 0.004$}
    			\addplot+[
    			mark=triangle*,
    			mark size=2pt, dashed, blue, mark options={solid}]table{Data/b_e_m008000.txt};
    			\addlegendentry{$dt = 0.008$}
    			\addplot+[
    			mark=diamond*,
    			mark size=2pt, dashed, teal, mark options={solid}]table{Data/b_e_m015000.txt};
    			\addlegendentry{$dt = 0.015$}
    			\addplot+[
    			mark=pentagon*,
    			mark size=2pt, dashed, violet, mark options={solid}]table{Data/b_e_m020100.txt};
    			\addlegendentry{$dt = 0.0201$}
    		\end{axis}
    	\end{tikzpicture}
    \end{center}
    Notiamo che in questo caso gli step $dt$ che ci avevano dato problemi nel caso del metodo di Eulero esplicito si comportano bene: infatti,
    \begin{enumerate}[(i)]
        \item non abbiamo più il problema delle oscillazioni: infatti, dato che $dt>0$ e $a>0$, abbiamo che
        \[
        (1+adt)>0 \ \text{per ogni scelta di} \ dt \implies \frac{1}{(1+adt)^i}>0 \ \text{per ogni scelta di} \ dt
        \]
        \item non abbiamo più il problema dell'amplificazione dell'errore: infatti,
        \[
        (1+adt)>1 \ \text{per ogni scelta di} \ dt>0 \implies \frac{1}{1+adt}<1 
        \]
        e di conseguenza non abbiamo più il fenomeno di amplificazione degli errori.
    \end{enumerate}
\end{example}
\begin{remark}
    Osserviamo che in questo caso siamo stati fortunati, poiché siamo riusciti ad ottenere un'espressione esplicita di $u_{i+1}$ a partire dalla (\ref{eq:12.2}). In generale, questo non è possibile, ed è quindi necessario utilizzare un metodo numerico (ad esempio il metodo di Newton) anche per determinare $u_{i+1}$. Per questo i metodo impliciti sono generalmente computazionalmente più intensi; si guadagna però la stabilità incondizionata del metodo.  
\end{remark}
\newpage
\clearpage
\pagestyle{empty}
\phantomsection
\addcontentsline{toc}{section}{Riferimenti bibliografici}
\printbibliography
\nocite{rudin1976principles}
\nocite{lang1999}
\nocite{paganisauce}
\end{document}