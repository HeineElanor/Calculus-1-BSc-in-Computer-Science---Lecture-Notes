\section{Lezione 2}
\subsection{Ripasso: maggioranti, minoranti, estremo superiore e inferiore}
\begin{definition}
    \label{def:2.1}
    Sia $S\subseteq \amsbb{R}$, $S\ne \varnothing$. Diremo che $x\in\amsbb{R}$ è un \emph{maggiorante} di $S$ se
    \[
    x\ge y \ \text{per ogni} \ y\in S
    \]
    Diremo invece che $x\in\amsbb{R}$ è un \emph{minorante} di $S$ se 
    \[
    x\le y \ \text{per ogni} \ y\in S
    \]
    Se $S$ ammette un minorante o un maggiorante, diremo rispettivamente che $S$ è \emph{limitato inferiormente} o \emph{superiormente}.
\end{definition}
\begin{definition}
    \label{def:2.2}
    Sia $S\subseteq \amsbb{R}$, $S\ne \varnothing$ un sottoinsieme limitato superiormente. Se esiste $\alpha\in\amsbb{R}$ tale che
    \begin{enumerate}[(i)]
        \item $\alpha$ è maggiorante di $S$
        \item se $\gamma <\alpha$ allora $\gamma$ \emph{non} è maggiorante di $S$
    \end{enumerate}
    allora $\alpha$ è unico ed è detto \emph{estremo superiore} di $S$, e scriveremo $\alpha = \sup S$.\\
    Analogamente, se $S$ è limitato inferiormente ed esiste $\beta\in\amsbb{R}$ tale che
    \begin{enumerate}[(i)]
        \item $\beta$ è minorante di $S$
        \item se $\gamma>\beta$ allora $\gamma$ \emph{non} è minorante di $S$
    \end{enumerate}
    allora $\beta$ è unico ed è detto \emph{estremo inferiore} di $S$, in notazione $\beta = \inf S$.\\
    Se $\alpha = \sup S \in S$, allora $\alpha$ verrà detto \emph{massimo}, e se $\beta = \inf S \in S$, $\beta$ verrà detto \emph{minimo}.
\end{definition}
\begin{theorem}[Least-upper-bound property]
    \label{th:2.1}
    Se $S\subseteq \amsbb{R}$, $S\ne \varnothing$ è limitato superiormente, allora $S$ ammette un estremo superiore.
\end{theorem}
\begin{corollary}
    \label{cor:2.1}
    Sia $S\subseteq \amsbb{R}$, $S\ne \varnothing$ limitato inferiormente. Se denotiamo con $L$ l'insieme dei minoranti di $L$, avremo che
    \[
    \beta = \sup L
    \]
    esiste e $\beta = \inf S$.
\end{corollary}
\begin{proof}
    Per ipotesi, sappiamo che $S$ è limitato inferiormente, e di conseguenza $S$ ammette almeno un minorante; abbiamo quindi che $L\ne \varnothing$. Inoltre, fissato un qualsiasi $y\in S$, 
    \[
    x\le y \ \text{per ogni} \ x\in L
    \]
    poiché $L$ è l'insieme dei minoranti di $S$ (cfr. definizione \ref{def:2.1}). Di conseguenza $y$ è un maggiorante di $L$ e $L$ è limitato superiormente, e per il teorema \ref{th:2.1} ammette un estremo superiore; sia $\beta = \sup L$.\\
    Vogliamo ora mostrare che $\beta$ è l'estremo inferiore di $S$, ossia che $\beta$ soddisfa i punti (i) e (ii) della seconda parte della  definizione \ref{def:2.2}. 
    \begin{enumerate}[(i)]
        \item Consideriamo la definizione \ref{def:2.2} di estremo superiore; per il punto (ii) vale che se $\gamma <\beta$ allora $\gamma$ non è maggiorante di $L$. Abbiamo notato prima che se $y\in S$, allora $y$ è un maggiorante di $L$, scritto in simboli
        \[
        \gamma\in S \implies \gamma \ \text{è un maggiorante di}\ L
        \]
        Questa scrittura è equivalente\footnote{In logica proposizionale $A\implies B$ è equivalente a $\neg B \implies \neg A$.} a
        \[
        \gamma\  \text{non è un maggiorante di} \ L \implies \gamma \notin S
        \]
        Di conseguenza, 
        \[
        \gamma < \beta \implies \gamma \ \text{non è un maggiornate di} \ L \implies \gamma \notin S
        \]
        che è equivalente a 
        \[
        \gamma \in S \implies \gamma \ge \beta 
        \]
        ossia $\beta$ è un minorante di $S$.
        \item Consideriamo ora $\gamma>\beta$; vogliamo mostrare che $\gamma$ non è un minorante di $S$. Ricordiamo che $\beta = \sup L$; di conseguenza se $\gamma>\beta$, allora
        \[
        \gamma > x \ \text{per ogni} \ x\in L
        \]
        In particolare ne consegue che $\gamma\notin L$; ma ricordiamo che $L$ è l'insieme dei minoranti di $S$, e quindi $\gamma$ non è un minorante di $S$. \qedhere
    \end{enumerate}
\end{proof}
\subsection{Esercizi}
\begin{exercise}
    \label{ex:2.1}
    Determinare $\sup$ e $\inf$, ed eventualmente massimo e minimo, dell'insieme
    \[
    A = \left\{ a_n = \frac{\cos(\pi n)}{n^2+1}, \ n\in\amsbb{N}\right\}
    \]
\end{exercise}
\begin{proof}[Soluzione]
    Per cercare di capire com'è fatto l'insieme, scriviamone i primi elementi:
    \[
    a_0 = \frac{\cos(0)}{0^2+1} = 1 \quad a_1 = \frac{\cos(\pi)}{1^2+1} = -\frac{1}{2}\quad a_2 = \frac{\cos(2\pi)}{2^2+1} = \frac{1}{5} \quad  a_3 = \frac{\cos(3\pi)}{3^2+1} = -\frac{1}{10} \ \dots
    \]
    Notiamo quindi che
     \[
     a_0, a_2 >0  \ \text{e} \ a_1, a_3<0
         \qquad \abs{a_0}>\abs{a_1}>\abs{a_2}>\abs{a_3}
    \]
    ossia sembrerebbe che per $n$ pari gli elementi dell'insieme siano positivi e che per $n$ dispari siano negativi, e che in valore assoluto questi decrescano al crescere di $n$. Vogliamo provare a dimostrare queste due proprietà.
    \begin{enumerate}[(i)]
        \item Notiamo che
        \[
        \cos(\pi n) = \begin{dcases}
            1\, & n = 2k, \ k\in \amsbb{N}\\
            -1\, & n=2k+1, \ k\in\amsbb{N}
        \end{dcases} = (-1)^n
        \]
        e di conseguenza la prima ipotesi sul comportamento dei termini dell'insieme è verificata.
        \item Vogliamo mostrare che
        \[
        \abs{a_n}>\abs{a_{n+1}} \ \text{per ogni} \ n\in\amsbb{N}
        \]
        Notiamo che grazie al punto (i) vale che
        \[
        a_n = \frac{\cos(\pi n)}{n^2+1} = \frac{(-1)^n}{n^2+1}
        \]
        ossia 
        \[
        \abs{a_n} = \frac{1}{n^2+1}
        \]
        Vogliamo quindi mostrare che 
        \[
        \frac{1}{n^2+1}>\frac{1}{(n+1)^2+1} \ \text{per ogni} \ n\in\amsbb{N}
        \]
        Consideriamo quindi la disequazione
        \[
        \frac{1}{n^2+1}>\frac{1}{(n+1)^2+1} \iff \frac{1}{n^2+1}-\frac{1}{(n+1)^2+1}>0
        \]
        di cui cerchiamo le soluzioni in $\amsbb{N}$. Abbiamo che
        \[
        \frac{1}{n^2+1}-\frac{1}{(n+1)^2+1} = \frac{(n+1)^2+1-(n^2+1)}{(n^2+1)((n+1)^2+1)} = \frac{2n+1}{(n^2+1)((n+1)^2+1)}
        \]
        e quindi la disequazione risulta essere
        \[
        \frac{\overbrace{2n+1}^{>0 \ \forall n\in\amsbb{N}}}{\underbrace{(n^2+1)}_{>0 \ \forall n\in\amsbb{N}}\underbrace{((n+1)^2+1)}_{>0 \ \forall n\in\amsbb{N}}}>0
        \]
        che è soddisfatta per ogni $n\in\amsbb{N}$; di conseguenza $\abs{a_n}>\abs{a_{n+1}}$ per ogni $n\in\amsbb{N}$.
    \end{enumerate}
    In conseguenza del punto (ii) concludiamo immediatamente che
    \[
    \abs{a_n}< \abs{a_{n-1}}<\dots < \abs{a_0} = 1
    \]
    ossia $1>a_n>-1$ per ogni $n\in\amsbb{N}$; quindi $A\subseteq [-1,1]$ è limitato superiormente e inferiormente, ed ammette quindi estremo superiore ed inferiore per il teorema \ref{th:2.1} e il corollario \ref{cor:2.1}.\\
    Notiamo che possiamo restringere l'intervallo in cui $A$ è incluso notando che
    \[
    a_1 = -\frac{1}{2}<a_n \ \text{per ogni} \ n\in\amsbb{N}
    \]
    Infatti, chiaramente $a_1 <a_{2k}$ per ogni $k\in\amsbb{N}$, mentre vale che
    \[
    \abs{a_1} \ge \abs{a_{2k+1}} \iff a_1 \le a_{2k+1} \ \text{per ogni} \ k\in\amsbb{N}
    \]
    ossia $A\subseteq \left[-\frac{1}{2}, 1\right]$; poiché $1=a_0\in A$ e $-\frac{1}{2}=a_1\in A$, vale che
    \[
    1 = \sup A = \max A \qquad -\frac{1}{2} = \inf A = \min A
    \]
\end{proof}
\begin{exercise}
    \label{ex:2.2}
    Determinare $\sup$ e $\inf$, ed eventualmente massimo e minimo, dell'insieme
    \[
    A = \left\{ a_n = \frac{n^2-5}{n^2+2}, \ n\in\amsbb{N}\right\}
    \]
\end{exercise}
\begin{proof}[Soluzione]
    Anche in questo caso per cercare di capire com'è fatto l'insieme scriviamone i primi elementi:
    \[
    a_0 = \frac{0^2-5}{0^2+2} = -\frac{5}{2} \quad a_1 = \frac{1^2-5}{1^2+2} = -\frac{4}{3} \quad a_3 = \frac{3^2-5}{3^2+2} = \frac{4}{11} \quad a_4 = \frac{4^2-5}{4^2+2} = \frac{11}{18} \ \dots
    \]
    In questo caso sembrerebbe che $a_{n+1}>a_n$ per ogni $n\in\amsbb{N}$; lo possiamo dimostrare come prima, ossia considerando la disequazione
    \[
    \frac{(n+1)^2-5}{(n+1)^2+2}>\frac{n^2-5}{n^2+2} \iff \frac{(n+1)^2-5}{(n+1)^2+2}-\frac{n^2-5}{n^2+2}>0
    \]
    e cercandone le soluzioni in $n\in\amsbb{N}$. Vale che
    \[
    \begin{split}
        &\frac{(n+1)^2-5}{(n+1)^2+2}-\frac{n^2-5}{n^2+2} = \frac{((n+1)^2-5)(n^2+2)-(n^2-5)((n+1)^2+2)}{(n^2+2)((n+1)^2+2)} = \\
        & = \frac{n^2(n+1)^2+2(n+1)^2-5n^2-10-n^2(n+1)^2-2n^2+5(n+1)^2+10}{(n^2+2)((n+1)^2+2)} = \\
        & = \frac{2n^2+4n+2-7n^2+5n^2+10n+5}{(n^2+2)((n+1)^2+2)} = \frac{14n+7}{(n^2+2)((n+1)^2+2)}
    \end{split}
    \]
    e quindi la disequazione risulta essere
    \[
    \frac{\overbrace{14n+7}^{>0 \ \forall n\in\amsbb{N}}}{\underbrace{(n^2+2)}_{>0 \ \forall n\in\amsbb{N}}\underbrace{((n+1)^2+2)}_{>0 \ \forall n\in \amsbb{N}}}>0
    \]
    che è soddisfatta per ogni $n\in\amsbb{N}$. Di conseguenza sappiamo che $A \subseteq \left[-\frac{5}{2}, +\infty\right)$, ed $A$ è quindi limitato inferiormente. Per quanto riguarda eventuali maggioranti invece, possiamo notare come gli elementi $a_n$ dell'insieme si avvicinino ad $1$ al crescere di $n$. Ragionando in maniera non rigorosa, al crescere di $n$ il contributo dei termini $-5$ a numeratore e $+2$ a denominatore diminuisce, e pertanto dominerebbero i termini di $n^2$; inoltre, poiché a numeratore abbiamo $n^2-5$ e a denominatore abbiamo invece $n^2+2$, il denominatore sarà sempre più grande del denominatore, e quindi i termini saranno sempre minori di $1$. Vogliamo quindi mostrare che
    \[
    \frac{n^2-5}{n^2+2}<1 \ \text{per ogni} \ n\in\amsbb{N}
    \]
    Notiamo che la disequazione è equivalente a
    \[
    \frac{n^2-5}{n^2+2}-1 <0 
    \]
    Consideriamo quindi
    \[
    \frac{n^2-5}{n^2+2}-1 = \frac{n^2-5-n^2+2}{n^2+2} = -\frac{3}{\underbrace{n^2+2}_{>0 \ \forall n\in\amsbb{N}}}
    \]
    che effettivamente è minore di 0 per ogni $n\in\amsbb{N}$. Quindi abbiamo che $A\subseteq\left[-\frac{5}{2}, 1\right)$, e di conseguenza $A$ è anche limitato superiormente; per il teorema \ref{th:2.1} e per il corollario \ref{cor:2.1} $A$ ammette di conseguenza estremo superiore ed inferiore.\\
    Ragionando come nell'esercizio \ref{ex:2.1} si può mostrare che $-\frac{5}{2} = a_0\in A$ è estremo inferiore e minimo di $A$; per quanto riguarda l'estremo superiore, ragionando sempre in modo non formale, al crescere di $n$ la frazione assomiglia sempre più ad un oggetto del tipo $\frac{n^2}{n^2}$, ossia si avvicina sempre più ad $1$. La nostra ipotesi è quindi che $1$ sia l'estremo superiore di $A$. Come facciamo a dimostrarlo? Sfruttiamo la seguente caratterizzazione dell'estremo superiore:
    \begin{tcolorbox}
        \begin{theorem}[Caratterizzazione dell'estremo superiore]
            \label{th:2.2}
            Sia $S\subseteq \amsbb{R}$, $S\ne \varnothing$ un insieme limitato superiormente; $\alpha\in \amsbb{R}$ è l'estremo superiore di $S$ se per ogni $\varepsilon>0$ esiste $s_{\varepsilon}\in A$ tale che $s_\varepsilon>\alpha-\varepsilon$.
        \end{theorem}
    \end{tcolorbox}
    Fissiamo quindi un generico $\varepsilon>0$; vogliamo mostrare che esiste un elemento $a_\varepsilon\in A$ tale che $a_\varepsilon>1-\varepsilon$. Poiché gli elementi di $A$ sono in corrispondenza biunivoca con i numeri naturali, $a_\varepsilon = a_{n_\varepsilon}$ per un qualche $n_\varepsilon \in \amsbb{N}$; dobbiamo quindi risolvere il seguente problema: dato $\varepsilon>0$, esiste $n_\varepsilon\in\amsbb{N}$ tale che
    \[
    1-\varepsilon<\frac{n_\varepsilon^2-5}{n_\varepsilon^2+2} \ \text{?}
    \]
    Consideriamo quindi la disequazione
    \[
    1-\varepsilon < \frac{n_\varepsilon^2-5}{n_\varepsilon^2+2} \iff 1-\varepsilon - \frac{n_\varepsilon^2-5}{n_\varepsilon^2+2} <0
    \]
    Possiamo scrivere
    \[
    1-\varepsilon-\frac{n_\varepsilon^2-5}{n_\varepsilon^2+2} = \frac{(1-\varepsilon)(n_\varepsilon^2+2)-n_\varepsilon^2+5}{n_\varepsilon^2+2} = \frac{-\varepsilon n_\varepsilon^2+7-2\varepsilon}{n_\varepsilon^2+2}
    \]
    e la disequazione è
    \[
    \frac{-\varepsilon n_\varepsilon^2 + 7 -2\varepsilon}{\underbrace{n_\varepsilon^2+2}_{>0 \ \forall n_\varepsilon \in \amsbb{N}}}<0
    \]
    Notiamo che se $7-2\varepsilon<=0$, ossia se $\varepsilon>=\frac{7}{2}$, il numeratore è sempre negativo, e quindi la disuguaglianza è verificata per ogni $n_\varepsilon\in\amsbb{N}$; se invece $\varepsilon<\frac{7}{2}$ bisogna scegliere $n_\varepsilon$ in maniera più oculata. Notiamo che il polinomio $-\varepsilon n_\varepsilon^2 +7-2\varepsilon$ ammette due radici,
    \[
    r_1 = -\sqrt{\frac{7}{\varepsilon}-2} \qquad r_2 = +\sqrt{\frac{7}{\varepsilon}-2}
    \]
    e che possiamo scrivere
    \[
    -\varepsilon n_\varepsilon^2 +7-2\varepsilon = -\varepsilon(n_\varepsilon-r_1)(n_\varepsilon-r_2)
    \]
    Studiamo quindi il segno del polinomio
    \begin{center}
        \begin{tikzpicture}
            \tikzset{h style/.style = {fill=black!30}}
            \tkzTabInit[lgt=4,espcl=2,deltacl=0]
            { /.8, $(n_\varepsilon-r_1)$ /.8, $(n_\varepsilon-r_2)$ /.8, $-\varepsilon(n_\varepsilon-r_1)(n_\varepsilon-r_2)$ /.8}
            {,$r_1$, $r_2$, } % four main references
            \tkzTabLine {,-,z,+, t, +, } % seven denotations
            \tkzTabLine {,-,t,-, z, +, }
            \tkzTabLine {,h,z,+,z, h, }
            \path (M13) -- (M14) node[black,midway]{-};
            \path (M33) -- (M34) node[black,midway]{-};
        \end{tikzpicture}
    \end{center}
    ossia $-\varepsilon n_\varepsilon^2+7-2\varepsilon$ è negativo se $n_\varepsilon <r_1$ o $n_\varepsilon> r_2$; poiché i numeri naturali sono contenuti nei numeri reali positivi, l'unica possibilità è che
    \[
    n_\varepsilon > \sqrt{\frac{7}{\varepsilon}-2}
    \]
    Quindi se prendiamo un numero naturale $n_\varepsilon > \sqrt{\frac{7}{\varepsilon}-2}$ vale che $1-\varepsilon <a_{n_\varepsilon}$; ad esempio, possiamo considerare
    \[
    n_\varepsilon = \left\lfloor\sqrt{\frac{7}{\varepsilon}-2}\right\rfloor+1
    \]
    ove $\amsbb{R}\ni x \mapsto \lfloor x \rfloor$ è la cosiddetta \emph{floor function}, che restituisce la parte intera di un numero reale ("approssimando verso $-\infty$")\footnote{Per i numeri reali positivi è analogo a fare il casting \texttt{n=(int)x} di una variabile \texttt{float} su C/C\texttt{++}.}. Per concludere quindi
    \[
    -\frac{5}{2} = \inf A = \min A \qquad 1 = \sup A
    \]
    e $A$ non ammette massimo, poiché non esiste $n\in\amsbb{N}$ tale che $a_n = 1$.
\end{proof}
\begin{exercise}
    \label{ex:2.3}
    Determinare $\sup$ e $\inf$, ed eventualmente massimo e minimo, dell'insieme
    \[
    A = \left\{ x\in\amsbb{R} \colon 2^{\frac{x+1}{x-1}}>4^x\right\}
    \]
\end{exercise}
\begin{proof}[Soluzione]
    Notiamo che la funzione $\amsbb{R}\ni x \mapsto 2^{\frac{x+1}{x-1}}$ è la composizione di un'esponenziale con una funzione razionale definita su $\amsbb{R}\setminus \{1\}$; pertanto sicuramente $1\notin A$. Inoltre, ricordiamo che
    \begin{tcolorbox}
        Dato $a\in\amsbb{R}$ con $a>1$, la funzione $\amsbb{R}\ni x \mapsto a^x$ è monotona strettamente crescente, ossia
        \[
        x > y \implies f(x) > f(y)
        \]
    \end{tcolorbox}
    ossia equivalentemente $f(x)<f(y) \implies x <y$. Di conseguenza,
    \[
    2^{\frac{x+1}{x-1}}>2^{2x} \iff \frac{x+1}{x-1}> 2x
    \]
    Per determinare $A$ è quindi necessario studiare la disequazione
    \[
    \frac{x+1}{x-1}>2x \iff \frac{x+1}{x-1}-2x >0
    \]
    che possiamo scrivere come 
    \begin{equation}
        \label{eq:2.1}
        \frac{x+1-2x^2+2x}{x-1}>0; \qquad \frac{-2x^2+3x+1}{x-1}
    \end{equation}
    Le radici del polinomio $-2x^2+3x+1$ sono
    \[
    r_1 = \frac{3-\sqrt{17}}{4} \qquad r_2 = \frac{3+\sqrt{17}}{4}
    \]
    tramite cui possiamo scrivere $-2x^2+3x+1 = -2(x-r_1)(x-r_2)$; notiamo che $r_1<0$ e $r_2>1$. Studiamo quindi il segno della funzione razionale in (\ref{eq:2.1}):
    \begin{center}
        \begin{tikzpicture}
            \tikzset{h style/.style = {fill=black!30}}
            \tkzTabInit[lgt=4,espcl=2,deltacl=0]
            { /.8, $(x-r_1)$ /.8, $(x-r_2)$ /.8, ${-2x^2+3x+1}$ /.8,  $(x-1)$ /.8, $\frac{-2x^2+3x+1}{x-1}$ /.8}
            {,$r_1$, $1$, $r_2$, } % four main references
            \tkzTabLine {,-,z,+, t, +,t, +, } % seven denotations
            \tkzTabLine {,-,t,-, t, -, z, +, }
            \tkzTabLine {,-,z,+,t, +,z, -,  }
            \tkzTabLine{, -, t, -, z, +, t, +, }
            \tkzTabLine{, h, z, -, t, h, z, -, }
            \path (N35) -- (N36) node[black,midway,inner sep=2pt,draw,circle,fill=white]{};
            \path (M15) -- (M16) node[black,midway]{+};
            \path (M35) -- (M36) node[black,midway]{+};
        \end{tikzpicture}
    \end{center}
    Le soluzioni della disequazione in (\ref{eq:2.1}) sono quindi
    \[
    x<\frac{3-\sqrt{17}}{4} \quad \lor \quad  1 < x < \frac{3+\sqrt{17}}{4}
    \]
    ossia possiamo scrivere l'insieme $A$ come
    \[
    A = \left(-\infty, \frac{3-\sqrt{17}}{4}\right)\cup \left(1, \frac{3+\sqrt{17}}{4}\right)
    \]
    Osserviamo quindi che $A$ non è limitato inferiormente; pertanto convenzionalmente diciamo che $\inf A = -\infty$. $A$ è invece limitato superiormente, e vale che $\sup A = \frac{3+\sqrt{17}}{4}$; poiché $\sup A \notin A$, $A$ non ammette massimo.
\end{proof}
\begin{exercise}
    \label{ex:2.4}
    Determinare $\sup$ e $\inf$, ed eventualmente massimo e minimo, dell'insieme
    \[
    A = \left\{ x\in\amsbb{R} \colon \sin(x)+3\cos(x)+1 \le 0 \right\}
    \]
\end{exercise}
\begin{proof}[Soluzione]
    Il metodo più immediato per risolvere il problema è utilizzare le formule parametriche
    \begin{equation}
            \label{eq:2.2}
            \sin(x) = \frac{2t}{1+t^2} \qquad \cos(x) = \frac{1-t^2}{1+t^2} \qquad t = \tan\left(\frac{x}{2}\right)
    \end{equation}
    Notiamo che, mentre $\amsbb{R}\ni x \mapsto \sin(x)$ e $\amsbb{R}\ni x \mapsto \cos(x)$ sono definite su tutto $\amsbb{R}$, la funzione $\amsbb{R}\ni x \mapsto \tan\left(\frac{x}{2}\right)$ è definita su $\amsbb{R} \setminus \left\{\pi +2k\pi, \ k\in\amsbb{Z}\right\}$; pertanto se riscriviamo la funzione che descrive l'insieme $A$ usando (\ref{eq:2.2}), escluderemo di default i punti del tipo $\{\pi + 2k\pi, \ k\in\amsbb{Z}\}$, che bisognerà controllare manualmente. Procediamo quindi con la risoluzione dell'esercizio:
    \begin{enumerate}[(i)]
        \item iniziamo a verificare i punti dell'insieme $\{\pi + 2k\pi, k\in\amsbb{Z}\}$; avremo
        \[
        \sin(\pi + 2k\pi) + 3\cos(\pi+2k\pi)+1 = -3+1 = -2 <0
        \]
        e quindi
        \[
        \left\{\pi + 2k\pi, k\in\amsbb{Z}\right\}\subseteq A
        \]
        \item cerchiamo le soluzioni
        di
        \begin{equation}
            \label{eq:2.3}
            \frac{2t}{1+t^2}+3\frac{1-t^2}{1+t^2}+1\le 0, \qquad t= \tan\left(\frac{x}{2}\right)
        \end{equation}
        in $\amsbb{R}\setminus \{\pi+2k\pi, \ k\in\amsbb{Z}\}$. Possiamo riscrivere la disequazione in (\ref{eq:2.3}) come
        \[
        \frac{2t+3-3t^2+1+t^2}{1+t^2}\le 0; \qquad \frac{-2t^2+2t+4}{1+t^2}\le 0; \qquad \frac{-2(t^2-t-2)}{1+t^2}\le 0
        \]
        e infine come
        \[
        \frac{-2(t-2)(t+1)}{\underbrace{1+t^2}_{>0 \ \forall t\in\amsbb{R}}}\le 0
        \]
        Studiamo come prima il segno della funzione razionale:
        \begin{center}
            \begin{tikzpicture}
                \tikzset{h style/.style = {fill=black!30}}
                \tkzTabInit[lgt=4,espcl=2,deltacl=0]
                { /.8, $(t+1)$ /.8, $(t-2)$ /.8, ${-2t^2+2t+4}$ /.8,   $\frac{-2t^2+2t+4}{1+t^2}$ /.8}
                {,$-1$, $2$, } % four main references
                \tkzTabLine {,-,z,+, t, +, } % seven denotations
                \tkzTabLine {,-,t,-, z, +, }
                \tkzTabLine{, -, z, +, z, -,  }
                \tkzTabLine{, h, z, +, z, h,  }
                \path (M14) -- (M15) node[black,midway]{-};
                \path (M34) -- (M35) node[black,midway]{-};
            \end{tikzpicture}
        \end{center}
        Le soluzioni di (\ref{eq:2.3}) sono quindi
        \[
        t\le-1 \quad \lor \quad t\ge 2
        \]
        e di conseguenza l'insieme $A$ conterrà gli insiemi
        \[
        \underbrace{\left\{x\in \amsbb{R} \colon \tan\left(\frac{x}{2}\right)\le -1\right\}}_{\stepcounter{equation}\mbox{(\theequation)}} \nonumber\cup \underbrace{\left\{x\in\amsbb{R}\colon \tan\left(\frac{x}{2}\right)\ge 2\right\}}_{\stepcounter{equation}\mbox{(\theequation)}}\nonumber
        \]
        \addtocounter{equation}{-2}\refstepcounter{equation}\label{eq:2.4}
        \addtocounter{equation}{0}\refstepcounter{equation}\label{eq:2.5}
        ove per brevità notazionale abbiamo trascurato le condizioni di esistenza della tangente.
        Le soluzioni di (\ref{eq:2.4}) sono
        \[
        \left\{x\in\amsbb{R}\colon \frac{\pi}{2}+k\pi<\frac{x}{2}\le\frac{3}{4}\pi + k\pi, \ k\in\amsbb{Z}\right\}
        \]
        mentre le soluzioni di (\ref{eq:2.5}) sono
        \[
        \left\{x\in\amsbb{R}\colon \arctan(2)+k\pi \le \frac{x}{2}<\frac{\pi}{2}+k\pi, \ k\in\amsbb{Z} \right\}
        \]
        Riscrivendo i due insiemi in termini di disuguaglianze per $x$ abbiamo
        \[
        \left\{x\in\amsbb{R}\colon \pi+2k\pi<x\le\frac{3}{2}\pi + 2k\pi \right\} \quad \left\{x\in\amsbb{R}\colon 2\arctan(2)+2k\pi\le x <\pi + 2k\pi \right\}
        \]
    \end{enumerate}
    con $k$ che varia in $\amsbb{Z}$. $A$ risulta quindi essere
    \[
    \begin{split}
        \left\{\pi + 2k\pi, \ k\in \amsbb{Z}\right\} &\cup \left\{x\in\amsbb{R}\colon \pi+2k\pi<x\le\frac{3}{2}\pi + 2k\pi, \ k\in\amsbb{Z}\right\} \cup\\
        & \cup\left\{x\in\amsbb{R}\colon 2\arctan(2)+2k\pi\le x <\pi + 2k\pi, \ k\in\amsbb{Z} \right\}
    \end{split}
    \]
    ossia
    \[
    A = \left\{x\in\amsbb{R}\colon 2\arctan(2)+2k\pi \le x \le \frac{3}{2}\pi + 2k\pi, \ k\in\amsbb{Z}\right\}
    \]
    Notiamo quindi che $A$ è illimitato\footnote{Questo lo si poteva notare già dal punto (i): infatti l'insieme $\{\pi+2k\pi, \ k\in\amsbb{Z}\}$ è illimitato, e poiché $A \supseteq \{\pi +2k\pi, k \in \amsbb{Z}\}$ anche $A$ è illimitato.}; pertanto per convenzione
    \[
    \sup A = +\infty \qquad \inf A = -\infty
    \]
\end{proof}
\newpage