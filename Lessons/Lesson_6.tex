\section{Lezione 6}
\subsection{Ripasso: derivabilità e differenziabilità}
\begin{definition}
    \label{def:6.1}
    Sia $f\colon(a,b)\to\amsbb{R}$, e sia $x\in(a,b)$; definiamo il \emph{rapporto incrementale di $f$ in $x$} come la funzione $\phi_x \colon (a,b) \setminus \{x\} \to \amsbb{R}$,
    \[
    \phi_x(t) = \frac{f(t)-f(x)}{t-x}
    \]
    Diremo che $f$ è \emph{derivabile in $x$} se esiste finito
    \begin{equation}
        \label{eq:6.1}
        \lim_{t\to x} \phi_x(t) = \lim_{t\to x} \frac{f(t)-f(x)}{t-x} = f'(x)
    \end{equation}
\end{definition}
\begin{remark}
    Detto $I\subseteq (a,b)$ l'insieme dei punti in cui $f$ è derivabile, possiamo definire una funzione $ I\ni x \mapsto f'(x)\in \amsbb{R}$, detta \emph{derivata prima} di $f$, che si indica con il simbolo $f'$.
\end{remark}
\begin{definition}
    \label{def:6.2}
    Dati $f\colon(a,b)\to\amsbb{R}$ e $x\in(a,b)$, diremo che $f$ è \emph{differenziabile in $x$} se esiste un'applicazione lineare $A_x\colon\amsbb{R}\to \amsbb{R}$ tale che
    \begin{equation}
        \label{eq:6.2}
        \lim_{\abs{h}\to0}\frac{\abs{f(x+h)-f(x)-A_x h}}{\abs{h}}=0
    \end{equation}
\end{definition}
\begin{remark}
    Ricordiamo che vale il teorema \ref{th:4.6}, e che quindi $\abs{h}\to 0$ se e solo se $h\to0$ e $\abs{f(h)}\to0$ se e solo se $f(h)\to0$. Possiamo quindi riscrivere la (\ref{eq:6.1}) come
    \[
    \lim_{h\to 0} \frac{f(x+h)-f(x)-A_x h}{h}=0
    \]
    Per la definizione \ref{def:5.4} vale quindi che
    \[
    f(x+h)-f(x)-A_xh = o(h) \iff f(x+h) = f(x) + A_x h+o(h)
    \]
    ossia $f$ è differenziabile in $x$ se in un intorno di $h$ possiamo approssimare $f$ con un'applicazione lineare $A_x$ commettendo un errore contenuto ($o(h)$).
\end{remark}
\begin{remark}
    Dai corsi di geometria sappiamo che, fissata una base di $\amsbb{R}$, esiste una corrispondenza biunivoca fra applicazioni lineari $A_x\in L(\amsbb{R}, \amsbb{R})$ e matrici $1\times 1$ a coefficienti reali, ossia numeri reali: in particolare, data l'applicazione lineare $A_x\colon \amsbb{R}\to\amsbb{R}$, esiste $m_{A_x}\in\amsbb{R}$ tale che
    \[
    A_x(h) = m_{A_x}h \ \text{per ogni} \ h\in\amsbb{R}
    \]
\end{remark}
\begin{theorem}
    \label{th:6.1}
    Data $f\colon (a,b)\to\amsbb{R}$, $f$ è differenziabile in $x$ se e solo se $f$ è derivabile in $x$.
\end{theorem}
\begin{proof}
    Dimostriamo le due implicazioni.
    \begin{enumerate}[(i)]
        \item Supponiamo che $f$ sia differenziabile in $x$, e mostriamo che $f$ è derivabile in $x$. Sappiamo che
        \[
        f(x+h)=f(x)+A_x h +o(h) = f(x)+m_{A_x}h +o(h)
        \]
        e di conseguenza
        \[
        \lim_{h\to 0} \frac{f(x+h)-f(x)}{h} = \lim_{h\to 0} \frac{f(x)+m_{A_x}h+o(h)-f(x)}{h} = m_{A_x}+\lim_{h\to 0} \frac{o(h)}{h} = m_{A_x} 
        \]
        Notiamo che il limite in (\ref{eq:6.1}) può essere scritto come
        \[
        \lim_{t\to x} \frac{f(t)-f(x)}{t-x} = \lim_{t\to x} \frac{f(t-x+x)-f(x)}{t-x}\overset{h=t-x}{=} \lim_{h\to 0} \frac{f(x+h)-f(x)}{h}
        \]
        in quanto $h\to 0$ se $t\to x$. Abbiamo quindi che il limite (\ref{eq:6.1}) esiste finito, e che $f'(x) = m_{A_x}$.
        \item Supponiamo ora che $f$ sia derivabile in $x$ e mostriamo che $f$ è differenziabile in x. Per definizione vale che
        \[
        \lim_{t\to x}\frac{f(t)-f(x)}{t-x} = \lim_{h\to 0}\frac{f(x+h)-f(x)}{h} = f'(x)
        \]
        ossia
        \[
        \lim_{h\to 0}\frac{f(x+h)-f(x)}{h}-f'(x) = 0
        \]
        Possiamo riscrivere il precedente limite come
        \[
        0=\lim_{h\to 0}\frac{f(x+h)-f(x)}{h}-f'(x) = \lim_{h\to 0} \frac{f(x+h)-f(x)-f'(x)h}{h}
        \]
        Per il teorema \ref{th:4.6} sappiamo che 
        \[
        \lim_{h\to 0}\abs{\frac{f(x+h)-f(x)-f'(x)h}{h}} = \lim_{h\to 0} \frac{\abs{f(x+h)-f(x)-f'(x)h}}{\abs{h}}=0
        \]
        e di conseguenza l'applicazione lineare $\amsbb{R}\ni h \mapsto f'(x)h$ indotta da $f'(x)\in\amsbb{R}$ soddisfa la definizione \ref{def:6.2}.
    \end{enumerate}
\end{proof}
\begin{remark}
    Questo teorema ci dice che per funzioni $f\colon \amsbb{R}\to \amsbb{R}$ differenziabilità e derivabilità sono due concetti equivalenti. Non è così per funzioni $f\colon \amsbb{R}^n\to \amsbb{R}$! Ci sono esempi\footnote{Ad esempio, la funzione $\amsbb{R}^2\ni (x,y)\mapsto \frac{x^2y}{x^2+y^2}$ per $(x,y)\ne (0,0)$ e $f(0,0)=0$ è derivabile (direzionalmente) in $(0,0)$ ma non è differenziabile.} di funzioni derivabili ma non differenziabili.
\end{remark}
\begin{theorem}
    \label{th:6.2}
    Siano $f,g\colon (a,b)\to \amsbb{R}$ due funzioni differenziabili in $x\in(a,b)$; allora
    \begin{enumerate}[(i)]
        \item $(f+g)'(x) = f'(x) + g'(x)$;
        \item $(\lambda f)'(x) = \lambda f'(x)$;
        \item (Regola di Leibniz)
        \begin{equation}
            \label{eq:6.3}
            (fg)'(x) = f'(x)g(x) + f(x)g'(x)
        \end{equation}
    \end{enumerate}
    Inoltre, se $g\colon (c,d) \to \amsbb{R}$ è tale per cui $f((a,b))\subseteq (c,d)$ e $g$ è differenziabile in $f(x)\in(c,d)$, allora $g\circ f \colon (a,b) \to \amsbb{R}$ è differenziabile in $x$ e vale la regola della catena
    \begin{equation}
        \label{eq:6.4}
        (g\circ f)'(x) = g'(f(x))f'(x)
    \end{equation}
\end{theorem}