\section{Lezione 12}
\subsection{Ripasso: integrazione di Riemann impropria}
\begin{definition}
    \label{def:10.1}
    Sia $f\colon[a,+\infty)\to \amsbb{R}$ una funzione tale che $f\in\mathscr{R}([a,x])$ per ogni $x>a$. Diremo che la quantità
    \[
    \int_a^{+\infty} f(x)\, dx
    \]
    esiste se esiste finito il limite
    \[
    \lim_{x\to +\infty} \int_a^x f(t)\, dt = \lim_{x\to+\infty} F(x)
    \]
\end{definition}
Allo stesso modo, 
\begin{definition}
    \label{def:10.2}
    Data una funzione $f\colon [a,b)\to\amsbb{R}$ con $f\in\mathscr{R}([a,x])$ per ogni $x\in(a,b)$, diremo che la quantità
    \[
    \int_a^b f(x)\, dx
    \]
    esiste se esiste finito
    \[
    \lim_{x\to b^-} \int_a^x f(x)\, dx = \lim_{x\to b^-} F(x)
    \]
\end{definition}
\begin{remark}
    Notiamo che in questo secondo caso possono presentarsi due casi: la funzione $f\colon[a,b)\to\amsbb{R}$ non è definita in $b$ ma ammette limite finito per $x\to b^-$ (ad esempio se $f$ fosse $[-1, 0)\ni x\mapsto \frac{\sin(x)}{x}$), oppure $f$ non è definita in $b$ e non ammette limite finito per $x\to b^-$ (ad esempio, $[1,0)\ni x\mapsto \frac{1}{x}$). Nel primo caso possiamo agire nel modo seguente: possiamo definire per continuità una funzione $\tilde{f}\colon[a,b]\to \amsbb{R}$ che è definita su tutto $[a,b]$ ed è tale che $\tilde{f}\in\mathscr{R}([a,b])$. In virtù del corollario \ref{cor:8.2}, poiché $\tilde{f}$ e $f$ differiscono solamente in un punto, possiamo dire che
    \[
    \int_a^b f(x)\, dx = \int_a^b \tilde{f}(x)\, dx
    \]
    Di conseguenza, nel primo caso il limite esiste sempre; di conseguenza, in generale quando si parla di integrazione di Riemann impropria si tende a considerare unicamente due casi:
    \begin{enumerate}[(i)]
        \item funzioni $f$ \textbf{definite su un intervallo illimitato};
        \item funzioni $f$ \textbf{definite su un intervallo limitato aventi immagine illimitata}.
    \end{enumerate}
\end{remark}
\begin{example}
    Consideriamo $f\colon[0,+\infty)\ni t\mapsto  \sin(t)\in\amsbb{R}$; ci chiediamo se esiste
    \[
    \int_0^{+\infty}\sin(t)\, dt
    \]
    Calcoliamo quindi
    \[
    F(x) = \int_0^{x}\sin(t)\, dt = -\cos(t)\bigg|_0^x = 1-\cos(x)
    \]
    Vogliamo determinare, se esiste,
    \[
    \lim_{x\to+\infty}F(x)
    \]
    Ci ricordiamo che vale il teorema \ref{th:5.1}: consideriamo quindi due successioni,
    \[
    (a_n = 2n\pi)_n \qquad (bn=\pi + 2n\pi)_n
    \]
    Entrambe divergono a $+\infty$ per $n\to \infty$, e possiamo considerare quindi
    \[
    \lim_{n\to\infty} F(a_n) = \lim_{n\to\infty}1-\cos(2n\pi) = 0 \qquad \lim_{n\to\infty} F(b_n) = \lim_{n\to\infty} 1-\cos(\pi+2n\pi) = 2
    \]
    Di conseguenza per il teorema \ref{th:5.1} sappiamo che
    \[
    \lim_{n\to\infty}F(x)
    \]
    non esiste.\\
    Questo esempio ammette un'interessante interpretazione grafica: se consideriamo infatti il grafico di $f(t) = \sin(t)$
    \begin{center}
        \begin{tikzpicture}
            \pgfplotsset{
                axis lines=middle,
                xlabel style=below,
                ylabel style=left
            }
            \begin{axis}[y=7mm, ymin=-1.2, ymax = 1.2, xmin = -.2, xmax = 13.2, xscale =1.3, yscale = 3, samples=201, xlabel={$x$}, ylabel = {$y$}, xtick={3.14159, 6.28318, 9.42477, 12.56637}, xticklabels={$\pi$, $2\pi$, $3\pi$, $4\pi$}]
        %\addplot [name path=A, teal] coordinates { (\Xmin, -\Xmin) (\Xmax,-\Xmax) };
        %\path [name path=B] (\Xmax,\Ymax)--(\Xmin,\Ymax);
        

            \addplot [name path=A, domain=0:3.14159, black] {sin(deg(x))};
            \path [name path=B] (0, 0)--(3.14159, 0);
            \addplot [red!30] fill between [of=A and B];
            \addplot [name path=C, domain=3.14159:6.28318, black] {sin(deg(x))};
            \path [name path=D] (3.14159, 0)--(6.28318, 0);
            \addplot [blue!30] fill between [of=C and D];
            \addplot [name path=E, domain=6.28318:9.42477, black] {sin(deg(x))};
            \path [name path=F] (6.28318, 0)--(9.42477,0);
            \addplot [red!30] fill between [of=E and F];
            \addplot [name path=G, domain=9.42477:12.56637, black] {sin(deg(x))};
            \path [name path=H] (9.42477, 0)--(12.56637, 0);
            \addplot [blue!30] fill between [of=G and H];
        %\addplot [white] fill between [of=C and D];
        %\endpgfonlayer
                \end{axis}
            \end{tikzpicture}
        \end{center}
        sappiamo che $F(x)$ rappresenta l'area sottesa dal grafico di $\sin(t)$ da $0$ a $x$. Per $x$ che varia da 0 a $\pi$ quest'area cresce fino a raggiungere il valore massimo di $2$ per $x=\pi$; dopodiché decresce sino a raggiungere il valore minimo di $0$ per $x=2\pi$, e il processo si ripete ogni periodo. Poiché l'area sottesa dal grafico oscilla in questo modo, non tende a stabilizzarsi verso un valore predefinito per $x\to +\infty$, e di conseguenza l'integrale improprio non esiste.
\end{example}
Per capire se un integrale improprio esiste o meno, possiamo appellarci ai seguenti risultati:
\begin{theorem}
    \label{th:10.1}
    Siano $f, g\colon[a,+\infty)\to \amsbb{R}$ due funzioni continue (in modo tale da soddisfare le richieste di integrabilità) e tali che esista $M>0$ tale che
    \[
    Mg(x)\ge f(x)\ge 0 \ \text{per ogni} \ x\in[a,+\infty)
    \]
    Allora:
    \begin{enumerate}[(i)]
        \item Se $\int_a^{+\infty}g(x)\, dx$ esiste allora $\int_a^{+\infty}g(x)\, dx$ esiste;
        \item Se $\int_a^{+\infty}f(x)\, dx$ diverge allora $\int_a^{+\infty}g(x)$ diverge.
    \end{enumerate}
\end{theorem}
\begin{theorem}
    \label{th:10.2}
    Siano $f,g\colon[a,+\infty)\to \amsbb{R}$ due funzioni continue tali che $f(x), g(x)\ge 0$ per ogni $x\in[a,+\infty)$ e tali che
    \[
    \lim_{x\to+\infty} \frac{f(x)}{g(x)} = l\in\amsbb{R}\setminus\{0\}
    \]
    Allora
    \[
    \int_a^{+\infty}f(x)\, dx \ \text{e} \ \int_a^{+\infty}g(x)\, dx
    \]
    hanno lo stesso comportamento.
\end{theorem}
Analoghi risultati valgono per nel caso di integrali impropri nel senso della definizione \ref{def:10.2}.
\subsection{Esercizi: integrazione di Riemann impropria}
\begin{exercise}
    \label{ex:10.1}
    Determinare se
    \[
    \int_0^1 \frac{1}{\sqrt{1-x^4}}\, dx
    \]
    esiste.
\end{exercise}
\begin{proof}[Soluzione]
    Notiamo che la funzione integranda $f\colon[0,1)\to\amsbb{R}$ può essere scritta come
    \[
    f(x) = \frac{1}{\sqrt{(1-x)(1+x)(1+x^2)}}
    \]
    e che $\lim_{x\to 1^-} f(x) = +\infty$. Pertanto l'integrale che stiamo considerando è un integrale di Riemann improprio; per determinare se questo esiste, conviene usare uno dei teoremi \ref{th:10.1} e \ref{th:10.2}.\\
    Notiamo che per la definizione \ref{def:10.2} vale che
    \[
    \begin{split}
        \int_0^1 \frac{1}{\sqrt{1-x^4}}\, dx &= \lim_{t\to1^-} \int_0^t \frac{1}{\sqrt{(1-x)(1+x)(1+x^2)}}\, dx \overset{\text{Teorema \ref{th:8.4}}}{=} \\
        &=\lim_{u\to0^+} -\int_1^u \frac{1}{\sqrt{y}\sqrt{2-y}\sqrt{1+(1-y)^2}}\, dy = \\
        & = \lim_{u\to 0^+}\int_u^1 \frac{1}{\sqrt{y}\sqrt{2-y}\sqrt{1+(1-y)^2}}\, dy
    \end{split}
    \]
    A questo punto possiamo provare a sviluppare con Taylor (cfr. teorema \ref{th:6.5}) parte del denominatore per provare a trovare una funzione $g(y)$ che soddisfi il teorema \ref{th:10.2}:
    \[
    \frac{1}{\sqrt{2-y}} = \frac{1}{\sqrt{2}}+\frac{1}{4\sqrt{2}}y +o(y) \qquad \frac{1}{\sqrt{1+(1-y)^2}} = \frac{1}{\sqrt{2}}+\frac{1}{2\sqrt{2}}y +o(y)
    \]
    La funzione integranda può essere quindi sviluppata in un intorno di $y=0$ come
    \[
    \begin{split}
        & \frac{1}{\sqrt{y}}\left(\frac{1}{\sqrt{2}}+\frac{1}{4\sqrt{2}}y+o(y)\right)\left(\frac{1}{\sqrt{2}}+\frac{1}{2\sqrt{2}}y+o(y)\right) = \\
        & = \frac{1}{\sqrt{y}}\left(\frac{1}{2}+\frac{3}{8}y+o(y)\right) = \frac{1}{2\sqrt{y}}+\frac{3}{8}\sqrt{y}+o(\sqrt{y})
    \end{split}
    \]
    Se consideriamo $g(y) = \frac{1}{2\sqrt{y}}$ abbiamo che
    \[
    \lim_{y\to 0^+} \frac{f(y)}{g(y)} = \lim_{y\to 0^+} \left(\frac{1}{2\sqrt{y}}+\frac{3}{8}\sqrt{y}+o(\sqrt{y})\right)2\sqrt{y} = \lim_{y\to 0^+} 1 +\frac{3}{4}y+o(y) = 1
    \]
    Notiamo che $f(y)$ e $g(y)$ sono continue in $(0,1]$ e positive; quindi per il teorema \ref{th:10.2} vale che
    \[
    \int_0^1 \frac{1}{\sqrt{1-x^4}}\, dx \ \text{e} \ \int_0^1 \frac{1}{2\sqrt{y}}\, dy
    \]
    hanno lo stesso comportamento. Sappiamo che $\int_0^1 \frac{1}{x^\alpha}\, dx$ esiste se $\alpha<1$; pertanto 
    \[
    \int_0^1 \frac{1}{\sqrt{1-x^4}}\, dx
    \]
    esiste.
\end{proof}
\begin{exercise}
    \label{ex:10.2}
    Determinare i valori di $\alpha, \beta>0$ tali che
    \[
    \int_0^\pi \frac{\abs{\cos(x)}^{2\alpha}}{\sin^\beta(x)}\, dx
    \]
    esiste.
\end{exercise}
\begin{proof}[Soluzione]
    Innanzitutto, osserviamo che la funzione è continua su $(0,\pi)$, e di conseguenza è Riemann-integrabile su $[x_1, x_2]$ per ogni scelta di $0<x_1<x_2<\pi$. Notiamo inoltre che in questo caso i punti problematici sono due: infatti, per ogni scelta di $\alpha, \beta>0$ vale che
    \[
    \lim_{x\to 0^+}\frac{\abs{\cos(x)}^{2\alpha}}{\sin^\beta(x)} = +\infty \qquad \lim_{x\to \pi^-} \frac{\abs{\cos(x)}^{2\alpha}}{\sin^\beta(x)} = +\infty
    \]
    Poiché sappiamo trattare funzioni con eventualmente un solo punto problematico, possiamo appellarci al punto (i) della seconda parte del corollario \ref{cor:8.1} per scrivere
    \[
    \int_0^\pi \frac{\abs{\cos(x)}^{2\alpha}}{\sin^\beta(x)}\, dx = \int_0^{\frac{\pi}{2}}\frac{\abs{\cos(x)}^{2\alpha}}{\sin^\beta(x)}\, dx + \int_{\frac{\pi}{2}}^\pi \frac{\abs{\cos(x)}^{2\alpha}}{\sin^\beta(x)}\, dx
    \]
    Questa scrittura è possibile solamente se esistono separatamente
    \[
    \underbrace{\lim_{y\to 0^+} \int_y^{\frac{\pi}{2}} \frac{\abs{\cos(x)}^{2\alpha}}{\sin^\beta(x)}\, dx}_{\stepcounter{equation}\mbox{(\theequation)}} \qquad \underbrace{\lim_{z\to \pi^-}\int_{\frac{\pi}{2}}^z \frac{\abs{\cos(x)}^{2\alpha}}{\sin^\beta(x)}\, dx}_{\stepcounter{equation}\mbox{(\theequation)}}
    \]
    \addtocounter{equation}{-2}\refstepcounter{equation}\label{eq:10.1}
    \addtocounter{equation}{0}\refstepcounter{equation}\label{eq:10.2}
    Consideriamo quindi i due limiti:
    \begin{enumerate}[(i)]
        \item per quanto riguarda (\ref{eq:10.1}), operiamo come nella soluzione dell'esercizio \ref{ex:10.1}, ossia sviluppiamo con Taylor la funzione integranda in un intorno di $x_0=0$, notando che $\cos(x)\ge 0$ per ogni $x\in\left[0,\frac{\pi}{2}\right]$:
        \[
        \abs{\cos(x)}^{2\alpha} = (\cos(x))^{2\alpha} = (1+o(x))^{2\alpha} \qquad \sin^\beta(x) = \left(x + o(x)\right)^\beta = x^\beta\left(1+\frac{o(x)}{x}\right)^\beta
        \]
        Quindi abbiamo che
        \[
        \frac{\abs{\cos(x)}^{2\alpha}}{\sin^\beta(x)} = \frac{(1+o(x))^{2\alpha}}{x^\beta\left(1+\frac{o(x)}{x}\right)^\beta}
        \]
        Se consideriamo $g(x) = \frac{1}{x^\beta}$ vale che
        \[
        \lim_{x\to 0^+}\frac{f(x)}{g(x)} = \lim_{x\to 0^+} \frac{(1+o(x))^{2\alpha}}{x^\beta\left(1+\frac{o(x)}{x}\right)^\beta} x^\beta = \lim_{x\to 0^+} \frac{(1+o(x))^{2\alpha}}{\left(1+\frac{o(x)}{x}\right)^\beta} = 1
        \]
        Entrambe le funzioni sono continue e positive su $\left(0, \frac{\pi}{2}\right]$, e per il teorema \ref{th:10.2} abbiamo che
        \[
        \int_0^{\frac{\pi}{2}} \frac{\abs{\cos(x)}^{2\alpha}}{\sin^\beta(x)}\, dx \ \text{e} \ \int_0^{\frac{\pi}{2}}\frac{1}{x^\beta}\, dx
        \]
        hanno lo stesso comportamento. Come prima, sappiamo che $\int_0^{\frac{\pi}{2}} \frac{1}{x^\beta}\, dx$ esiste se $\beta<1$.
        \item Consideriamo ora il limite (\ref{eq:10.2}): notiamo innanzitutto che $\cos(x)\le0$ per ogni $x\in\left[\frac{\pi}{2}, \pi\right]$ e quindi
        \[
        \abs{\cos(x)}^{2\alpha} = (-\cos(x))^{2\alpha}
        \]
        Procedendo come prima e sviluppando numeratore e denominatore della funzione integranda in un intorno di $x_0=\pi$ abbiamo
        \[
        (-\cos(x))^{2\alpha} = (1+o(x-\pi))^{2\alpha}
        \]
        e
        \[
        \sin^\beta(x) = \left(-(x-\pi)+o((x-\pi)) \right)^\beta = (\pi-x)^\beta\left(1+\frac{o(\pi-x)}{\pi-x}\right)^\beta
        \]
        e se consideriamo $g(x) = \frac{1}{(\pi-x)^\beta}$ vale che
        \[
        \lim_{x\to \pi^-} \frac{f(x)}{g(x)} = \lim_{x\to \pi^-} \frac{(1+o(x-\pi))^{2\alpha}}{(\pi-x)^\beta \left(1+\frac{o(\pi-x)}{\pi-x}\right)^\beta} (\pi-x)^\beta = \lim_{x\to \pi^-} \frac{(1+o(x-\pi))^{2\alpha}}{\left(1+\frac{o(\pi-x)}{\pi-x}\right)^\beta} = 1 
        \]
        con $g(x), f(x)$ continue e positive in $\left[\frac{\pi}{2}, \pi\right)$. Quindi per il teorema \ref{th:10.2}
        \[
        \int_{\frac{\pi}{2}}^\pi \frac{\abs{\cos(x)}^{2\alpha}}{\sin^\beta(x)}\, dx \ \text{e} \ \int_{\frac{\pi}{2}}^\pi \frac{1}{(\pi-x)^\beta}\, dx 
        \]
        hanno lo stesso comportamento. Notiamo che
        \[
        \int_{\frac{\pi}{2}}^\pi \frac{1}{(\pi-x)^\beta}\, dx = \lim_{z\to \pi^-}\int_{\frac{\pi}{2}}^z \frac{1}{(\pi-x)^\beta}\, dx \overset{\text{Teorema \ref{th:8.4}}}{=} \lim_{u\to 0^+}\int_u^{\frac{\pi}{2}} \frac{1}{y^\beta}\, dy
        \]
        che esiste finito per $\beta<1$.
    \end{enumerate}
    Quindi entrambi i limiti (\ref{eq:10.1}) e (\ref{eq:10.2}) esistono finiti se $\beta<1$ e per ogni $\alpha$; quindi l'integrale
    \[
    \int_0^\pi \frac{\abs{\cos(x)}^{2\alpha}}{\sin^\beta(x)}\, dx
    \]
    esiste se $\alpha>0$, $0<\beta<1$.
\end{proof}
\begin{exercise}
    \label{ex:10.3}
    Calcolare, se esiste,
    \[
    \int_0^{+\infty}\frac{1}{e^x+e^{-x}}\, dx
    \]
\end{exercise}
\begin{proof}[Soluzione]
    Notiamo che $e^x+e^{-x}>0$ per ogni $x\in\amsbb{R}$; pertanto l'unica criticità dell'integrale improprio è il dominio di integrazione illimitato. Procediamo quindi secondo la definizione \ref{def:10.1}, ossia
    \[
    \int_0^{+\infty} \frac{1}{e^x+e^{-x}}\, dx = \lim_{t\to +\infty} \int_0^t \frac{1}{e^x + e^{-x}}\, dx 
    \]
    Notiamo che
    \[
    \int_0^t \frac{1}{e^x+e^{-x}}\, dx = \int_0^{t}\frac{e^x}{e^{2x}+1}\, dx = \int_0^t \frac{1}{(e^x)^2+1}e^x\, dx \overset{\text{Teorema \ref{th:8.4}}}{=} \int_1^{e^t} \frac{1}{y^2+1}\, dy
    \]
    Quindi
    \[
    \begin{split}
        \lim_{t\to +\infty} \int_0^t \frac{1}{e^x+e^{-x}}\, dx &= \lim_{t\to +\infty} \int_1^{e^t} \frac{1}{y^2+1}\, dy = \lim_{t\to +\infty} \left(\arctan(e^t)-\arctan(1)\right) = \\
        & = \frac{\pi}{2}-\frac{\pi}{4} = \frac{\pi}{4}
    \end{split}
    \]
\end{proof}
\begin{exercise}
    \label{ex:10.4}
    Dato $\alpha\in\amsbb{R}$, si consideri
    \[
    I_\alpha = \int_{-\infty}^{+\infty} \abs{x}^{-\alpha}\arctan\left(\frac{x}{x^4+1}\right)\, dx
    \]
    \begin{enumerate}[(i)]
        \item Determinare per quali valori di $\alpha\in\amsbb{R}$ l'integrale improprio esiste;
        \item per questi $\alpha$, calcolare $I_\alpha$.
    \end{enumerate}
\end{exercise}
\begin{proof}[Soluzione]
    Innanzitutto, detta $f(x)$ la funzione 
    \[
    f(x) = \abs{x}^{-\alpha}\arctan\left(\frac{x}{x^4+1}\right)
    \]
    notiamo che per $\alpha\le 0$ la funzione è ben definita su tutto $\amsbb{R}$, mentre per $\alpha>0$ la funzione è definita su $\amsbb{R}\setminus\{0\}$. In entrambi i casi, notiamo che 
    \[
    f(-x) = \abs{-x}^{-\alpha}\arctan\left(\frac{-x}{(-x)^4+1}\right) = -\abs{x}^{-\alpha} \arctan\left(\frac{x}{x^4+1}\right) = -f(x)
    \]
    ossia $f$ è dispari sul proprio dominio di definizione. Consideriamo quindi $I_\alpha$: questo integrale improprio consiste in realtà, come nel caso dell'esercizio \ref{ex:10.2}, nella somma di 4 integrali impropri ``di base'', ossia
    \[
    I_\alpha =  \underbrace{\int_{-\infty}^{-1} f(x)\, dx}_{\stepcounter{equation}\mbox{(\theequation)}} + \underbrace{\int_{-1}^0 f(x)\, dx}_{\stepcounter{equation}\mbox{(\theequation)}} + \underbrace{\int_0^1 f(x)\, dx}_{\stepcounter{equation}\mbox{(\theequation)}} + \underbrace{\int_1^{+\infty} f(x)\, dx}_{\stepcounter{equation}\mbox{(\theequation)}}
    \]
    \addtocounter{equation}{-4}\refstepcounter{equation}\label{eq:10.3}
    \addtocounter{equation}{0}\refstepcounter{equation}\label{eq:10.4}
    \addtocounter{equation}{0}\refstepcounter{equation}\label{eq:10.5}
    \addtocounter{equation}{0}\refstepcounter{equation}\label{eq:10.6}
    Quindi affinché $I_\alpha$ esista devono esistere questi singolarmente questi 4 integrali. Consideriamo ora i fatti seguenti:
    \begin{enumerate}[(i)]
        \item se consideriamo gli integrali (\ref{eq:10.3}) e (\ref{eq:10.6}), che esplicitamente sono rispettivamente
        \[
        \lim_{y\to-\infty} \int_y^{-1} f(x)\, dx \ \text{e} \ \lim_{u\to +\infty} \int_1^u f(x)\, dx
        \]
        notiamo che
        \[
        \begin{split}
            \lim_{y\to-\infty} \int_{y}^{-1} f(x)\, dx &= \lim_{u\to+\infty} \int_{-u}^{-(1)} f(x)\, dx \overset{\text{Teorema \ref{th:8.4}}}{=} \lim_{u\to +\infty} \int_u^{1} -f(-x)\, dx = \\
            & = \lim_{u\to+\infty} \int_1^u f(-x)\, dx = -\lim_{u\to+\infty} \int_1^u f(x)\, dx
        \end{split}
        \]
        Quindi se esiste (\ref{eq:10.6}) automaticamente esiste (\ref{eq:10.3}). Possiamo quindi studiare unicamente il limite (\ref{eq:10.6}). Un analogo risultato vale per i limiti (\ref{eq:10.4}) e (\ref{eq:10.5}).
        \item In virtù del punto (i), notiamo che se (\ref{eq:10.5}) e (\ref{eq:10.6}) esistono allora $I_\alpha$ esiste e vale che
        \[
        I_\alpha = 0
        \]
        Di conseguenza quando $I_\alpha$ esiste, questo vale 0.
    \end{enumerate}
    Determiniamo ora i valori di $\alpha\in\amsbb{R}$ tali che (\ref{eq:10.5}) e (\ref{eq:10.6}) esistono finiti.
    \begin{enumerate}[(i)]
        \item Consideriamo (\ref{eq:10.5}), ossia
        \[
        \lim_{y\to 0^+} \int_y^1 f(x)\, dx 
        \]
        Per determinare se esiste, procediamo come nel caso degli esercizi \ref{ex:10.1} e \ref{ex:10.2}, ossia proviamo ad applicare il teorema \ref{th:10.2} sfruttando l'espansione di Taylor di $f(x)$ in un intorno di $x_0=0$:
        \[
        \arctan\left(\frac{x}{x^4+1}\right) = x+o(x) = x\left(1+\frac{o(x)}{x}\right)
        \]
        quindi
        \[
        f(x) = \abs{x}^{-\alpha} \arctan\left(\frac{x}{x^4+1}\right) = {x}^{-\alpha} x \left(1+\frac{o(x)}{x}\right) = \frac{1}{x^{\alpha-1}} \left(1+\frac{o(x)}{x}\right) 
        \]
        Di conseguenza, se consideriamo $g(x) = \frac{1}{x^{\alpha-1}}$ abbiamo che
        \[
        \lim_{x\to 0^+} \frac{f(x)}{g(x)} = \lim_{x\to 0^+}\frac{1}{x^{\alpha-1}}\left(1+\frac{o(x)}{x}\right)x^{\alpha-1} = \lim_{x\to 0^+} \left(1+\frac{o(x)}{x}\right) = 1
        \]
        e poiché $f$ e $g$ sono entrambe continue e positive in $(0,1]$ abbiamo che
        \[
        \int_0^1 f(x)\, dx \ \text{e} \ \int_0^1 \frac{1}{x^{\alpha-1}}\, dx
        \]
        hanno lo stesso comportamento alla luce del teorema \ref{th:10.2}. Poiché sappiamo che $\int_0^1 \frac{1}{x^{\alpha-1}}\, dx$ esiste se $\alpha<2$, lo stesso vale per (\ref{eq:10.5}).
        \item Se consideriamo (\ref{eq:10.6}), 
        \[
        \lim_{y\to +\infty} \int_1^y f(x)\, dx 
        \]
        per operare come prima bisogna stimare il comportamento di $\arctan\left(\frac{x}{x^4+1}\right)$ per grandi $x$; chiaramente 
        \[
        \lim_{x\to+\infty} \arctan\left(\frac{x}{x^4+1}\right)=0
        \]
        però per decidere se l'integrale esiste o meno dobbiamo stimare la velocità di decadimento della funzione. Per fare questo dobbiamo fare uno ``sviluppo all'$\infty$'' della funzione. A questo scopo, consideriamo i seguenti passaggi:
        \[
        \lim_{x\to +\infty} \arctan\left(\frac{x}{x^4+1}\right) = \lim_{x\to +\infty} \arctan\left(\frac{x}{x^4\left(1+\frac{1}{x^4}\right)}\right ) \overset{y=\frac{1}{x}}{=} \lim_{y\to 0^+} \arctan\left(\frac{y^3}{1+y^4}\right)
        \]
        Quindi il comportamento all'infinito di $\arctan\left(\frac{x}{x^4+1}\right)$ è analogo al comportamento in un intorno di $0$ di $\arctan\left(\frac{y^3}{y^4+1}\right)$. Consideriamone quindi lo sviluppo di Taylor in un intorno di $x_0=0$:
        \[
        \arctan\left(\frac{y^3}{y^4+1}\right) = y^3 + o(y^3)
        \]
        e di conseguenza 
        \[
        \arctan\left(\frac{x}{x^4+1}\right) = \frac{1}{x^3}+o\left(\frac{1}{x^3}\right) \ \text{per} \ x\to +\infty
        \]
        Quindi
        \[
        f(x) = \abs{x}^{-\alpha}\frac{1}{x^3}\left(1+x^3o\left(\frac{1}{x^3}\right)\right) = \frac{1}{x^{3+\alpha}}\left(1+x^3o\left(\frac{1}{x^3}\right)\right)
        \]
        e se definiamo $g(x) = \frac{1}{x^{3+\alpha}}$ abbiamo che
        \[
        \lim_{x\to +\infty} \frac{f(x)}{g(x)} = \lim_{x\to+\infty}\frac{1}{x^{3+\alpha}}\left(1+x^3o\left(\frac{1}{x^3}\right)\right) x^{3+\alpha} = \lim_{x\to+\infty} \left(1+x^3o\left(\frac{1}{x^3}\right)\right) = 1
        \]
        Dato che $f$ e $g$ sono continue e positive su $[1, +\infty)$, per il teorema \ref{th:10.2} abbiamo che
        \[
        \int_1^{+\infty} f(x)\, dx \ \text{e} \ \int_1^{+\infty} \frac{1}{x^{\alpha+3}}\, dx
        \]
        hanno lo stesso comportamento. Dato che $\int_1^{+\infty}\frac{1}{x^{\alpha+3}}\, dx$ esiste se $\alpha>-2$, lo stesso vale per (\ref{eq:10.6}).
    \end{enumerate}
    Possiamo quindi concludere dicendo che $I_\alpha$ esiste per $\alpha\in(-2,2)$ e che per $\alpha$ in questo intervallo $I_\alpha = 0$.
\end{proof}
\begin{exercise}
    \label{ex:10.5}
    Determinare per quali $\alpha, \beta\in\amsbb{R}$ l'integrale
    \[
    \int_1^{+\infty} \frac{\log(x)e^{-x^2}}{(x+1)^\alpha(x-1)^\beta}\, dx
    \]
    esiste.
\end{exercise}
\begin{proof}[Soluzione]
    Come nel caso dell'esercizio \ref{ex:10.4}, i problemi di questo integrale improprio sono due: la singolarità della funzione integranda
    \[
    f(x) = \frac{\log(x)e^{-x^2}}{(x+1)^{\alpha}(x-1)^\beta}
    \]
    in $x=1$ e l'illimitatezza del dominio di integrazione. \`E quindi necessario considerare separatamente i due casi
    \[
    I = \underbrace{\int_1^e f(x)\, dx}_{\stepcounter{equation}\mbox{(\theequation)}} + \underbrace{\int_e^{+\infty}f(x)\, dx}_{\stepcounter{equation}\mbox{(\theequation)}}
    \]
    \addtocounter{equation}{-2}\refstepcounter{equation}\label{eq:10.7}
    \addtocounter{equation}{0}\refstepcounter{equation}\label{eq:10.8}
    \begin{enumerate}[(i)]
        \item Consideriamo l'integrale improprio (\ref{eq:10.7}), che possiamo scrivere come
        \[
        \lim_{y\to 1^+} \int_y^e f(x)\, dx
        \]
        Per poter applicare il teorema \ref{th:10.2} dobbiamo, come nei casi precedenti, determinare il comportamento in un intorno di $x_0=1$ di $f(x)$ sfruttando l'espansione in serie di Taylor:
        \[
        \log(x)=\log(1+(x-1))= (x-1)+o(x-1) \qquad e^{-x^2} = \frac{1}{e} -\frac{2}{e}(x-1) +o(x-1)
        \]
        e
        \[
        \frac{1}{(x+1)^\alpha} = \frac{1}{2^\alpha } -\frac{\alpha}{2^{\alpha+1}}(x-1) +o(x-1)
        \]
        Quindi
        \[
        \begin{split}
            f(x) & = \frac{(x-1)^{1-\beta}}{2^{\alpha}e}\left(1+\frac{o(x-1)}{x-1}\right)\left(1-2(x-1)+o(x-1)\right)\times \\
            & \times \left(1-\frac{\alpha}{2}(x-1)+o(x-1)\right)
        \end{split}
        \]
        e definendo $g(x) = \frac{(x-1)^{1-\beta}}{2^{\alpha}e}$ abbiamo che
        \[
        \lim_{x\to 1^+} \frac{f(x)}{g(x)} = 1
        \]
        Dato che $f$ e $g$ sono positive e continue in $(1, e]$ per il teorema \ref{th:10.2} 
        \[
        \int_1^e \frac{\log(x)e^{-x^2}}{(x+1)^\alpha(x-1)^\beta}\, dx \ \text{e} \ \int_1^e \frac{(x-1)^{1-\beta}}{2^{\alpha}e}\, dx
        \]
        hanno lo stesso comportamento; effettuando passaggi simili a quelli dell'esercizio \ref{ex:10.2} possiamo quindi dire che (\ref{eq:10.7}) esiste se $\beta<2$.
        \item Per quanto riguarda l'integrale (\ref{eq:10.8}), notiamo innanzitutto che 
        \begin{equation}
            \label{eq:10.9}
            \frac{\log(x)e^{-x^2}}{(x+1)^\alpha(x-1)^\beta} \le \frac{xe^{-x^2}}{(x+1)^\alpha(x-1)^\beta} \ \text{per ogni} \ x\in [e,+\infty)
        \end{equation}
        e, se consideriamo $g(x) = x^{1-\beta-\alpha}e^{-x^2}$, abbiamo che
        \[
        \lim_{x\to +\infty} \frac{xe^{-x^2}}{(x+1)^\alpha (x-1)^\beta} \frac{1}{g(x)} = \lim_{x\to +\infty} \frac{x^\alpha}{(x+1)^\alpha} \frac{x^\beta}{(x-1)^\beta} = 1
        \]
        e di conseguenza $\frac{xe^{-x^2}}{(x+1)^\alpha (x-1)^\beta}$ e $g$ hanno lo stesso comportamento per il teorema \ref{th:10.2}. Determiniamo quindi per quali valori di $\alpha, \beta\in\amsbb{R}$ l'integrale
        \[
        \int_e^{+\infty} x^{1-\alpha-\beta} e^{-x^2}
        \]
        esiste. A questo scopo, consideriamo separatamente i casi $1-\alpha-\beta\le 0$ e $1-\alpha-\beta>0$. Nel caso $1-\alpha-\beta=0$ il risultato segue dal fatto che l'integrale $\int_0^{+\infty} e^{-x^2}\, dx$ esiste finito, mentre nel caso $1-\alpha-\beta<0$ possiamo ragionare come segue: la funzione
        \[
        [e,+\infty)\ni x \mapsto x^{1-\alpha-\beta}
        \]
        è monotona decrescente, e quindi
        \[
        x^{1-\alpha-\beta}e^{-x^2}\le e^{1-\alpha-\beta}e^{-x^2} \ \forall x\in[e, +\infty)
        \]
        Dato che $\int_0^{+\infty}e^{-x^2}\, dx$ esiste, grazie al teorema \ref{th:10.1} abbiamo che
        \[
        \int_e^{+\infty} x^{1-\alpha-\beta} e^{-x^2}\, dx
        \]
        esiste finito; per il teorema \ref{th:10.2} segue che
        \[
        \int_e^{+\infty} \frac{xe^{-x^2}}{(x+1)^\alpha(x-1)^\beta}\, dx
        \]
        esiste e quindi grazie al teorema \ref{th:10.1} e alla stima (\ref{eq:10.9}) possiamo concludere che
        \[
        \int_e^{+\infty}\frac{\log(x)e^{-x^2}}{(x+1)^\alpha(x-1)^\beta}\, dx
        \]
        esiste finito; quindi l'integrale improprio esiste se $\alpha+\beta\ge 1$.\\
        Se invece $1-\alpha-\beta>0$, bisogna operare diversamente. Vogliamo mostrare che
        \[
        \int_e^{+\infty}x^{1-\alpha-\beta} e^{-x^2}\, dx 
        \]
        esiste. A questo scopo, restringiamoci inizialmente al caso $1-\alpha-\beta\in\amsbb{N}$ e procediamo per induzione:
        \begin{enumerate}[(a)]
            \item \emph{caso base $n=1$}: vale che
            \[
            \begin{split}
                \int_e^{+\infty} x e^{-x^2}\, dx & = \lim_{y\to +\infty} \int_e^y x e^{-x^2}\, dx = \lim_{y\to +\infty} -\frac{1}{2}\int_e^y e^{-x^2} \frac{d}{dx}(-x^2)\, dx \overset{\text{Teorema \ref{th:8.4}}}{=} \\
                & = \lim_{y\to+\infty} -\frac{1}{2}\int_{-e^2}^{-t^2} e^z\, dz = \frac{1}{2}e^{-e^2} - \lim_{t\to +\infty} \frac{e^{-t^2}}{2} = \frac{1}{2}e^{-e^{2}}
            \end{split} 
            \]
            \item \emph{passo induttivo}: supponiamo che valga per $n\ge 1$, e mostriamo che vale per $n+1$. Consideriamo quindi
            \[
            \begin{split}
                & \int_e^{+\infty} x^{n+1}e^{-x^2}\, dx = \lim_{y\to +\infty} -\frac{1}{2} \int_e^y x^n \frac{d}{dx}(e^{-x^2})\, dx \overset{\text{Teorema \ref{th:8.5}}}{=} \\
                & = \frac{1}{2}e^n e^{-e^2}-\lim_{y\to +\infty}\left(\frac{1}{2}y^ne^{-y^2}-\frac{n}{2}\int_e^{y} x^{n-1} e^{-x^2}\, dx\right)
            \end{split} 
            \]
            Notiamo che 
            \[
            \lim_{y\to+\infty} \frac{1}{2}y^n e^{-y^2} = 0 \ \text{per ogni} \ n\in\amsbb{N}
            \]
            e che
            \[
            \lim_{y\to+\infty}-\frac{n}{2}\int_e^y x^{n-1}e^{-x^2}\, dx  
            \]
            esiste finito per l'ipotesi induttiva\footnote{$x^n\ge x^{n-1}$ per ogni scelta di $x\in[e,+\infty)$ per $n\ge 1$; di conseguenza l'asserzione segue dal teorema \ref{th:10.1}.}. Quindi possiamo applicare il teorema \ref{th:4.2} e concludere che
            \[
            \int_e^{+\infty} x^{n+1} e^{-x^2}\, dx
            \]
            esiste finito.
        \end{enumerate}
        Quindi 
        \[
        \int_e^{+\infty} x^n e^{-x^2}\, dx 
        \]
        esiste finito per ogni $n\in\amsbb{N}$. Noi vogliamo mostrare che
        \[
        \int_e^{+\infty} x^{1-\alpha-\beta} e^{-x^2}\, dx
        \]
        converge per ogni scelta di $\alpha, \beta$ tali che $1-\alpha-\beta >0$. Ma questo segue dalla proprietà archimedea dei numeri reali e dal teorema \ref{th:10.1}: infatti, per ogni scelta di $\alpha,\beta$ tali che $1-\alpha-\beta>0$, esiste un numero naturale $n\in\amsbb{N}$ tale che
        \[
        n>1-\alpha-\beta
        \]
        e ne consegue che possiamo trovare $n\in\amsbb{N}$ tale che
        \[
        x^n e^{-x^2}\ge x^{1-\alpha-\beta} e^{-x^2} \ \text{per ogni} \ x\in[e,+\infty)
        \]
        Per il teorema \ref{th:10.1} vale quindi che
        \[
        \int_e^{+\infty} x^{1-\alpha-\beta} e^{-x^2}\, dx
        \]
        esiste finito per ogni scelta di $\alpha, \beta\in\amsbb{R}$; il teorema \ref{th:10.2} a questo punto ci dice che 
        \[
        \int_e^{+\infty} \frac{xe^{-x^2}}{(x+1)^\alpha (x-1)^\beta}\, dx
        \]
        esiste finito per ogni scelta di $\alpha, \beta\in\amsbb{R}$. Possiamo quindi concludere, grazie alla stima (\ref{eq:10.9}) e al teorema \ref{th:10.2}, che
        \[
        \int_e^{+\infty} \frac{\log(x)e^{-x^2}}{(x+1)^\alpha (x-1)^\beta}\, dx
        \]
        esiste finito per ogni scelta di $\alpha, \beta\in\amsbb{R}$.
    \end{enumerate}
    Quindi per il punto (i) sappiamo che (\ref{eq:10.7}) esiste se $\beta<2$, mentre per il punto (ii) l'integrale (\ref{eq:10.8}) esiste per ogni scelta di $\alpha,\beta\in\amsbb{R}$. Quindi l'integrale
    \[
    \int_1^{+\infty}  \frac{\log(x)e^{-x^2}}{(x+1)^\alpha (x-1)^\beta}\, dx
    \]
    esiste per ogni $\alpha\in\amsbb{R}$ e $\beta<2$.
\end{proof}
\begin{exercise}
    \label{ex:10.6}
    Determinare per quali valori del parametro $x\in\amsbb{R}$ l'integrale
    \[
    \Gamma(x) = \int_0^{+\infty} t^{x-1} e^{-t}\, dt
    \]
    esiste finito.
\end{exercise}
\begin{proof}[Soluzione]
    
    Consideriamo un generico $x\in\amsbb{R}$, con $x\ge1$; possiamo scrivere
    \[
    \lim_{y\to +\infty} \int_0^{y} t^{x-1} e^{-t}\, dt = \lim_{y\to+\infty} \int_{0^2}^{(\sqrt{y})^2} t^{x-1}e^{-t}\, dt = \lim_{y\to +\infty} 2\int_0^{\sqrt{y}} u^{2x-1} e^{-u^2} \, du 
    \]
    Nell'esercizio precedente avevamo già mostrato che, nel caso $2x-1>0$, l'ultimo limite della catena di uguaglianze esiste; pertanto
    \[
    \Gamma(x) = \int_0^{+\infty} t^{x-1} e^{-t}\, dt
    \]
    esiste sicuramente per $x\ge 1$.\\
    Se $x<1$, abbiamo un potenziale problema in $x=0$; dobbiamo pertanto considerare
    \[
    \int_0^{+\infty} t^{x-1} e^{-t}\, dt = \underbrace{\int_0^1 t^{x-1}e^{-t}\, dt}_{\stepcounter{equation}\mbox{(\theequation)}} + \underbrace{\int_1^{+\infty}t^{x-1}e^{-t}\, dt}_{\stepcounter{equation}\mbox{(\theequation)}} 
    \]
    \addtocounter{equation}{-2}\refstepcounter{equation}\label{eq:10.10}
    \addtocounter{equation}{0}\refstepcounter{equation}\label{eq:10.11}
    Seguendo una procedura analoga a quella dell'esercizio \ref{ex:10.5}, è possibile dimostrare che l'integrale (\ref{eq:10.11}) esiste per ogni scelta di $x<1$. Concentriamoci quindi su (\ref{eq:10.10}): possiamo sviluppare con Taylor la funzione integranda per provare ad applicare il teorema \ref{th:10.2}. In particolare abbiamo che
    \[
    e^{-t} = 1-t +o(t)
    \]
    e quindi 
    \[
    \frac{e^{-t}}{t^{1-x}} = \frac{1}{t^{1-x}}(1-t+o(t))
    \]
    Se definiamo $g(t) = \frac{1}{t^{1-x}}$ abbiamo che
    \[
    \lim_{t\to 0^+} \frac{e^{-t}}{t^{1-x}}t^{1-x} = \lim_{t\to0^+} e^{-t} = 1
    \]
    e per il teorema \ref{th:10.2}
    \[
    \int_0^1 \frac{e^{-t}}{t^{1-x}}\, dt \ \text{e} \ \int_0^1 \frac{1}{t^{1-x}}\, dt
    \]
    hanno lo stesso comportamento. Sappiamo che $\int_0^1 \frac{1}{t^{1-x}}\, dt$ esiste se $x>0$, e pertanto anche (\ref{eq:10.10}) esiste se $x>0$. Di conseguenza, $\Gamma(x)$ è definita per $x>0$.
\end{proof}
\begin{remark}
    La funzione $\amsbb{R}^+\ni x \mapsto \Gamma(x)$ è la \emph{funzione Gamma di Eulero}, e ha un'interessante proprietà: infatti, valutandola sui naturali $\mathbb{N}$ e procedendo per induzione:
    \begin{enumerate}[(i)]
        \item \emph{caso base $n=1$}: abbiamo che
        \[
        \Gamma(1) = \int_0^{+\infty} e^{-t}\, dt = \lim_{x\to +\infty} \int_0^x e^{-t}\, dt = 1-\lim_{x\to+\infty}e^{-x} = 1 
        \]
        \item \emph{passo induttivo}: supponiamo che $\Gamma(n)$ sia finito, e mostriamo che anche $\Gamma(n+1)$ lo è. Consideriamo
        \[
        \begin{split}
            \Gamma(n+1) & = \int_0^{+\infty} t^n e^{-t}\, dt = \lim_{x\to +\infty} -\int_0^x t^{n}\frac{d}{dt}(e^{-t})\, dt \overset{\text{Teorema \ref{th:8.5}}}{=} \\
            & = -\lim_{x\to\infty}\left(x^ne^{-x}\right)+\lim_{x\to+\infty} n\int_0^x t^{n-1} e^{-t}\, dt = n \int_0^{+\infty} t^{n-1}e^{-t}\, dt = n\Gamma(n)
        \end{split}
        \]
        Di conseguenza 
        \[
        \Gamma(n+1)=n\Gamma(n) \ \text{per ogni} \ n\in\mathbb{N}
        \]
    \end{enumerate}
    ossia $\Gamma(n+1)=n!$. Quindi in quest'ottica la funzione Gamma di Eulero costituisce una generalizzazione del fattoriale ai numeri reali positivi.
\end{remark}
\newpage