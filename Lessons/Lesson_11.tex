\section{Lezione 11}
\subsection{Ripasso: equazioni differenziali ordinarie}
\begin{definition}
    \label{def:11.1}
    Un'\emph{equazione differenziale ordinaria} di ordine $n$ è un'equazione del tipo
    \[
    F(t, u(t), u'(t), \dots, u^{(n)}(t)) = 0
    \]
    con $F\colon D_F\subseteq \amsbb{R}^{n+2}\to \amsbb{R}$ una data funzione, definita su di un certo dominio $D_F$.\\
    Diremo che un'equazione differenziale ordinaria di ordine $n$ è scritta \emph{in forma normale} se è scritta come
    \[
    u^{(n)}(t) = f(t, u(t), \dots, u^{(n-1)}(t))
    \]
\end{definition}
\begin{remark}
    In generale, data un'equazione differenziale di ordine $n$
    \[
    F(t, u(t), u'(t), \dots, u^{(n)}(t))=0
    \]
    non è sempre possibile ricondursi ad un'equazione in forma normale: ci sono però delle condizioni sufficienti che la funzione $F$ deve soddisfare affinché ciò sia possibile (teorema della funzione implicita per funzioni scalari, in particolare $\partial_{n+2}F \ne 0$).
\end{remark}
\begin{definition}
    \label{def:11.2}
    Data un'equazione differenziale ordinaria di ordine $n$, diremo che la funzione $\phi\colon I \to \amsbb{R}$, definita sull'intervallo $I\subseteq \amsbb{R}$, è una \emph{soluzione} se:
    \begin{enumerate}[(i)]
        \item $\phi\colon I \to \amsbb{R}$ è differenziabile $n$ volte in $I$;
        \item $(t, \phi(t), \dots, \phi^{(n)}(t))\in D_F$ per ogni $t\in I$;
        \item $F(t, \phi(t), \dots, \phi^{(n)}(t))=0$ per ogni $t\in I$.
    \end{enumerate}
\end{definition}
\begin{remark}
    In generale, la proprietà di essere soluzione di un'equazione differenziale è una proprietà locale: ad esempio, consideriamo l'equazione differenziale
    \[
    u'(t) = \frac{1}{\cos(t)+2}
    \]
    La funzione $F\colon \amsbb{R^3}\to \amsbb{R}$ è definita su tutto $\amsbb{R}^3$; tuttavia sappiamo che la soluzione che è possibile trovare (cfr. esercizio \ref{ex:8.7}) è data da
    \[
    u(t) = \frac{2}{\sqrt{3}}\arctan\left(\frac{1}{\sqrt{3}}\tan\left(\frac{x}{2}\right)\right)+c, \quad c\in\amsbb{R}
    \]
    e di conseguenza $u(t)$ è definita su $\amsbb{R}\setminus\{\pi + 2k\pi, \ k\in\amsbb{Z}\}$. Poiché vogliamo che la soluzione sia definita su di un intervallo, $u$ sarà definita su di un intervallo del tipo $(\pi + 2k\pi, \pi + 2(k+1)\pi)$ per qualche $k\in\amsbb{Z}$. La scelta di $k\in\amsbb{Z}$ e del valore di $c\in\amsbb{R}$ dipendono dalle \emph{condizioni iniziali}.
\end{remark}
\begin{definition}
    \label{def:11.3}
    Data un'equazione differenziale ordinaria scritta in forma normale
    \[
    u^{(n)}(t) = f(t, u(t), \dots, u^{(n-1)}(t))
    \]
    un \emph{problema di Cauchy} ad essa associato è un sistema
    \[
    \begin{dcases}
        u^{(n)}(t) = f(t, u(t), \dots, u^{(n-1)}(t))\\
        u(t_0) = u_0\\
        u'(t_0) = u_1\\
        \dots\\
        u^{(n-1)}(t_0) = u_{n-1}
    \end{dcases}
    \]
    ove $(u(t_0), u'(t_0), \dots, u^{(n-1)}(t_0)) = (u_0, u_1, \dots, u_{n-1})$ è detta \emph{condizione iniziale}.
\end{definition}
\begin{remark}
    La soluzione $\phi\colon I \to \amsbb{R}$ deve chiaramente essere compatibile con la condizione iniziale: deve valere che $t_0\in I$ e che $(t_0, \phi(t_0), \dots, \phi^{(n)}(t_0))\in D_F$. Ad esempio, consideriamo il problema di Cauchy
    \[
    \begin{dcases}
        u'(t) = \frac{1}{\cos(t)+2}\\
        u(0) = 0
    \end{dcases}
    \]
    Sappiamo che la soluzione generale dell'equazione differenziale è data da
    \[
    u(t) = \frac{2}{\sqrt{3}}\arctan\left(\frac{1}{\sqrt{3}}\tan\left(\frac{x}{2}\right)\right)+c
    \]
    definita su $(\pi +2k\pi, \pi + 2(k+1)\pi)$. Poiché in questo caso l'istante iniziale $t_0=0$, l'intervallo del tipo precedente che contiene $0$ è quello per $k=-1$, ossia $(-\pi, \pi)$; pertanto $u$ sarà definita su $(-\pi, \pi)$. Inoltre, poiché $u(0) = 0$, vale che
    \[
    0=u(0) = \frac{2}{\sqrt{3}}\arctan\left(\frac{1}{\sqrt{3}}\tan\left(\frac{0}{2}\right)\right)+c = c
    \]
    Quindi la soluzione sarà data da
    \[
    u\colon (-\pi, \pi)\ni t \mapsto \frac{2}{\sqrt{3}}\arctan\left(\frac{1}{\sqrt{3}}\tan\left(\frac{t}{2}\right)\right)
    \]
    In generale, questo è il massimo che possiamo fare: possiamo cioè trovare una soluzione di un problema di Cauchy solo localmente, ossia in un intorno aperto della condizione iniziale. Talvolta però possiamo ``incollare'' soluzioni locali ed ottenere una soluzione globale: è il caso dello svolgimento dell'esercizio \ref{ex:8.7}.
\end{remark}
\subsection{Diverse tipologie di equazioni differenziali ordinarie di ordine 1}
\begin{definition}
    \label{def:11.4}
    Un'equazione differenziale lineare non-omogenea a coefficienti non-costanti è un'equazione differenziale del tipo
    \[
    u'(t) = P(t)u(t) + Q(t)
    \]
    con $P, Q\in\mathscr{C}(I)$. La soluzione generale è una funzione $u\colon I \to \amsbb{R}$ data da
    \begin{equation}
        \label{eq:11.1}
        u(t) = e^{\int P(t)\, dt}\left(c+\int Q(t) e^{-\int P(t)\, dt}\, dt\right)
    \end{equation}
    ove con $\int P(t)\, dt$ denotiamo una generica primitiva di $P$.
\end{definition}
\begin{example}
    Consideriamo il problema di Cauchy
    \[
    \begin{dcases}
        u'(t) = \frac{1}{t}u(t) + 3t^3\\
        u(-1) = 2
    \end{dcases}
    \]
    In questo caso $P(t) = \frac{1}{t}$ e $Q(t) = 3t^3$. Notiamo che $P$ non è definita in $0$; pertanto, dato che vogliamo che $P$ e $Q$ siano continue sull'intervallo $I$ di definizione, dobbiamo restringerci a $(-\infty, 0)$ o $(0, +\infty)$. Dato che l'istante iniziale $t_0=-1$ appartiene al primo insieme, ci restringiamo a $(-\infty, 0)$. In questo caso, le primitive saranno
    \[
    \int P(t)\, dt = \int \frac{1}{t}\, dt = \log\abs{t} = \log(-t)
    \]
    e
    \[
    \int Q(t) e^{-\int P(t)\, dt}\, dt = \int 3t^3 e^{-\log(-t)}\, dt = -\int 3t^2\, dt = -t^3
    \]
    Inserendo i risultati in (\ref{eq:11.1}) otteniamo che la soluzione generale dell'equazione differenziale nell'intervallo $(-\infty, 0)$ è 
    \[
    u(t) = e^{\log(-t)}(c-t^3) = t(t^3-c)
    \]
    Fissiamo ora la costante $c\in\amsbb{R}$ imponendo le condizioni iniziali:
    \[
    2 = u(-1) = (-1)((-1)^3-c) = (c+1) \iff c = 1
    \]
    La soluzione del problema di Cauchy sarà quindi
    \[
    u\colon (-\infty, 0)\ni t \mapsto t(t^3-1)
    \]
\end{example}
\begin{definition}
    \label{def:11.5}
    Un'equazione differenziale non-lineare esatta è un'equazione differenziale del tipo
    \[
    u'(t) = -\frac{P(t,u)}{Q(t,u)}
    \]
    con $P, Q \colon A\subseteq\amsbb{R}^2 \to \amsbb{R}$ dove $A$ è un sottoinsieme semplicemente connesso\footnote{Vuol dire ``senza buchi''.} di $\amsbb{R}^2$, $Q\ne 0$ su $A$, $P,Q$ sono sufficientemente regolari su $A$ e sono tali\footnote{Le notazioni $\partial_t$ e $\partial_u$ denotano le derivate parziali rispetto a $t$ e $u$; all'atto pratico dovete derivare nel primo caso rispetto a $t$ trattando i termini in $u$ come costanti, e nel secondo caso i termini in $u$ trattando come costanti i termini dipendenti da $t$.} che
    \[
    \partial_u P(t,u) = \partial_t Q(t,u)
    \]
     Dato un punto $(\tau, \varepsilon)\in A$ esiste una funzione definita su di un intorno aperto $U$ di $(\tau, \varepsilon)$, $F\colon U\to\amsbb{R}$, data da
     \begin{equation}
         \label{eq:11.2}
         F(t,u) = \int_\tau^t P(s,u)\, ds + \int_\varepsilon^u Q(\tau, s)\, ds
     \end{equation}
     tale che una soluzione generica dell'equazione differenziale esatta è data in forma \emph{implicita} da $F(t,u)=c$.\\
     Se stiamo considerando un problema di Cauchy
     \[
     \begin{dcases}
         u'(t) = -\frac{P(t,u)}{Q(t,u)} \\
         u(t_0) = u_0
     \end{dcases}
     \]
     allora è possibile scegliere $(\tau, \varepsilon) = (t_0, u_0)$, e in quel caso le soluzioni saranno date in forma implicita da $F(t,u) = 0$.
\end{definition}
\begin{example}
    Consideriamo il problema di Cauchy
    \[
    \begin{dcases}
        u'(t) = -\frac{t+u}{t-3u}\\
        u(0) = 1
    \end{dcases}
    \]
    Notiamo innanzitutto che la funzione a denominatore si annulla se $t=3u$; dobbiamo pertanto considerare uno dei due insiemi
    \[
    \left\{(t,u)\in\amsbb{R}^2\colon t<3u\right\} \qquad \left\{(t,u)\in\amsbb{R}^2\colon t>3u\right\}
    \]
    Poiché la condizione iniziale $(t_0, u_0) = (0, 1)$ appartiene al primo insieme, scegliamo quello come dominio $A$ (che è semplicemente connesso). Notiamo inoltre che se definiamo
    \[
    P(t) = t+u \qquad Q(t) = t-3u
    \]
    abbiamo che
    \[
    \partial_u P(t,u) = \partial_u(t+u) = 1 \qquad \partial_t Q(t,u) = \partial_t(t-3u) = 1
    \]
    L'equazione differenziale è quindi un'equazione differenziale esatta (cfr. \ref{def:11.5}), e la soluzione è data in forma implicita da $F(t,u) = 0$, con $F(t,u)$ data dalla (\ref{eq:11.2}):
    \[
    F(t,u) = \int_0^t (s+u)\, ds -\int_1^{u}3s\, ds = \frac{t^2}{2} + ut +\frac{3}{2}u^2-\frac{3}{2}
    \]
    ossia l'equazione implicita è
    \[
    \frac{t^2}{2} +ut-\frac{3}{2}u^2 +\frac{3}{2}=0\iff t^2 +2ut -3u^2+3=0
    \]
    In questo caso è possibile esplicitare la soluzione $u(t)$: consideriamo infatti
    \[
    u(t) = \frac{-t\pm \sqrt{t^2+3t^2+9}}{-3} = \frac{t\pm \sqrt{4t^2+9}}{3}
    \]
    Per capire quale fra i due segni prima della radice è opportuno tenere, consideriamo la condizione iniziale: abbiamo che
    \[
    1 = u(0) = \frac{\pm\sqrt{9}}{3}
    \]
    La soluzione corretta dell'equazione è quella con il segno positivo; pertanto la soluzione dell'equazione differenziale è data da
    \[
    u(t) = \frac{t+\sqrt{4t^2+9}}{3}
    \]
\end{example}
\begin{definition}
    \label{def:11.6}
    Un'equazione differenziale a variabili separabili è un'equazione del tipo 
    \[
    u'(t) = f(t)g(u)
    \]
    con $f\colon I\to \amsbb{R}$ e $g\colon J \to \amsbb{R}$ sufficientemente regolari, e $g\ne 0$ su $J$. Poiché $g(u)\ne 0$ per ogni $u\in J$, possiamo definire
    \[
    P(t,u) = f(t) \qquad Q(t,u) = -\frac{1}{g(u)}
    \]
    con $P, Q \colon I \times J \to \amsbb{R}$. Notiamo che
    \begin{enumerate}[(i)]
        \item $P$ e $Q$ sono definite sul rettangolo $I \times J$ che è semplicemente connesso;
        \item $\partial_u P(t,u) = \partial_u f(t) = 0$ e $\partial_t Q(t,u) = \partial_t g(u)^{-1}=0$;
        \item $u'(t) = f(t)g(u) = -\frac{P(t,u)}{Q(t,u)}$.
    \end{enumerate}
    Di conseguenza, possiamo dire che le equazioni differenziali a variabili separabili sono un sottocaso di quelle esatte; pertanto dato un punto $(t_0, u_0)\in I\times J$ la soluzione generale è data in forma implicita da $F(t,u)=c$, con $F(t,u)$ definita dalla (\ref{eq:11.2}) come
    \begin{equation}
        \label{eq:11.3}
        F(t,u) = \int_{t_0}^t P(s,u)\, ds + \int_{u_0}^u Q(t_0, s)\, ds = \int_{t_0}^t f(s) - \int_{u_0}^u \frac{1}{g(s)}\, ds
    \end{equation}
\end{definition}
\begin{example}
    Consideriamo il problema di Cauchy
    \[
    \begin{dcases}
        u'(t)=2t\sqrt{1-u^2}\\
        u(0) = \frac{1}{2}
    \end{dcases}
    \]
    Notiamo che in questo caso abbiamo
    \[
    f(t) = 2t \qquad g(u) = \sqrt{1-u^2}
    \]
    con $g\colon[-1,1]\to \amsbb{R}^+$. Chiaramente $g(u)\ge 0$ per ogni $u\in[-1,1]$, e $g(u)=0$ se e solo se $u=\pm 1$. Poiché dobbiamo necessariamente avere $g(u)\ne 0$, dobbiamo restringerci all'intervallo aperto $(-1,1)$. $f$ è invece definita su tutto $\amsbb{R}$; pertanto il dominio di definizione dell'equazione a variabili separabili, intesa come equazione differenziale esatta, è $\amsbb{R}\times (-1,1)$, e la soluzione è data da (\ref{eq:11.3}):
    \[
    0=F(t,u) = \int_{0}^t 2s\, ds -\int_{\frac{1}{2}}^u\frac{1}{\sqrt{1-s^2}}\, ds = t^2 - \arcsin(u)+ \arcsin\left(\frac{1}{2}\right) = t^2-\arcsin(u)+\frac{\pi}{6}
    \]
    L'equazione che da la soluzione del problema in forma implicita è quindi
    \[
    t^2+\frac{\pi}{6}=\arcsin(u)
    \]
    Dato che l'immagine di $\arcsin(u)$ per $u\in(-1,1)$ è $\left(-\frac{\pi}{2}, \frac{\pi}{2}\right)$, anche $t^2+\frac{\pi}{6}$ appartiene a questo insieme; pertanto dobbiamo restringere il dominio di $t$ a $\left(-\sqrt{\frac{\pi}{3}}, \sqrt{\frac{\pi}{3}}\right)$. \\
    Anche in questo caso possiamo esprimere $u$ in forma esplicita come
    \[
    u(t)=\sin\left(t^2+\frac{\pi}{6}\right), \quad t\in\left(-\sqrt{\frac{\pi}{3}}, \sqrt{\frac{\pi}{3}}\right)
    \]
\end{example}
\begin{remark}
    Per risolvere le equazioni differenziali a variabili separabili, è possibile effettuare delle manipolazioni algebriche informali per facilitarne la risoluzione. Innanzitutto, possiamo scrivere
    \[
    u'(t) = \frac{du}{dt}
    \]
    L'equazione differenziale a questo punto può essere scritta come
    \[
    \frac{du}{dt} = 2t\sqrt{1-u^2} 
    \]
    Se immaginiamo di trattare $\frac{du}{dt}$ come una frazione otteniamo, supponendo $u\ne \pm1$,
    \[
    \frac{du}{dt} = 2t\sqrt{1-u^2} \iff \frac{1}{\sqrt{1-u^2}}du = (2t) dt
    \]
    A questo punto possiamo immaginare di poter scrivere
    \[
    \int \frac{1}{\sqrt{1-u^2}}\, du = \int 2t\, dt
    \]
    ossia
    \[
    \arcsin(u) = t^2+c
    \]
    La soluzione a cui siamo giunti è quindi fondamentalmente la stessa di prima.
\end{remark}
\begin{definition}
    \label{def:11.7}
    Un'equazione differenziale ordinaria è un'equazione di Bernoulli se è un'equazione del tipo
    \[
    u'(t) = P(t)u(t)+Q(t)(u(t))^\alpha
    \]
    con $\alpha\in \amsbb{R}\setminus\{0,1\}$ un parametro reale e $P,Q$ continue su $I$. Notiamo che se $\alpha\notin \amsbb{Z}$, allora necessariamente $u(t)\ge 0$ sul proprio dominio di definizione.\\
    In questo caso, chiaramente $u(t) \equiv 0$ è una soluzione banale dell'equazione. Supponiamo ora $u(t)\ne 0$ nel dominio di definizione della soluzione; possiamo moltiplicare ambo i membri per $u(t)^{-\alpha}$, ottenendo
    \[
    u(t)^{-\alpha} u'(t) = P(t) u(t)^{1-\alpha} + Q(t)
    \]
    Notiamo inoltre che 
    \[
    \frac{d}{dt}(u(t))^{1-\alpha} \overset{(\ref{eq:6.4})}{=} (1-\alpha) u(t)^{-\alpha} u'(t)
    \]
    Quindi
    \[
    \frac{1}{1-\alpha} \frac{d}{dt}(u(t)^{1-\alpha}) = P(t) u(t)^{1-\alpha} + Q(t)
    \]
    Se definiamo $\varphi(t) = u(t)^{1-\alpha}$ abbiamo quindi che l'equazione differenziale risulta essere
    \[
    \varphi'(t) = (1-\alpha)P(t)\varphi(t) + (1-\alpha)Q(t)
    \]
    L'equazione che abbiamo ottenuto è descritta dalla definizione \ref{def:11.4}, e la soluzione è data da (\ref{eq:11.1}):
    \[
    \varphi(t) = e^{\int (1-\alpha)P(t)\, dt} \left(c+\int (1-\alpha)Q(t) e^{-\int(1-\alpha)P(t)\, dt}\right)
    \]
    Una volta trovata $\varphi(t)$, è possibile ricondursi a $u(t)$ considerando 
    \[
    u(t) = \varphi(t)^{\frac{1}{1-\alpha}}
    \]
    ove ben definita.
\end{definition}
\begin{example}
    Consideriamo il problema di Cauchy
    \[
    \begin{dcases}
        u'(t) = -t u(t) + t^3u(t)^2\\
        u(1) = 1
    \end{dcases}
    \]
    Questa è un'equazione di Bernoulli in cui l'esponente $\alpha$ è $2$, con $P(t) = -t$ e $Q(t) = t^3$ definite e continue su tutto $\amsbb{R}$. Di conseguenza, consideriamo $\varphi(t) = u(t)^{1-\alpha} = u(t)^{-1}$; l'equazione differenziale diventa quindi
    \[
    \varphi'(t) = t\varphi(t)-t^3
    \]
    La soluzione è data dalla (\ref{eq:11.1}): consideriamo quindi
    \[
    \int t\, dt = \frac{t^2}{2} \quad -\int t^3 e^{-\frac{t^2}{2}}\, dt = \int t^2 \frac{d}{dt}e^{-\frac{t^2}{2}}\, dt \overset{\text{Teorema \ref{th:8.5}}}{=} t^2e^{-\frac{t^2}{2}}-2\int e^{-\frac{t^2}{2}} t\, dt = t^2e^{-\frac{t^2}{2}}+2e^{-\frac{t^2}{2}}
    \]
    Di conseguenza
    \[
    \varphi(t) = e^{\frac{t^2}{2}}\left(c+t^2 e^{-\frac{t^2}{2}}+2e^{-\frac{t^2}{2}}\right) = ce^{\frac{t^2}{2}}+t^2+2
    \] 
    Poiché $u(t)= \varphi(t)^{-1}$ abbiamo che la soluzione generale dell'equazione differenziale è
    \[
    u(t) = \frac{1}{t^2+2+ce^{\frac{t^2}{2}}}
    \]
    Imponendo la condizione iniziale abbiamo
    \[
    1 = u(1) = \frac{1}{1+2+ce^{\frac{1}{2}}} \iff c = -2e^{-\frac{1}{2}}
    \]
    e quindi la soluzione del problema di Cauchy è data da
    \[
    u(t) = \frac{1}{t^2+2-2e^{\frac{t^2}{2}-\frac{1}{2}}}
    \]
    Notiamo che in questo caso abbiamo che il denominatore è
    \[
    q(t) = t^2+2-2e^{\frac{t^2}{2}-\frac{1}{2}}
    \]
    e questa funzione è tale per cui $q(0)>0$ e $\lim_{t\to\pm\infty}q(t) = -\infty$. Pertanto esistono almeno due punti $t_0, t_1\in\amsbb{R}$ tali che $t_0<0$, $t_1>0$ e $q(t_0) = q(t_1) = 0$. Studiando la derivata di $q$ è possibile mostrare che $t_1>1$; di conseguenza $u(t)$ sarà definita su $(t_0, t_1)$.
\end{example}
\subsection{Esercizi: equazioni differenziali ordinarie e problemi di Cauchy}
\begin{exercise}
    \label{ex:11.1}
    Sia $u\colon I \to\amsbb{R}$ la soluzione del seguente problema di Cauchy:
    \begin{equation}
        \label{eq:11.4}
        \begin{dcases}
            u'(t)e^t +\sin(t)u(t) = \cos(u(t))\\
            u(0) = 0
        \end{dcases}
    \end{equation}
    Determinare il comportamento locale della soluzione $u$ in un intorno di $t_0=0$.
\end{exercise}
\begin{proof}[Soluzione]
    Ricordiamo che per determinare il comportamento locale di $u$ dobbiamo capire se è crescente o decrescente e concava o convessa in un intorno di $t_0=0$. Dobbiamo quindi determinare $u'(0)$ e $u''(0)$. Poiché $u'(t)$ è determinato dall'equazione differenziale in (\ref{eq:11.4}), possiamo valutare ambo i membri dell'equazione in $t_0=0$ ottenendo
    \[
    u'(0)e^0 +\sin(0)u(0)=\cos(u(0)) \iff u'(0) = \cos(0) \iff u'(0) = 1
    \]
    Di conseguenza $u'(0)>0$, e quindi la funzione $u$ è crescente in un intorno di $t_0=0$.\\
    Per calcolare $u''(0)$, possiamo sfruttare l'equazione differenziale in (\ref{eq:11.4}) nel modo seguente: possiamo derivare ambo i membri dell'equazione ottenendo così
    \[
    u''(t)e^t + u'(t)e^t + \cos(t)u(t) + \sin(t)u'(t) = -\sin(u(t))u'(t)
    \]
    Se valutiamo l'equazione precedente in $t_0=0$ otteniamo
    \[
    u''(0) e^0 +u'(0)e^0 +\cos(0)u(0)+\sin(0)u'(0) = -\sin(u(0))u'(0) 
    \]
    ossia
    \[
    u''(0) + 1 = 0 \iff u''(0)=-1
    \]
    Quindi $u''(0)<0$, e di conseguenza $u$ è concava in un intorno di $t_0=0$.
\end{proof}
\begin{exercise}
    \label{ex:11.2}
    Dato il problema di Cauchy
    \begin{equation}
        \label{eq:11.5}
        \begin{dcases}
            n'(t) = r n(t)\left(1-\frac{n(t)}{k}\right)\\
            n(0) = n_0
    \end{dcases}
    \end{equation}
    determinarne la soluzione $n(t)$ in funzione dei parametri $r>0$ e $k>0$ e della condizione iniziale $n_0\ge 0$.
\end{exercise}
\begin{proof}[Soluzione]
    Notiamo che possiamo scrivere l'equazione differenziale come
    \[
    n'(t) = \frac{r}{k}n(t)(k-n(t))
    \]
    Notiamo che la funzione $n(t)\equiv 0$ è una soluzione del problema di Cauhcy se $n_0=0$; supponiamo quindi $n_0>0$. L'equazione differenziale scritta come prima è un'equazione differenziale a variabili separabili (cfr. definizione \ref{def:11.6}), con
    \[
    f(t) = \frac{r}{k} \qquad g(n) = n(k-n)
    \]
    Notiamo che $g(n)\ne 0$ se $n\ne k$ e $n\ne 0$; quindi dobbiamo restringerci ad uno degli insiemi
    \[
    \{n\in\amsbb{R}\colon n>k\} \qquad \{n\in\amsbb{R}\colon 0<n<k\} \qquad \{n\in\amsbb{R}\colon n<0\}
    \]
    a seconda della condizione iniziale $n_0$. Chiaramente escludiamo l'ultimo insieme, visto che $n_0>0$. 
    \begin{enumerate}[(i)]
        \item Caso $n_0>k$: allora l'insieme a cui dobbiamo restringerci è $\{n\in\amsbb{R}\colon n>k\}$; il dominio in cui cerchiamo soluzioni è quindi $\amsbb{R}\times \{n\in\amsbb{R} \colon n>k\}$; applicando la formula risolutiva (\ref{eq:11.3}) otteniamo che la soluzione è data in forma implicita da
        \[
        \begin{split}
            F(t,n) &= \int_{0}^t \frac{r}{k}\, ds -\int_{n_0}^n \frac{1}{s(k-s)}\, ds = \frac{r}{k}t-\int_{n_0}^n \frac{1}{k}\frac{1}{s} +\frac{1}{k}\frac{1}{k-s}\, ds = \\
            & = \frac{r}{k}t-\frac{1}{k}\left(\log\abs{n}-\log\abs{n_0}-\log\abs{k-n}+\log\abs{k-n_0}\right) = \\
            & = \frac{r}{k}t -\frac{1}{k}\log\abs{\frac{n}{k-n}}-\frac{1}{k}\log\abs{\frac{k-n_0}{n_0}} = \\
            & = \frac{r}{k}t+\frac{1}{k}\log\left(\frac{n-k}{n}\right)-\frac{1}{k}\log\left(\frac{n_0-k}{n_0}\right) =0
        \end{split}
        \]
        Possiamo esplicitare la soluzione scrivendo
        \[
        -rt = \log\left(\frac{n-k}{n}\frac{n_0}{n_0-k}\right)
        \]
        e quindi
        \[
        n(t) = \frac{k}{1-\frac{n_0-k}{n_0}e^{-rt}}
        \]
        Notiamo che $n(t)$ è definita per $t\ne -\frac{1}{r}\log\left(\frac{n_0}{n_0-k}\right)<0$; dato che $t_0=0$, l'intervallo di esistenza della soluzione è $\left(-\frac{1}{r}\log\left(\frac{n_0}{n_0-k}\right), +\infty\right)$.
        \item Caso $n_0<k$: in questo caso dobbiamo restringerci a $\{n\in\amsbb{R}\colon 0<n<k\}$, e quindi il dominio in cui cerchiamo soluzioni è $\amsbb{R}\times \{n\in\amsbb{R}\colon 0<n<k\}$. La funzione $F(t,n)$ si ottiene allo stesso modo, ma in questo caso abbiamo
        \[
        F(t,n) = \frac{r}{k}t+\frac{1}{k}\log\left(\frac{k-n}{n}\right)-\frac{1}{k}\log\left(\frac{k-n_0}{n_0}\right) = 0
        \]
        ossia, esplicitando $n(t)$,
        \[
        n(t) = \frac{k}{1+\frac{k-n_0}{n_0}e^{-rt}}
        \]
        \item Infine, notiamo che per $n_0=k$ l'equazione differenziale ammette una soluzione: infatti se $n(t)\equiv k$ allora $n$ soddisfa il problema di Cauchy.
    \end{enumerate}
    In tutti i casi, notiamo che 
    \[
    \lim_{t\to+\infty} n(t) = k
    \]
    \begin{center}
        \begin{tikzpicture}
        \pgfplotsset{
            axis lines=middle,
            xlabel style=below,
            ylabel style=left
        }
            \begin{axis}[y=7mm, ymin=-.2, ymax = 21, xmin = -.2, xmax = 11, xscale =1.5, yscale = .5, samples=201, xlabel={$t$}, ylabel = {$n$}, xtick={1, 2, ..., 10}, ytick ={2, 4, ..., 20}, legend style={at={(axis cs: 8,15)},anchor=west}, legend cell align=left]
            \addplot [name path=A, domain=0:10, black] {10};
            \addlegendentry{$n_0=k$}
            \addplot[domain=0:10, red]{10/(1+(10-8)/8*exp(-0.5*x))};
            \addlegendentry{$n_0=0.8k$}
            \addplot[domain=0:10, teal]{10/(1+(10-2)/2*exp(-0.5*x))};
            \addlegendentry{$n_0=0.2k$}
            \addplot[domain=0:10, blue]{10/(1+(10-0.2)/0.2*exp(-0.5*x))};
            \addlegendentry{$n_0=0.02k$}
            \addplot[domain=0:10, violet]{10/(1-(20-10)/20*exp(-0.5*x))};
            \addlegendentry{$n_0=2k$}
            \addplot[domain=0:10, orange]{10/(1-(12-10)/12*exp(-0.5*x))};
            \addlegendentry{$n_0=1.2k$}
            \end{axis}
        \end{tikzpicture}
    \end{center}
\end{proof}
\begin{remark}
    L'equazione differenziale dell'esercizio \ref{ex:11.2} descrive una popolazione di individui che compete per un delle risorse limitate e che si riproduce ad un certo tasso $r$. $k$ rappresenta il numero di individui che il dato insieme di risorse può sostenere stabilmente. Le sue soluzioni sono dette \emph{curve logistiche}. 
\end{remark}
\newpage